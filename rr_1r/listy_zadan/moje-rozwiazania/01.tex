\documentclass{article}

\usepackage{../../../notatki}
\pgfplotsset{samples=800}

\begin{document}

\subsection*{ZADANIE 1.}
\emph{Sprawdź, czy podana funkcja jest rozwiązaniem podanego równania różniczkowego:}
\smallskip

{\color{dark-green}(a) $x(t)=\tan t$, $x'=1+x^2$} //YUP

$$x'(t)={1\over cos^2t}$$
$$1+x^2=1+{\sin^2t\over \cos^2t}={\cos^2t+\sin^2t\over\cos^2t}={1\over\cos^2t}=x'$$

{\color{dark-green}(b) $x(t)={\sin t\over t}$, $tx'+x=\cos t$} //YUP

$$x'(t)={t\cos t-\sin t\over t^2}$$
$$tx'+x={t\cos t-\sin t\over t}+{\sin t\over t}={t\cos t\over t}=\cos t$$

\subsection*{ZADANIE 2.}
\emph{Znaleźć rozwiązania ogólne (tzn. rozwiązania zależne od pewnej stałej $C$) następujących równań różniczkowych o rozdzielonych zmiennych i naszkicować ich wykresy dla różnych stałych $C$:}
\smallskip

{\color{dark-green}(a) $y'=e^{x+y}$}

\begin{align*}
    {dy\over dx}&=e^{x+y}\\
    e^{-y}dy&=e^xdx\\
    \int e^{-y}dy&=\int e^xdx\\
    -e^{-y}&=e^x+c\\
    e^{-y}&=c-e^x\\
    \ln(e^{-y})&=\ln(c-e^x)\\
    y&=-\ln(c-e^x)
\end{align*}

Niech $c=e^m$ dla pewnego $m$, bo wiadomo, że aby ta funkcja była gdziekolwiek określona, to $(c-e^x>0)$ na pewnym przedziale, czyli $c>0$. Rozważmy teraz dwa przypadki: $x<0$ i $x\geq 0$. Dla $x\in(-\infty, 0)$ funkcja będzie coraz bardziej zbliżać się do wartości $m$, bo $e^x$ będzie dążyć do $0$, ale nigdy go nie osiągnie. Dla $x\in [0, m)$ funkcja będzie maleć z $\lim_{x\to m}\ln(c-e^x)=-\infty$. Czyli wykres wygląda tak dla $m=5$: (i tutaj użyję sobie paczuszki do rysowania grafów bo czemu nie XD)

\begin{illustration}
    \begin{my-axis}
        \addplot[domain=-6:4.999, very thick, orange] {ln(e^5-e^x)};
    \end{my-axis}
\end{illustration}

{\color{dark-green}(b) $y'={\sqrt{x}\over y}$}

\begin{align*}
    {dy\over dx}&={\sqrt{x}\over y}\\
    ydy&=\sqrt{x}dx\\
    \int ydy&=\int\sqrt{x}dx\\
    \frac12y^2&=\frac23x^{\frac32}+c\\
    y&=\pm\sqrt{\frac43x^{\frac32}+c}
\end{align*}

W tym przypadku możemy mieć dodatnie i ujemne $c$, ale $x$ musi być liczbą dodatnią, inaczej $x^{\frac32}$ nie istnieje. Dla dodatniego $c$ wiemy, że wartość w $x=0$ będzie wynosić $\sqrt{c}$. Wtedy mamy funkcję wyglądającą jak:
\begin{illustration}
    \begin{my-axis}
        \addplot[domain=-0.1:5, very thick, orange] {sqrt(4/3*x^(3/2)+6)};
    \end{my-axis}
\end{illustration}

Natomiast dla $c<0$ wiemy, że będziemy ruszać gdzie graf się "zaczyna". To znaczy dla $x<\frac34c^{\frac23}$ funkcja jest nieokreślona, a dla $x\geq \frac34c^{\frac23}$ wygląda troszkę jak funkcja pierwiastka $4$-tego stopnia z $x^3$?
\begin{illustration}
    \begin{my-axis}
        \addplot[domain=0:5, very thick, orange] {sqrt(4/3*x^(3/2)-6)};
    \end{my-axis}
\end{illustration}

{\color{dark-green}(c) $y'=\sqrt{{y\over x}}$}
\begin{align*}
    {dy\over dx}&={\sqrt{y}\over\sqrt{x}}\\
    {dy\over \sqrt{y}}&={dx\over\sqrt{x}}\\
    \int y^{-\frac12}dy&=\int x^{-\frac12}dx\\
    2y^{\frac12}&=2x^\frac12+c\\
    y&=x+2c\sqrt{x}+c^2
\end{align*}

Dla $c<0$ to jakaś potężna okejka, a dla $c\geq 0$ to se leeeeci w górę. Już mi się nie chce rozrysowywać.

\subsection*{ZADANIE 4.}
\emph{Szybkość zmiany temperatury rozgrzanego czajnika jest proporcjonalna do różnicy między jego temperaturą a temperaturą powietrza (prawo Newtona). Niech $S(t)$ oznacza temperaturę czajnika w chwili $t$. Zakładamy, że $S(0)=100^oC$ w temperaturze otoczenia $20^oC$. Po dziesięciu minutach temperatura czajnika wynosiła $60^oC$. Po ilu minutach czajnik będzie miał temperaturę $25^oC$?}

Wiemy, że pochodna funkcji temperatury od czasu jest wprost proporcjonalna do różnicy w temperaturze:
$${dS\over dt}=a(S-20).$$
Rozwiążmy to równanie
\begin{align*}
    {dS\over dt}&=a(20-S(t))\\
    {dS\over 20-S(t)}&=adt\\
    \int {dS\over 20-S(t)}&=\int adt\\
    -\ln|20-S(t)|&=at+c\\
    |20-S(t)|&=e^{-at-c}
\end{align*}
Tutaj ciało stygnie, więc $S(t)>20$, czyli
$$|20-S(t)|=S(t)-20.$$
Z treści zadania wiemy, że $S(0)=100$, co po podstawieniu da nam wartość $c$:
\begin{align*}
    100-20&=e^{-c}\\
    80&=e^{-c}\\
    \ln80&=-c\\
    c&=-\ln80
\end{align*}
Dalej, podstawiając $S(10)=60$ możemy poznać wartość $a$:
\begin{align*}
    60-20&=e^{-10a+\ln80}\\
    40&=e^{-10a}\cdot 80\\
    \frac12&=e^{-10a}\\
    \ln\frac12&=-10a\\
    a&=\ln2^{\frac1{10}}
\end{align*}
Dostajemy wzór:
$$S(t)=20+e^{-t\ln2^{1\over10}+\ln80}$$
po ilu minutach będziemy mieli temperaturę $25^oC$? Jeszu nie wiem i nie chce tego zmieniać tbh.
\begin{align*}
    25-20&=e^{-ta-c}\\
    \ln5&=-ta-c\\
    \ln5+\ln80&=-ta\\
    10\ln400&=t\ln2\\
    10\ln20^2&=t\ln2\\
    \ln20^{20}&=t\ln2
\end{align*}

\subsection*{ZADANIE 5.}
\emph{Modelujemy rozprzestrzenianie się plotki w populacji liczącej 1000 osób. Załóżmy, że $5$ osób rozprzestrzenia plotkę i po jednym dniu wie o niej już $10$ osób. Ile czasu potrzeba, aby o plotce dowiedziało się $850$ osób, przy założeniu, że:}

\indent 1. \emph{Plotka rozprzestrzenia się z prędkością proporcjonalną do iloczynu liczby osób, które już słyszały tę plotkę oraz liczby osób, które jeszcze nie słyszały tej plotki.}

\indent 2. \emph{Plotka rozprzestrzenia się według prawa Gompertza: $y'=ke^{-(73/520)t}$.}

\emph{Porównaj te dwa modele i otrzymane wyniki.}
\smallskip

{\color{dark-green}Wersja 1:}

Będziemy modelować $y(t)$, czyli ilość osób, które plotkę już usłyszały w zależności od czasu $t$. Wiemy, że prędkość rozprzestrzeniania się plotki, to znaczy $y'$, jest wprost proporcjonalna do $y(1000-y)$. Rozwiążmy to cudeńko
\begin{align*}
    {dy\over dt}&=ay(1000-y)\\
    {dy\over y(1000-y)}&=adt\\
    \int {dy\over y(1000-y)}&=\int adt\\
    {\ln y-\ln(1000-y)\over 1000}&=at+c\\
    \ln{y\over 1000-y}&=at+c\\
    {y\over 1000-y}&=e^{at+c}\\
    y&=(1000-y)e^{at+c}\\
    y(1+e^{at+c})&=1000e^{at+c}\\
    y&={1000e^{at+c}\over 1+e^{at+c}}\\
    y&={1000\over e^{-at-c}+1}
\end{align*}

To teraz będzie podstawianko-sranko. Gdy $t=0$ mamy $y=5$, czyli
\begin{align*}
    5&={1000\over e^{-c}+1}\\
    e^{-c}+1&=200\\
    e^{-c}&=199\\
    -c&=\ln1999\\
    c&=-\ln199
\end{align*}
natomiast, gdy $t=1$, to $y=10$, czyli
\begin{align*}
    10&={1000\over e^{-a+\ln199}+1}\\
    e^{-a}\cdot199+1&=100\\
    e^{-a}&={99\over199}\\
    -a&=\ln{99\over199}\\
    a&=\ln{199\over99}
\end{align*}
Teraz chce rozwiązać
\begin{align*}
    850&={1000\over e^{t\ln{99\over199}+\ln199}+1}\\
    199\cdot\left({99\over199}\right)^t&={1000\over 850}-1\\
    {99^t\over199^{t-1}}&={3\over17}\\
    99^t\cdot17&=3\cdot199^{t-1}\\
    \ln(99^t\cdot17)&=\ln(3\cdot199^{t-1})\\
    t\ln99+\ln17&=\ln3+(t-1)\ln199\\
    t(\ln99-\ln199)&=\ln{3\over199\cdot17}\\
    t\ln{99\over199}&=\ln{3\over199\cdot17}
\end{align*}
\smallskip

{\color{dark-green}Wersja 2:}

Teraz chcemy rozwiązać równanie już nam dane:
\begin{align*}
    {dy\over dx}&=ke^{-(73/520) t}\\
    dy&=ke^{-(73/520)t}dt\\
    \int dy&=\int ke^{-(73/520)t}dt\\
    y&=-{520\over 73}ke^{-(75/520)t}+c
\end{align*}

To lecimy z podstawianiem:
\begin{align*}
    &\begin{cases}
        5=-{520\over73}k+c\\
        10=-{520\over73}ke^{-73/520}+c
    \end{cases}\\
    &5=-{520\over73}k+{520\over73}ke^{-73\over520}\\
    &365=k(520e^{-73/520}-520)\\
    &{365\over 520e^{-73/520}-520}=k
\end{align*}

\end{document}