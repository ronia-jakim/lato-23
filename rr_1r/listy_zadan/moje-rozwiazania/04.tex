\documentclass{article}

\usepackage{../../../lecture_notes}

\title{Lista 4\\{\normalsize Równania różniczkowe 1R}}
\author{}
\date{}

\begin{document}
\maketitle
\thispagestyle{empty}

\begin{problem}{}
Załóżmy, że funkcja $f=f(t,y)$ jest klasy $C^1$ na zbiorze $t_0\leq t<\infty$, $-\infty<y<\infty$ oraz spełnia dodatkowe oszacowanie $|f(t,y)|\leq K$ na całym tym zbiorze dla pewnej stałej $K>0$. Udowodnić, że rozwiązanie zagadnienia
$$y'=f(t,y),\quad y(t_0)=y_0$$
istnieje dla wszystkich $t\geq t_0$.
\end{problem}

Euler generalnie robi małe kroczki i w ten sposób dostaje szacowanie rozwiązania równania.

Jeśli $h\to 0$, to spełniona jest równość:
$$y'(t)=\frac{y(t+h)-y(t)}{h}$$
To ja tak sobie mogę dojść od $t_0$ do $t$ lols



\end{document}
