\documentclass{article}

\usepackage{../../../notatki}

\author{Julia Walczuk, Weronika Jakimowicz}

\begin{document}

Pokazać, że istnieją $A_1,...,A_n$ takie, że:

$$\prod\limits_{i=1}^n{1\over x-x_i}=\sum\limits_{i=1}^n{A_i\over x-x_i}$$

% Pomnóżmy sobie obie strony przez $(x-x_1)...(x-x_n)$
% \begin{align*}
%     1&=\sum\limits_{i=1}^nA_i\prod\limits_{j=1,j\neq i}^n(x-x_j)
% \end{align*}

Pierścień wielomianów $K[X]$ nad ciałem $K$ jest zawsze domeną Euklidesową, a ponieważ $\R$ zdecydowanie jest ciałem, to śmiga, bo mamy rozkład
$$c=Q(x)A+(x-x_1)B$$
$${c\over(x-x_1)Q(x)}={A\over (x-x_1)}+{B\over Q(x)}$$
gdzie $Q(x)=(x-x_2)...(x-x_n)$, a resztę mamy z trywialnej indukcji

\end{document}