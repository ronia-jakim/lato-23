\documentclass{article}

\usepackage{../../../notatki}

\begin{document}

\subsection*{ZADANIE 11.}
\emph{\color{pink}Spadek kamienia pod wpływem siły grawitacji, z uwzględnienim oporu powietrza, jest opisany równaniem}
$$\color{pink}x''(t)=-g+k(x'(t))^2,\quad k>0$$
\emph{\color{pink}Pokaż, że po długim czasie porusza się on z prędkością graniczną, tzn. $\lim\limits_{t\to\infty}x'(t)=-\left({g\over k}\right)^\frac12$.}
\smallskip

Duuupa dupa dupa

Ułatwię sobie życie i skoro $x'$ to prędkość, to będę je oznaczać przez $v$. Wtedy mam
$$v'=-g+kv^2$$

Czy ja nie mam po prostu pokazać, że to ma asymptotę poziomą?


% Mam, że
% \begin{align*}
%     \int_{t_0}^tv'ds=&\int_{t_0}^t[-g+kv^2]ds\\
%     v-v_0&=\int_{t_0}^t[-g+kv^2]ds=-gt+gt_0+k\int_{t_0}^tv^2ds\\
%     v&=v_0+gt_0-gt+k\int_{t_0}^tv^2ds\\
%     \int v^2ds&=\begin{bmatrix}
%         u=v^2&du=2v\\
%         w=s&dw=1
%     \end{bmatrix}=sv^2-2\int svds\\
%     \int svds&=\begin{bmatrix}
%         u=v&du=v'\\
%         w=\frac12 s^2&dw=s
%     \end{bmatrix}=\frac12s^2v-\frac12\int v's^2ds
% \end{align*}

\end{document}