\documentclass{article}

\usepackage{../../notatki}

\title{Algebra 2R\smallskip\\{\scriptsize a voyage into unknown}}
\author{koteczek}
\date{$\sim$}

\begin{document}
\maketitle

\subsection*{SYLABUS:}

\deff{I. Podstawy teorii równań algebraicznych}

\indent 1. Rozszerzenia ciał. Rozszerzenia o pierwiastek wielomianu nierozkładalnego. Ciało rozkładu wielomianu: istnieje, jedyność.

\indent 2. Ciało algebraicznie domknięte: definicja. Każde ciało zawiera się w ciele algebraicznie domkniętym (konstrukcja). Podciało proste: istnienie, jedyność. Ciała proste.

\indent 3. Pierwiastki z jedności, pierwiastki pierwotne. Grupa pierwiastków z jedności w ciele: każda jej skończona podgrupa jest cykliczna. Wielomiany podziału koła. Funkcja Frobeniusa. Ciała skończone: własności.
\smallskip

\deff{II. Teoria Galois}

\indent 1. Rozszerzenia [elementy] algebraiczne, przestępne: definicja. Stopień rozszerzenia. Warunki równoważne algebraiczności. Wielomian minimalny elementu ciała nad podciałem, własności.

\indent 2. Algebraiczne domknięcie ciała: definicja, istnienie, jedyność, własności (jednorodność). Istnienie rzeczywistych liczb przestępnych, liczby Liouville'a.

\indent 3. Rozszerzenia normalne: definicja, własności. Rozszerzenia [elementy, wielomiany] rozdzielcze. Twierdzenie Abela o elemencie pierwotnym. Rozszerzenia czysto nierozdzielcze (radykalne): definicja, własności. Stopień rozdzielczy [radykalny] rozszerzenia: definicja, własności.

\end{document}