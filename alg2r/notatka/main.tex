\documentclass{article}

\usepackage{../../notatki}

\title{Algebra 2R\smallskip\\{\scriptsize a voyage into the unknown}}
\author{koteczek}
\date{$\sim$}

\makeatletter
\let\latexl@section\l@section
\def\l@section#1#2{\begingroup\let\numberline\@gobble\latexl@section{#1}{#2}\endgroup}
\makeatother

\begin{document}
\maketit

%Pomoce dydaktyczne:
%\href{https://www.youtube.com/playlist?list=PL8yHsr3EFj53Zxu3iRGMYL_89GDMvdkgt}{playlista z losowymi wykladami}

% \subsection*{SYLABUS:}

% \deff{I. Podstawy teorii równań algebraicznych}

% \indent 1. Rozszerzenia ciał. Rozszerzenia o pierwiastek wielomianu nierozkładalnego. Ciało rozkładu wielomianu: istnieje, jedyność.

% \indent 2. Ciało algebraicznie domknięte: definicja. Każde ciało zawiera się w ciele algebraicznie domkniętym (konstrukcja). Podciało proste: istnienie, jedyność. Ciała proste.

% \indent 3. Pierwiastki z jedności, pierwiastki pierwotne. Grupa pierwiastków z jedności w ciele: każda jej skończona podgrupa jest cykliczna. Wielomiany podziału koła. Funkcja Frobeniusa. Ciała skończone: własności.
% \smallskip

% \deff{II. Teoria Galois}

% \indent 1. Rozszerzenia [elementy] algebraiczne, przestępne: definicja. Stopień rozszerzenia. Warunki równoważne algebraiczności. Wielomian minimalny elementu ciała nad podciałem, własności.

% \indent 2. Algebraiczne domknięcie ciała: definicja, istnienie, jedyność, własności (jednorodność). Istnienie rzeczywistych liczb przestępnych, liczby Liouville'a.

% \indent 3. Rozszerzenia normalne: definicja, własności. Rozszerzenia [elementy, wielomiany] rozdzielcze. Twierdzenie Abela o elemencie pierwotnym. Rozszerzenia czysto nierozdzielcze (radykalne): definicja, własności. Stopień rozdzielczy [radykalny] rozszerzenia: definicja, własności.

%\newpage

\tableofcontents

\newpage

\includegraphics[width=\textwidth]{./kytel.jpg}
\newpage

\section{Teoria równań algebraicznych}

Przez $R, S$ będziemy oznaczać pierścienie przemienne z $1\neq0$, natomiast $K, L$ będziemy rezerwować dla oznaczeń ciał.

\subsection{Układy równań}

Rozważmy funkcje $f_1,...,f_m\in R[X_1, ..., X_n]$. Dla wygody będziemy oznaczać krotki przez $\overline X$, czyli $R[X_1,...,X_n]=R[\overline X]$. Pojawia się problem: \dyg{czy istnieje rozszerzenie pierścieni z jednością $R\subseteq S$ takie, że układ $U:f_1(\overline X)=...=f_m(\overline X)=0$ ma rozwiązanie w pierścieniu $S$?}

\begin{fakt}
    $\overline a=(a_1,...,a_n)\subseteq S$, gdzie $S$ jest rozszerzeniem pierścienia $R$, jest \acc{rozwiązaniem układu równań} $U\iff g(\overline a)=0$ dla każdego wielomianu $\color{blue}g\in (f_1,...,f_m)\normalsubgroup R[X]$.
\end{fakt}

\textbf{Dowód:} 

$\impliedby$ Implikacja jest dość trywialna, jeśli każdy wielomian z $(f_1,...,f_m)$, czyli wytworzony za pomocą sumy i produktu wielomianów $f_1,...,f_m$ zeruje się na $\overline a$, to musi zerować się też na każdym z tych wielomianów.

$\implies$ Rozważamy dwa przypadki:

\indent 1. $(f_1,...,f_m)\ni b\neq 0$ i $b\in R$. 

To znaczy w $(f_1,...,f_m)$ mamy pewien niezerowy wyraz wolny. Wtedy mamy wielomian $g\in(f_1,...,f_m)$ taki, że $g(\overline a)\neq 0$. Ale przecież $g$ jest kombinacką wielomianów $f_1,...,f_m$, która na $\overline a$ przyjmują wartość $0$. W takim razie dostajemy układ sprzeczny i przypadek jest do odrzucenia.

\indent 2. $(f_1,...,f_m)\cap R=\{0\}$. (nie ma wyrazów wolnych różnych od $0$)

Teraz wiemy, że układ $U$ jest niesprzeczny, a więc możemy skonstruować pierścień z $1$ $S$ będący rozszerzeniem $R$ [$S\supseteq R$] oraz rozwiązanie $\overline a\subseteq S$ spełniające nasz układ równań.

Niech $S=R[\overline X]/(f_1,...,f_m)$ i rozważmy
$$j:R[\overline X]\to S=R[\overline{X}]/(f_1,...,f_m)$$
nazywane \acc{przekształceniem ilorazowym}. Po pierwsze, zauważmy, że $j\obciete R$ jest $1-1$, bo
$$\ker(j\obciete R)=\ker(j)\cap R=(f_1,...,f_m)\cap R=\{0\}$$
i dlatego
$$j\obciete R:R\izo{}j[R]\subseteq S.$$
Z uwagi na ten izomorfizm, będziemy utożsamiać $R, j[R]$. W takim razie, $S$ jest rozszerzeniem pierścienia $R$. Czyli mamy rozszerzenie pierścienia $R$.

Niech 
$$\overline a=(a_1,...,a_m)=(j(X_1),...,j(X_n))\subseteq S,$$
czyli jako potencjalne rozwiązanie rozważamy zbiór obrazów wielomianów stopnia $1$ przez wcześniej zdefiniowaną funkcję $j:R[\overline X]\to S$. Tak zdefiniowane $\overline a$ jest rozwiązaniem układu $U$ w pierścieniu $S$, bo dla funkcji wielomianowej (czyli zapisywalnej jako wielomian) $\hat{f_i}\in(f_1,...,f_m)$ mamy
$$\hat{f_i}(\overline a)=\hat{f_i}(j(X_1),...,j(X_m))=j(\hat{f_i}(X_1,...,X_m))=j(f_i)=0.$$

{\color{orange}\large TUTAJ TRZEBA POUZASADNIAĆ KILKA RÓWNOŚCI, ALE MOŻE NIE BĘDĘ TEGO ROBIŁA NA AISD}

\begin{uwaga}
    \label{uwaga1:1:2-warunek-rozwiazanie-ogolne}
    Skonstruowane powyżej {rozwiązanie $\overline a$} układu $U$ ma następującą własność {uniwersalności}:

    ($\coffee$) Jeżeli $S'\supseteq R$ jest rozszerzeniem pierścienia z $1$ i $\overline a'=(a_1',...,a_m')\subseteq S$ jest rozwiązaniem $U$ w $S'$, to istnieje jedyny homomorfizm 
    $$h:R[\overline a]\to R[\overline a']$$ 
    taki, że $h\obciete R$ jest identycznością na $R$ i $h(\overline a)=\overline a'$. \acc{Wszystkie rozwiązania układów są homomorficzne.}
\end{uwaga}

\begin{illustration}
    \node (R) at (0, 0) {$R$};
    \node (R[a]) at (4, 0) {$R[\overline a]\subseteq S$};
    \node (R[a']) at (0, -3) {$R[\overline a']\subseteq S'$};
    \draw [ ->] (R)--(R[a]) node [midway, above] {$\subseteq$};
    \draw [->] (R)--(R[a']) node [midway, left] {$\subseteq$};
    \draw[->] (R[a])--(R[a']) node [midway, right] {$h$};
\end{illustration}

Tutaj \deff{$R[\overline a]\subseteq S$ jest podpierścieniem generowanym przez $R\cup\{\overline a\}$}, czyli zbiór:
$$R[\overline a]=\{f(\overline a)\;:\;f(\overline X)\in R[\overline X]\}\subseteq S$$

\textbf{Dowód:} Niech $I=\{g\in R[\overline X]\;:\;g(\overline a')=0\}\subseteq S'$. Oczywiście mamy, że $I\normalsubgroup R[\overline X]$, a więc
$$(f_1,...,f_m)\subseteq I.$$
Z twierdzenia o faktoryzacji wie
\begin{illustration}
    \node (R[x]) at (0, 0) {$R[\overline X]$};
    \node (S) at (4, 0) {$S=R[\overline X]/(f_1,...,f_m)$};
    \node (R[a']) at (0, -3) {$S'\supseteq R[\overline a']$};
    \draw[->] (R[x])--(S) node [midway, above] {$j$};
    \draw[->] (R[x])--(R[a']) node [midway, left] {$\phi$};
    \draw[->, dashed] (S)--(R[a']) node [midway, right] {$(\exists\;!h)\;h(\overline a)=\overline a'$};
\end{illustration}
Homomorfizm $\phi:R[\overline X]\to R[\overline a']$ określamy wzorem
$$\phi(w)=w(\overline a),$$
a homomorfizm $j$ jest jak wyżej odwzorowaniem ilorazowym. Widzimy, że 
$$I=\ker(\phi)$$
$$\ker(j)=(f_1,...,f_m).$$
Z twierdzenia o homomorfizmie pierścieni dostajemy jedyny homomorfizm 
$$h:R[X]/(f_1,...,f_m)\to R[\overline a]$$
taki, że $h(\overline a)=\overline a'$.

\begin{uwaga}
    Jeśli $I=(f_1,...,f_m)$, to $h:R[\overline a]\izo R[\overline a']$.
\end{uwaga}

Wtedy mamy $\ker\phi=\ker j$, czyli $\ker(h\circ j)=\ker\phi=\ker j$, no a z tego wynika, że $\ker h$ jest trywialne, czyli $h$ jest apimorfizmem (1-1). Z drugiej strony, $Im \phi=Im(h\circ j)$, a $\phi$ jest epimorfizmem ("na"), więc również $h$ musi być "na".
\medskip

\begin{important}
Załóżmy, że $S\supseteq R$ jest rozszerzeniem pierścienia oraz $\overline a\in S^n$. Wtedy:

\indent 1. ideał $\overline a$ nad $R$ definiujemy jako 
$$\color{blue}I(\overline a/R)=\{g\in R[\overline X]\;:\;g(\overline a)=0\}$$

\indent 2. $\overline a$ nazywamy \deff{rozwiązaniem ogólnym} układu $U$, jeśli ideał 
$$I(\overline a/R)=(f_1,...,f_m).$$
\end{important}

\begin{uwaga}
    W sytuacji jak z definicji wyżej, gdy $U$ jest układem niesprzecznym, wtedy 
    
    $\overline a$ jest rozwiązaniem ogólnym układu $U$ $\iff$ zachodzi warunek \hyperref[uwaga1:1:2-warunek-rozwiazanie-ogolne]{(\coffee)}.
\end{uwaga}

\textbf{Dowód:} Ćwiczenia.

\subsection[Rozszerzenia ciał]{Zasadniczy temat: ciała}

Dla $K\subseteq L$ ciał i $\overline a\subseteq L$ definiujemy \deff{ideał $\overline a$ nad $K$} jako:
$$\color{blue}I(\overline a /L):=\{f(X_1,...,X_n)\in K[\overline X]:f(\overline a)=0\},$$
to znaczy generujemy ideał w wielomianach nad $K$ zawierający wszystkie wielomiany (niekoniecznie tylko jednej zmiennej) zerujące się w $\overline a$. 
\medskip

\textbf{Przykład:}
\smallskip

Dla $K=\Q,L=\R,n=1,a_1=\sqrt{2}$ mamy
$$I(\sqrt{2}/\Q)=\{f(x^2-2)\;:\;f\in\Q[X]\}=(x^2-2)\normalsubgroup \Q[X]$$

Dalej, definiujemy
$$\color{blue}K[\overline a]:=\{f(\overline a)\;:\;f\in K[X]\}$$
czyli \deff{podpierścień $L$ generowany przez $K\cup\{\overline a\}$} oraz \deff{$K(\overline a)$, czyli podciało $L$} generowane przez $K\cup\{\overline a\}$:
$$\color{blue}K(\overline a):=\{f(\overline a)\;:\;f\in K(X_1,...,X_n)\;i\;f(\overline a)\text{ dobrze określone}\}.$$
Tutaj $K(X_1,...,X_n)$ to \dyg{ciało ułamków pierścienia} $K[\overline a]$ w ciele $L$ (czyli najmniejsze ciało, że pierścień może być w nim zanurzony). Czasami oznaczamy to przez $K[\overline a]_0$.
\medskip

% \textbf{Przykład:}
% \smallskip

% Dla $K=\Q,L=\R$ zachodzi:
% $$K[\sqrt2]=\Q[\sqrt2]=\{q+p\sqrt2\;:\;q,p\in\Q\}$$
% $$K[\sqrt2,\sqrt3]=\Q[\sqrt2,\sqrt3]$$
% $$K(\sqrt2)=\Q[\sqrt2]$$
% to ostatnie to usuwanie niewymierności z mianownika.
% \medskip

\begin{uwaga}
    \label{uwga:1:1:5}
    Niech $K\subseteq L_1,K\subseteq L_2$ będą ciałami. Wybieramy $\overline a_1\in L_1$ i $\overline a_2\in L_2$, $|\overline a_1|=|\overline a_2|=n$. Wtedy następujące warunki są równoważne:

\indent 1. istnieje izomorfizm $\phi:K[\overline a_1]\to K[\overline a_2]$ taki, że $\phi\obciete K=id_K$ oraz $\phi(\overline a_1)=\overline a_2$.

\indent 2. $I(\overline a_1/K)=I(\overline a_2/K)$.
\end{uwaga}

\textbf{Dowód:}

%\begin{center}
% $K[\overline a]\isomorphism K[\overline b]$ \acc{$\implies$} $I(\overline a/K)=I(\overline b/K)$

% Niech $\omega\in K[\overline X]$. Wtedy $\omega\in I(\overline a/K)$ wtedy i tylko wtedy, gdy $\omega(\overline a)=0$, to mamy z definicji $I(\overline a/K)$. Wiemy też, że $\phi(a)\in K[\overline X]$ wtedy, gdy $\omega(\phi(\overline a))=0$, a ponieważ $\phi(\overline a)=\overline b$, to również $\omega(\overline b)=0$ i mamy, że $\omega\in I(\overline b/K)$. Czyli izomorfizm między $K[\overline a]=K[\overline b]$ implikuje, że $I(\overline a/K)=I(\overline b/K)$.
% \smallskip

% $K[\overline a]\isomorphism K[\overline b]$ \acc{$\impliedby$} $I(\overline a/K)=I(\overline b/K)$

% Spróbujmy zdefiniować izomorfizm $\phi$ tak, że dla $\omega\in K[\overline X]$ mamy $\phi(\omega(\overline a))=\omega(\overline b)$

% 1. $\phi$ jest homomorfizmem: 
% $$\phi(\omega(\overline a)\cdot v(\overline a))=f((\omega\cdot v)(\overline a))=(\omega\cdot v)(\overline b)=\omega(\overline b)\cdot v(\overline b)=\phi(\omega(\overline a))\cdot \phi(v(\overline a))$$

% 2. $\phi$ jest różnowartościowe:
% $$\phi(\omega(\overline a))=\phi(v(\overline a))\iff \omega(\overline b)=v(\overline b)\iff (\omega-v)(\overline b)=0\iff \omega-v\in I(\overline b/K)=I(\overline a/K)\iff (\omega-v)(\overline a)=0\iff \omega(\overline a)=v(\overline a)$$

% 3. $\phi$ jest dobrze zdefiniowane (czyli przyjmuje tylko jedną wartość dla jednego argumentu):
% $$\omega(\overline a)-v(\overline a)=0\iff (\omega-v)(\overline a)=0\iff \omega -v\in I(\overline a/K)\iff \omega-v\in I(\overline b/K)\iff (\omega-v)(\overline b)=0\iff \omega(\overline b)-v(\overline b)=0$$

% \podz{dark-green}
% \bigskip

% Możemy teraz zapytać, czy każdy ideał w pierścieniu wielomianów $K[X]$ jest postaci $I(\overline a/K)$ dla pewnego $\overline a\in L\supset K$? Albo ogólniej, czy dla pierścienia przemiennego $R$ z $1_R\neq0_R$ oraz ideału $I=(f_1,...,f_m)=I(\overline a/R)\normalsubgroup R[X]$, czy istnieje nadpierścień $S$ taki, że $1_S=1_R$ i $0_S=0_R$ oraz układ
% $$f_1(\overline x)=...=f_m(\overline m)=0$$
% ma rozwiązanie w $S$? Takie rozwiązanie spełniałoby $\overline a\in S\iff (\forall\;g\in(f_1,...,f_m))\;g(\overline a)=0$.
%\end{center}

$1\implies 2$

Implikacja jest jasna, bo dla $g(\overline X)\in K[\overline X]$, bo $g(\overline a_1)=0$ w $K[\overline a_1]$ $\iff$ $g(f(\overline a_1))=0$, a $f(\overline a_1)=\overline a_2$.

$1\impliedby 2$

Zwróćmy uwagę na odwzorowanie ewaluacji $\overline a_1$
$$\phi_{\overline a_1}:K[\overline X]\xrightarrow[]{"na"}K[a_1]$$
zadane wzorem
$$\phi(w(\overline X))=w(\overline a_1).$$
Mamy
$$\ker(\phi_{\overline a_1})=I(\overline a_1/K).$$

Tak samo dla $\overline a_2$ możemy określić analogicznie odwzorowanie ewaluacyjne $\phi_{\overline a_2}:K[\overline X]\to K[\overline a_2]$. Wtedy
$$I(\overline a_2/K)=\ker(\phi_{\overline a_2}),$$
ale ponieważ $I(\overline a_1/K)=I(\overline a_2/K)$, to $\ker(\phi_{\overline a_1})=\ker(\phi_{\overline a_2})$. Oznaczmy $I=I(\overline a_1/K)=I(\overline a_2/K)$. Widzimy, że $\phi_{\overline a_i}\obciete K=id_k$.

\begin{illustration}
    \node (K) at (4, 0) {$K[\overline X]$};
    \node (K[a1]) at (0, -2) {$K[\overline a_1]$};
    \node (K[a2]) at (8, -2) {$K[\overline a_2]$};
    \node (K[X]/I) at (4, -2.5) {$K[X]/I$};
    \draw[->] (K)--(K[a1]) node [midway, left] {$\phi_{\overline a_1}$};
    \draw[->] (K)--(K[a2]) node [midway, right] {$\phi_{\overline a_2}$};
    \draw[->] (K)--(K[X]/I) node [midway, right] {$j${\scriptsize - ilorazowe}};
    \draw[->, dashed] (K[X]/I)--(K[a1]) node [midway, below] {$f_1$} node [midway, above] {$\cong$};
    \draw[->, dashed] (K[X]/I)--(K[a2]) node [midway, below] {$f_2$} node [midway, above] {$\cong$};
\end{illustration}

Niech $f=f_2f_1^{-1}:K[\overline a_1]\to K[\overline a_2]$ jest funkcją spełniającą warunki punktu 1.

{\large\color{orange}MOŻE TUTAJ ŁADNIE SPRAWDZIĆ ŻE NAPRAWDĘ JEST TO DOBRZE SPEŁNIAJĄCA WARUNKI FUNKCJA?}

\begin{uwaga}
    Niech $I\normalsubgroup K[\overline X]$ \acc{noetherowskiego} pierścienia $K[\overline X]$. Niech $I=(f_1,...,f_m)$ dla pewnych $f_i\in K[\overline X]$. Wtedy istnieje rozszerzenie pierścienia $S\supseteq K$ oraz $\overline a\subseteq S$ - rozwiązanie ogólne układu $f_1(\overline X)=...=f_m(\overline X)=0$ takie, że $\color{blue}I(\overline a/K)=I$.
\end{uwaga}

\textbf{Dowód:} Wcześniejsze uwagi {\large\color{orange}KTÓRE KONKRETNIE?}

\begin{tw}
    Niech $I\normalsubgroup K[\overline X]$. Wtedy istnieje ciało $L\supseteq K$ oraz $\overline a=(a_1,...,a_n)\subseteq L$ takie, że $f(\overline a)=0$ dla każdego $f\in I$.
\end{tw}

\textbf{Dowód:} Niech $I\subseteq M\normalsubgroup K[\overline X]$ będzie ideałem maksymalnym. Niech $L=K[\overline X]/M$ i określmy przekształcenie ilorazowe
$$j:K[\overline X]/M\to L=K[\overline X]/M.$$
Ponieważ $M\cap K=\{0\}$ (bo inaczej w ideale byłby wielomian odwracalny), to $j\obciete K:K\to L$ jest funkcją $1-1$, czyli
$$j\obciete K:K\xrightarrow[]{1-1}j[K]\subseteq L.$$
Możemy utożsamić $K$ z $j[K]$, czyli $K\subseteq L$. Niech $\overline a=(a_1,..., a_n)$ takie, że dla każdego $i\in[n]$ 
$$a_i=j(X_i)\in L.$$
Wtedy $g(\overline a)=0$ dla każdego $g(\overline X)\in M\supseteq I$ (bo inaczej mielibyśmy wyrazy wolne).

\begin{wniosek}
    \label{wniosek1:2:4}
    Niech $f\in K[X]$ stopnia $>0$. Wtedy istnieje ciało $L\supseteq K$ rozszerzające ciało $K$ takie, że $f$ ma pierwiastek w ciele $L$.
\end{wniosek}

\textbf{Przykłady:}

\indent 1. Rozpatrzmy ciało $K=\Q$ i $f(X)=X-2$. Wtedy $I=(f)\normalsubgroup\Q[X]$ jest ideałem maksymalnym, bo jest on pierwszy (w tym wypadku nierozkładalny). Równanie $f=0$ ma rozwiązanie ogólne w pierścieniu ilorazowym
$$\Q[X]/I\cong \Q.$$
Czyli nie zawsze musimy rozszerzać ciało do czegoś nowego.

\indent 2. $\C=\R[i]=\R(i)=\R[z]$ dla każdego $z\in\C\setminus\R$, co jest na liście zadań.
\medskip

Załóżmy, że $K\subseteq L_1, K\subseteq L_2$ są rozszerzeniami ciała. Wtedy mówimy, że \deff{$L_1$ jest izomorficzne z $L_2$ nad $K$} [\acc{$L_1\cong_KL_2$}] $\iff$ istnieje izomorfizm $f:L_1\to L_2$ taki, że $f\obciete K=id_K$.

\begin{fakt}{\color{back}dupa }
\label{fakt:1:2:5}

\indent 1. Załóżmy, że $f(X)\in K[X]$ jest nierozkładalny. Niech $L_1=K(a_1)$, $L_2=K(a_2)$ i $f(a_i)=0$ w $L_i$. Wtedy $L_1\cong_KL_2$.

\indent 2. Ogółniej: załóżmy, że $\phi:K_1\to K_2$ jest izomorfizmem i $f_1\in K_1[X],f_2\in K_2[X]$, $\phi(f_1)=f_2$, $f_i$ jest nierozkładalne. Dodatkowo załóżmy, że $L_1=K_1(a_1)$ i $L_2=K_2(a_2)$, gdzie $f_i(a_i)=0$ w $L_i$. Wtedy istnieje \acc{izomorfizm $\phi\in\psi:L_1\to L_2$ taki, że $\psi(a_1)=a_2$.}
\end{fakt}

\textbf{Dowód:}

\indent 1. $I(a_1/K)=(f)=I(a_2/K)$, stąd na mocy \ref{uwga:1:1:5} mamy $K(a_1)\cong_KK(a_2)$. Po dowodzie przypadku 2. możemy uzasadniać, że jest to szczególny przypadek tego ogólniejszego stwierdzenia właśnie.

\indent 2. Zacznijmy od rozrysowania tej sytuacji:

\begin{illustration}
    \draw (-2, 0) ellipse (1.5 and 0.7) node [below] {$K_1$};
    \draw (2, 0) ellipse (1.5 and 0.7) node [below] {$K_2$};
    \draw (-3.5, 0)..controls (-3, 3) and (-1, 3)..(-0.5, 0) node [midway, above] {$L_1$};
    \draw (3.5, 0)..controls (3, 3) and (1, 3)..(0.5, 0) node [midway, above] {$L_2$};
    \draw[->] (-2, 1)..controls (-1, 2) and (1, 2)..(2, 1) node [midway, below] {$\psi$} node [midway, above] {$\cong$};
    \draw[->] (-1.2, -0.6)..controls (-0.8, -1) and (0.8, -1)..(1.2, -0.6) node [midway, below] {$\phi$} node [midway, above] {$\cong$};
\end{illustration}

Izomorfizm $\phi:K_1[X]\izo K_2[X]$ indukuje nam przekształcenie
$$K_1[X]/(f_1)\izo{\phi} K_2[X]/(f_2),$$
bo $\phi(f_1)=f_2$. Wiemy, że $f_i$ jest nierozkładalne, czyli
$$I(a_i/K_i)=(f_i)\normalsubgroup K_i[X]$$
jest ideałem maksymalnym. Mamy
$$L_i=K_i(a_i)=K_i[a_i]\cong K[X]/I(a_i/K_i).$$

\begin{illustration}
    \node (K1) at (0, 0) {$K_1[X]$};
    \node (K2) at (4, 0) {$K_2[X]$};
    \draw[->] (K1)--(K2) node [midway, above] {$\cong$} node [midway, below] {$\phi$};
    \node (K1f) at (0, -2) {$K_1[X]/(f_1)$};
    \node (K2f) at (4, -2) {$K_2[X]/(f_2)$};
    \draw[->] (K1f)--(K2f) node [midway, below] {$\phi$} node [midway, above] {$\cong$};
    \draw[->] (2, -.7)--(2, -1.3);
    \node (L1) at (0, -4) {$L_1=K_1(a_1)$};
    \node (L2) at (4, -4) {$L_2=K_2(a_2)$};
    \draw[->] (L1)--(L2) node [midway, above] {$\cong$} node [midway, below] {$\psi$};
    \node (K1) at (0, -5) {$K_1$};
    \node (K2) at (4, -5) {$K_2$};
    \draw[->] (K1)--(K2) node [midway, below] {$\phi$};
    \node[rotate=90] at (0, -4.5) {$\subseteq$};
    \node[rotate=90] at (4, -4.5) {$\subseteq$};
    \draw[->] (K1f)--(L1) node [midway, left] {$\cong$} node [midway, right] {$h_1$};
    \draw[->] (K2f)--(L2) node [midway, left] {$\cong$} node [midway, right] {$h_2$};
\end{illustration}

\begin{important}
Ciało $L\supseteq K$ nazywamy \deff{ciałem rozkładu nad $K$} wielomianu $f\in K[X]$, gdy spełnione są warunki:

\indent 1. $f$ rozkłada się w pierścieniu $L[X]$ na czynniki liniowe (stopnia $1$)

\indent 2. Ciało $L$ jest rozszerzeniem ciała $K$ o elementy $a_1,...,a_n$, gdzie $a_1,...,a_n$ to wszystkie pierwiastki $f$ w $L$.
\end{important}

\textbf{Przykład:} Jeżeli $deg(f)=0$, to nie istnieje ciało rozkładu $f$.

\begin{wniosek}
    Załóżmy, że $f\in K[X]$ jest wielomianem stopnia $>0$. Wtedy

\indent 1. istnieje $L$: ciało rozkładu $f$ nad $K$,

\indent 2. to ciało jest jedyne z dokładnością do izomorfizmy nad $K$.
\end{wniosek}

\textbf{Dowód:}

\indent 1. Dowód przez indukcje względem stopnia $f$

Jako przypadek bazowy rozważmy $f$ takie, że $deg(f)=1$. Wtedy $L=K$ i wszystko wniosek jest spełniony.

Załóżmy teraz, że stopień wielomianu $f$ jest $>1$ i tez zachodzi dla wszystkich wielomianów stopnia $<deg(f)$ i wszystkich ciał $K'$. Teraz z \ref{wniosek1:2:4} wiemy, że istnieje rozszerzenie ciała $L\supseteq K$ takie, że $f$ ma pierwiastek w $L$. Nazwijmy ten pierwiastek $a_0$ i niech
$$K'=K(a_0).$$
Ponieważ $K'[X]$ wielomian $f$ ma pierwiastek $a_0$, to możemy zapisać
$$f=(x-a_0)f_1$$
dla pewnego $f_1\in K'[X]$ i $deg(f_1)<deg(f)$. Z założenia indukcyjnego dla $f_a$ istnieje $L'=K'(a_1,...,a_r)$ - ciało rozkładu wielomianu $f_1$ nad $K'$. Wtedy 
$$L=K(a_0,...,a_r)$$
jest ciałem rozkładu $f$ nad $K$.

\indent 2. Udowodnimy wersję ogólniejszą: 
\phantomsection
\label{stwierdzenie:wniosek}

\emph{(\bat) Jeśli $\phi:K_1\izo{} K_2$ jest izomorfizmem nad ciałem i $f_i\in K_i[X]$ jest wielomianem stopnia $>0$, $\phi(f_1)=f_2$, to wtedy istnieje $ \psi:L_1\izo{} L_2$ izomorfizm nad ciałami rozkładu $f_i$ w $K_i$ rozszerzający izomorfizm $\phi$ (to znaczy $\phi\subseteq \psi$).}

Wykorzystamy indukcję po $deg(f)$. W przypadku bazowym mamy $deg(f)=1$, czyli $L_1=K_1,L_2=K_2$ i $\phi=\psi$.

Teraz niech $deg(f)>1$ i załóżmy, że dla wszystkich ciał $K'$ oraz wielomianów stopnia $<deg(f)$ jest to prawdą. Niech
$$f_i=f_i'\cdot g_i,$$
gdzie $f_i',g_i\in K_i[X]$ i $g_i$ jest wielomianem nierozkładalnym w $K$. Wiemy już, że istnieje $a_i\in L_i$ będące pierwiastkiem wielomianu $g_i$.

Z faktu \ref{fakt:1:2:5}:(2), wiemy, że istnieje wtedy izomorfizm
$$\psi_0:K_1(a_1)\izo{}K_2(a_2)$$
taki, że $\psi_0(a_1)=a_2$ i $\phi\subseteq\psi_0$.

\begin{illustration}
    \node (K1) at (0, 0) {$K_1(a_1)$};
    \node (K2) at (4, 0) {$K_2(a_2)$};
    \node (L1) at (0, -2.3) {$L_1$};
    \node (L2) at (4, -2.3) {$L_2$};
    \draw[->] (K1)--(K2) node [midway, above] {$\cong$} node [midway, below] {$\exists\;\psi_0$};
    \draw[->] (L1)--(L2) node [midway, above] {$\cong$} node [midway, below] {$\exists\;\psi_1$};
    \node [rotate=90] (=1) at (0, -0.5) {$=$};
    \node [rotate=90] (=1) at (4, -0.5) {$=$};
    \node (K1') at (0, -1) {$K_1'$};
    \node (K2') at (4, -1) {$K_2'$};
    \node [rotate=90] at (0, -1.65) {$\supseteq$};
    \node [rotate=90] at (4, -1.65) {$\supseteq$};
\end{illustration}

Mamy, że {\large\color{orange}$L_i$ to ciało rozkładu $f_i'$ nad $K_i$}. W takim razie z założenia indukcyjnego istnieje izomorfizm
$$\psi_1:L_1\izo{}L_2$$
taki, że $\psi\subseteq\psi_0$ i to już jest koniec.

\begin{wniosek}
    Jeśli $f_1\in K_1[X]$ i $f_2\in K_2[X]$ są nierozkładalnymi wielomianami, $\phi:K_1\izo{}K_2$ izomorfizmem i $\phi(f_1)=f_2$, a $L_1,L_2$ to ciała rozkładu $f_1,f_2$ odpowiednio nad $K_1$ i $K_2$, $a_i\in L_i$ to pierwiastek $f_i$, to wtedy istnieje $\psi:L_1\izo{}L_2$ takie, że $\psi(a_1)=a_2$.
\end{wniosek}

\textbf{Dowód:} Wynika z dowodu stwierdzenia \hyperref[stwierdzenie:wniosek]{\bat}.
\newpage

\subsection{Domknięcia ciał}

Ciało $L$ jest \deff{algebraicznie domknięte} $\iff$ dla każdego $f\in L[X]$ o stopniu $>0$ istnieje pierwiastek $f$ w $L$, to znaczy każdy wielomian rozkłada się na czynniki liniowe nad $L$.

\textbf{Przykład:}

\indent \point $\C$ jest algebraicznie domknięte.

\indent \point $\R$ nie jest algebraicznie domknięte, gdyż $x^2+1$ nie ma pierwiastka rzeczywistego.

\indent \point $\Q[i]$ nie jest algebraicznie domknięte, bo $x^2-2$ nie ma pierwiastka.

\begin{tw}
    Każde ciało zawiera się w pewnym ciele algebraicznie domkniętym.
\end{tw}

\textbf{Dowód:}

\acc{Lemat:} Dla każdego ciała $K$ istnieje $K'\supseteq K$ takie, że $(\forall\;f\in K[X])$ stopnia $>0$ $f$ ma pierwiastek w $K'$.

Rozważmy dobry porządek na zbiorze wielomianów z $K[X]$ stopnia $>0$
$$\{f\in K[X]\;:\;deg(f)>0\}=\{f_\alpha\;:\;\alpha\subset x\}$$
KSkonstruujmy rosnący ciąg ciał $\{K_\alpha\;:\;\alpha\subset x\}$ taki, że 

\indent \point $K\subseteq K_\alpha\subseteq K_\beta$ dla $\alpha<\beta< x$

\indent \point $f_\alpha$ ma pierwiastek w $K_{\alpha+1}$.

Załóżmy, że $\alpha<x$ i mamy $\{K_\beta\;:\;\beta<\alpha\}$.

\indent 1. $\alpha$ to liczba graniczna, wtedy $K_\alpha=\ubigcup\limits_{\beta<\alpha}K_\beta$

\indent 2. $\alpha=\beta+1$ to następnik, wtedy $K_\alpha=K_\beta(a)$, gdzie $a$ to pierwiastek wielomianu $f_\beta$.

Czyli lemat jest prawdziwy.

Wracamy teraz do dowodu twierdzenia i niech $(L_n,n<\omega)$ będzie rosnącym ciągiem ciał takim, że

\indent \point $L_0=K$

\indent \point $L_{n+1}\supseteq L_n$, gdzie $L_{n+1}$ dane jest przez lemat, to znaczy $(\forall\;f\in L_n[X])$ $f$ ma pierwiastek $L_{n+1}$.

Niech
$$\hat{K}=L_\infty=\ubigcup\limits_{n<\omega}L_n.$$
Jest to ciało, ponieważ suma rosnącego ciągu ciał jest ciałem. Dalej mamy, że również
$$L[X]=\ubigcup\limits_{n<\omega}L_n[X]$$
i $L[X]$ jest algebraicznie domknięte.

\begin{uwaga}
    Załóżmy, że mamy ciała $K\subseteq L$. Wtedy

\indent \point $char(K)=char(L)$

\indent \point $0_K=0_L$ oraz $1_K=1_L$

\indent \point $K^*=K\setminus\{0\}<L^*=L\setminus\{0\}$
\end{uwaga}

\subsection{Ciała proste}

$K$ jest \deff{ciałem prostym} wtedy i tylko wtedy, gdy $K$ nie zawierza żadnego właściwego podciała. 

\textbf{Przykład:}

\indent \point $\Q$, gdzie $char(\Q)=0$ to ciało proste nieskończone.

\indent \point Ciałem prostym skończonym jest na przykład $\Z_p$ dla liczby pierwszej $p$, wtedy $char(\Z_p)=p$.
\medskip

Niech $R$ będzie pierścieniem przemiennym z $1\neq0$. Mamy następujące definicje:

\indent 1. $a\in R$ jest \deff{pierwiastkiem z $1$ }stopnia $n>0$ $\iff$ $a^n=1$

\indent 2. $\mu_n(R)=\{a\in R\;:\;a^n=1\}$ jest \deff{grupą pierwiastków z $1$} stopnia $n$

\indent 3. $\mu(R)=\bigcup\limits_{n>0}\mu_n(R)$ jest \deff{grupą pierwiastków z $1$}

\indent 4. $a$ jest \deff{pierwiastkiem pierwotnym} stopnia $n$ z $1$ $\iff$ $a\in\mu_n(R)$ oraz $(\forall\;k<n)a\notin\mu_k(R)$.

\begin{uwaga}{\color{back}dupa}

\indent 1. $\mu_n(R)\normalsubgroup R^X$ jest grupą jednostek pierścienia

\indent 2.$\mu(R)\normalsubgroup R^X$

\indent 3. $\mu(R)$ jest \acc{torsyjną grupą abelową} (każdy element jest pierwiastkiem z $1$).
\end{uwaga}

\textbf{Przykłady}

\indent 1. $\mu(\C)=\bigcup\limits_{n>0}\mu_n(\C)\lneq(\{z\in\C\;:\;|z|=1\},\cdot\})<\C^x=C\setminus\{0\}$

\indent 2. $\mu(\C)\cong(\Q,+)/(\Z,+)$, bo $f:\Q\xrightarrow[homo]{"na"}\mu(\C)$ taki, że $f(w)=\cos(w2\pi)+i\sin(w2\pi)$ ma jądro $ker(f)=\Z$.

\indent 3. $\mu(\R)=\{\pm1\}$

\indent 4. $\mu_n(K)=\{\text{zera wielomianu }w_n(x)=x^n-1\}$

\begin{uwaga}{\color{back}dupa}

\indent 1. Jeśli $char(K)=0$, to $w_n(x)=x^n-1$ ma tylko pierwiastki jednokrotne w $K$

\indent 2. Jeśli $char(K)=p>0$ i $n=p^ln_1$ takie, że $p\nmid n_1$, to wszystkie pierwiastki $w_n(x)=x^n-1$ mają krotność $p^l$ w $K$.
\end{uwaga}

\textbf{Dowód:}

\indent 1. Niech $a\in K$ takie, że $w_n(a)=0$. Z twierdzenia Bezouta mamy, że
$$w_n(x)=x^n-1=x^n-a^n=(x-a)(x^{n-1}+ax^{n-2}+...+a^{n-2}x+a^{n-1})=(x-a)v_n(x),$$
gdzie $v_n(x)=x^{n-1}+ax^{n-2}+...+a^{n-2}x+a^{n-1}$.

Z tego, że $char(K)=0$ wynika, że $v_n(a)=na^{n-1}+0$, skąd wynika, że $a$ jest jednokrotnym pierwiastkiem $w_n(x)$.

\begin{fakt}
    Załóżmy, że $char(K)=p>0$. Wtedy funkcja $f:K\to K$ taka, że $f(x)=x^p$ jest homomorfizmem ciał oraz monomorfizmem zwanym \deff{funkcją Frobeniusa}.
\end{fakt}

\begin{uwaga}
    $x\mapsto x^p$ nie musi być funkcją "na" (automorfizmem). Na przykład $K=\Z_p(f)$, wtedy $x\mapsto x^p$ nie jest "na".
\end{uwaga}

\indent 2. Mając powyższy fakt i uwagę z tyłu, przechodzimy do dowodu 2.

Niech $f:K[X]\to K[X]$, $f(h(x))=w(x)^p$ i 
$$f(\sum a_kx^k)=\sum a_k^px^{n\cdot p}$$
Z faktu wyżej mamy, że $f$ jest $1-1$.

\newpage


\section{Ciała proste, pierwiastki z jedności}

\subsection{Ciała proste}

\textbf{\large\color{blue}Uwaga 3.0.}
    \emph{Załóżmy, że mamy ciała $K\subseteq L$. Wtedy}

\begin{itemize}
\item $char(K)=char(L)$
\item \emph{$0_K=0_L$ oraz $1_K=1_L$}
\item  $K^*=K\setminus\{0\}<L^*=L\setminus\{0\}$ oraz dla $x\in K$ $-x$ w $K$ jest równe $-x$ w $L$.
\end{itemize}

$K$ jest \important{ciałem prostym} wtedy i tylko wtedy, gdy $K$ nie zawierza żadnego właściwego podciała. 

\textbf{Przykład:}
\begin{itemize}
\item $\Q$, gdzie $char(\Q)=0$ to ciało proste nieskończone.
\item Ciałem prostym skończonym jest na przykład $\Z_p$ dla liczby pierwszej $p$, wtedy $char(\Z_p)=p$.
\end{itemize}

\begin{remark}{\color{pagColor}uwu dupa meow meow}

\indent 1. Każde ciało zawiera jedyne podciało proste

\indent 2. Z dokładnościa do $\cong$ $\Q,\Z_p$ to wszystkie ciała proste.
\end{remark}

\textbf{Przykład:} Załóżmy, że $K$ jest skończone. Wtedy $K^*$ też jest skończone rzędu $|K^*|=n<\infty$. Później dowiemy się, że $|K|=p^k$, a więc $|K^*|=p^k-1$. Wiemy, że dla każdego $x\in K^*$ zachodzi $x^n=1$.

\subsection{Pierwiastki z jedności}

Niech $R$ będzie pierścieniem przemiennym z $1\neq0$. Mamy następujące definicje:
\begin{enumerate}
\item $a\in R$ jest \important{pierwiastkiem z $1$ }stopnia $n>0$ $\iff$ $a^n=1$
\item $\mu_n(R)=\{a\in R\;:\;a^n=1\}$ jest \important{grupą pierwiastków z $1$} stopnia $n$
\item $\mu(R)=\{a\in R\;:\;(\exists\;n)\;a^n=1\}=\bigcup\limits_{n>0}\mu_n(R)$ jest \important{grupą pierwiastków z $1$}
\item $a$ jest \important{pierwiastkiem pierwotnym} [primitive root] stopnia $n$ z $1$ $\iff$ $a\in\mu_n(R)$ oraz dla każdego $k<n$ $a\notin\mu_k(R)$.
\end{enumerate}

\begin{remark}{\color{pagColor}dupa}
\begin{enumerate}
\item $\mu_n(R)\triangleleft R^*$ jest grupą jednostek pierścienia
\item $\mu(R)\triangleleft R^*$
\item $\mu(R)$ jest \acc{torsyjną grupą abelową} (każdy element jest pierwiastkiem z $1$).
\end{enumerate}
\end{remark}

\textbf{Przykłady}

\indent 1. $\mu(\C)=\bigcup\limits_{n>0}\mu_n(\C)\lneq(\{z\in\C\;:\;|z|=1\},\cdot)<\C^*=C\setminus\{0\}$ jest nieskończona.

\indent 2. $\mu(\C)\cong(\Q,+)/(\Z,+)$, bo $f:\Q\xrightarrow[homo]{"na"}\mu(\C)$ taki, że $f(w)=\cos(w2\pi)+i\sin(w2\pi)$ ma jądro $ker(f)=\Z$.

\indent 3. $\mu(\R)=\{\pm1\}$

\indent 4. $\mu_n(K)=\{\text{zera wielomianu }x^n-1\}$. Ten wielomian będziemy oznaczali $\color{blue}w_n(x)=x^n-1$.

\begin{remark}{\color{pagColor}dupa}
    \label{uwaga:2:6}

\indent 1. Jeśli $char(K)=0$, to $w_n(x)=x^n-1$ ma tylko pierwiastki jednokrotne w $K$ [simple roots]

\indent 2. Jeśli $char(K)=p>0$ i $n=p^ln_1$ takie, że $p\nmid n_1$, to wszystkie pierwiastki $w_n(x)=x^n-1$ mają krotność $p^l$ w $K$.
\end{remark}

\textbf{Dowód:}

\indent 1. Niech $a\in K$ takie, że $w_n(a)=0$. Z twierdzenia Bezouta mamy, że
$$w_n(x)=x^n-1=x^n-a^n=(x-a)(x^{n-1}+ax^{n-2}+...+a^{n-2}x+a^{n-1})=(x-a)v_n(x),$$
gdzie $v_n(x)=x^{n-1}+ax^{n-2}+...+a^{n-2}x+a^{n-1}$.

Z tego, że $char(K)=0$ wynika, że $v_n(a)=na^{n-1}\neq0$, skąd wynika, że $a$ jest jednokrotnym pierwiastkiem $w_n(x)$.

% \textbf{\large\color{yellow}Fakt}
%     \emph{Załóżmy, że $char(K)=p>0$. Wtedy funkcja $f:K\to K$ taka, że $f(x)=x^p$ jest homomorfizmem ciał oraz monomorfizmem zwanym \important{funkcją Frobeniusa}.}

% \textbf{\large\color{yellow}Uwaga}
%     \emph{$x\mapsto x^p$ nie musi być funkcją "na" (automorfizmem). Na przykład $K=\Z_p(f)$, wtedy $x\mapsto x^p$ nie jest "na".}

\indent 2. Jesteśmy w ciele $K$ o $char(K)=p$. Niech $n=p^ln_1$. Rozważmy wielomian
$$w_n(X)=X^n-1=(X^{n_1})^{p^l}-1^{p^l}=(X^n-1)^{p^l}=w_{n_1}(X)^{p^l}.$$
Czyli $\mu_n(K)=\mu_{n_1}(K)$. Załóżmy, że $a\in K$ to pierwiastek wielomianu $w_n(X)$. Wtedy $a$ jest też pierwiastkiem wielomianu $w_{n_1}$ w ciele $K$. Wtedy
$$w_{n_1}(X)=(X-a)v_{n_1}(X),$$
$v_{n_1}$ jak w przypadku wyżej. Wówczas
$$v_{n_1}(a)=n_1a^{n_1-1}\neq0,$$
bo $p\nmid n_1$. 
Jeśli $a$ jest $1$-krotnym pierwiastkiem $w_{n_1}(X)$, to jest on $p^l$-krotnym pierwiastkiem $w_n(X)$.

% Niech $f:K[X]\to K[X]$, $f(h(x))=w(x)^p$ i 
% $$f(\sum a_kx^k)=\sum a_k^px^{n\cdot p}$$
% Z faktu wyżej mamy, że $f$ jest $1-1$. Ponieważ $n=p^ln_1$, to mamy 
% $$w_n(x)=x^n-1=x^n-1^n=(x^{n_1})^{p^l}-(1^{n_1})^{p^l}=...l\text{ razy}...=(x^{n_1}-1)^{p^l}=\underbrace{w_{n_1}\cdot...\cdot w_{n_1}}_{p^l},$$
% zatem każdy pierwiastek $w_n(x)$ ma krotność co najmniej $p^l$. Wystarczy więc pokazać, że każdy pierwiastek $w_{n_1}(x)$ jest jednokrotny.

% Niech $a\in K$ takie, że $w_{n_1}(a)=0$. Wtedy 
% $$w_{n_1}(x)=x^{n_1}-a^{n_1}=(x-1)(x^{n_1-1}+...+a^{n_1-1})=(x-a)v_{n_1}(x),$$
% gdzie $v_{n_1}$ jest analogiczne jak w dowodzie 1. Ale przecież $v_{n_1}(a)=n_1\cdot a^{n_1-1}\neq0$, bo $p\nmid n_1$.

\begin{theorem}
    Niech $G<\mu(K)$ i $G$ jest podgrupą skończoną o $|G|=n$. Wtedy

\indent 1. $G=\mu_n(K)$

\indent 2. $G$ jest cykliczna

\indent 3. Jeśli $char(K)=p>0$, to $p\nmid n$.
\end{theorem}

\begin{proof}{\color{pagColor}dupa}

\begin{enumerate}
\item 1. Jeśli $|G|=n$, to dla każdego $x\in G$ mamy $x^n=1$. Z tego wynika, że $G\subseteq \mu_n(K)$, ale $|\mu_n(K)|\leq n$, czyli $G=\mu_n(K)$.

\item 2. Chcemy pokazać, że dla wielomianu $w_n(X)$ mamy $n$ różnych pierwiastków. Wystarczy pokazać, że istnieje $x\in G$ taki, że $ord(x)=n$.

Załóżmy nie wprost, że dla każdego $x\in G$ $ord(x)<n$. Niech 
$$k=\max\{ord(x)\;:\;x\in G\}.$$ 
Niech $x_0\in G$ takie, że $ord(x_0)=k$. Wtedy 
$$(\forall\;y\in G)\;ord(y)\;|\;k.$$ 
Gdyby tak nie było, to istniałby $y\in G$, $ord(y)\nmid k$. Czyli istnieje liczba pierwsza $p$ taka, że $l$ jest podzielne przez wyższą potęgę $p$ niż $k$. To oznacza, że $l=p^{\alpha}l'$ i $k=p^\beta k'$, gdzie $p\nmid l'$ i $\alpha>\beta$. Rozważmy $y'=y^{l'}$. Skoro $y$ ma rząd $l$, to $ord(y')=p^\alpha$, a dla $x_0'=x_0^{p^\beta}$ mamy $ord(x')=k'$. Wobec tego $ord(x_0'y')=p^\alpha\cdot k'$, ale to jest większe od $k$ i dostajemy sprzeczność.


% Czyli
% $$(\forall\;y\in G)\;y^k=1,$$
% co pociąga $G\subseteq \mu_k(K)$ i $|G|\leq k<n$. Sprzeczność.

\item 3. Wiemy, że wszystkie pierwiastki $w_n=x^n-1$ są jednokrotne, bo jest ich w tym przypadku dokładnie $n$ (z poprzedniego punktu). Z uwagi \ref{uwaga:2:6}, że jeśli $n=p^ln_1$, to pierwiastki wielomianu $w_n(x)$ mają krotność $p^l$. Ale w tym przypadku pierwiastki mają krotność jeden, czyli $p^l=1$ i $n=1\cdot n_1$, gdzie $p\nmid n_1$.
\end{enumerate}
\end{proof}

\begin{conclusion}
    Jeśli $a\in \mu_n(K)$ jest pierwiastkiem pierwotnym z $1$ stopnia $n>1$, to $a$ generuje $\mu_n(K)$.
\end{conclusion}

\begin{proof}

$\mu_n(K)\supseteq\langle a\rangle =\mu_k(K)$ dla pewnego $k\in\N$. Ale ponieważ $a$ było pierwiastkiem pierwotnym z $1$, to musimy mieć $n=k$.
\end{proof}

\subsection{Ciała skończone}

\begin{theorem}
    Niech $K$ będzie ciałem skończonym. Wtedy

\indent 1. $char(K)=p\implies |K|=p^n$ dla pewnego $n\in\N$

\indent 2. Dla każdego $n>0$ istnieje dokładnie jedno ciało $K$ takie, że $|K|=p^n$ z dokładnością do izomorfizmu.

Ciało mocy $p^n$ będziemy oznaczać $\color{blue}F(p^n)$.
\end{theorem}

\begin{proof}

\indent 1. Skoro $char(K)=p$, to $\Z_p\subseteq K$ jest najmniejszym podciałem prostym ciała $K$. W takim razie, $K$ jest skończoną przestrzenią liniową nad $\Z_p$. Jeśli $n=dim_{\Z_p}(K)$, to $K$ jest izomorficzne z $\Z_p^n$, jako przestrzenie liniowe nad $\Z_p$. W takim razie $|K|=p^n$.

\indent 2. 

\emph{Istnienie:}

Niech $n>0$. Rozważmy 
$$w_{p^{n}-1}(x)=x^{p^n-1}\in\Z_p[X].$$
Niech $L\supseteq\Z_p$ będzie ciałem rozkładu wielomianu $w_{p^n-1}$, a $K=\{0\}\cup\{\text{ pierwiastki }w_{p^n-1}\}$. Wtedy
$$|K|=1+p^n-1=p^n,$$
czyli mamy potencjalne ciało rzędu $p^n$. Wystarczy więc pokazać, że $K$ jest ciałem.

Niech $f:L\xrightarrow[]{1-1}L$ będzie funkcją Frobeniusa $x\mapsto x^p$. Teraz niech $f^n=f\circ...\circ f$, $f^n(x)=x^{p^n}$. Jest to monomorfizm, bo składamy ze sobą $n$ takich samych funkcji $1-1$. Dla $a\in L$ mamy 
$$(a^{p^n-1}=1\;\lor\;a=0)\iff a\in K.$$
Co więcej, $a^{p^n-1}=1\iff a^{p^n}=a\iff f^n(a)=a$, czyli $K=\{a\in L\;:\;f^n(a)=a\}$ jest zbiorem punktów stałych morfizmu $f^n$, czyli jest ciałem, czego dowód jest pozostawiony na ćwiczenia. 

\emph{Jedyność $K$:}

Ciało $K$ stworzone jak wyżej jest ciałem rozkładu $w_{p^n-1}(x)$ nad $\Z_p$. 

Załóżmy nie wprost, że $K'$ to inne ciało mocy $p^n$. Bes straty ogólności $\Z_p\subseteq K'$. Niech $x\in K'$. wiemy, że $x=0$ lub $x^{p^n-1}=1$. W takim razie $w_{p^n-1}$ rozkłada się nad $K'$ na czynniki liniowe. Zatem $K'$ jest również ciałem rozkładu $w_{p^n-1}$ nad $\Z_p$.

Z wniosku \ref{wniosek:2.1}.(2) mamy, że dwa ciała rozkładu nad jednym wielomianem są izomorficzne i $K\cong K'$ nad $\Z_p$ i mamy sprzeczność.
\end{proof}

\newpage

\section{Rozszerzenia ciał}

\begin{definicja}
Niech $K\subseteq L$ będą ciałami i $a\in L\setminus K$.

\indent \point Jeżeli \acc{$a$ jest algebraiczny nad $K$}, to istnieje $f\in K[X]$ stopnia $>0$ i $f(a)=0$

\indent \point $a$ jest \acc{przestępny nad $K$} [transcendental] $\iff$ $a$ nie jest algebraiczny.

\indent \point \deff{Rozszerzenie} $L\supseteq K$ jest \deff{algebraiczne} $\iff$ dla każdego $a\in L$ $a$ jest algebraiczny nad $K$.

\indent \point \deff{Rozszerzenie jest przestępne} $\iff$ nie jest algebraiczne.

\indent \point Niech $a\in \C$. Wtedy $a$ jest algebraiczna, gdy $a$ jest algebraiczna nad $\Q$.

\end{definicja}

\textbf{Przykłady}:

\indent 1. W $\C$ na $i$ jest pierwiastkiem algebraicznym wielomianu $x^2+1$, a $\sqrt[n]{d}$ jest pierwiastkiem $x^n-d$. 

\indent 2. Ciało $F(p^n)$ ma charakterystykę $p$ i $F(p)\subseteq F(p^n)$ jest rozszerzeniem ciał, które jest algebraiczne. Dla dowolnego $a\in F(p^n)$ to jest ono pierwiastkiem wielomianu $X^{p^n}-X$, czyli $a$ jest algebraiczne nad $F(p)$.

\indent 3. Pierwiastki przestępne to na przykład $e,\pi,E^\pi$, aczkolwiek nie jesteśmy pewni tego ostatniego [doczytać w S. Lang, Algebra].

\indent 4. Rozważamy $K\subseteq L=K(X)$, czyli pierścień ułamków. Weźmy $x\in K(X)$ - przestępny nad $K$. Załóżmy, że istnieje wielomian $f\in K[X]$ rózny od $0$. I załóżmy, że $0=\hat{f}(X)$ to funkcja wielomianowa. 
$$0=\hat{f}(X)=f\neq0$$
i jest to sprzeczność.


\begin{uwaga}
    Niech $a$ jak wyżej. Wtedy $a$ jest algebraiczny nad $K$ $\iff$ $I(a/K)\neq\{0\}$ jako ideał $K[X]$.
\end{uwaga}

\subsection{Wymiar przestrzeni liniowej}

Niech $K\subseteq L$ będzie rozszerzeniem ciała $K$. Wtedy $L$ jest \deff{przestrzenią liniową nad $K$}. Definiujemy stopień rozszerzenia [coś innego jak indeks przy grupach]
$$[L:K]:=\dim_K(L)$$
jako \acc{wymiar przestrzeni liniowej} nad $K$.

\begin{uwaga}
    Niech $a\in L\setminus K$. Następujące warunki są równoważne:

\indent 1. $a$ jest algebraiczny nad $K$

\indent 2. $K[a]=K(a)$, to znaczy $K[a]$ jest ciałem (usuwanie niewymierności z mianownika)

\indent 3. $[K(a):K]=\dim_K(a)<\infty$
\end{uwaga}

\textbf{Dowód:}

$1\implies2$

Wiemy, że $K[X]$ jest euklidesowy (bo $K$ to ciało), więc $K[X]$ jest też PID.

Skoro $a$ jest algebraiczny nad $K$, to istnieje $f\in K[X]$ takie, że $f(a)=0$, a więc
$$0\neq I(\overline a/K)\normalsubgroup K[X]$$
czyli $I(a/K)$ jest maksymalnym ideałem głównym. Teraz, jeśli $I\normalsubgroup R$ jest ideałem maksymalnym pierścienia $R$, to $R/I$ jest ciałem. Czyli
$$K[a]\cong K[X]/I(a/K)$$
jest ciałem.

$2\implies 3$

Załóżmy, że $a\neq 0$. Wtedy $a^{-1}\in K[a]$, czyli istnieje wielomian $f\in K[X]$ taki, że 
$$f(x)=\sum\limits_{i=1}^n b_ix^i,\quad b_n\neq 0$$
i $a^{-1}=f(a)$. Wobec tego mamy
$$1=f(a)\cdot a$$
$$0=f(a)a-1=b_na^{n+1}+b_aa^2+...+b_0a-1,$$
stąd mamy, że
$$a^{n+1}=-\frac1{b_n}(b_{n-1}a^n+...+b_0a-1)\in Lin_K(1, a, ..., a^n)$$
jest w domknięciu liniowym $(1, a,..., a^n)$. Indukcyjnie można pokazać, że
$$(\forall\;m\geq0)\;a^m\in Lin_K(1,a,...,a^n),$$
czyli 
$$K[a]=K(a)=Lin_K(1,a,...,a^n),$$ 
co daje, że $[K(a):K]\leq n<\infty$.

$3\implies 1$

$[K(a):K]<\infty$, z czego wynika, że
$$\{1,a,...,a^n,...,\}=\{a^t\;:\;t\in\N\}\subseteq K(a)$$
jest zbiorem liniowo zależnym. Z liniowej zależności wiemy, że
$$(\exists\;n\in\N)(\exists\;b_{n-1},...,b_0)\;a^{n}=b_{n-1}a^{n-1}+...+b_1a+b_0.$$
Stąd dla $f\in K[X]$ zadanego wzorem
$$f(x)=x^n+b_{n-1}x^{n-1}+...+b_0$$
mamy $f(a)=0$, zatem $a$ jest algebraiczny nad $K$.
\medskip

Niech $a\in L\supseteq K$ będzie algebraicznym pierwiastkiem nad $K$, $I(a/K)=\{w\in K[X]\;:\;w(a)=0\}=(f)$, $f\neq 0$, $f\in K[X]$, $f$ unormowany (\dyg{czyli współczynnik przy wyrazie wiodącym jest $1$?})

\indent \point $f$ jest nazywany wielomianem \deff{minimalnym} $a$ nad $K$ (wyznaczony jednoznacznie)

\indent \point \acc{stopień $a$} nad $K$ jest definiowany jako $deg(f)$.
\medskip

\textbf{Przykład:}

\indent 1. $\sqrt{2}\in\R\supseteq\Q$, wtedy $f(x)=x^2-2$ jest wielomianem minimalnym $\sqrt2$ nad $\Q$ i stopień $\sqrt{2}$ nad $\Q$ jest równy $2$.

\indent 2. $\pi\in\R$ nie ma stopnia, bo $\pi$ nie jest liczbą algebraiczną nad $\Q$

\indent 3. $\sqrt[7]{7+\sqrt[3]{3}}-\sqrt[6]{6}\in\R$, czy jest to algebraiczne nad $\Q$? Tak i ma stopień $126$.

\begin{uwaga}[$I(a/K)=(f)\implies deg(f)=\begin{bmatrix}K(a):K\end{bmatrix}$]
    Załóżmy, że $I(a/K)=(f)$ i $f$ jest unormowany. Wówczas:

    \indent 1. $f$ jest unormowanym wielomianem minimalnego stopnia takim, że $f(a)=0$

    \indent 2. $deg(f)=[K(a):K]$, czyli stopień tego wielomianu jest równy stopniu przestrzeni liniowej $K(a)$ nad $K$.
\end{uwaga}

\textbf{Dowód:}

Niech $n=deg(f)$, 
$$f(x)=x^n+\sum\limits_{k<n}b_kx^k$$
Z tego, że $f(a)=0$ mamy, że 
$$a^n=-\sum\limits_{k<n}b_kx^k\in Lin_K(1,a,...,a^{n-1})\subseteq L.$$
Czyli $K(a)=Lin_K(1,a,...,a^{n-1})$ i wystarczy zobaczyć, że $\{1,..., a^{n-1}\}$ jest liniowo niezależny nad $K$, to znaczy jest bazą $K(a)$ nad $K$. Jest, bo $f$ jest minimalnego stopnia.

\begin{fakt}[$dim_K(M)=dim_L(M)\cdot dim_K(L)$]\label{fakt:4:5}
    Niech $K\subseteq L\subseteq M$ będą rozszerzeniami ciał. Wtedy 
    $$[M:K]=[M:L]\cdot [L:K]$$
\end{fakt}

\textbf{Dowód:}

Niech $\{e_i\;:\;i\in I\}$ będzie bazą $L$ nad $K$, a $\{f_j\;:\;j\in J\}$ będzie bazą $M$ nad $L$. Stąd $|I|=[L:K]$ i $|J|=[M:L]$.

Chcemy za pomocą tych dwóch zbiorków zrobić bazę $M$ nad $K$. Rozważmy zbiór
$$X=\{e_i\cdot f_j\;:\;i\in I,j\in J\}.$$
Musimy pokazać, że 

1. $|X|=|I|\cdot|J|$

2. $X$ jest liniowo niezależny

3. $X$ jest bazą $M$ nad $K$

Te dwa ostatnie mówią, że $X$ jest bazą.

1. Załóżmy, nie wprost, że dla $i\neq i'$ i $j\neq j'$ i $e_if_j=e_{i'}f_{j'}$. Czyli
$$e_if_j-e_{i'}f_{j'}=0,$$
czyli $f_j,f_{j'}$ są liniowo zależne nad $L$, czyli mamy, że $f_j=f_{j'}$ i
$$0=e_if_j-e_{i'}f_{j}=(e_i-e_{i'})f_j\implies e_i-e_{i'}=0\implies i=i'$$

2. Załóżmy nie wprost, że $X$ nie jest lnz, czyli istnieją $k_{ij}\in K$ takie, że
$$\sum\limits_{j\in J}\sum\limits_{i\in I}k_{ij}e_if_j=0,$$
ale $\sum\limits_ik_ije_i=l_j$ są elementami $L$, czyli
$$\sum\limits_{j\in J}l_jf_j=0$$
więc $f_j$ są liniowo zależne, a przecież były bazowe, w takim razie
$$0=l_j=\sum\limits_{i\in I}k_{ij}e_i,$$
$e_i\neq0$, czyli $k_{ij}=0$ i koniec.

3. $X$ generuje $M$ nad $K$, bo dla $m\in M$ mam
$$m=\sum l_jf_j=\sum\left(\sum a_{ij}e_i\right)f_j=\sum\sum a_{ij}e_if_j=\sum \sum k_{ij}e_if_j$$
Z tego wynika, że $[M:K]=|X|=|I||J|=[L:K][M:L]$.

\begin{wniosek}[$K^{alg}$ - podciałem]
    Niech $K\subseteq L$ będzie rozszerzeniem skończonego ciała. Niech 
    $$K^{alg}(L)=\{a\in L\;:\;a\text{ jest algebraiczny nad }K\}.$$
    Okazuje się, że $K^{alg}$ jest podciałem.
\end{wniosek}

\textbf{Dowód:}

Weźmy $a,b\in K^{alg}$. Wiemy, że $[K(a):K]$ i $[K(b):K]$ są skończone. Mamy, że
$$K\subseteq K(a)\subseteq K(a, b)$$
Z faktu \ref{fakt:4:5} wiemy, że 
$$[K(a, b):K]=[K(a,b):K(a)]\cdot[K(a):K]$$
czyli również $K(a,b)$ jest skończone. Zatem dla $x\in K(a,b)$ mamy
$$[K(x):K]\leq[K(a,b):K]$$
też jest skończone, zatem $x$ jest algebraiczny nad $K$.
\medskip

Dla $x\in K(a, b)$ mamy $[K(x):K]\leq[K(a):K]$, czyli również jest skończone. W takim razie, $x$ jest algebraiczny nad $K$ i należy do $K^{alg}$.
\medskip

\begin{definicja}[(relatywne) algebraiczne domknięcie]{\color{back}spaaaać}
    \begin{enumerate}
        \item $K^{alg}(L)$ nazywamy \deff{algebraicznym domknięciem} $K$ w $L$.
        \item $K$ jest \deff{relatywnie algebraicznie domknięte} w $L$ $\iff$ $K^{alg}(L)=K$.
    \end{enumerate}
\end{definicja}

\begin{wniosek}[algebraiczne rozszerzenia ciał, $K^{alg}$]{\color{back}kot}
    \begin{itemize}
        \item Niech $K\subseteq L\subseteq M$ będą rozszerzeniami ciał. $K\subseteq M$ jest algebraiczne $\iff$ $K\subseteq L$ i $L\subseteq M$ są algebraiczne
        \item $K^{alg}(L)$ jest relatywnie algebraicznie domknięte w $L$, tzn. $K^{alg}(L)=[K^{alg}(L)]^{alg}(L)$ 
    \end{itemize}
\end{wniosek}

\subsection{Wielomian rozkładu koła}

Rozważamy wielomian
$$w_m(x)=x^m-1$$
dla $m\in\N$. Wiemy, że
\begin{itemize}
    \item[\point] pierwiastki $w_m$ w $\C$ są jednokrotne
    \item[\point] $\mu_m(\C)$ jest grupą cykliczną
    \item[\point] $a\in\mu_m(\C)$ jest generatorem $\mu_m(\C)=\{a^i\;:\;0\leq i\leq m\}\cong(\Z_m,+)$
    \item[\point] $a^k$ generuje $\mu_m(\C)$ $\iff$$NWD(k, m)=1$
\end{itemize}
Zatem $\mu_m(\C)$ ma $\phi(m)$ generatorów.{\large\color{red}?????}

Niech
$$\{k\in\N\;:\;0<k<m, NWD(k, m)=1\}=\{m_1,...,m_{\phi(n)}\}$$
i zdefiniujmy
$$\color{blue}F_m(x):=(x-a^{m_1})...(x-a^{m_{\phi(n)}})\in\C[X]$$

\begin{uwaga}[$F_m\in\Z\begin{bmatrix}X\end{bmatrix}$]{\color{back}ddd}
    \begin{enumerate}
        \item $w_m(x)=x^m-1=F_m(x)\cdot\prod\limits_{\substack{d<m\\d|m}}F_d(x)$
        \item $F_m(x)\in\Z[X]$
    \end{enumerate}
\end{uwaga}

\textbf{Dowód:}

1. Wiemy, że wielomian jest iloczynem dwumianów $x-$\emph{pierwiastek}, te dla $w_m(x)$ są schowane w $\mu_m(\C)$, czyli
\begin{align*}
    w_m(x)&=\prod\limits_{b\in\mu_m(\C)}(x-b)=\prod\limits_{d|m}\prod\limits_{\substack{b\in\mu_m(\C)\\ord(b)=d}}(x-b)=\prod\limits_{d|m}F_d(x)=F_m(x)\prod\limits_{\substack{d<m\\d|m}}F_d(m)
\end{align*}

2. Dowód przez indukcję względem $m$. Dla $m=1$ mamy $F_m(x)=x-1\in\Z[X]$. Teraz zakładamy, że dla $d<m$ jest $F_d(x)\in\Z[X]$. Z punktu (1) wiemy, że
$$x^m-1=w_m(x)=F_m(x)v_m(x)$$
z założenia indukcyjnego $v_m(x)\in\Z[X]$, bo każdy z nich ma stopień mniejszy niż $m$ i $v_m(x)$ jest unormowany.

$w_m(x)$ dzielimy z resztą w $\Z[X]$ i dostajemy:
$$w_m(x)=v_m(x)\cdot L(x)+R(x)$$
ale w $\C[X]\supseteq\Z[X]$ było
$$w_m(x)=v_m(x)\cdot F_m(x),$$
czyli 
\begin{align*}
    F_m(x)v_m(x)&=v_m(x)L(x)+R(x)\\
    R(x)&=v_m(x)[F_m(x)-L(x)]
\end{align*}
ale $deg(R)<deg(v)$, czyli $R=0$ i $F_m=L\in\Z[X]$.

\begin{uwaga}[$F_m$ nierozkładalny w $\Q$]
    $F_m(x)$ jest wielomianem nierozkładalnym w $\Q[X]$.
\end{uwaga}

\textbf{Dowód:} 

Po pierwsze zauważmy, że $F_m$ jest nierozkładalny w $\Q[X]$ $\iff$ nierozkładalny w $\Z[X]$. 

Załóżmy nie wprost, że
$$F_m(x)=G_1(x)\cdot G_2(x)$$
dla $G_1,G_2\in\Z[X]$. Możemy założyć, że $G_1(x)$ jest dalej nierozkładalny w $\Z[X]$ oraz $0<deg(G_1)<deg(F_m)=\phi(m)$

\acc{Lemat:} Istnieje $b$-pierwiastek pierwotny stopnia $n$ oraz liczba pierwsza $p$ taka, że $p\nmid m$ i $G_1(b)=G_2(b^p)=0$.

\textbf{Dowód lematu:}

Z tego, że $0<deg(G_1)$ i $G_1|F_m$, $0<deg(G_2)$ i $G_2|F_m$ mamy, że istnieje pierwiastek pierwotny $b$ stopnia $m$ taki, że $G_1(b)=0$ oraz pierwiastek pierwotny $b'$ stopnia $m$ taki, że $G_2(b')=0$. Zatem istnieje $k\in\N$, $NWD(k, m)=1$ takie, że $b'=b^k$, bo grupa $\mu_m(\C)$ jest cykliczna i $b$ jest jej generatorem.

Niech $k=p_1\cdot...p_s$ będzie rozkładem na liczby pierwsze. Wtedy mamy ciąg różnych liczb
$$b, b^{p_1},b^{p_1p_2},...,b^{p_1,...,p_s}=b^k$$
które są pierwiastkami pierwotnymi stopnia $m$. Z tego wynika, że każda z tych liczb jest pierwiastkiem $G_1$ lub $G_2$, czyli istnieje taka pozycja $i$, że
$$G_1(b^{p_1...p_i})=0,$$
$$G_2(b^{p_1...p_{i+1}})=0$$
wtedy $b:=b^{p_1...p_i}$ oraz $p=p_{i+1}$ i lemat jest spełniony.
\medskip

Wimy już, że $G_1(b)=0$ i $G_1\in\Z[X]$ jest wielomianem nierozkładalnym. Niech $p$ będzie liczbą pierwszą z lematu. Rozważmy
$$G_3(x)=G_2(x^p).$$
Wtedy $G_2(b^p)=G_3(b)=0$, ale stąd wynika, że $G_1(x)$ dzieli $G_3(x)$. Niech więc 
$$G_3(x)=G_1(x)H(x)\in\Z[X].$$

Rozważmy homomorfizm
$$f:\Z\to\Z_p$$
i indukowany przez niego
$$\overline f:\Z[X]\to\Z_p[X].$$
Z założenia $F_m=G_1G_2$ mamy, że
$$\overline f(F_m)=\overline f(G_1)\overline f(G_2)$$
a z rozumowania powyżej ($G_3=G_1H$)
$$\overline f(G_3)=\overline f(G_1)\overline f(H)$$
ale
$$\overline f(G_3(x))=\overline f(G_2(x^p))=\overline f(G_2(x))^p,$$
bo współczynniki $f(G_2(x^p))$ są w $\Z_p$, a $(\sum c_ix^i)^p=\sum c_ix^{pi}$, bo $c_i^{kp}=c_i^k$ dla $c_i\in\Z_p$.

Stąd wiemy, że
$$f(G_2(x))^p=\overline f(G_1)\overline f(H).$$
Pierścień $\Z_p[X]$ jest UFD, więc $\overline f(G_1)$ i $\overline f(G_2)$ mają wspólny dzielnik w $\Z_p[X]$, stopnia co najmniej $1$. Zatem z
$$\overline f(F_m)=\overline f(G_1)\overline f(G_2)$$
ma co najmniej pierwiastek wielokrotny. Jest to sprzeczność, bo nie mogą istnieć pierwiastki wielokrotne: $\overline f(F_m)|x^m-1=w_m$, a $x^m-1$ ma pierwiastki jednokrotne w $\Q$.

\begin{wniosek}[pierwiastek pierwotny a $dim_\Q(\Q(b))$]
    Jeżeli $b\in\C$ jest pierwiastkiem pierwotnym z $1$ stopnia $m$, to $[\Q(b):\Q]=\phi(m)$.
\end{wniosek}

\textbf{Dowód:} $F_m(x)$ jest wielomianem minimalnym dla $b$ nad $\Q$. Mamy, że $[\Q(b):\Q]=deg F_m=\phi(m)$.

\begin{lemat}[twierdzenie Liouville'a o aproksymacji diofantycznej] \dyg{[twierdzenie Liouville'a o aproksymacji diofantycznej]}: Jeżeli $a\in\R$ jest liczbą algebraiczną stopnia $N>1$, to istnieje $c\in\R_+$ takie, że dla każdego $r=\frac pq\in\Q$ zachodzi
    $$\left|a-\frac pq\right|\geq{c\over q^N}$$
\end{lemat}

\begin{definicja}[algebraiczne domknięcie]
    Ciało $L\supseteq K$ jest \deff{algebraicznym domknięciem} $K$ wtedy i tylko wtedy, gdy:
    \begin{enumerate}
        \item $L$ jest algebraicznie domknięte
        \item $L\supseteq K$ jest rozszerzeniem algebraicznym, to znaczy dla każdego $a\in L$ $a$ jest pierwiastkiem algebraicznym nad $K$
    \end{enumerate}
    Takie $L$ oznaczamy przez $\color{blue}\hat{K}$.
\end{definicja}

\begin{wniosek}[istnieje algebraiczne domknięcie]
    Dla każdego $K$ istnieje algebraiczne domknięcie $\hat{K}$.
\end{wniosek}

\textbf{Dowód:} Rozważmy $K_\infty\supseteq K$ - ciało algebraicznie domknięte (twierdzenie z początku wykładu). Pokażemy, że
$$\hat{K}=K^{alg}(K_\infty)=\{a\in K_\infty\;:\;a\text{ algebraiczny nad }K\}$$

1. $\hat{K}$ jest algebraicznie domknięte:

Jeżeli $f\in\hat{K}[X]$, to $f$ ma pierwiastek w $K$, ale $\hat{K}\subseteq K_\infty$, to znaczy, że $a\in\hat{K}$ jest algebraiczne nad $K$.

2. $K\subseteq\hat{K}$ jest rozszerzeniem algebraicznym:

$K\subseteq\widehat{K}=K^{alg}(K_\infty)$ z definicji jest rozszerzeniem algebraicznym.

\begin{tw}[jedyność domknięcia algebraicznego]\label{tw:4:15}
    $\hat{K}$ jest jedyne z dokładnością do izomorfizmu nad $K$.
\end{tw}

\begin{center}
    \begin{tikzcd}
        L_1\arrow[rr, "(\exists!\;f)\;f\obciete K=id_K" above, "\cong" below] & & L_2\\
        & K\arrow[ul, "\supseteq" lablb]\arrow[ur, "\subseteq" labl] &
    \end{tikzcd}
\end{center}

\textbf{Dowód:}

Niech 
$$\mathfrak{K}=\{(k',f')\;:\;K\subseteq K'\subseteq L_1,f':K'\xrightarrow[]{1-1}L_2,\;f'\obciete K=id_k\}$$
\begin{illustration}
    \draw (0, 0) ellipse(1.2 and 2);
    \node at (0, -2.5) {$L_1$};
    \draw (5, 0) ellipse (1.2 and 2);
    \draw (0, -1) ellipse (0.8 and 1);
    \node at (-0.3, -1) {$K$};
    \draw (0, -0.55) ellipse (1 and 1.4);
    \node at (-0.3, 0.3) {$K'$}; 
    \draw (5, -1) ellipse (0.8 and 1);
    \draw (5, -0.55) ellipse (1 and 1.4); 
    \node at (5.2, 0.3) {$f'[K']$}; 
    \node at (5.3, -1) {$K$};
    \node at (5, -2.5) {$L_2$};
    \draw[->] (0.5, -1)..controls (1, 0) and (4, 0)..(4.5, -1) node [midway, above] {$id_K$};
    \draw[->] (0.5, 0.2)..controls (1, 1.2) and (4, 1.2)..(4.5, 0.2) node [midway, above] {$f'$};
\end{illustration}

W $\mathfrak{K}$ definiujemy relację porządku w naturalny sposób, to znaczy
$$(K', f')\leq(K'', f'')\iff K'\subseteq K''\;\land\;f''\obciete K'=f''.$$
Wtedy $(\mathfrak{K},\leq)$ jest zbiorem częściowo uporządkowanym i niepustym (bo jest $(K,id_K)\in\mathfrak{K}$). Ponadto każdy wstępujący łańcuch $(\mathfrak{K},\leq)$ ma ograniczenie górne. Na mocy lematu Kuratowskiego-Zorna w tej rodzinie istnieje element maksymalny, nazwijmy go $(K_1,f_1)$. Pokażemy, że $K_1=L_1$.

Załóżmy nie wprost, że istnieje $a\in L_1\setminus K_1$. Niech $w(x)\in K_1[X]$ będzie wielomianem minimalnym elementu $a$ nad $K_1$. Niech
$$K_2=f_1[K_1]$$
$$v(x)=f_1(a_0)+f_1(a_1)x+...+f_1(a_n)x^n\in K_2[X].$$
$v(x)$ też jest nierozkładalny nad $K_2$, bo $w(x)$ był nierozkładalny nad $K_1$. Niech $b\in L_2$ będzie pierwiastkiem wielomianu $v$.

Zauważmy, że $K_1(a)=K_1[a]$, bo $w(x)$ jest nierozkładalny nad $K_1$, ale
$$K_1[a]\simeq K_1[X]/(w)\simeq K_2[X]/(v)\simeq K_2[b]\simeq K_2(b).$$
Czyli $K_1(a)\simeq K_2(b)$ i $f_2:K_1(a)\izo{}K_2(b)$ jest izomorfizmem rozszerzającym $f_1$. Wtedy mamy $(K_1,f_1)\lneq(K_1(a),f_2)$, co daje sprzeczność z maksymalnością $(K_1,f_1)$. Zatem $L_1=K_2$.

Niech $K_2=f[K_1]=f[L_1]$. Pokażemy nie wprost, że $K_2=L_2$. Załóżmy, że istnieje $a\in L_2\setminus K_2$. Niech $w(x)\in K_2[X]$ wielomian minimalny dla $a$ nad $K_2$. Wtedy $w(x)$ nie ma pierwiastka w $K_2$, ale $K_2=f_1[L_1]$ jest algebraicznie domknięte, bo $L_1$ jest algebraicznie domknięte, co daje sprzeczność.

\begin{wniosek}[$K\cong L\implies\hat{K}\cong\hat{L}$]
    Jeśli $K\cong L$, to $\hat{K}\cong \hat{L}$. Dokładniej, jeżeli $f_0LK\to L$ jest izomorfizmem ciał, to istnieje izomorfizm $f:\hat{K}\to\hat{L}$ taki, że $f\obciete K=f_0$.
\end{wniosek}

\begin{wniosek}[algebraiczne rozszerzenie $1-1$ $\to$ $\hat{K}$]
    Jeśli $K\subseteq L$ jest algebraicznym rozszerzeniem ciał, to istnieje monomorfizm $f:L\to \hat{K}$ taki, że $f\obciete K=id_K$.
\end{wniosek}

\textbf{Dowód:} Mamy dane $K\subseteq L\subseteq\hat{L}$ rozszerzenia algebraiczne, zatem rozszerzenie $K\subseteq\hat{L}$ jest algebraiczne. Stąd $\hat{L}$ jest algebraicznym domknięciem $K$. Z twierdzenia \ref{tw:4:15} istnieje izomorfizm $g:\hat{L}\to\hat{K}$ taki, że $g\obciete K=id_K$. Wtedy $f=g\obciete L$ jest szukanym monomorfizmem.

%\section{Równania w pierścieniach}

\subsection{Układy równań}

\deff{Notacja:} przez $R, S$ oznaczamy pierścienie przemienne z $1\neq0$. Przez $K, L$ oznaczamy ciała.

Niech $f_1,...,f_n\in R[X_1,..., X_n]=R[\overline X]$.

\deff{Problem:} Czy istnieje rozszerzenie pierścieni z jednością $R\subseteq S$ takie, że układ $U:f_1(\overline X)=...=f_m(\overline X)=0$ ma rozwiązanie w pierścieniu $S$?

$\overline a=(a_1,...,a_n)\subseteq S\supseteq R$ jest rozwiązaniem układu równań $U$ $\iff$ $g(\overline a)=0$ dla każdego wielomianu $g\in(f_1,...,f_m)\normalsubgroup R[X]$.

\textbf{Dowód:} Rozważmy przypadki:

\indent 1. $(f_1,...,f_m)\ni b\neq 0$ i $b\in R$. Wtedy układ $U$ jest sprzeczny i nie ma rozwiązania w żadnym pierścieniu rozszerzającym $R$, więc możemy ten przypadek odrzucić.

\indent 2. $(f_1,...,f_m)\cap R=\{0\}$, czyli negacja pierwszego przypadku. Teraz układ $U$ jest niesprzeczny i skonstruujemy pierścień $S\supseteq R$ z jednością (czyli rozszerzenie pierścienia $S$) i rozwiązanie $\overline a\subseteq S$.

Niech $S=R[\overline X]/(f_1,...,f_m)$ i rozważmy $jR[\overline X]\to S$ ilorazowe. Po pierwsze zauważmy, że $j\obciete R$ jest $1-1$, bo
$$ker(j\obciete R)=ker(j)\cap R=(f_1,...,f_m)\cap R=\{0\}$$
i dlatego 
$$j\obciete R:R\xrightarrow[\cong]{} j[R]\subseteq S.$$
Z uwagi na ten izomorfizm utożsamiamy $R$ z $j[R]$ i $S$ jest więc rozszerzeniem pierścienia $R$.

Niech $\overline a=(a_1,...,a_m)=(j(X_1),...,j(X_m))$, czyli zbiór obrazów wielomianów stopnia $1$ z peirścienia $S$. Wtedy $\overline a$ jest rozwiązaniem układu $U$ w pierścieniu $S$. Oznaczmy funkcję wielomianową przez
$$\hat{f_i}(\overline a)=\hat{f_i}(j(X_1),...,j(X_m))=j(\hat{f_i}(X_1,...,X_m))=j(f_i)=0$$
powyższe równości należy sprawdzić w ramach ćwiczenia.

\deff{Uwaga:} Skonstruowane powyżej rozwiązanie $\overline a$ układu $U$ ma następującą własność uniwersalności. Jeśli $S'\supseteq R$ jest rozszerzeniem pierścieni z 1 i $\overline a'=(a_1',...,a_n')\subseteq S$ jest rozwiązaniem $U$ w $S'$, to istnieje jedyny homomorfizm $h:R[\overline a]\to R[\overline a']$ taki, że $h\obciete R$ jest identycznością na $R$ i $h(\overline a)=\overline a'$. Wszystkie rozwiązania układów sa homomorficzne.

$R[\overline a]\subseteq S$ to podpierścień generowany przez $R\cup\{\overline a\}$, czyli
$$R[\overline a]=\{f(\overline a)\;:\;f(\overline X)\in R[\overline X]\}\subseteq S$$

\textbf{Dowód:} Niech $I=\{g\in R{\overline X}\;:\;g(\overline a')=0\}$ w $S'$. Oczywiście $I\normalsubgroup R[\overline X]$. Znaczy to, że 
$$(f_1,...,f_m)\subseteq I$$
z twierdzenia o faktoryzacji wielomianów w pierściueniu (????) dostajemy od razu
$$R[X]\xrightarrow[]{j} S=R[\overline X]/(f_1,...,f_m)$$
i $R[\overline a']\subseteq S'$. Widzimy, że $I=ker\phi\subseteq ker j=(f_1,...,f_m)$. Z twierdzenia o homomorfizmie peirścieni dostajemy jedyne $h:R[X]/(f_1,...,f_m)\to R[\overline a']$ taki, że $h(\overline a)=\overline a'$.

\deff{Uwaga:} Jeśli $I=(f_1,...,f_m)$ to $h:R[\overline a]\xrightarrow[]{\cong}R[\overline a']$

\deff{Definicja:} Załóżmy, że $S\supseteq R$ jest rozszerzeniem pierścienia oraz $\overline a\in S^n$. Wtedy

\indent I. $I(\overline a/R)=\{g\in R[\overline X]\;:\;g(\overline a)=0\}$

\indent II. $\overline a$: \acc{rozwiązanie ogólne} układu $U$ gdy ideał $I(\overline a/R)=(f_1,...,f_m)$.

\deff{Uwaga:} W sytuacji z definicji powyżej, gdy $U$ jest niesprzeczne, wtedy $\overline a$ jest rozwiązaniem ogólnym układu $U$ $\iff$ zachodzi warunek z gwizdką.

\textbf{Dowód:} ćwiczenia.

\subsection{Ciała}

$K\subseteq L$ i $\overline a\subseteq L$. Definiujemy \acc{ideał $\overline a$ nad $K$} jako
$$I(\overline a/K)=\{g\in K[\overline X]\;:\;g(\overline a)=0\}$$

Wtedy $K[\overline a]=$ podpierścień ciała $L$ generowany przez $K\cup\{a_1,...,a_m\}=\{g(\overline a)\;:\;g\in K[\overline X]$.

$K(\overline a)$ to podciało ciała $L$ generowane przez $K\cup\{a_1,...,a_m\}$. Czyli jest to ciało ułamków pierścienia $K[\overline a]$ w ciele $L$. Inaczej piszemy $K[\overline a]_0$ 
$$K(\overline a)=\{g(\overline a\;:\;g\in K(\overline X)\text{ i } g(\overline a)\text{ jest dobrze określone})\}$$

\deff{Uwaga:} Załóżmy, że $K\subseteq L_1,K\subseteq L_2$ są to rozszerzenia ciał i $\overline a_1\subseteq L_1$, $\overline a_2\in L_2$ i $|\overline a_1|=|\overline a_2|=n$. Wtedy następujące warunki są równoważne:

\indent 1. $(\exists\;f:K[\overline a_1]\xrightarrow[]{\cong}K[\overline a_2])\;f(\overline a_1)=\overline a_2$ i $f\obciete K=id_k$

\indent 2. $I(\overline a_1/K)=I(\overline A_2/K)$

\textbf{Dowód:}

$1\implies 2$ jest jasne, bo dla $g(\overline x)\in K[\overline x]$ takie, że $g(\overline a_1)=0$ w $K[\overline a_1]$ $\iff$ $g(f(\overline a_1))=0$ dla w $K[\overline a_2]$.

$\impliedby$ Zwróćmy uwagę na odwzorowanie ewaluacji $\overline a_1$
$$\phi_{\overline a_1}:K[\overline X]\xrightarrow[]{epi} K[\overline a_1]$$

mamy $\phi_{\overline a_1}(w(\overline x))=w(\overline a_1)$, czyli do wielomianu $\phi$ podstawia $\overline a_1$. Oczywiście, 
$$ker(\phi_{\overline a_1})=I(\overline a_1/K)=I(\overline a_2/K)=ker \phi_{\overline a_2}$$

\deff{Uwaga:} Niech $I\normalsubgroup K[\overline X]$ noetherowskiego pierścienia $K[\overline X]$. I niech $I=(f_1,...,f_m)$  dla pewnych $f_i\in K[\overline X]$. Wtedy istnieje rozszerzenie pierścienia $S\supseteq K$ oraz $\overline a\subseteq S$: rozwiązanie ogólne układu $f_1(\overline X)=..,.= f_m(\overline X)=0$ takie, że $I(\overline a/K)=I$

\textbf{Dowód:} Patrz na poprzednie uwagi, których było już dość dużo.

\deff{Twierdzenie:} Niech $I\normalsubgroup K[\overline X]$. Wtedy istnieje ciało $L\supseteq K$ oraz $\overline a=(a_1,...,a_n)\subseteq L$ takie, że $f(\overline a)=0$ dla każdego $f\in I$.

\textbf{Dowód:} Niech $I\subseteq M\normalsubgroup K[X]$ będzie ideałem maksymalnym. Niech $L=K[\overline X]/M$, $j:K[\overline X]\to L$ ilorazowe, $M\cap K=\{0\}$, więc $j\obciete K:K\to L$ jest $1-1$, a więc
$$j\obciete K:K\xrightarrow[]{1-1}j[K]\subseteq L.$$
Utożsamiamy $K$ z $j[K]$, to znaczy $K\subseteq L$. Niech $\overline a=(a_1,...,a_n)$, $a_i=j(X_i)\in L$. $g(\overline a)=0$ dla każdego $g(\overline X)\in M\subseteq I$.

\textbf{Wniosek:} Niech $f\in K[X]$ stopnia $>0$. Wtedy istnieje ciało $L\supseteq K$ rozszerzające ciało $K$ taki, że $f$ ma pierwiastek w ciele $L$.

\textbf{Przykład:} 

\indent 1. Popatrzmy na ciało $K=\Q$ i $f(X)=X-2$. Wtedy $I=(f)\normalsubgroup \Q[X]$ jest ideałem maksymalnym, bo jest on pierwsz (czyli w tym wypadku nierozkładalny). Równanie $f=0$ ma rozwiązanie ogólne w pierścieniu ilorazowym
$$\Q[X]/I\cong \Q$$

\indent 2. $\C=\R[i]=\R(i)=\R[z]$ dla każdej $z\in \C\setminus\R$.

\deff{Załóżmy, że $k\subseteq L_1$}, $K\subseteq L_2$ to rozszerzenia ciała. Wtedy mówimy, że $L_1$ jest izomorficzne z $L_2$ nad $K$ [$L_1\cong_K L_2$] $\iff$ gdy istnieje izomorfizm $f:L_1\to L_2$ taki, że $f\obciete K=id_k$.

\deff{Fakt:}

\indent 1. Załóżmy, że $f(X)\in K[X]$ jest nierozkładalny. Niech $L_1=K(a_1)$, $L_2=K(a_2)$ $f(a_i)=0$ w $L_i$. Wtedy $L_1\cong_K L_2$.

\indent 2. Ogólnie: załóżmy, że $\phi:K_1\to K_2$ jest izomorfizmem i $f_1\in K_1[X]$, $f_2\in K_2[X]$ i $\phi(f_1)=f_2$, $f_i$ jest nierozkładalne. Dodatkowo załóżmy, że $L_1$ jest rozszerzeniem ciała $K_1$ o element $a_1$ i $L_2=K(a_2)$, gdzie $f_i(a_i)=0$ w $L_i$. Wtedy istnieje izomorfizm $\phi\in \psi:L_1\to L_2$ taki, że $\psi(a_1)=a_2$.

Podpunkt pierwszy jest szczególnym przypadkiem podpunktu 2, gdy $\phi=id$.

\textbf{Dowód:}

\indent 1. $T(a_1/K)=(f)=I(a_2/K)$, stąd na mocy faktu 1.5 mamy $K(a_1)\cong_K K(a_2)$.

\indent 2. Popatrzmy najpierw na izomomfizm $K_1[X]]xrightarrow[\phi]{\cong}K_2[X]$ Wtedy ten $\phi$ indukuje $K_1[X]/(f_1\xrightarrow[\phi]{\cong}K_2[X]/(f_2)$, bo $\phi(f_1)=f_2$. Zatem
$$I(\overline a_i/K_i)=(f_i)\normalsubgroup K_i[X]$$
$$L_i=K_i(a_i)=K_i[a_i]\cong K_i[\overline X]/I(a_j/K_j)$$

Ciało $L\supseteq K$ jest \deff{ciałem rozkładu} [decomposition field] nad $K$ wielomianu $f\in K[X]$, gdy spełnione są warunki:

\indent 1. $f$ rozkłada się w pierścieniu $L[X]$ na czynniki liniowe stopnia $1$

\indent 2. Ciało $L$ jest rozszerzeniem ciała $K$ o elementy $a_1,..., a_n$, gdzie $a_1,..., a_n$ to wszystkie pierwiastki $f$ w $L$.

Nie są warunkami równoważnymi, bo $1$ może być spełnione przez coś większego niż 2, a my chcemy najmniejsze takie ciało.

\textbf{Przykład:} Jeżeli $deg(f)=0$, to nie istnieje ciało rozkładu $f$.

\textbf{Wniosek:} Załóżmy, że $f\in K[X]$ jest wielomianem stopnia $>0$. Wtedy 

\indent 1. istnieje $L$: ciało rozkładu $f$ nad $K$,

\indent 2. ciało to jest jedyne z dokładnością do izomorfizmu nad $K$.

\textbf{Dowód:}

\indent 1. Dowód przez indukcję względem stopnia $f$.

$deg(f)=1\implies L=K$ i jest OK

Załóżmy, że stopień $f>1$ i teza zachodzi dla wszystkich wielomianów stopnia $<deg(f)$ i wszystkich ciał $K'$. Teraz z wniosku 1.7. wiemy, że istnieje rozszerzenie ciała $K$, w którym wielomian $f$ ma pierwiastek, powiedzmy $a_0$ to ten pierwastek:
$$K'=K(a_0)$$
w $K'[X]$ ma pierwastek $a_0$, więc dzieli się przez $(x-a_0)$, więc
$$f=(x-a_0)f_1$$
gdzie $f_1\in K'[X]$, $0<deg(f_1)<deg(f)$. Z założenia indukcyjnego dla $f_1$ istnieje $L'=K'(a_1,..., a_r)$ - ciało rozkładu wielomianu $f_1$ nad $K'$. Wtedy $L=K(a_0,...,a_r)$ jest ciałek rozkładu $f$ nad $K$.

\indent 2. Udowodnimy wersję ogólniejszą: Jeśli $\phi:K_1\to K_2$ jest izomorfizmem nad ciałem i $f_i\in K_i[X]$ jest wielomianem stopnia $>0$, $\phi(f_1)=f_2$, to wtedy istnieje $\psi:L_1\to L_2$ izoorfizm nad ciałami rozkładu tych $K_i$. 

\end{document}