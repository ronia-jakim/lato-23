\documentclass{article}

\usepackage{../../notatki}

\title{Algebra 2R\smallskip\\{\scriptsize a voyage into the unknown}}
\author{koteczek}
\date{$\sim$}

\begin{document}
\maketitle

Pomoce dydaktyczne:
\href{https://www.youtube.com/playlist?list=PL8yHsr3EFj53Zxu3iRGMYL_89GDMvdkgt}{playlista z losowymi wykladami}

\subsection*{SYLABUS:}

\deff{I. Podstawy teorii równań algebraicznych}

\indent 1. Rozszerzenia ciał. Rozszerzenia o pierwiastek wielomianu nierozkładalnego. Ciało rozkładu wielomianu: istnieje, jedyność.

\indent 2. Ciało algebraicznie domknięte: definicja. Każde ciało zawiera się w ciele algebraicznie domkniętym (konstrukcja). Podciało proste: istnienie, jedyność. Ciała proste.

\indent 3. Pierwiastki z jedności, pierwiastki pierwotne. Grupa pierwiastków z jedności w ciele: każda jej skończona podgrupa jest cykliczna. Wielomiany podziału koła. Funkcja Frobeniusa. Ciała skończone: własności.
\smallskip

\deff{II. Teoria Galois}

\indent 1. Rozszerzenia [elementy] algebraiczne, przestępne: definicja. Stopień rozszerzenia. Warunki równoważne algebraiczności. Wielomian minimalny elementu ciała nad podciałem, własności.

\indent 2. Algebraiczne domknięcie ciała: definicja, istnienie, jedyność, własności (jednorodność). Istnienie rzeczywistych liczb przestępnych, liczby Liouville'a.

\indent 3. Rozszerzenia normalne: definicja, własności. Rozszerzenia [elementy, wielomiany] rozdzielcze. Twierdzenie Abela o elemencie pierwotnym. Rozszerzenia czysto nierozdzielcze (radykalne): definicja, własności. Stopień rozdzielczy [radykalny] rozszerzenia: definicja, własności.

\newpage

\tableofcontents

\newpage

\section{Teoria równań algebraicznych}

\subsection{Ciała}

Dla $K\subseteq L$ ciał i $a_1,...,a_n=\overline a\in L$ definiujemy ideał $I({\overline a}/L)$ w $K[X_1,...,X_n]$ jako:
$$\color{blue}I(\overline a /L):=\{f(X_1,...,X_n)\in K[\overline X]:f(\overline a)=0\},$$
to znaczy generujemy ideał w wielomianach nad $K$ zawierający wszystkie wielomiany (niekoniecznie tylko jednej zmiennej) zerujące się w $\overline a$. 
\medskip

\textbf{Przykład:}
\smallskip

Dla $K=\Q,L=\R,n=1,a_1=\sqrt{2}$ mamy
$$I(\sqrt{2}/\Q)=\{f(x^2-2)\;:\;f\in\Q[X]\}=(x^2-2)\normalsubgroup \Q[X]$$

Dalej, definiujemy
$$\color{blue}K[\overline a]:=\{f(\overline a)\;:\;f\in K[X]\}$$
czyli podpierścień $L$ generowany przez $K\cup\{\overline a\}$ oraz $K(\overline a)$, czyli podciało $L$ generowane przez $K\cup\{\overline a\}$:
$$\color{blue}K(\overline a):=\{f(\overline a)\;:\;f\in K(X_1,...,X_n)\;i\;f(\overline a)\text{ dobrze określone}\}.$$
Tutaj $K(X_1,...,X_n)$ to \dyg{ciało ułamków pierścienia} $K[\overline X]$ (czyli najmniejsze ciało, że pierścień może być w nim zanurzony).
\medskip

\textbf{Przykład:}
\smallskip

{\color{red}\large ??? chyba nie rozumiem tego w pełni, a powinnam po przeczytaniu ???}
\smallskip

??? chyba nie rozumiem tego w pełni, a powinnam po przeczytaniu ???
Dla $K=\Q,L=\R$ zachodzi:
$$K[\sqrt2]=\Q[\sqrt2]=\{q+p\sqrt2\;:\;q,p\in\Q\}$$
$$K[\sqrt2,\sqrt3]=\Q[\sqrt2,\sqrt3]$$
$$K(\sqrt2)=\Q[\sqrt2]$$
to ostatnie to usuwanie niewymierności z mianownika.
\medskip

\textbf{\large\deff{Twierdzenie:}} Niech $K\subseteq L_1,K\subseteq L_2$ będą ciałami. Wybieramy $\{a_1,...,a_n\}\in L_1$ i $\{b_1,...,b_n\}\in L_2$. Wtedy następujące warunki są równoważne:

\indent \point istnieje izomorfizm $\phi:K[a_1,...,a_n]\to K[b_1,...,b_n]$ taki, że $\phi\obciete K=id_K$ oraz $\phi(a_i)=b_i$.

\indent \point $I(\overline a/K)=I(\overline b/K)$.
\smallskip

\textbf{Dowodzik:}

$K[\overline a]\isomorphism K[\overline b]$ \acc{$\implies$} $I(\overline a/K)=I(\overline b/K)$

Niech $\omega\in K[\overline X]$. Wtedy $\omega\in I(\overline a/K)$ wtedy i tylko wtedy, gdy $\omega(\overline a)=0$, to mamy z definicji $I(\overline a/K)$. Wiemy też, że $\phi(a)\in K[\overline X]$ wtedy, gdy $\omega(\phi(\overline a))=0$, a ponieważ $\phi(\overline a)=\overline b$, to również $\omega(\overline b)=0$ i mamy, że $\omega\in I(\overline b/K)$. Czyli izomorfizm między $K[\overline a]=K[\overline b]$ implikuje, że $I(\overline a/K)=I(\overline b/K)$.
\smallskip

$K[\overline a]\isomorphism K[\overline b]$ \acc{$\impliedby$} $I(\overline a/K)=I(\overline b/K)$

Spróbujmy zdefiniować izomorfizm $\phi$ tak, że dla $\omega\in K[\overline X]$ mamy $\phi(\omega(\overline a))=\omega(\overline b)$

1. $\phi$ jest homomorfizmem: 
$$\phi(\omega(\overline a)\cdot v(\overline a))=f((\omega\cdot v)(\overline a))=(\omega\cdot v)(\overline b)=\omega(\overline b)\cdot v(\overline b)=\phi(\omega(\overline a))\cdot \phi(v(\overline a))$$

2. $\phi$ jest różnowartościowe:
$$\phi(\omega(\overline a))=\phi(v(\overline a))\iff \omega(\overline b)=v(\overline b)\iff (\omega-v)(\overline b)=0\iff \omega-v\in I(\overline b/K)=I(\overline a/K)\iff (\omega-v)(\overline a)=0\iff \omega(\overline a)=v(\overline a)$$

3. $\phi$ jest dobrze zdefiniowane (czyli przyjmuje tylko jedną wartość dla jednego argumentu):
$$\omega(\overline a)-v(\overline a)=0\iff (\omega-v)(\overline a)=0\iff \omega -v\in I(\overline a/K)\iff \omega-v\in I(\overline b/K)\iff (\omega-v)(\overline b)=0\iff \omega(\overline b)-v(\overline b)=0$$



\end{document}