\documentclass{article}

\usepackage{../../notatki}

\title{Algebra 2R\smallskip\\{\scriptsize a voyage into the unknown}}
\author{koteczek}
\date{$\sim$}

\begin{document}
\maketit

%Pomoce dydaktyczne:
%\href{https://www.youtube.com/playlist?list=PL8yHsr3EFj53Zxu3iRGMYL_89GDMvdkgt}{playlista z losowymi wykladami}

% \subsection*{SYLABUS:}

% \deff{I. Podstawy teorii równań algebraicznych}

% \indent 1. Rozszerzenia ciał. Rozszerzenia o pierwiastek wielomianu nierozkładalnego. Ciało rozkładu wielomianu: istnieje, jedyność.

% \indent 2. Ciało algebraicznie domknięte: definicja. Każde ciało zawiera się w ciele algebraicznie domkniętym (konstrukcja). Podciało proste: istnienie, jedyność. Ciała proste.

% \indent 3. Pierwiastki z jedności, pierwiastki pierwotne. Grupa pierwiastków z jedności w ciele: każda jej skończona podgrupa jest cykliczna. Wielomiany podziału koła. Funkcja Frobeniusa. Ciała skończone: własności.
% \smallskip

% \deff{II. Teoria Galois}

% \indent 1. Rozszerzenia [elementy] algebraiczne, przestępne: definicja. Stopień rozszerzenia. Warunki równoważne algebraiczności. Wielomian minimalny elementu ciała nad podciałem, własności.

% \indent 2. Algebraiczne domknięcie ciała: definicja, istnienie, jedyność, własności (jednorodność). Istnienie rzeczywistych liczb przestępnych, liczby Liouville'a.

% \indent 3. Rozszerzenia normalne: definicja, własności. Rozszerzenia [elementy, wielomiany] rozdzielcze. Twierdzenie Abela o elemencie pierwotnym. Rozszerzenia czysto nierozdzielcze (radykalne): definicja, własności. Stopień rozdzielczy [radykalny] rozszerzenia: definicja, własności.

%\newpage

\tableofcontents

\newpage

\section{Wykład 1: Teoria równań algebraicznych}

Przez $R, S$ będziemy oznaczać pierścienie przemienne z $1\neq0$, natomiast $K, L$ będziemy rezerwować dla oznaczeń ciał.

\subsection{Rozwiązywanie układów równań}

Rozważmy funkcje $f_1,...,f_m\in R[X_1, ..., X_n]$. Dla wygody będziemy oznaczać krotki przez $\overline X$, czyli $R[X_1,...,X_n]=R[\overline X]$. Pojawia się problem: \dyg{czy istnieje rozszerzenie pierścieni z jednością $R\subseteq S$ takie, że układ $U:f_1(\overline X)=...=f_m(\overline X)=0$ ma rozwiązanie w pierścieniu $S$?}

\begin{fakt}
    $\overline a=(a_1,...,a_n)\subseteq S$, gdzie $S$ jest rozszerzeniem pierścienia $R$, jest \acc{rozwiązaniem układu równań} $U\iff g(\overline a)=0$ dla każdego wielomianu $\color{blue}g\in (f_1,...,f_m)\normalsubgroup R[X]$.
\end{fakt}

\textbf{Dowód:} 

$\impliedby$ Implikacja jest dość trywialna, jeśli każdy wielomian z $(f_1,...,f_m)$, czyli wytworzony za pomocą sumy i produktu wielomianów $f_1,...,f_m$ zeruje się na $\overline a$, to musi zerować się też na każdym z tych wielomianów.

$\implies$ Rozważamy dwa przypadki:

\indent 1. $(f_1,...,f_m)\ni b\neq 0$ i $b\in R$. 

To znaczy w $(f_1,...,f_m)$ mamy pewien niezerowy wyraz wolny. Wtedy mamy wielomian $g\in(f_1,...,f_m)$ taki, że $g(\overline a)\neq 0$. Ale przecież $g$ jest kombinacką wielomianów $f_1,...,f_m$, która na $\overline a$ przyjmują wartość $0$. W takim razie dostajemy układ sprzeczny i przypadek jest do odrzucenia.

\indent 2. $(f_1,...,f_m)\cap R=\{0\}$. (nie ma wyrazów wolnych różnych od $0$)

Teraz wiemy, że układ $U$ jest niesprzeczny, a więc możemy skonstruować pierścień z $1$ $S$ będący rozszerzeniem $R$ [$S\supseteq R$] oraz rozwiązanie $\overline a\subseteq S$ spełniające nasz układ równań.

Niech $S=R[\overline X]/(f_1,...,f_m)$ i rozważmy
$$j:R[\overline X]\to S=R[\overline{X}]/(f_1,...,f_m)$$
nazywane \acc{przekształceniem ilorazowym}. Po pierwsze, zauważmy, że $j\obciete R$ jest $1-1$, bo
$$\ker(j\obciete R)=\ker(j)\cap R=(f_1,...,f_m)\cap R=\{0\}$$
i dlatego
$$j\obciete R:R\izo{}j[R]\subseteq S.$$
Z uwagi na ten izomorfizm, będziemy utożsamiać $R, j[R]$. W takim razie, $S$ jest rozszerzeniem pierścienia $R$. Czyli mamy rozszerzenie pierścienia $R$.

Niech 
$$\overline a=(a_1,...,a_m)=(j(X_1),...,j(X_n))\subseteq S,$$
czyli jako potencjalne rozwiązanie rozważamy zbiór obrazów wielomianów stopnia $1$ przez wcześniej zdefiniowaną funkcję $j:R[\overline X]\to S$. Tak zdefiniowane $\overline a$ jest rozwiązaniem układu $U$ w pierścieniu $S$, bo dla funkcji wielomianowej (czyli zapisywalnej jako wielomian) $\hat{f_i}\in(f_1,...,f_m)$ mamy
$$\hat{f_i}(\overline a)=\hat{f_i}(j(X_1),...,j(X_m))=j(\hat{f_i}(X_1,...,X_m))=j(f_i)=0.$$

{\color{orange}\large TUTAJ TRZEBA POUZASADNIAĆ KILKA RÓWNOŚCI, ALE MOŻE NIE BĘDĘ TEGO ROBIŁA NA AISD}

\begin{uwaga}
    \label{uwaga1:1:2-warunek-rozwiazanie-ogolne}
    Skonstruowane powyżej {rozwiązanie $\overline a$} układu $U$ ma następującą własność {uniwersalności}:

    ($\coffee$) Jeżeli $S'\supseteq R$ jest rozszerzeniem pierścienia z $1$ i $\overline a'=(a_1',...,a_m')\subseteq S$ jest rozwiązaniem $U$ w $S'$, to istnieje jedyny homomorfizm 
    $$h:R[\overline a]\to R[\overline a']$$ 
    taki, że $h\obciete R$ jest identycznością na $R$ i $h(\overline a)=\overline a'$. \acc{Wszystkie rozwiązania układów są homomorficzne.}
\end{uwaga}

\begin{illustration}
    \node (R) at (0, 0) {$R$};
    \node (R[a]) at (4, 0) {$R[\overline a]\subseteq S$};
    \node (R[a']) at (0, -3) {$R[\overline a']\subseteq S'$};
    \draw [ ->] (R)--(R[a]) node [midway, above] {$\subseteq$};
    \draw [->] (R)--(R[a']) node [midway, left] {$\subseteq$};
    \draw[->] (R[a])--(R[a']) node [midway, right] {$h$};
\end{illustration}

Tutaj \deff{$R[\overline a]\subseteq S$ jest podpierścieniem generowanym przez $R\cup\{\overline a\}$}, czyli zbiór:
$$R[\overline a]=\{f(\overline a)\;:\;f(\overline X)\in R[\overline X]\}\subseteq S$$

\textbf{Dowód:} Niech $I=\{g\in R[\overline X]\;:\;g(\overline a')=0\}\subseteq S'$. Oczywiście mamy, że $I\normalsubgroup R[\overline X]$, a więc
$$(f_1,...,f_m)\subseteq I.$$
Z twierdzenia o faktoryzacji wie
\begin{illustration}
    \node (R[x]) at (0, 0) {$R[\overline X]$};
    \node (S) at (4, 0) {$S=R[\overline X]/(f_1,...,f_m)$};
    \node (R[a']) at (0, -3) {$S'\supseteq R[\overline a']$};
    \draw[->] (R[x])--(S) node [midway, above] {$j$};
    \draw[->] (R[x])--(R[a']) node [midway, left] {$\phi$};
    \draw[->, dashed] (S)--(R[a']) node [midway, right] {$(\exists\;!h)\;h(\overline a)=\overline a'$};
\end{illustration}
Homomorfizm $\phi:R[\overline X]\to R[\overline a']$ określamy wzorem
$$\phi(w)=w(\overline a),$$
a homomorfizm $j$ jest jak wyżej odwzorowaniem ilorazowym. Widzimy, że 
$$I=\ker(\phi)$$
$$\ker(j)=(f_1,...,f_m).$$
Z twierdzenia o homomorfizmie pierścieni dostajemy jedyny homomorfizm 
$$h:R[X]/(f_1,...,f_m)\to R[\overline a]$$
taki, że $h(\overline a)=\overline a'$.

\begin{uwaga}
    Jeśli $I=(f_1,...,f_m)$, to $h:R[\overline a]\izo R[\overline a']$.
\end{uwaga}

Wtedy mamy $\ker\phi=\ker j$, czyli $\ker(h\circ j)=\ker\phi=\ker j$, no a z tego wynika, że $\ker h$ jest trywialne, czyli $h$ jest apimorfizmem (1-1). Z drugiej strony, $Im \phi=Im(h\circ j)$, a $\phi$ jest epimorfizmem ("na"), więc również $h$ musi być "na".
\medskip

\begin{important}
Załóżmy, że $S\supseteq R$ jest rozszerzeniem pierścienia oraz $\overline a\in S^n$. Wtedy:

\indent 1. ideał $\overline a$ nad $R$ definiujemy jako 
$$\color{blue}I(\overline a/R)=\{g\in R[\overline X]\;:\;g(\overline a)=0\}$$

\indent 2. $\overline a$ nazywamy \deff{rozwiązaniem ogólnym} układu $U$, jeśli ideał 
$$I(\overline a/R)=(f_1,...,f_m).$$
\end{important}

\begin{uwaga}
    W sytuacji jak z definicji wyżej, gdy $U$ jest układem niesprzecznym, wtedy 
    
    $\overline a$ jest rozwiązaniem ogólnym układu $U$ $\iff$ zachodzi warunek \hyperref[uwaga1:1:2-warunek-rozwiazanie-ogolne]{(\coffee)}.
\end{uwaga}

\textbf{Dowód:} Ćwiczenia.

\subsection{Rozszerzanie ciał}

Dla $K\subseteq L$ ciał i $\overline a\subseteq L$ definiujemy \deff{ideał $\overline a$ nad $K$} jako:
$$\color{blue}I(\overline a /L):=\{f(X_1,...,X_n)\in K[\overline X]:f(\overline a)=0\},$$
to znaczy generujemy ideał w wielomianach nad $K$ zawierający wszystkie wielomiany (niekoniecznie tylko jednej zmiennej) zerujące się w $\overline a$. 
\medskip

\textbf{Przykład:}
\smallskip

Dla $K=\Q,L=\R,n=1,a_1=\sqrt{2}$ mamy
$$I(\sqrt{2}/\Q)=\{f(x^2-2)\;:\;f\in\Q[X]\}=(x^2-2)\normalsubgroup \Q[X]$$

Dalej, definiujemy
$$\color{blue}K[\overline a]:=\{f(\overline a)\;:\;f\in K[X]\}$$
czyli \deff{podpierścień $L$ generowany przez $K\cup\{\overline a\}$} oraz \deff{$K(\overline a)$, czyli podciało $L$} generowane przez $K\cup\{\overline a\}$:
$$\color{blue}K(\overline a):=\{f(\overline a)\;:\;f\in K(X_1,...,X_n)\;i\;f(\overline a)\text{ dobrze określone}\}.$$
Tutaj $K(X_1,...,X_n)$ to \dyg{ciało ułamków pierścienia} $K[\overline a]$ w ciele $L$ (czyli najmniejsze ciało, że pierścień może być w nim zanurzony). Czasami oznaczamy to przez $K[\overline a]_0$.
\medskip

% \textbf{Przykład:}
% \smallskip

% Dla $K=\Q,L=\R$ zachodzi:
% $$K[\sqrt2]=\Q[\sqrt2]=\{q+p\sqrt2\;:\;q,p\in\Q\}$$
% $$K[\sqrt2,\sqrt3]=\Q[\sqrt2,\sqrt3]$$
% $$K(\sqrt2)=\Q[\sqrt2]$$
% to ostatnie to usuwanie niewymierności z mianownika.
% \medskip

\begin{uwaga}
    \label{uwga:1:1:5}
    Niech $K\subseteq L_1,K\subseteq L_2$ będą ciałami. Wybieramy $\overline a_1\in L_1$ i $\overline a_2\in L_2$, $|\overline a_1|=|\overline a_2|=n$. Wtedy następujące warunki są równoważne:

\indent 1. istnieje izomorfizm $\phi:K[\overline a_1]\to K[\overline a_2]$ taki, że $\phi\obciete K=id_K$ oraz $\phi(\overline a_1)=\overline a_2$.

\indent 2. $I(\overline a_1/K)=I(\overline a_2/K)$.
\end{uwaga}

\textbf{Dowód:}

%\begin{center}
% $K[\overline a]\isomorphism K[\overline b]$ \acc{$\implies$} $I(\overline a/K)=I(\overline b/K)$

% Niech $\omega\in K[\overline X]$. Wtedy $\omega\in I(\overline a/K)$ wtedy i tylko wtedy, gdy $\omega(\overline a)=0$, to mamy z definicji $I(\overline a/K)$. Wiemy też, że $\phi(a)\in K[\overline X]$ wtedy, gdy $\omega(\phi(\overline a))=0$, a ponieważ $\phi(\overline a)=\overline b$, to również $\omega(\overline b)=0$ i mamy, że $\omega\in I(\overline b/K)$. Czyli izomorfizm między $K[\overline a]=K[\overline b]$ implikuje, że $I(\overline a/K)=I(\overline b/K)$.
% \smallskip

% $K[\overline a]\isomorphism K[\overline b]$ \acc{$\impliedby$} $I(\overline a/K)=I(\overline b/K)$

% Spróbujmy zdefiniować izomorfizm $\phi$ tak, że dla $\omega\in K[\overline X]$ mamy $\phi(\omega(\overline a))=\omega(\overline b)$

% 1. $\phi$ jest homomorfizmem: 
% $$\phi(\omega(\overline a)\cdot v(\overline a))=f((\omega\cdot v)(\overline a))=(\omega\cdot v)(\overline b)=\omega(\overline b)\cdot v(\overline b)=\phi(\omega(\overline a))\cdot \phi(v(\overline a))$$

% 2. $\phi$ jest różnowartościowe:
% $$\phi(\omega(\overline a))=\phi(v(\overline a))\iff \omega(\overline b)=v(\overline b)\iff (\omega-v)(\overline b)=0\iff \omega-v\in I(\overline b/K)=I(\overline a/K)\iff (\omega-v)(\overline a)=0\iff \omega(\overline a)=v(\overline a)$$

% 3. $\phi$ jest dobrze zdefiniowane (czyli przyjmuje tylko jedną wartość dla jednego argumentu):
% $$\omega(\overline a)-v(\overline a)=0\iff (\omega-v)(\overline a)=0\iff \omega -v\in I(\overline a/K)\iff \omega-v\in I(\overline b/K)\iff (\omega-v)(\overline b)=0\iff \omega(\overline b)-v(\overline b)=0$$

% \podz{dark-green}
% \bigskip

% Możemy teraz zapytać, czy każdy ideał w pierścieniu wielomianów $K[X]$ jest postaci $I(\overline a/K)$ dla pewnego $\overline a\in L\supset K$? Albo ogólniej, czy dla pierścienia przemiennego $R$ z $1_R\neq0_R$ oraz ideału $I=(f_1,...,f_m)=I(\overline a/R)\normalsubgroup R[X]$, czy istnieje nadpierścień $S$ taki, że $1_S=1_R$ i $0_S=0_R$ oraz układ
% $$f_1(\overline x)=...=f_m(\overline m)=0$$
% ma rozwiązanie w $S$? Takie rozwiązanie spełniałoby $\overline a\in S\iff (\forall\;g\in(f_1,...,f_m))\;g(\overline a)=0$.
%\end{center}

$1\implies 2$

Implikacja jest jasna, bo dla $g(\overline X)\in K[\overline X]$, bo $g(\overline a_1)=0$ w $K[\overline a_1]$ $\iff$ $g(f(\overline a_1))=0$, a $f(\overline a_1)=\overline a_2$.

$1\impliedby 2$

Zwróćmy uwagę na odwzorowanie ewaluacji $\overline a_1$
$$\phi_{\overline a_1}:K[\overline X]\xrightarrow[]{"na"}K[a_1]$$
zadane wzorem
$$\phi(w(\overline X))=w(\overline a_1).$$
Mamy
$$\ker(\phi_{\overline a_1})=I(\overline a_1/K).$$

Tak samo dla $\overline a_2$ możemy określić analogicznie odwzorowanie ewaluacyjne $\phi_{\overline a_2}:K[\overline X]\to K[\overline a_2]$. Wtedy
$$I(\overline a_2/K)=\ker(\phi_{\overline a_2}),$$
ale ponieważ $I(\overline a_1/K)=I(\overline a_2/K)$, to $\ker(\phi_{\overline a_1})=\ker(\phi_{\overline a_2})$. Oznaczmy $I=I(\overline a_1/K)=I(\overline a_2/K)$. Widzimy, że $\phi_{\overline a_i}\obciete K=id_k$.

\begin{illustration}
    \node (K) at (4, 0) {$K[\overline X]$};
    \node (K[a1]) at (0, -2) {$K[\overline a_1]$};
    \node (K[a2]) at (8, -2) {$K[\overline a_2]$};
    \node (K[X]/I) at (4, -2.5) {$K[X]/I$};
    \draw[->] (K)--(K[a1]) node [midway, left] {$\phi_{\overline a_1}$};
    \draw[->] (K)--(K[a2]) node [midway, right] {$\phi_{\overline a_2}$};
    \draw[->] (K)--(K[X]/I) node [midway, right] {$j${\scriptsize - ilorazowe}};
    \draw[->, dashed] (K[X]/I)--(K[a1]) node [midway, below] {$f_1$} node [midway, above] {$\cong$};
    \draw[->, dashed] (K[X]/I)--(K[a2]) node [midway, below] {$f_2$} node [midway, above] {$\cong$};
\end{illustration}

Niech $f=f_2f_1^{-1}:K[\overline a_1]\to K[\overline a_2]$ jest funkcją spełniającą warunki punktu 1.

{\large\color{orange}MOŻE TUTAJ ŁADNIE SPRAWDZIĆ ŻE NAPRAWDĘ JEST TO DOBRZE SPEŁNIAJĄCA WARUNKI FUNKCJA?}

\textbf{\large\color{yellow}Uwaga.}
    \emph{Niech $I\normalsubgroup K[\overline X]$ \acc{noetherowskiego} pierścienia $K[\overline X]$. Niech $I=(f_1,...,f_m)$ dla pewnych $f_i\in K[\overline X]$. Wtedy istnieje rozszerzenie pierścienia $S\supseteq K$ oraz $\overline a\subseteq S$ - rozwiązanie ogólne układu $f_1(\overline X)=...=f_m(\overline X)=0$ takie, że $\color{blue}I(\overline a/K)=I$.}

\textbf{Dowód:} Wcześniejsze uwagi {\large\color{orange}KTÓRE KONKRETNIE?}

\begin{tw}
    Niech $I\normalsubgroup K[\overline X]$. Wtedy istnieje ciało $L\supseteq K$ oraz $\overline a=(a_1,...,a_n)\subseteq L$ takie, że $f(\overline a)=0$ dla każdego $f\in I$.
\end{tw}

\textbf{Dowód:} Niech $I\subseteq M\normalsubgroup K[\overline X]$ będzie ideałem maksymalnym. Niech $L=K[\overline X]/M$ i określmy przekształcenie ilorazowe
$$j:K[\overline X]/M\to L=K[\overline X]/M.$$
Ponieważ $M\cap K=\{0\}$ (bo inaczej w ideale byłby wielomian odwracalny), to $j\obciete K:K\to L$ jest funkcją $1-1$, czyli
$$j\obciete K:K\xrightarrow[]{1-1}j[K]\subseteq L.$$
Możemy utożsamić $K$ z $j[K]$, czyli $K\subseteq L$. Niech $\overline a=(a_1,..., a_n)$ takie, że dla każdego $i\in[n]$ 
$$a_i=j(X_i)\in L.$$
Wtedy $g(\overline a)=0$ dla każdego $g(\overline X)\in M\supseteq I$ (bo inaczej mielibyśmy wyrazy wolne).

\begin{wniosek}
    \label{wniosek1:2:4}
    Niech $f\in K[X]$ stopnia $>0$. Wtedy istnieje ciało $L\supseteq K$ rozszerzające ciało $K$ takie, że $f$ ma pierwiastek w ciele $L$.
\end{wniosek}

\textbf{Przykłady:}

\indent 1. Rozpatrzmy ciało $K=\Q$ i $f(X)=X-2$. Wtedy $I=(f)\normalsubgroup\Q[X]$ jest ideałem maksymalnym, bo jest on pierwszy (w tym wypadku nierozkładalny). Równanie $f=0$ ma rozwiązanie ogólne w pierścieniu ilorazowym
$$\Q[X]/I\cong \Q.$$
Czyli nie zawsze musimy rozszerzać ciało do czegoś nowego.

\indent 2. $\C=\R[i]=\R(i)=\R[z]$ dla każdego $z\in\C\setminus\R$, co jest na liście zadań.
\medskip

Załóżmy, że $K\subseteq L_1, K\subseteq L_2$ są rozszerzeniami ciała. Wtedy mówimy, że \deff{$L_1$ jest izomorficzne z $L_2$ nad $K$} [\acc{$L_1\cong_KL_2$}] $\iff$ istnieje izomorfizm $f:L_1\to L_2$ taki, że $f\obciete K=id_K$.

\begin{fakt}{\color{back}dupa }
\label{fakt:1:2:5}

\indent 1. Załóżmy, że $f(X)\in K[X]$ jest nierozkładalny. Niech $L_1=K(a_1)$, $L_2=K(a_2)$ i $f(a_i)=0$ w $L_i$. Wtedy $L_1\cong_KL_2$.

\indent 2. Ogółniej: załóżmy, że $\phi:K_1\to K_2$ jest izomorfizmem i $f_1\in K_1[X],f_2\in K_2[X]$, $\phi(f_1)=f_2$, $f_i$ jest nierozkładalne. Dodatkowo załóżmy, że $L_1=K_1(a_1)$ i $L_2=K_2(a_2)$, gdzie $f_i(a_i)=0$ w $L_i$. Wtedy istnieje \acc{izomorfizm $\phi\in\psi:L_1\to L_2$ taki, że $\psi(a_1)=a_2$.}
\end{fakt}

\textbf{Dowód:}

\indent 1. $I(a_1/K)=(f)=I(a_2/K)$, stąd na mocy \ref{uwga:1:1:5} mamy $K(a_1)\cong_KK(a_2)$. Po dowodzie przypadku 2. możemy uzasadniać, że jest to szczególny przypadek tego ogólniejszego stwierdzenia właśnie.

\indent 2. Zacznijmy od rozrysowania tej sytuacji:

\begin{illustration}
    \draw (-2, 0) ellipse (1.5 and 0.7) node [below] {$K_1$};
    \draw (2, 0) ellipse (1.5 and 0.7) node [below] {$K_2$};
    \draw (-3.5, 0)..controls (-3, 3) and (-1, 3)..(-0.5, 0) node [midway, above] {$L_1$};
    \draw (3.5, 0)..controls (3, 3) and (1, 3)..(0.5, 0) node [midway, above] {$L_2$};
    \draw[->] (-2, 1)..controls (-1, 2) and (1, 2)..(2, 1) node [midway, below] {$\psi$} node [midway, above] {$\cong$};
    \draw[->] (-1.2, -0.6)..controls (-0.8, -1) and (0.8, -1)..(1.2, -0.6) node [midway, below] {$\phi$} node [midway, above] {$\cong$};
\end{illustration}

Izomorfizm $\phi:K_1[X]\izo K_2[X]$ indukuje nam przekształcenie
$$K_1[X]/(f_1)\izo{\phi} K_2[X]/(f_2),$$
bo $\phi(f_1)=f_2$. Wiemy, że $f_i$ jest nierozkładalne, czyli
$$I(a_i/K_i)=(f_i)\normalsubgroup K_i[X]$$
jest ideałem maksymalnym. Mamy
$$L_i=K_i(a_i)=K_i[a_i]\cong K[X]/I(a_i/K_i).$$

\begin{illustration}
    \node (K1) at (0, 0) {$K_1[X]$};
    \node (K2) at (4, 0) {$K_2[X]$};
    \draw[->] (K1)--(K2) node [midway, above] {$\cong$} node [midway, below] {$\phi$};
    \node (K1f) at (0, -2) {$K_1[X]/(f_1)$};
    \node (K2f) at (4, -2) {$K_2[X]/(f_2)$};
    \draw[->] (K1f)--(K2f) node [midway, below] {$\phi$} node [midway, above] {$\cong$};
    \draw[->] (2, -.7)--(2, -1.3);
    \node (L1) at (0, -4) {$L_1=K_1(a_1)$};
    \node (L2) at (4, -4) {$L_2=K_2(a_2)$};
    \draw[->] (L1)--(L2) node [midway, above] {$\cong$} node [midway, below] {$\psi$};
    \node (K1) at (0, -5) {$K_1$};
    \node (K2) at (4, -5) {$K_2$};
    \draw[->] (K1)--(K2) node [midway, below] {$\phi$};
    \node[rotate=90] at (0, -4.5) {$\subseteq$};
    \node[rotate=90] at (4, -4.5) {$\subseteq$};
    \draw[->] (K1f)--(L1) node [midway, left] {$\cong$} node [midway, right] {$h_1$};
    \draw[->] (K2f)--(L2) node [midway, left] {$\cong$} node [midway, right] {$h_2$};
\end{illustration}


\newpage

\section{Równania w pierścieniach}

\subsection{Układy równań}

\deff{Notacja:} przez $R, S$ oznaczamy pierścienie przemienne z $1\neq0$. Przez $K, L$ oznaczamy ciała.

Niech $f_1,...,f_n\in R[X_1,..., X_n]=R[\overline X]$.

\deff{Problem:} Czy istnieje rozszerzenie pierścieni z jednością $R\subseteq S$ takie, że układ $U:f_1(\overline X)=...=f_m(\overline X)=0$ ma rozwiązanie w pierścieniu $S$?

$\overline a=(a_1,...,a_n)\subseteq S\supseteq R$ jest rozwiązaniem układu równań $U$ $\iff$ $g(\overline a)=0$ dla każdego wielomianu $g\in(f_1,...,f_m)\normalsubgroup R[X]$.

\textbf{Dowód:} Rozważmy przypadki:

\indent 1. $(f_1,...,f_m)\ni b\neq 0$ i $b\in R$. Wtedy układ $U$ jest sprzeczny i nie ma rozwiązania w żadnym pierścieniu rozszerzającym $R$, więc możemy ten przypadek odrzucić.

\indent 2. $(f_1,...,f_m)\cap R=\{0\}$, czyli negacja pierwszego przypadku. Teraz układ $U$ jest niesprzeczny i skonstruujemy pierścień $S\supseteq R$ z jednością (czyli rozszerzenie pierścienia $S$) i rozwiązanie $\overline a\subseteq S$.

Niech $S=R[\overline X]/(f_1,...,f_m)$ i rozważmy $jR[\overline X]\to S$ ilorazowe. Po pierwsze zauważmy, że $j\obciete R$ jest $1-1$, bo
$$ker(j\obciete R)=ker(j)\cap R=(f_1,...,f_m)\cap R=\{0\}$$
i dlatego 
$$j\obciete R:R\xrightarrow[\cong]{} j[R]\subseteq S.$$
Z uwagi na ten izomorfizm utożsamiamy $R$ z $j[R]$ i $S$ jest więc rozszerzeniem pierścienia $R$.

Niech $\overline a=(a_1,...,a_m)=(j(X_1),...,j(X_m))$, czyli zbiór obrazów wielomianów stopnia $1$ z peirścienia $S$. Wtedy $\overline a$ jest rozwiązaniem układu $U$ w pierścieniu $S$. Oznaczmy funkcję wielomianową przez
$$\hat{f_i}(\overline a)=\hat{f_i}(j(X_1),...,j(X_m))=j(\hat{f_i}(X_1,...,X_m))=j(f_i)=0$$
powyższe równości należy sprawdzić w ramach ćwiczenia.

\deff{Uwaga:} Skonstruowane powyżej rozwiązanie $\overline a$ układu $U$ ma następującą własność uniwersalności. Jeśli $S'\supseteq R$ jest rozszerzeniem pierścieni z 1 i $\overline a'=(a_1',...,a_n')\subseteq S$ jest rozwiązaniem $U$ w $S'$, to istnieje jedyny homomorfizm $h:R[\overline a]\to R[\overline a']$ taki, że $h\obciete R$ jest identycznością na $R$ i $h(\overline a)=\overline a'$. Wszystkie rozwiązania układów sa homomorficzne.

$R[\overline a]\subseteq S$ to podpierścień generowany przez $R\cup\{\overline a\}$, czyli
$$R[\overline a]=\{f(\overline a)\;:\;f(\overline X)\in R[\overline X]\}\subseteq S$$

\textbf{Dowód:} Niech $I=\{g\in R{\overline X}\;:\;g(\overline a')=0\}$ w $S'$. Oczywiście $I\normalsubgroup R[\overline X]$. Znaczy to, że 
$$(f_1,...,f_m)\subseteq I$$
z twierdzenia o faktoryzacji wielomianów w pierściueniu (????) dostajemy od razu
$$R[X]\xrightarrow[]{j} S=R[\overline X]/(f_1,...,f_m)$$
i $R[\overline a']\subseteq S'$. Widzimy, że $I=ker\phi\subseteq ker j=(f_1,...,f_m)$. Z twierdzenia o homomorfizmie peirścieni dostajemy jedyne $h:R[X]/(f_1,...,f_m)\to R[\overline a']$ taki, że $h(\overline a)=\overline a'$.

\deff{Uwaga:} Jeśli $I=(f_1,...,f_m)$ to $h:R[\overline a]\xrightarrow[]{\cong}R[\overline a']$

\deff{Definicja:} Załóżmy, że $S\supseteq R$ jest rozszerzeniem pierścienia oraz $\overline a\in S^n$. Wtedy

\indent I. $I(\overline a/R)=\{g\in R[\overline X]\;:\;g(\overline a)=0\}$

\indent II. $\overline a$: \acc{rozwiązanie ogólne} układu $U$ gdy ideał $I(\overline a/R)=(f_1,...,f_m)$.

\deff{Uwaga:} W sytuacji z definicji powyżej, gdy $U$ jest niesprzeczne, wtedy $\overline a$ jest rozwiązaniem ogólnym układu $U$ $\iff$ zachodzi warunek z gwizdką.

\textbf{Dowód:} ćwiczenia.

\subsection{Ciała}

$K\subseteq L$ i $\overline a\subseteq L$. Definiujemy \acc{ideał $\overline a$ nad $K$} jako
$$I(\overline a/K)=\{g\in K[\overline X]\;:\;g(\overline a)=0\}$$

Wtedy $K[\overline a]=$ podpierścień ciała $L$ generowany przez $K\cup\{a_1,...,a_m\}=\{g(\overline a)\;:\;g\in K[\overline X]$.

$K(\overline a)$ to podciało ciała $L$ generowane przez $K\cup\{a_1,...,a_m\}$. Czyli jest to ciało ułamków pierścienia $K[\overline a]$ w ciele $L$. Inaczej piszemy $K[\overline a]_0$ 
$$K(\overline a)=\{g(\overline a\;:\;g\in K(\overline X)\text{ i } g(\overline a)\text{ jest dobrze określone})\}$$

\deff{Uwaga:} Załóżmy, że $K\subseteq L_1,K\subseteq L_2$ są to rozszerzenia ciał i $\overline a_1\subseteq L_1$, $\overline a_2\in L_2$ i $|\overline a_1|=|\overline a_2|=n$. Wtedy następujące warunki są równoważne:

\indent 1. $(\exists\;f:K[\overline a_1]\xrightarrow[]{\cong}K[\overline a_2])\;f(\overline a_1)=\overline a_2$ i $f\obciete K=id_k$

\indent 2. $I(\overline a_1/K)=I(\overline A_2/K)$

\textbf{Dowód:}

$1\implies 2$ jest jasne, bo dla $g(\overline x)\in K[\overline x]$ takie, że $g(\overline a_1)=0$ w $K[\overline a_1]$ $\iff$ $g(f(\overline a_1))=0$ dla w $K[\overline a_2]$.

$\impliedby$ Zwróćmy uwagę na odwzorowanie ewaluacji $\overline a_1$
$$\phi_{\overline a_1}:K[\overline X]\xrightarrow[]{epi} K[\overline a_1]$$

mamy $\phi_{\overline a_1}(w(\overline x))=w(\overline a_1)$, czyli do wielomianu $\phi$ podstawia $\overline a_1$. Oczywiście, 
$$ker(\phi_{\overline a_1})=I(\overline a_1/K)=I(\overline a_2/K)=ker \phi_{\overline a_2}$$

\deff{Uwaga:} Niech $I\normalsubgroup K[\overline X]$ noetherowskiego pierścienia $K[\overline X]$. I niech $I=(f_1,...,f_m)$  dla pewnych $f_i\in K[\overline X]$. Wtedy istnieje rozszerzenie pierścienia $S\supseteq K$ oraz $\overline a\subseteq S$: rozwiązanie ogólne układu $f_1(\overline X)=..,.= f_m(\overline X)=0$ takie, że $I(\overline a/K)=I$

\textbf{Dowód:} Patrz na poprzednie uwagi, których było już dość dużo.

\deff{Twierdzenie:} Niech $I\normalsubgroup K[\overline X]$. Wtedy istnieje ciało $L\supseteq K$ oraz $\overline a=(a_1,...,a_n)\subseteq L$ takie, że $f(\overline a)=0$ dla każdego $f\in I$.

\textbf{Dowód:} Niech $I\subseteq M\normalsubgroup K[X]$ będzie ideałem maksymalnym. Niech $L=K[\overline X]/M$, $j:K[\overline X]\to L$ ilorazowe, $M\cap K=\{0\}$, więc $j\obciete K:K\to L$ jest $1-1$, a więc
$$j\obciete K:K\xrightarrow[]{1-1}j[K]\subseteq L.$$
Utożsamiamy $K$ z $j[K]$, to znaczy $K\subseteq L$. Niech $\overline a=(a_1,...,a_n)$, $a_i=j(X_i)\in L$. $g(\overline a)=0$ dla każdego $g(\overline X)\in M\subseteq I$.

\textbf{Wniosek:} Niech $f\in K[X]$ stopnia $>0$. Wtedy istnieje ciało $L\supseteq K$ rozszerzające ciało $K$ taki, że $f$ ma pierwiastek w ciele $L$.

\textbf{Przykład:} 

\indent 1. Popatrzmy na ciało $K=\Q$ i $f(X)=X-2$. Wtedy $I=(f)\normalsubgroup \Q[X]$ jest ideałem maksymalnym, bo jest on pierwsz (czyli w tym wypadku nierozkładalny). Równanie $f=0$ ma rozwiązanie ogólne w pierścieniu ilorazowym
$$\Q[X]/I\cong \Q$$

\indent 2. $\C=\R[i]=\R(i)=\R[z]$ dla każdej $z\in \C\setminus\R$.

\deff{Załóżmy, że $k\subseteq L_1$}, $K\subseteq L_2$ to rozszerzenia ciała. Wtedy mówimy, że $L_1$ jest izomorficzne z $L_2$ nad $K$ [$L_1\cong_K L_2$] $\iff$ gdy istnieje izomorfizm $f:L_1\to L_2$ taki, że $f\obciete K=id_k$.

\deff{Fakt:}

\indent 1. Załóżmy, że $f(X)\in K[X]$ jest nierozkładalny. Niech $L_1=K(a_1)$, $L_2=K(a_2)$ $f(a_i)=0$ w $L_i$. Wtedy $L_1\cong_K L_2$.

\indent 2. Ogólnie: załóżmy, że $\phi:K_1\to K_2$ jest izomorfizmem i $f_1\in K_1[X]$, $f_2\in K_2[X]$ i $\phi(f_1)=f_2$, $f_i$ jest nierozkładalne. Dodatkowo załóżmy, że $L_1$ jest rozszerzeniem ciała $K_1$ o element $a_1$ i $L_2=K(a_2)$, gdzie $f_i(a_i)=0$ w $L_i$. Wtedy istnieje izomorfizm $\phi\in \psi:L_1\to L_2$ taki, że $\psi(a_1)=a_2$.

Podpunkt pierwszy jest szczególnym przypadkiem podpunktu 2, gdy $\phi=id$.

\textbf{Dowód:}

\indent 1. $T(a_1/K)=(f)=I(a_2/K)$, stąd na mocy faktu 1.5 mamy $K(a_1)\cong_K K(a_2)$.

\indent 2. Popatrzmy najpierw na izomomfizm $K_1[X]]xrightarrow[\phi]{\cong}K_2[X]$ Wtedy ten $\phi$ indukuje $K_1[X]/(f_1\xrightarrow[\phi]{\cong}K_2[X]/(f_2)$, bo $\phi(f_1)=f_2$. Zatem
$$I(\overline a_i/K_i)=(f_i)\normalsubgroup K_i[X]$$
$$L_i=K_i(a_i)=K_i[a_i]\cong K_i[\overline X]/I(a_j/K_j)$$

Ciało $L\supseteq K$ jest \deff{ciałem rozkładu} [decomposition field] nad $K$ wielomianu $f\in K[X]$, gdy spełnione są warunki:

\indent 1. $f$ rozkłada się w pierścieniu $L[X]$ na czynniki liniowe stopnia $1$

\indent 2. Ciało $L$ jest rozszerzeniem ciała $K$ o elementy $a_1,..., a_n$, gdzie $a_1,..., a_n$ to wszystkie pierwiastki $f$ w $L$.

Nie są warunkami równoważnymi, bo $1$ może być spełnione przez coś większego niż 2, a my chcemy najmniejsze takie ciało.

\textbf{Przykład:} Jeżeli $deg(f)=0$, to nie istnieje ciało rozkładu $f$.

\textbf{Wniosek:} Załóżmy, że $f\in K[X]$ jest wielomianem stopnia $>0$. Wtedy 

\indent 1. istnieje $L$: ciało rozkładu $f$ nad $K$,

\indent 2. ciało to jest jedyne z dokładnością do izomorfizmu nad $K$.

\textbf{Dowód:}

\indent 1. Dowód przez indukcję względem stopnia $f$.

$deg(f)=1\implies L=K$ i jest OK

Załóżmy, że stopień $f>1$ i teza zachodzi dla wszystkich wielomianów stopnia $<deg(f)$ i wszystkich ciał $K'$. Teraz z wniosku 1.7. wiemy, że istnieje rozszerzenie ciała $K$, w którym wielomian $f$ ma pierwiastek, powiedzmy $a_0$ to ten pierwastek:
$$K'=K(a_0)$$
w $K'[X]$ ma pierwastek $a_0$, więc dzieli się przez $(x-a_0)$, więc
$$f=(x-a_0)f_1$$
gdzie $f_1\in K'[X]$, $0<deg(f_1)<deg(f)$. Z założenia indukcyjnego dla $f_1$ istnieje $L'=K'(a_1,..., a_r)$ - ciało rozkładu wielomianu $f_1$ nad $K'$. Wtedy $L=K(a_0,...,a_r)$ jest ciałek rozkładu $f$ nad $K$.

\indent 2. Udowodnimy wersję ogólniejszą: Jeśli $\phi:K_1\to K_2$ jest izomorfizmem nad ciałem i $f_i\in K_i[X]$ jest wielomianem stopnia $>0$, $\phi(f_1)=f_2$, to wtedy istnieje $\psi:L_1\to L_2$ izoorfizm nad ciałami rozkładu tych $K_i$. 

\end{document}