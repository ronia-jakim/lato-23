\section{Wstęp do teorii Galois}

\subsection{Grupy Galois}
    Niech $K$ będzie ciałem, $\hat{K}$ jego algebraicznym domknięciem. Niech $K\subseteq L$ będzie rozszerzeniem algebraicznym ciał [BSO: $L\subseteq\hat{K}$]. \important{Grupą Galois} rozszerzenia $K\subseteq L$ nazywamy 
    $$G(L/K)=Gal(L/K)=\{f\in Aut(L)\;:\;f\restriction K=id_k\}=Aut(L/K)$$
    ze składaniem jako działaniem. Jest to jednocześnie podgrupa wszystkich automorfizmów.

\textbf{Przykład:}

\begin{enumerate}
    \item  Niech $K$ będzie ciałem prostym ($\cong$ z $\Q$ lub z $\Z_p$). Wtedy $Gal(L/K)=Aut(L)$, bo
    \begin{itemize}
        \item  Niech $char(K)=char(L)=p>0$ i niech $f\in Aut(L)$. Wtedy $f(1)=1$, $f(\underbrace{1+....+1}_k)=\underbrace{1+....+1}_k$, a ponieważ $K=\{\underbrace{1+...+1}_k\;:\;k\in\{1,...,p\}\}$, zatem $f\restriction K=id_K$, czyli $f\in Gal(L/K)$.
        \item Niech $char(K)=char(L)=0$, wtedy $K\cong\Q$. Niech $f\in Aut(L)$. Wtedy $f(0)=0, f(1)=1$, a dla dowolnego $k\in\N$ $f\underbrace{1+....+1}_k=\underbrace{1+....+1}_k$, stąd dostajemy, że $f(n)=n$ dla $n\in\Z$, a z własności $\Q$ dostajemy, że $f(\frac mn)=\frac mn$, zatem $f\restriction K=id_K$.
    \end{itemize}
    \item $Gal(\Q(\sqrt2)/\Q)=Aut(\Q(\sqrt2))=\{f_0,f_1\}\cong\Z$, bo $\sqrt{2}$ może przejść na siebie albo na $-\sqrt{2}$. Wtedy $f_0=id$, a $f_1(-\sqrt{2})$ 
\end{enumerate}

    Grupę Galois $Gal(\hat{K}/K)$ nazywamy \important{absolutną grupą Galois} ciała $K$.

\emph{Czy każda grupa skończona jest izomorficzna z $Gal(L/\Q)$ dla pewnego $\Q\subseteq L$?} Jest to otwarty problem teorii Galois.


\begin{remark}[jednorodność $\hat{K}$]\label{uwaga:6:1}
    $a,b\in\hat{K}$, takie, że $I(a/K)=I(b/K)$, to wtedy istnieje $f\in Gal(\hat{K}/K)$ takie, że $f(a)=b$.
\end{remark}

\begin{proof}{\color{pagColor}dupa}

\begin{center}
\begin{tikzcd}
    K[a]\arrow[r, "\cong" above, "f" below]\arrow[d,"\subseteq"] & K[b]\arrow[d, "\subseteq"]\\
    K[a]^{alg}=\hat{K}\arrow[r, "\exists\;f'" above, "\cong" below] & \hat{K}=K[b]^{alg}
\end{tikzcd}
\end{center}

Co jest wnioskiem z wniosku \ref{wniosek:5:10}.
\end{proof}

\subsection{Rozszerzenia algebraiczne normalne}

\phantomsection
\label{aaaa}
$\hat{K}$ jest największym algebraicznym rozszerzeniem $K$ tzn. $K\subseteq L$ oznacza, że istnieje $f:L\to\hat{K}$ monomorfizm ciał taki, że $f\restriction K=id_K$. \hyperref[aaaa]{{\color{yellow}(\Coffeecup)}}
\medskip

Mówmy, że rozszerzenie algebraiczne $K\subseteq L$ jest \important{normalne}, gdy w \hyperref[aaaa]{{\color{yellow}(\Coffeecup)} }$f[L]\subseteq\hat{K}$ dla wszystkich $f:L\to K$.

\textbf{Przykład} Rozszerzenie $K\subseteq\hat{K}$ jest normalne.

\begin{remark}\label{uwaga:6:2}
Załóżmy, że $K\subseteq L\subseteq\hat{K}$. Wtedy rozszerzenie $K\subseteq L$ jest normalne $\iff$ dla każdego $f\in Gal(\hat{K}/K)$ $f[L]=L$.
\end{remark}

\begin{proof}
$\implies$ z definicji, bo $id_K[L]=L$.

$\impliedby$ z definicji.

\end{proof}

Czyli $K\subseteq L_1\subseteq L$ i $K\subseteq L$ jest normalna, to $L_1\subseteq L(\subseteq \hat{K})$, więc $Gal(\hat{L_1}/L_1)\leq Gal(\hat{K}/K)$.

\begin{theorem}[rozszerzenie jest normalne]
Dla $K\subseteq L$ algebraicznego rozszerzenia jest normalne $\iff$ dla każdego $b\in L$ wielomian minimalny $f\in K[X]$ rozkłada się w $L[X]$ na iloczyn czynników liniowych.
\end{theorem}
\begin{proof}
Bez straty ogólności rozważamy $L\subseteq\hat{K}$. 

$\implies$ 

Dowód nie wprost, to znaczy załóżmy, że istnieje $b\in L$ takie, że $w_b(x)$ ma pierwiastek $a\in \hat{K}\setminus L$. Ale wtedy z Uwagi \ref{uwaga:6:1}. na jednorodność $\hat{K}$ istnieje $f\in Gal(\hat{K}/K)$ takie, że $f(b)=a$, więc $f[L]=L$ co jest sprzeczne z \ref{uwaga:6:2}.

$\impliedby$

Załóżmy nie wprost, że na mocy \ref{uwaga:6:2}. istnieje $f\in Gal(\hat{K}/K)$ takie, że $f[L]\neq L$. Ale $L$ i $f[L]$ są wzajemnie sprzężone, więc wybierzmy $a\in L\setminus f[L]$. Symetrycznie, $a'\in f[L]\setminus L$, $f':f[L]\isomorphism L$ spełnia warunek \hyperref[aaaa]{{\color{yellow}(\Coffeecup)}}.

Niech $w_a(x)$ jest wielomianem minimalnym $a$ nad $K$. Wtedy $w_a(X)=f(w_a(x))$, bo $f\restriction K=id_K$. Czyli $w_a$ jest wielomianem minimalnym dla $b=f(a)/K$. Czyli $L\overset{f}{\cong}_Kf[L]$. Z \hyperref[aaaa]{{\color{yellow}(\Coffeecup)}} wiemy, że $w_a(x)$ rozkłada się nad $L$ na czynniki liniowe. Czyli $w_a(x)....f[L]...$, co daje nam sprzeczność, bo $a$ jest pierwiastkiem $w_a(X)$, ale $a\notin f[L]$.
\end{proof}

Rozszerzenie ciał $K\subseteq L$ jest \important{skończone}, jeśli $[L:K]<\infty$.

\begin{theorem}[skończone i normalne $\iff$ ciało rozkładu wielomianu]
    Niech $K\subseteq L$ będą rozszerzeniami ciał. Wtedy następujące warunki są równoważne:
    \begin{enumerate}
        \item rozszerzenie $K\subseteq L$ jest skończone i normalne
        \item $L$ jest ciałem rozkładu pewnego wielomianu
    \end{enumerate}
\end{theorem}
\begin{proof} Bez straty ogólności załóżmy, że $K\subseteq L\subseteq\hat{K}$.

$(2)\implies(1)$

Załóżmy, że $L$ jest ciałem rozkłądu pewnego wielomianu. Wtedy $L=K(a_1,...,a_n)$, gdzie $a_1,...,a_n$ to wszystkie pierwiastki wielomianu $w(x)$ w $\hat{K}$.

Niech $f\in Gal(\hat{K}/K)$, wtedy $f(a_1,...,f(a_n)$ to też wszystkie pierwiastki wielomianu $w(x)$. Stąd 
$$f[L]=K(f(a_1),...,f(a_n))=K(a_1,...,a_n)=L,$$
zatem rozszerzenie $K\subseteq L$ jest normalne i skończone.

$(1)\implies(2)$

Niech $K\subseteq L$ będzie skończone i normalne. Wtedy $L=K(a_1,...,a_n)$ dla pewnych $a_1,...,a_n\in L$ i $\{a_1,...,a_n\}$ będzie bazą $L$ nad $K$. Wtedy istnieje $w\in K[X]\setminus\{0\}$ takie, że $w(a_1)=...=w(a_n)=0$, zatem 
$$L\supseteq\{\text{ pierwiastki }w\}\supseteq\{a_1,...,a_n\}.$$

{\large\color{orange}COŚ TUTAJ JEST NIE TAK}
\end{proof}

\textbf{Przykłady:} 
\begin{enumerate}
    \item Niech $K\subseteq L$ będą ciałami skończonymi, wtedy $K\subseteq L$ jest ciałem normalnym, bo $|L|=p^n$, $w_{p^n-1}(x)=x^{p^n-1}-1$ i $L$ jest ciałem rozkładu $w$ nad $K$.
    \item $\Q\subseteq\Q(\sqrt[3]{2})$ to rozszerzenie skończone, ale nie normalne. Jest tak, bo
        \begin{itemize}
            \item $x^3-2$ jest nierozkładalny nad $\Q$ (kryterium Eisteina)
            \item W ciele $\C$ $x^3-2$ ma $3$ pierwiastki, z których tylko jeden jest w $\Q(\sqrt[3]{2})\subseteq\R$a
        \end{itemize}
\end{enumerate}

\begin{remark}
    Niech $K\subseteq L\subseteq \hat{K}$ i niech $L_1$ będzie ciałem generowanym przez $\bigcup\{f[L]\;:\;f\in Gal(\hat{K}/K)\}$. Wtedy $L_1$ to \important{normalne domknięcie ciała $L$ w $\hat{K}$}. Wtedy
    \begin{enumerate}
        \item Rozszerzenie $K\subseteq L_1$ jest normalne
        \item Jeśli $K\subseteq L_2$ i $L\subseteq L_2$ są normalne, to istnieje monomorfizm $L_1\to L_2$ taki, że $f\restriction K=id$.
    \end{enumerate}
\end{remark}
\begin{proof}
    (1) Z \ref{uwaga:6:2}

    (2)

    Bez straty ogólności załóżmy, że $K\subseteq L\subseteq L_2\subseteq\hat{K}$ i $K\subseteq L\subseteq L_2\subseteq\hat{K}$. Niech $f\in Gal(\hat{K}/K),f[L]\subseteq L_2$. W takim razkie $\bigcup\{f[L]\;:\;f\in Gal(\hat{K}/K)\}\subseteq L_2$, z czego wynika, że $L_1\subseteq L_2$.
\end{proof}

\subsection{Rozszerzenia rozdzielcze}

\begin{bbox}
\begin{itemize}[leftmargin=*]
    \item Niech $K$ będzie ciałem i $a\in \hat{K}$. Mówimy, że $a$ jest \important{rozdzielczy nad $K$}, gdy wielomian minimalny $a$, $w_a(x)\in K[X]$ ma tylko pierwiastki jednokrotne w $\hat{K}$.
    \item Algebraiczne rozszerzenie $K\subseteq L$ jest \important{rozszerzeniem rozdzielczym}, gdy dla każdego $a\in L$ $a$ jest rozdzielcze nad $K$.
    \item Wielomian $w(x)\in K[X]$ jest \important{rozdzielczy}, gdy $w$ ma tylko pierwiastki jednokrotne w $\hat{K}$.
\end{itemize}
\end{bbox}

\begin{remark}[nierozkładalny a rozdzielczy]
Załóżmy, że $w(x)\in K[X]$ jest wielomianem nierozkładalnym stopnia $>0$. Wtedy
\begin{enumerate}
    \item $w(x)$ jest rozdzielczy $\iff$ $w(x)$ i $w'(x)$ są względnie pierwsze
    \item Jeśli $char(K)=0$, to $w$ jest rozdzielczy
    \item Jeśli $char(K)=p>0$, to $w$ jest nierozdzielczy $\iff$ $w(x)\in K[X^p]$, to znaczy $w(x)=v(x^p$ dla pewnego $v(x)\in K[X]$).
\end{enumerate}
\end{remark}
\begin{proof}
    Dowód zadanie  z listy 4
\end{proof}

\textbf{Przykłady:}
\begin{enumerate}
    \item Niech $K\subseteq L$ będzie rozdzielcze i $K\subseteq L_1\subseteq L$. Wtedy $L_1\subseteq L$ też jest rozdzielcze [ćwiczenia]
    \item Jeśli $char(K)=0$, to każde rozszerzenie algebraiczne ciała $K$ jest rozdzielcze.
    \item Niech $K\subseteq L$ będą ciałami skończonymi. Wtedy $K\subseteq L$ jest rozdzielcze.
    
    Ciał $L$ rozkładu wielomianu $x^{p^n}-x$ o pierwiastkach jednokrotnych.
    \item Rozszerzeni nierozdzielnicze: niech $K=F_p(X)\subseteq L=K(\sqrt[p]{x})$. Niech $w_a(T)=T^p-x\in K[T]$ będzie wielomianem minimalnym $a=\sqrt[p]{x}$. Wtedy $w_a'=0$, czyli w ciele $L$ istnieje $p$-krotny pierwiastek $w_a$: $w_a(T)=(t-a)^p$.a
\end{enumerate}

\begin{lemma}\label{lemat:6:7}
    $ $\newline
    \begin{enumerate}
        \item Jeśli $a\in\hat{K}$, to $|\{f(a)\;:\;f\in Gal(\hat{K}/K)\}|\leq$ stopień $a$ nad $K$
        \item $a$ jest rozdzielczy nad $K$ $\iff$ w podpunkcie (1) jest równość. 
    \end{enumerate}
\end{lemma}
\begin{proof}
    $$\{f(a)\;:\;f\in Gal(\hat{K}/K)\}\overset{\ref{uwaga:6:1s}}{=}\{\text{pierwiastki wielomianu minimalnego }w_a\in K[X]\text{ nad } K\}$$
    czyli $deg(a/K)=deg(w_a)$.
\end{proof}

\begin{bbox}
    Element $a\in L$ nazywamy \important{elementem pierwotnym} rozszerzenia $K\subseteq L$, gdy $L=K(a)$.
\end{bbox}

\begin{theorem}[Abela o elemencie pierwotnym]
    Niech $K\subseteq L$ będzie rozszerzeniem skończonym, $L=K(a_1,...,a_n)$ i $a_1,...,a_n$ rozdzielcze nad $K$. Wtedy istnieje $a^*\in L$ rozdzielczy nad $K$ taki, że $L=K(a^*)$.
\end{theorem}
\begin{proof}
    Bez starty ogólności załóżmy, że $K\subseteq L\subseteq \hat{K}$. Rozważmy dwa przypadki:
    \begin{enumerate}
        \item $K$ jest skończone. Wtedy $L$ także jest skończone, a $L^*$ jest cykliczna. Niech więc $a^*\in L^*$ będzie generatorem $L^*$. Wtedy $L=K(a^*)$.
        \item $K$ jest nieskończone.
        
        Dowód przez indukcję względem $n$. Dla $n=1$ jest oczywiste. Robimy więc krok indukcyjny $(n-1)\implies n$:
        $$K(a_1,...,a_{n-1})=K(b)$$
        $$K(a_1,...,a_{n-1}, a_n)=K(b, a_n)$$
        Niech teraz $k$ będzie stopniem $b$ nad $K$, a $m$ - stopniem $a_n$ nad $K(b)$. Z lematu \ref{lemat:6:7} wiemy, że istnieją $f_1,...,f_k\in Gal(\hat{K}/K)$ takie, że $f_1(b),...,v_k(b)$ są parami różne.  Niech więc $f_{1,1},...,f_{1,m}\in G(\hat{K}/K(b))$ takie, że $f_{1,1}(a),...,f_{1,m}(a)$ są parami różne. 

        Dla $i=1,...,k,j=1,...,m$ niech $f_{i,j}=f_i\circ f_{1,j}\in Gal(\hat{K}/K)$.
        \begin{center}
            \begin{tikzcd}
                K(b)(a)\arrow[r, "f_{i,j}"] & K(b, f_{1,j}(a))\arrow[r, "f_i"] & K(f_i(b), f_i(f_{1,j}(a)))\\
                K(b)\arrow[r]\arrow[u, "\subseteq"]\arrow[ur, "\subseteq"] & K(f_i(b))\arrow[ur, "\subseteq"]\\
                K\arrow[u,"\subseteq"] & K\arrow[u, "\subseteq"]
            \end{tikzcd}
        \end{center}
        Zauważmy, że 
        $$\langle i,j\rangle\neq\langle i',j'\rangle\implies\langle f_{i,j}(a),f_{i,j}(b)\rangle\neq\langle f_{i',j'}(a), f_{i',j'}(b)\rangle,$$
        bo są dwie możliwości: 
        \begin{itemize}
            \item $i\neq i'$, wtedy $f_{i,j}=f_i(b)\neq f_{i'}(b)=f_{i',j'}(b)$
            \item $i=i'\;\land\;j\neq j'$, wtedy $f_{ij}(a)=f_i(f_{1,j}(a))\neq f_{i'}(f_{1,j}(a))=f_{i'j'}(a)$, bo $f'_{1,j}(a)\neq f'_{1, j'}(a)$.
        \end{itemize}
        Skoro $K$ było nieskończone, to istnieje $c\in K$ takie, że dla $\langle i,j\rangle\neq\langle i',j'\rangle$ mamy
        $$f_{i,j}(b)+f_{i,j}(a)\cdot c\neq f_{i',j'}(b)+f_{i',j'}(a)\cdot c,$$
        bo
        $$F(x)=\prod\limits_{\langle i,j\rangle\neq\langle i',j'\rangle}[f_{i,j}(b)+f_{ij}(a)x-(f_{i'j'}(b)+f_{i'j'}(a)x)]$$
        i $c$ po prostu nie jest pierwiastkiem $F$.

        Postulujemy, że $K(b,a_n)=K(a^*)$, gdzie $a^*=b+a_nc$ jest elementem pierwotnym.

        $\supseteq$ jest jasne

        $\subseteq$ $f_{ij}(a^*)$, $1\leq i\leq k,1\leq j\leq m$ parami różne.

        Wiemy, że $deg(a^*/K)\geq k\cdot m$, z drugiej strony
        $$k\cdot m\leq [K(a^*):K]\leq [K(a_b,b):K]=[K(b):K][K(a_n,b):K(b)]=km$$
        czyli wszędzie wyżej są równości i mamy $K(a^*)=K(a_n,b)$.
    \end{enumerate}
\end{proof}

\begin{conclusion}\label{wniosek:6:9}
    $ $\newline
    \begin{enumerate}
        \item Jeśli $L=K(a_1,...,a_n)$ i $a_i$ są rozdzielcze nad $K$, to $L\supseteq K$ też jest rozdzielcze.
        \item $K\subseteq L$ jest rozdzielcze i $L\subseteq M$ jest rozdzielcze, to $K\subseteq M$ też jest rozdzielcze.
    \end{enumerate}
\end{conclusion}
\begin{proof}
    \begin{enumerate}[leftmargin=*]
        \item Niech $L=K(a)$ i $a$ jest rozdzielczy nad $K$. Załóżmy, że $b\in L$ nie jest rozdzielczy nad $K$. Wtedy $L=K(b,a)$.
        \begin{illustration}
        \node (L1) at (0, 0) {$n\cdot m$};
        \node (C1) at (2.3, 0) {$n$};
        \node (R1) at (4, 0) {$m$};
        \node (L2) at (0, -1) {$deg(a/K)$};
        \node (C2) at (3.5, -1) {$deg(b/K)\cdot deg(a/K(b))$};
        \node (L3) at (0, -2) {$[K(a):K]$};
        \node (C3) at (3.6, -2) {$[K(b):K]\cdot [K(a,b):K(b)]$};
        \node[rotate=90] at (0,-.5) {$=$};
        \node[rotate=90] at (2.3,-0.5) {$=$};
        \node[rotate=90] at (4,-.5){$=$};
        \node[rotate=90] at (0,-1.5){$=$};
        \node[rotate=90] at (2.3,-1.5) {$=$};
        \node[rotate=90] at (4,-1.5) {$=$};
        \node at (1.2, -1) {$=$};
        \node at (1.2, -2) {$=$};
        \end{illustration}
        Wybierzmy teraz $g\in K[X]$ takie, że $g(a)=b$. Wtedy
        $$n\cdot m=|\{f(a)\;:\;f\in Gal(\hat{K}/K)\}|=(\star),$$
        bo $a$ jest rozdzielczy nad $K$. Dalej, 
        $$(\star)=|\{(f(b), f(a))\;:\;f\in Gal(\hat{K}/K)\}|=(\star\star),$$
        bo $f(b)$ ma $k<n$ możliwości, gdyż $b$ nie jest rozdzielczy nad $K$ i korzystamy z \ref{lemat:6:7}. Przy ustalonym $f(b)$ skakać po $f(a)$ możemy na co najwyżej $m$ sposobów, bo $deg(a/K(b))=m=deg(f(a)/K(f(b))$. Czyli koniec końców
        $$(\star\star)\leq k\cdot m<n\cdot m,$$
        co daje sprzeczność.
        \item Podobny dowód zostawiony studentowi do pokiwania głową, że rozumie a w duszy płacz bo co się dzieje?
    \end{enumerate}
\end{proof}
































