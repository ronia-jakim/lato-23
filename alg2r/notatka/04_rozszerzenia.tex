\section{Rozszerzenia ciał}

\begin{definicja}
Niech $K\subseteq L$ będą ciałami i $a\in L\setminus K$.

\indent \point Jeżeli \acc{$a$ jest algebraiczny nad $K$}, to istnieje $f\in K[X]$ stopnia $>0$ i $f(a)=0$

\indent \point $a$ jest \acc{przestępny nad $K$} [transcendental] $\iff$ $a$ nie jest algebraiczny.

\indent \point \deff{Rozszerzenie} $L\supseteq K$ jest \deff{algebraiczne} $\iff$ dla każdego $a\in L$ $a$ jest algebraiczny nad $K$.

\indent \point \deff{Rozszerzenie jest przestępne} $\iff$ nie jest algebraiczne.

\indent \point Niech $a\in \C$. Wtedy $a$ jest algebraiczna, gdy $a$ jest algebraiczna nad $\Q$.

\end{definicja}

\textbf{Przykłady}:

\indent 1. W $\C$ na $i$ jest pierwiastkiem algebraicznym wielomianu $x^2+1$, a $\sqrt[n]{d}$ jest pierwiastkiem $x^n-d$. 

\indent 2. Ciało $F(p^n)$ ma charakterystykę $p$ i $F(p)\subseteq F(p^n)$ jest rozszerzeniem ciał, które jest algebraiczne. Dla dowolnego $a\in F(p^n)$ to jest ono pierwiastkiem wielomianu $X^{p^n}-X$, czyli $a$ jest algebraiczne nad $F(p)$.

\indent 3. Pierwiastki przestępne to na przykład $e,\pi,E^\pi$, aczkolwiek nie jesteśmy pewni tego ostatniego [doczytać w S. Lang, Algebra].

\indent 4. Rozważamy $K\subseteq L=K(X)$, czyli pierścień ułamków. Weźmy $x\in K(X)$ - przestępny nad $K$. Załóżmy, że istnieje wielomian $f\in K[X]$ rózny od $0$. I załóżmy, że $0=\hat{f}(X)$ to funkcja wielomianowa. 
$$0=\hat{f}(X)=f\neq0$$
i jest to sprzeczność.


\begin{uwaga}
    Niech $a$ jak wyżej. Wtedy $a$ jest algebraiczny nad $K$ $\iff$ $I(a/K)\neq\{0\}$ jako ideał $K[X]$.
\end{uwaga}

\subsection{Wymiar przestrzeni liniowej}

Niech $K\subseteq L$ będzie rozszerzeniem ciała $K$. Wtedy $L$ jest \deff{przestrzenią liniową nad $K$}. Definiujemy stopień rozszerzenia [coś innego jak indeks przy grupach]
$$[L:K]:=\dim_K(L)$$
jako \acc{wymiar przestrzeni liniowej} nad $K$.

\begin{uwaga}
    Niech $a\in L\setminus K$. Następujące warunki są równoważne:

\indent 1. $a$ jest algebraiczny nad $K$

\indent 2. $K[a]=K(a)$, to znaczy $K[a]$ jest ciałem (usuwanie niewymierności z mianownika)

\indent 3. $[K(a):K]=\dim_K(a)<\infty$
\end{uwaga}

\textbf{Dowód:}

$1\implies2$

Wiemy, że $K[X]$ jest euklidesowy (bo $K$ to ciało), więc $K[X]$ jest też PID.

Skoro $a$ jest algebraiczny nad $K$, to istnieje $f\in K[X]$ takie, że $f(a)=0$, a więc
$$0\neq I(\overline a/K)\normalsubgroup K[X]$$
czyli $I(a/K)$ jest maksymalnym ideałem głównym. Teraz, jeśli $I\normalsubgroup R$ jest ideałem maksymalnym pierścienia $R$, to $R/I$ jest ciałem. Czyli
$$K[a]\cong K[X]/I(a/K)$$
jest ciałem.

$2\implies 3$

Załóżmy, że $a\neq 0$. Wtedy $a^{-1}\in K[a]$, czyli istnieje wielomian $f\in K[X]$ taki, że 
$$f(x)=\sum\limits_{i=1}^n b_ix^i,\quad b_n\neq 0$$
i $a^{-1}=f(a)$. Wobec tego mamy
$$1=f(a)\cdot a$$
$$0=f(a)a-1=b_na^{n+1}+b_aa^2+...+b_0a-1,$$
stąd mamy, że
$$a^{n+1}=-\frac1{b_n}(b_{n-1}a^n+...+b_0a-1)\in Lin_K(1, a, ..., a^n)$$
jest w domknięciu liniowym $(1, a,..., a^n)$. Indukcyjnie można pokazać, że
$$(\forall\;m\geq0)\;a^m\in Lin_K(1,a,...,a^n),$$
czyli 
$$K[a]=K(a)=Lin_K(1,a,...,a^n),$$ 
co daje, że $[K(a):K]\leq n<\infty$.

$3\implies 1$

$[K(a):K]<\infty$, z czego wynika, że
$$\{1,a,...,a^n,...,\}=\{a^t\;:\;t\in\N\}\subseteq K(a)$$
jest zbiorem liniowo zależnym. Z liniowej zależności wiemy, że
$$(\exists\;n\in\N)(\exists\;b_{n-1},...,b_0)\;a^{n}=b_{n-1}a^{n-1}+...+b_1a+b_0.$$
Stąd dla $f\in K[X]$ zadanego wzorem
$$f(x)=x^n+b_{n-1}x^{n-1}+...+b_0$$
mamy $f(a)=0$, zatem $a$ jest algebraiczny nad $K$.
\medskip

Niech $a\in L\supseteq K$ będzie algebraicznym pierwiastkiem nad $K$, $I(a/K)=\{w\in K[X]\;:\;w(a)=0\}=(f)$, $f\neq 0$, $f\in K[X]$, $f$ unormowany (\dyg{czyli współczynnik przy wyrazie wiodącym jest $1$?})

\indent \point $f$ jest nazywany wielomianem \deff{minimalnym} $a$ nad $K$ (wyznaczony jednoznacznie)

\indent \point \acc{stopień $a$} nad $K$ jest definiowany jako $deg(f)$.
\medskip

\textbf{Przykład:}

\indent 1. $\sqrt{2}\in\R\supseteq\Q$, wtedy $f(x)=x^2-2$ jest wielomianem minimalnym $\sqrt2$ nad $\Q$ i stopień $\sqrt{2}$ nad $\Q$ jest równy $2$.

\indent 2. $\pi\in\R$ nie ma stopnia, bo $\pi$ nie jest liczbą algebraiczną nad $\Q$

\indent 3. $\sqrt[7]{7+\sqrt[3]{3}}-\sqrt[6]{6}\in\R$, czy jest to algebraiczne nad $\Q$? Tak i ma stopień $126$.

\begin{uwaga}[$I(a/K)=(f)\implies deg(f)=\begin{bmatrix}K(a):K\end{bmatrix}$]
    Załóżmy, że $I(a/K)=(f)$ i $f$ jest unormowany. Wówczas:

    \indent 1. $f$ jest unormowanym wielomianem minimalnego stopnia takim, że $f(a)=0$

    \indent 2. $deg(f)=[K(a):K]$, czyli stopień tego wielomianu jest równy stopniu przestrzeni liniowej $K(a)$ nad $K$.
\end{uwaga}

\textbf{Dowód:}

Niech $n=deg(f)$, 
$$f(x)=x^n+\sum\limits_{k<n}b_kx^k$$
Z tego, że $f(a)=0$ mamy, że 
$$a^n=-\sum\limits_{k<n}b_kx^k\in Lin_K(1,a,...,a^{n-1})\subseteq L.$$
Czyli $K(a)=Lin_K(1,a,...,a^{n-1})$ i wystarczy zobaczyć, że $\{1,..., a^{n-1}\}$ jest liniowo niezależny nad $K$, to znaczy jest bazą $K(a)$ nad $K$. Jest, bo $f$ jest minimalnego stopnia.

\begin{fakt}[$dim_K(M)=dim_L(M)\cdot dim_K(L)$]\label{fakt:4:5}
    Niech $K\subseteq L\subseteq M$ będą rozszerzeniami ciał. Wtedy 
    $$[M:K]=[M:L]\cdot [L:K]$$
\end{fakt}

\textbf{Dowód:}

Niech $\{e_i\;:\;i\in I\}$ będzie bazą $L$ nad $K$, a $\{f_j\;:\;j\in J\}$ będzie bazą $M$ nad $L$. Stąd $|I|=[L:K]$ i $|J|=[M:L]$.

Chcemy za pomocą tych dwóch zbiorków zrobić bazę $M$ nad $K$. Rozważmy zbiór
$$X=\{e_i\cdot f_j\;:\;i\in I,j\in J\}.$$
Musimy pokazać, że 

1. $|X|=|I|\cdot|J|$

2. $X$ jest liniowo niezależny

3. $X$ jest bazą $M$ nad $K$

Te dwa ostatnie mówią, że $X$ jest bazą.

1. Załóżmy, nie wprost, że dla $i\neq i'$ i $j\neq j'$ i $e_if_j=e_{i'}f_{j'}$. Czyli
$$e_if_j-e_{i'}f_{j'}=0,$$
czyli $f_j,f_{j'}$ są liniowo zależne nad $L$, czyli mamy, że $f_j=f_{j'}$ i
$$0=e_if_j-e_{i'}f_{j}=(e_i-e_{i'})f_j\implies e_i-e_{i'}=0\implies i=i'$$

2. Załóżmy nie wprost, że $X$ nie jest lnz, czyli istnieją $k_{ij}\in K$ takie, że
$$\sum\limits_{j\in J}\sum\limits_{i\in I}k_{ij}e_if_j=0,$$
ale $\sum\limits_ik_ije_i=l_j$ są elementami $L$, czyli
$$\sum\limits_{j\in J}l_jf_j=0$$
więc $f_j$ są liniowo zależne, a przecież były bazowe, w takim razie
$$0=l_j=\sum\limits_{i\in I}k_{ij}e_i,$$
$e_i\neq0$, czyli $k_{ij}=0$ i koniec.

3. $X$ generuje $M$ nad $K$, bo dla $m\in M$ mam
$$m=\sum l_jf_j=\sum\left(\sum a_{ij}e_i\right)f_j=\sum\sum a_{ij}e_if_j=\sum \sum k_{ij}e_if_j$$
Z tego wynika, że $[M:K]=|X|=|I||J|=[L:K][M:L]$.

\begin{wniosek}[$K^{alg}$ - podciałem]
    Niech $K\subseteq L$ będzie rozszerzeniem skończonego ciała. Niech 
    $$K^{alg}(L)=\{a\in L\;:\;a\text{ jest algebraiczny nad }K\}.$$
    Okazuje się, że $K^{alg}$ jest podciałem.
\end{wniosek}

\textbf{Dowód:}

Weźmy $a,b\in K^{alg}$. Wiemy, że $[K(a):K]$ i $[K(b):K]$ są skończone. Mamy, że
$$K\subseteq K(a)\subseteq K(a, b)$$
Z faktu \ref{fakt:4:5} wiemy, że 
$$[K(a, b):K]=[K(a,b):K(a)]\cdot[K(a):K]$$
czyli również $K(a,b)$ jest skończone. Zatem dla $x\in K(a,b)$ mamy
$$[K(x):K]\leq[K(a,b):K]$$
też jest skończone, zatem $x$ jest algebraiczny nad $K$.
\medskip

Dla $x\in K(a, b)$ mamy $[K(x):K]\leq[K(a):K]$, czyli również jest skończone. W takim razie, $x$ jest algebraiczny nad $K$ i należy do $K^{alg}$.
\medskip

\begin{definicja}[(relatywne) algebraiczne domknięcie]{\color{back}spaaaać}
    \begin{enumerate}
        \item $K^{alg}(L)$ nazywamy \deff{algebraicznym domknięciem} $K$ w $L$.
        \item $K$ jest \deff{relatywnie algebraicznie domknięte} w $L$ $\iff$ $K^{alg}(L)=K$.
    \end{enumerate}
\end{definicja}

\begin{wniosek}[algebraiczne rozszerzenia ciał, $K^{alg}$]{\color{back}kot}
    \begin{itemize}
        \item Niech $K\subseteq L\subseteq M$ będą rozszerzeniami ciał. $K\subseteq M$ jest algebraiczne $\iff$ $K\subseteq L$ i $L\subseteq M$ są algebraiczne
        \item $K^{alg}(L)$ jest relatywnie algebraicznie domknięte w $L$, tzn. $K^{alg}(L)=[K^{alg}(L)]^{alg}(L)$ 
    \end{itemize}
\end{wniosek}

\subsection{Wielomian rozkładu koła}

Rozważamy wielomian
$$w_m(x)=x^m-1$$
dla $m\in\N$. Wiemy, że
\begin{itemize}
    \item[\point] pierwiastki $w_m$ w $\C$ są jednokrotne
    \item[\point] $\mu_m(\C)$ jest grupą cykliczną
    \item[\point] $a\in\mu_m(\C)$ jest generatorem $\mu_m(\C)=\{a^i\;:\;0\leq i\leq m\}\cong(\Z_m,+)$
    \item[\point] $a^k$ generuje $\mu_m(\C)$ $\iff$$NWD(k, m)=1$
\end{itemize}
Zatem $\mu_m(\C)$ ma $\phi(m)$ generatorów.{\large\color{red}?????}

Niech
$$\{k\in\N\;:\;0<k<m, NWD(k, m)=1\}=\{m_1,...,m_{\phi(n)}\}$$
i zdefiniujmy
$$\color{blue}F_m(x):=(x-a^{m_1})...(x-a^{m_{\phi(n)}})\in\C[X]$$

\begin{uwaga}[$F_m\in\Z\begin{bmatrix}X\end{bmatrix}$]{\color{back}ddd}
    \begin{enumerate}
        \item $w_m(x)=x^m-1=F_m(x)\cdot\prod\limits_{\substack{d<m\\d|m}}F_d(x)$
        \item $F_m(x)\in\Z[X]$
    \end{enumerate}
\end{uwaga}

\textbf{Dowód:}

1. Wiemy, że wielomian jest iloczynem dwumianów $x-$\emph{pierwiastek}, te dla $w_m(x)$ są schowane w $\mu_m(\C)$, czyli
\begin{align*}
    w_m(x)&=\prod\limits_{b\in\mu_m(\C)}(x-b)=\prod\limits_{d|m}\prod\limits_{\substack{b\in\mu_m(\C)\\ord(b)=d}}(x-b)=\prod\limits_{d|m}F_d(x)=F_m(x)\prod\limits_{\substack{d<m\\d|m}}F_d(m)
\end{align*}

2. Dowód przez indukcję względem $m$. Dla $m=1$ mamy $F_m(x)=x-1\in\Z[X]$. Teraz zakładamy, że dla $d<m$ jest $F_d(x)\in\Z[X]$. Z punktu (1) wiemy, że
$$x^m-1=w_m(x)=F_m(x)v_m(x)$$
z założenia indukcyjnego $v_m(x)\in\Z[X]$, bo każdy z nich ma stopień mniejszy niż $m$ i $v_m(x)$ jest unormowany.

$w_m(x)$ dzielimy z resztą w $\Z[X]$ i dostajemy:
$$w_m(x)=v_m(x)\cdot L(x)+R(x)$$
ale w $\C[X]\supseteq\Z[X]$ było
$$w_m(x)=v_m(x)\cdot F_m(x),$$
czyli 
\begin{align*}
    F_m(x)v_m(x)&=v_m(x)L(x)+R(x)\\
    R(x)&=v_m(x)[F_m(x)-L(x)]
\end{align*}
ale $deg(R)<deg(v)$, czyli $R=0$ i $F_m=L\in\Z[X]$.

\begin{uwaga}[$F_m$ nierozkładalny w $\Q$]
    $F_m(x)$ jest wielomianem nierozkładalnym w $\Q[X]$.
\end{uwaga}

\textbf{Dowód:} 

Po pierwsze zauważmy, że $F_m$ jest nierozkładalny w $\Q[X]$ $\iff$ nierozkładalny w $\Z[X]$. 

Załóżmy nie wprost, że
$$F_m(x)=G_1(x)\cdot G_2(x)$$
dla $G_1,G_2\in\Z[X]$. Możemy założyć, że $G_1(x)$ jest dalej nierozkładalny w $\Z[X]$ oraz $0<deg(G_1)<deg(F_m)=\phi(m)$

\acc{Lemat:} Istnieje $b$-pierwiastek pierwotny stopnia $n$ oraz liczba pierwsza $p$ taka, że $p\nmid m$ i $G_1(b)=G_2(b^p)=0$.

\textbf{Dowód lematu:}

Z tego, że $0<deg(G_1)$ i $G_1|F_m$, $0<deg(G_2)$ i $G_2|F_m$ mamy, że istnieje pierwiastek pierwotny $b$ stopnia $m$ taki, że $G_1(b)=0$ oraz pierwiastek pierwotny $b'$ stopnia $m$ taki, że $G_2(b')=0$. Zatem istnieje $k\in\N$, $NWD(k, m)=1$ takie, że $b'=b^k$, bo grupa $\mu_m(\C)$ jest cykliczna i $b$ jest jej generatorem.

Niech $k=p_1\cdot...p_s$ będzie rozkładem na liczby pierwsze. Wtedy mamy ciąg różnych liczb
$$b, b^{p_1},b^{p_1p_2},...,b^{p_1,...,p_s}=b^k$$
które są pierwiastkami pierwotnymi stopnia $m$. Z tego wynika, że każda z tych liczb jest pierwiastkiem $G_1$ lub $G_2$, czyli istnieje taka pozycja $i$, że
$$G_1(b^{p_1...p_i})=0,$$
$$G_2(b^{p_1...p_{i+1}})=0$$
wtedy $b:=b^{p_1...p_i}$ oraz $p=p_{i+1}$ i lemat jest spełniony.
\medskip

Wimy już, że $G_1(b)=0$ i $G_1\in\Z[X]$ jest wielomianem nierozkładalnym. Niech $p$ będzie liczbą pierwszą z lematu. Rozważmy
$$G_3(x)=G_2(x^p).$$
Wtedy $G_2(b^p)=G_3(b)=0$, ale stąd wynika, że $G_1(x)$ dzieli $G_3(x)$. Niech więc 
$$G_3(x)=G_1(x)H(x)\in\Z[X].$$

Rozważmy homomorfizm
$$f:\Z\to\Z_p$$
i indukowany przez niego
$$\overline f:\Z[X]\to\Z_p[X].$$
Z założenia $F_m=G_1G_2$ mamy, że
$$\overline f(F_m)=\overline f(G_1)\overline f(G_2)$$
a z rozumowania powyżej ($G_3=G_1H$)
$$\overline f(G_3)=\overline f(G_1)\overline f(H)$$
ale
$$\overline f(G_3(x))=\overline f(G_2(x^p))=\overline f(G_2(x))^p,$$
bo współczynniki $f(G_2(x^p))$ są w $\Z_p$, a $(\sum c_ix^i)^p=\sum c_ix^{pi}$, bo $c_i^{kp}=c_i^k$ dla $c_i\in\Z_p$.

Stąd wiemy, że
$$f(G_2(x))^p=\overline f(G_1)\overline f(H).$$
Pierścień $\Z_p[X]$ jest UFD, więc $\overline f(G_1)$ i $\overline f(G_2)$ mają wspólny dzielnik w $\Z_p[X]$, stopnia co najmniej $1$. Zatem z
$$\overline f(F_m)=\overline f(G_1)\overline f(G_2)$$
ma co najmniej pierwiastek wielokrotny. Jest to sprzeczność, bo nie mogą istnieć pierwiastki wielokrotne: $\overline f(F_m)|x^m-1=w_m$, a $x^m-1$ ma pierwiastki jednokrotne w $\Q$.

\begin{wniosek}[pierwiastek pierwotny a $dim_\Q(\Q(b))$]
    Jeżeli $b\in\C$ jest pierwiastkiem pierwotnym z $1$ stopnia $m$, to $[\Q(b):\Q]=\phi(m)$.
\end{wniosek}

\textbf{Dowód:} $F_m(x)$ jest wielomianem minimalnym dla $b$ nad $\Q$. Mamy, że $[\Q(b):\Q]=deg F_m=\phi(m)$.

\begin{lemat}[twierdzenie Liouville'a o aproksymacji diofantycznej] \dyg{[twierdzenie Liouville'a o aproksymacji diofantycznej]}: Jeżeli $a\in\R$ jest liczbą algebraiczną stopnia $N>1$, to istnieje $c\in\R_+$ takie, że dla każdego $r=\frac pq\in\Q$ zachodzi
    $$\left|a-\frac pq\right|\geq{c\over q^N}$$
\end{lemat}

\begin{definicja}[algebraiczne domknięcie]
    Ciało $L\supseteq K$ jest \deff{algebraicznym domknięciem} $K$ wtedy i tylko wtedy, gdy:
    \begin{enumerate}
        \item $L$ jest algebraicznie domknięte
        \item $L\supseteq K$ jest rozszerzeniem algebraicznym, to znaczy dla każdego $a\in L$ $a$ jest pierwiastkiem algebraicznym nad $K$
    \end{enumerate}
    Takie $L$ oznaczamy przez $\color{blue}\hat{K}$.
\end{definicja}

\begin{wniosek}[istnieje algebraiczne domknięcie]
    Dla każdego $K$ istnieje algebraiczne domknięcie $\hat{K}$.
\end{wniosek}

\textbf{Dowód:} Rozważmy $K_\infty\supseteq K$ - ciało algebraicznie domknięte (twierdzenie z początku wykładu). Pokażemy, że
$$\hat{K}=K^{alg}(K_\infty)=\{a\in K_\infty\;:\;a\text{ algebraiczny nad }K\}$$

1. $\hat{K}$ jest algebraicznie domknięte:

Jeżeli $f\in\hat{K}[X]$, to $f$ ma pierwiastek w $K$, ale $\hat{K}\subseteq K_\infty$, to znaczy, że $a\in\hat{K}$ jest algebraiczne nad $K$.

2. $K\subseteq\hat{K}$ jest rozszerzeniem algebraicznym:

$K\subseteq\widehat{K}=K^{alg}(K_\infty)$ z definicji jest rozszerzeniem algebraicznym.

\begin{tw}[jedyność domknięcia algebraicznego]\label{tw:4:15}
    $\hat{K}$ jest jedyne z dokładnością do izomorfizmu nad $K$.
\end{tw}

\begin{center}
    \begin{tikzcd}
        L_1\arrow[rr, "(\exists!\;f)\;f\obciete K=id_K" above, "\cong" below] & & L_2\\
        & K\arrow[ul, "\supseteq" lablb]\arrow[ur, "\subseteq" labl] &
    \end{tikzcd}
\end{center}

\textbf{Dowód:}

Niech 
$$\mathfrak{K}=\{(k',f')\;:\;K\subseteq K'\subseteq L_1,f':K'\xrightarrow[]{1-1}L_2,\;f'\obciete K=id_k\}$$
\begin{illustration}
    \draw (0, 0) ellipse(1.2 and 2);
    \node at (0, -2.5) {$L_1$};
    \draw (5, 0) ellipse (1.2 and 2);
    \draw (0, -1) ellipse (0.8 and 1);
    \node at (-0.3, -1) {$K$};
    \draw (0, -0.55) ellipse (1 and 1.4);
    \node at (-0.3, 0.3) {$K'$}; 
    \draw (5, -1) ellipse (0.8 and 1);
    \draw (5, -0.55) ellipse (1 and 1.4); 
    \node at (5.2, 0.3) {$f'[K']$}; 
    \node at (5.3, -1) {$K$};
    \node at (5, -2.5) {$L_2$};
    \draw[->] (0.5, -1)..controls (1, 0) and (4, 0)..(4.5, -1) node [midway, above] {$id_K$};
    \draw[->] (0.5, 0.2)..controls (1, 1.2) and (4, 1.2)..(4.5, 0.2) node [midway, above] {$f'$};
\end{illustration}

W $\mathfrak{K}$ definiujemy relację porządku w naturalny sposób, to znaczy
$$(K', f')\leq(K'', f'')\iff K'\subseteq K''\;\land\;f''\obciete K'=f''.$$
Wtedy $(\mathfrak{K},\leq)$ jest zbiorem częściowo uporządkowanym i niepustym (bo jest $(K,id_K)\in\mathfrak{K}$). Ponadto każdy wstępujący łańcuch $(\mathfrak{K},\leq)$ ma ograniczenie górne. Na mocy lematu Kuratowskiego-Zorna w tej rodzinie istnieje element maksymalny, nazwijmy go $(K_1,f_1)$. Pokażemy, że $K_1=L_1$.

Załóżmy nie wprost, że istnieje $a\in L_1\setminus K_1$. Niech $w(x)\in K_1[X]$ będzie wielomianem minimalnym elementu $a$ nad $K_1$. Niech
$$K_2=f_1[K_1]$$
$$v(x)=f_1(a_0)+f_1(a_1)x+...+f_1(a_n)x^n\in K_2[X].$$
$v(x)$ też jest nierozkładalny nad $K_2$, bo $w(x)$ był nierozkładalny nad $K_1$. Niech $b\in L_2$ będzie pierwiastkiem wielomianu $v$.

Zauważmy, że $K_1(a)=K_1[a]$, bo $w(x)$ jest nierozkładalny nad $K_1$, ale
$$K_1[a]\simeq K_1[X]/(w)\simeq K_2[X]/(v)\simeq K_2[b]\simeq K_2(b).$$
Czyli $K_1(a)\simeq K_2(b)$ i $f_2:K_1(a)\izo{}K_2(b)$ jest izomorfizmem rozszerzającym $f_1$. Wtedy mamy $(K_1,f_1)\lneq(K_1(a),f_2)$, co daje sprzeczność z maksymalnością $(K_1,f_1)$. Zatem $L_1=K_2$.

Niech $K_2=f[K_1]=f[L_1]$. Pokażemy nie wprost, że $K_2=L_2$. Załóżmy, że istnieje $a\in L_2\setminus K_2$. Niech $w(x)\in K_2[X]$ wielomian minimalny dla $a$ nad $K_2$. Wtedy $w(x)$ nie ma pierwiastka w $K_2$, ale $K_2=f_1[L_1]$ jest algebraicznie domknięte, bo $L_1$ jest algebraicznie domknięte, co daje sprzeczność.

\begin{wniosek}[$K\cong L\implies\hat{K}\cong\hat{L}$]
    Jeśli $K\cong L$, to $\hat{K}\cong \hat{L}$. Dokładniej, jeżeli $f_0LK\to L$ jest izomorfizmem ciał, to istnieje izomorfizm $f:\hat{K}\to\hat{L}$ taki, że $f\obciete K=f_0$.
\end{wniosek}

\begin{wniosek}[algebraiczne rozszerzenie $1-1$ $\to$ $\hat{K}$]
    Jeśli $K\subseteq L$ jest algebraicznym rozszerzeniem ciał, to istnieje monomorfizm $f:L\to \hat{K}$ taki, że $f\obciete K=id_K$.
\end{wniosek}

\textbf{Dowód:} Mamy dane $K\subseteq L\subseteq\hat{L}$ rozszerzenia algebraiczne, zatem rozszerzenie $K\subseteq\hat{L}$ jest algebraiczne. Stąd $\hat{L}$ jest algebraicznym domknięciem $K$. Z twierdzenia \ref{tw:4:15} istnieje izomorfizm $g:\hat{L}\to\hat{K}$ taki, że $g\obciete K=id_K$. Wtedy $f=g\obciete L$ jest szukanym monomorfizmem.