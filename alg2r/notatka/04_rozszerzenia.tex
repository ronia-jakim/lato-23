\section{Rozszerzenia ciał}

\begin{definicja}
Niech $K\subseteq L$ będą ciałami i $a\in L\setminus K$.

\indent \point Jeżeli \acc{$a$ jest algebraiczny nad $K$}, to istnieje $f\in K[X]$ stopnia $>0$ i $f(a)=0$

\indent \point $a$ jest \acc{przestępny nad $K$} [transcendental] $\iff$ $a$ nie jest algebraiczny.

\indent \point \deff{Rozszerzenie} $L\supseteq K$ jest \deff{algebraiczne} $\iff$ dla każdego $a\in L$ $a$ jest algebraiczny nad $K$.

\indent \point \deff{Rozszerzenie jest przestępne} $\iff$ nie jest algebraiczne.

\indent \point Niech $a\in \C$. Wtedy $a$ jest algebraiczna, gdy $a$ jest algebraiczna nad $\Q$.

\end{definicja}

\textbf{Przykłady}:

\indent 1. W $\C$ na $i$ jest pierwiastkiem algebraicznym wielomianu $x^2+1$, a $\sqrt[n]{d}$ jest pierwiastkiem $x^n-d$. 

\indent 2. Ciało $F(p^n)$ ma charakterystykę $p$ i $F(p)\subseteq F(p^n)$ jest rozszerzeniem ciał, które jest algebraiczne. Dla dowolnego $a\in F(p^n)$ to jest ono pierwiastkiem wielomianu $X^{p^n}-X$, czyli $a$ jest algebraiczne nad $F(p)$.

\indent 3. Pierwiastki przestępne to na przykład $e,\pi,E^\pi$, aczkolwiek nie jesteśmy pewni tego ostatniego [doczytać w S. Lang, Algebra].

\indent 4. Rozważamy $K\subseteq L=K(X)$, czyli pierścień ułamków. Weźmy $x\in K(X)$ - przestępny nad $K$. Załóżmy, że istnieje wielomian $f\in K[X]$ rózny od $0$. I załóżmy, że $0=\hat{f}(X)$ to funkcja wielomianowa. 
$$0=\hat{f}(X)=f\neq0$$
i jest to sprzeczność.


\begin{uwaga}
    Niech $a$ jak wyżej. Wtedy $a$ jest algebraiczny nad $K$ $\iff$ $I(a/K)\neq\{0\}$ jako ideał $K[X]$.
\end{uwaga}

\subsection{Wymiar przestrzeni liniowej}

Niech $K\subseteq L$ będzie rozszerzeniem ciała $K$. Wtedy $L$ jest \deff{przestrzenią liniową nad $K$}. Definiujemy stopień rozszerzenia [coś innego jak indeks przy grupach]
$$[L:K]:=\dim_K(L)$$
jako \acc{wymiar przestrzeni liniowej} nad $K$.

\begin{uwaga}
    Niech $a\in L\setminus K$. Następujące warunki są równoważne:

\indent 1. $a$ jest algebraiczny nad $K$

\indent 2. $K[a]=K(a)$, to znaczy $K[a]$ jest ciałem (usuwanie niewymierności z mianownika)

\indent 3. $[K(a):K]=\dim_K(a)<\infty$
\end{uwaga}

\textbf{Dowód:}

$1\implies2$

Wiemy, że $K[X]$ jest euklidesowy (bo $K$ to ciało), więc $K[X]$ jest też PID.

Skoro $a$ jest algebraiczny nad $K$, to istnieje $f\in K[X]$ takie, że $f(a)=0$, a więc
$$0\neq I(\overline a/K)\normalsubgroup K[X]$$
czyli $I(a/K)$ jest maksymalnym ideałem głównym. Teraz, jeśli $I\normalsubgroup R$ jest ideałem maksymalnym pierścienia $R$, to $R/I$ jest ciałem. Czyli
$$K[a]\cong K[X]/I(a/K)$$
jest ciałem.

$2\implies 3$

Załóżmy, że $a\neq 0$. Wtedy $a^{-1}\in K[a]$, czyli istnieje wielomian $f\in K[X]$ taki, że 
$$f(x)=\sum\limits_{i=1}^n b_ix^i,\quad b_n\neq 0$$
i $a^{-1}=f(a)$. Wobec tego mamy
$$1=f(a)\cdot a$$
$$0=f(a)a-1=b_na^{n+1}+b_aa^2+...+b_0a-1,$$
stąd mamy, że
$$a^{n+1}=-\frac1{b_n}(b_{n-1}a^n+...+b_0a-1)\in Lin_K(1, a, ..., a^n)$$
jest w domknięciu liniowym $(1, a,..., a^n)$. Indukcyjnie można pokazać, że
$$(\forall\;m\geq0)\;a^m\in Lin_K(1,a,...,a^n),$$
czyli 
$$K[a]=K(a)=Lin_K(1,a,...,a^n),$$ 
co daje, że $[K(a):K]\leq n<\infty$.

$3\implies 1$

$[K(a):K]<\infty$, z czego wynika, że
$$\{1,a,...,a^n,...,\}=\{a^t\;:\;t\in\N\}\subseteq K(a)$$
jest zbiorem liniowo zależnym. Z liniowej zależności wiemy, że
$$(\exists\;n\in\N)(\exists\;b_{n-1},...,b_0)\;a^{n}=b_{n-1}a^{n-1}+...+b_1a+b_0.$$
Stąd dla $f\in K[X]$ zadanego wzorem
$$f(x)=x^n+b_{n-1}x^{n-1}+...+b_0$$
mamy $f(a)=0$, zatem $a$ jest algebraiczny nad $K$.
\medskip

Niech $a\in L\supseteq K$ będzie algebraicznym pierwiastkiem nad $K$, $I(a/K)=\{w\in K[X]\;:\;w(a)=0\}=(f)$, $f\neq 0$, $f\in K[X]$, $f$ unormowany (\dyg{czyli współczynnik przy wyrazie wiodącym jest $1$?})

\indent \point $f$ jest nazywany wielomianem \deff{minimalnym} $a$ nad $K$ (wyznaczony jednoznacznie)

\indent \point \acc{stopień $a$} nad $K$ jest definiowany jako $deg(f)$.
\medskip

\textbf{Przykład:}

\indent 1. $\sqrt{2}\in\R\supseteq\Q$, wtedy $f(x)=x^2-2$ jest wielomianem minimalnym $\sqrt2$ nad $\Q$ i stopień $\sqrt{2}$ nad $\Q$ jest równy $2$.

\indent 2. $\pi\in\R$ nie ma stopnia, bo $\pi$ nie jest liczbą algebraiczną nad $\Q$

\indent 3. $\sqrt[7]{7+\sqrt[3]{3}}-\sqrt[6]{6}\in\R$, czy jest to algebraiczne nad $\Q$? Tak i ma stopień $126$.

\begin{uwaga}
    Załóżmy, że $I(a/K)=(f)$ i $f$ jest unormowany. Wówczas:

    \indent 1. $f$ jest unormowanym wielomianem minimalnego stopnia takim, że $f(a)=0$

    \indent 2. $deg(f)=[K(a):K]$, czyli stopień tego wielomianu jest równy stopniu przestrzeni liniowej $K(a)$ nad $K$.
\end{uwaga}

\textbf{Dowód:}

Niech $n=deg(f)$, 
$$f(x)=x^n+\sum\limits_{k<n}b_kx^k$$
Z tego, że $f(a)=0$ mamy, że 
$$a^n=-\sum\limits_{k<n}b_kx^k\in Lin_K(1,a,...,a^{n-1})\subseteq L.$$
Czyli $K(a)=Lin_K(1,a,...,a^{n-1})$ i wystarczy zobaczyć, że $\{1,..., a^{n-1}\}$ jest liniowo niezależny nad $K$, to znaczy jest bazą $K(a)$ nad $K$. Jest, bo $f$ jest minimalnego stopnia.

\begin{fakt}
    Niech $K\subseteq L\subseteq M$ będą rozszerzeniami ciał. Wtedy 
    $$[M:K]=[M:L]\cdot [L:K]$$
\end{fakt}

\textbf{Dowód:}

Niech $\{e_i\;:\;i\in I\}$ będzie bazą $L$ nad $K$, a $\{f_j\;:\;j\in J\}$ będzie bazą $M$ nad $L$. Stąd $|I|=[L:K]$ i $|J|=[M:L]$.

Chcemy za pomocą tych dwóch zbiorków zrobić bazę $M$ nad $K$. Rozważmy zbiór
$$X=\{e_i\cdot f_j\;:\;i\in I,j\in J\}.$$
Musimy pokazać, że 

1. $|X|=|I|\cdot|J|$

2. $X$ jest liniowo niezależny

3. $X$ jest bazą $M$ nad $K$

Te dwa ostatnie mówią, że $X$ jest bazą.

1. Załóżmy, nie wprost, że dla $i\neq i'$ i $j\neq j'$ i $e_if_j=e_{i'}f_{j'}$. Czyli
$$e_if_j-e_{i'}f_{j'}=0,$$
czyli $f_j,f_{j'}$ są liniowo zależne nad $L$, czyli mamy, że $f_j=f_{j'}$ i
$$0=e_if_j-e_{i'}f_{j}=(e_i-e_{i'})f_j\implies e_i-e_{i'}=0\implies i=i'$$

2. Załóżmy nie wprost, że $X$ nie jest lnz, czyli istnieją $k_{ij}\in K$ takie, że
$$\sum\limits_{j\in J}\sum\limits_{i\in I}k_{ij}e_if_j=0,$$
ale $\sum\limits_ik_ije_i=l_j$ są elementami $L$, czyli
$$\sum\limits_{j\in J}l_jf_j=0$$
więc $f_j$ są liniowo zależne, a przecież były bazowe, w takim razie
$$0=l_j=\sum\limits_{i\in I}k_{ij}e_i,$$
$e_i\neq0$, czyli $k_{ij}=0$ i koniec.

3. $X$ generuje $M$ nad $K$, bo dla $m\in M$ mam
$$m=\sum l_jf_j=\sum\left(\sum a_{ij}e_i\right)f_j=\sum\sum a_{ij}e_if_j=\sum \sum k_{ij}e_if_j$$