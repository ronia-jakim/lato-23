\section{Rozszerzenia ciał}

\begin{definicja}
Niech $K\subseteq L$ będą ciałami i $a\in L\setminus K$.

\indent \point Jeżeli \acc{$a$ jest algebraiczny nad $K$}, to istnieje $f\in K[X]$ stopnia $>0$ i $f(a)=0$

\indent \point $a$ jest \acc{przestępny nad $K$} [transcendental] $\iff$ $a$ nie jest algebraiczny.

\indent \point \deff{Rozszerzenie} $L\supseteq K$ jest \deff{algebraiczne} $\iff$ dla każdego $a\in L$ $a$ jest algebraiczny nad $K$.

\indent \point \deff{Rozszerzenie jest przestępne} $\iff$ nie jest algebraiczne.

\indent \point Niech $a\in \C$. Wtedy $a$ jest algebraiczna, gdy $a$ jest algebraiczna nad $\Q$.

\end{definicja}

\textbf{Przykłady}:

\indent 1. W $\C$ na $i$ jest pierwiastkiem algebraicznym wielomianu $x^2+1$, a $\sqrt[n]{d}$ jest pierwiastkiem $x^n-d$. 

\indent 2. Ciało $F(p^n)$ ma charakterystykę $p$ i $F(p)\subseteq F(p^n)$ jest rozszerzeniem ciał, które jest algebraiczne. Dla dowolnego $a\in F(p^n)$ to jest ono pierwiastkiem wielomianu $X^{p^n}-X$, czyli $a$ jest algebraiczne nad $F(p)$.

\indent 3. Pierwiastki przestępne to na przykład $e,\pi,E^\pi$, aczkolwiek nie jesteśmy pewni tego ostatniego [doczytać w S. Lang, Algebra].

\indent 4. Rozważamy $K\subseteq L=K(X)$, czyli pierścień ułamków. Weźmy $x\in K(X)$ - przestępny nad $K$. Załóżmy, że istnieje wielomian $f\in K[X]$ rózny od $0$. I załóżmy, że $0=\hat{f}(X)$ to funkcja wielomianowa. 
$$0=\hat{f}(X)=f\neq0$$
i jest to sprzeczność.


\begin{uwaga}
    Niech $a$ jak wyżej. Wtedy $a$ jest algebraiczny nad $K$ $\iff$ $I(a/K)\neq\{0\}$ jako ideał $K[X]$.
\end{uwaga}

Niech $K\subseteq L$ będzie rozszerzeniem ciała $K$. Wtedy $L$ jest \deff{przestrzenią liniową nad $K$}. Definiujemy stopień rozszerzenia [coś innego jak indeks przy grupach]
$$[L:K]:=\dim_K(L)$$
jako \acc{wymiar przestrzeni liniowej} nad $K$.

\begin{uwaga}
    Niech $a\in L\setminus K$. Następujące warunki są równoważne:

\indent 1. $a$ jest algebraiczny nad $K$

\indent 2. $K[a]=K(a)$, to znaczy $K[a]$ jest ciałem (usuwanie niewymierności z mianownika)

\indent 3. $[K(a):K]=\dim_K(a)<\infty$
\end{uwaga}

\textbf{Dowód:}

$1\implies2$
