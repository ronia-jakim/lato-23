\section{Rozszerzenia ciał}

\begin{definition}
Niech $K\subseteq L$ będą ciałami i $a\in L\setminus K$.
\begin{itemize}
\item Jeżeli \acc{$a$ jest algebraiczny nad $K$}, to istnieje $f\in K[X]$ stopnia $>0$ i $f(a)=0$
\item $a$ jest \acc{przestępny nad $K$} [transcendental] $\iff$ $a$ nie jest algebraiczny.
\item \important{Rozszerzenie} $L\supseteq K$ jest \important{algebraiczne} $\iff$ dla każdego $a\in L$ $a$ jest algebraiczny nad $K$.
\item \important{Rozszerzenie jest przestępne} $\iff$ nie jest algebraiczne.
\item Niech $a\in \C$. Wtedy $a$ jest algebraiczna, gdy $a$ jest algebraiczna nad $\Q$.
\end{itemize}
\end{definition}

\textbf{Przykłady}:

\indent 1. W $\C$ na $i$ jest pierwiastkiem algebraicznym wielomianu $x^2+1$, a $\sqrt[n]{d}$ jest pierwiastkiem $x^n-d$. 

\indent 2. Ciało $F(p^n)$ ma charakterystykę $p$ i $F(p)\subseteq F(p^n)$ jest rozszerzeniem ciał, które jest algebraiczne. Dla dowolnego $a\in F(p^n)$ to jest ono pierwiastkiem wielomianu $X^{p^n}-X$, czyli $a$ jest algebraiczne nad $F(p)$.

\indent 3. Pierwiastki przestępne to na przykład $e,\pi,E^\pi$, aczkolwiek nie jesteśmy pewni tego ostatniego [doczytać w S. Lang, Algebra].

\indent 4. Rozważamy $K\subseteq L=K(X)$, czyli pierścień ułamków. Weźmy $x\in K(X)$ - przestępny nad $K$. Załóżmy, że istnieje wielomian $f\in K[X]$ rózny od $0$. I załóżmy, że $0=\hat{f}(X)$ to funkcja wielomianowa. 
$$0=\hat{f}(X)=f\neq0$$
i jest to sprzeczność.


\begin{remark}
    Niech $a$ jak wyżej. Wtedy $a$ jest algebraiczny nad $K$ $\iff$ $I(a/K)\neq\{0\}$ jako ideał $K[X]$.
\end{remark}

\subsection{Wymiar przestrzeni liniowej}

Niech $K\subseteq L$ będzie rozszerzeniem ciała $K$. Wtedy $L$ jest \important{przestrzenią liniową nad $K$}. Definiujemy stopień rozszerzenia [coś innego jak indeks przy grupach]
$$[L:K]:=\dim_K(L)$$
jako \acc{wymiar przestrzeni liniowej} nad $K$.

\begin{remark}\label{uwaga:4:3}
    Niech $a\in L\setminus K$. Następujące warunki są równoważne:

\indent 1. $a$ jest algebraiczny nad $K$

\indent 2. $K[a]=K(a)$, to znaczy $K[a]$ jest ciałem (usuwanie niewymierności z mianownika)

\indent 3. $[K(a):K]=\dim_K(a)<\infty$
\end{remark}

\begin{proof}
$1\implies2$

Wystarczy pokazać, że $K[a]$ jest ciałem. Rozważamy $I(a/K)\triangleleft K[X]$. Wiemy, że $K[X]$ jest PID, więc potrzebujemy, aby $I(a/K)$ było ideałem pierwszym.
$$f\cdot g\in I(a/K)\iff0=\hat{f\cdot g}(a)$$
gdzie daszek oznacza homomorfizm ewaluacji, który jest również homomorfizmem w punkcie. Czyli
$$\hat{f\cdot g}(a)=\hat{f}(a)\hat{g}(a)=0\iff\hat{f}(a)=0\;\lor\;\hat{g}(a)=0.$$
Czyli $I(a/K)$ jest ideałem pierwszym w pierścieniu PID, więc jest ideałem maksymalnym. Mamy więc, że
$$K[a]/I(a/K)$$
jest ciałem, więc jest izomorficzne z $K(a)$, bo $K[a]$ to najmniejszy pierścień generowany przez $K\cup\{a\}$ (tutaj pierścień), a $K(a)$ to najmniejsze ciało generowane przez $K\cup\{a\}$.

$2\implies 3$

Załóżmy, że $a\neq 0$. Wtedy $a^{-1}\in K[a]$, czyli istnieje wielomian $f\in K[X]$
$$f(x)=\sum\limits_{i=1}^n b_ix^i,\quad b_n\neq 0$$
taki, że $a^{-1}=f(a)$. Wobec tego mamy
$$1=f(a)\cdot a$$
$$0=f(a)a-1=b_na^{n+1}+b_aa^2+...+b_0a-1,$$
stąd mamy, że
$$a^{n+1}=-\frac1{b_n}(b_{n-1}a^n+...+b_0a-1)\in Lin_K(1, a, ..., a^n)$$
jest w domknięciu liniowym $(1, a,..., a^n)$. Indukcyjnie pokazujemy, że
$$(\forall\;m\geq0)\;a^m\in Lin_K(1,a,...,a^n).$$

1. $m=0,...,n+1$ bo one są już w $Lin_k(1,a,...,a^n)$.

2. Zakładamy teraz, że dla $m$ mamy 
$$a^m=\sum\limits_{i=0}^nc_ia^i$$ 
i pokazujemy dla $m+1$.
\begin{align*}
    a^{m+1}&=a\cdot a^m=a\sum\limits_{i=0}^nc_ia^{i}=\sum\limits_{i=0}^nc_ia^{i+1}\in Lim_K(1,a,...,a^n),
\end{align*}
bo $a^{n+1}\in Lim_K(1,a,...,a^m)$.

Czyli 
$$K[a]=K(a)=Lin_K(1,a,...,a^n),$$ 
co daje, że $[K(a):K]\leq n<\infty$.

$3\implies 1$

$[K(a):K]<\infty$, z czego wynika, że
$$\{1,a,...,a^n,...,\}=\{a^t\;:\;t\in\N\}\subseteq K(a)$$
jest zbiorem liniowo zależnym. Z liniowej zależności wiemy, że
$$(\exists\;n\in\N)(\exists\;b_{n-1},...,b_0)\;a^{n}=b_{n-1}a^{n-1}+...+b_1a+b_0.$$
Stąd dla $f\in K[X]$ zadanego wzorem
$$f(x)=b_{n-1}x^{n-1}+...+b_0-x^n$$
mamy $f(a)=0$, zatem $a$ jest algebraiczny nad $K$.
\end{proof}

\begin{definition}[wielomian minimalny, stopień pierwiastka]
Niech $a\in L\supseteq K$ będzie algebraicznym pierwiastkiem nad $K$, $I(a/K)=\{w\in K[X]\;:\;w(a)=0\}=(f)$, $f\neq 0$, $f\in K[X]$, $f$ unormowany (ang. monic)
\begin{itemize}
    \item $f$ jest nazywany wielomianem \important{minimalnym} $a$ nad $K$ (wyznaczony jednoznacznie)
    \item \acc{stopień $a$} nad $K$ jest definiowany jako $deg(f)$.
\end{itemize}
\end{definition}


\begin{remark}[$I(a/K)=(f)\implies deg(f)=\begin{bmatrix}K(a):K\end{bmatrix}$]
    Załóżmy, że $I(a/K)=(f)$ i $f$ jest unormowany. Wówczas:
    \begin{enumerate}
    \item $f$ jest unormowanym wielomianem minimalnego stopnia takim, że $f(a)=0$

    \item $deg(f)=[K(a):K]$, czyli stopień tego wielomianu jest równy stopniu przestrzeni liniowej $K(a)$ nad $K$.
    \end{enumerate}
\end{remark}

\begin{proof}{\color{pagColor}dupa}

\begin{enumerate}
\item Oczywiste. $f$ jest wielomianem nieredukowalnym, stąd jeśli istniałby $g$ taki, że $g(a)=0$ oraz $deg(g)<deg(f)$, to wtedy $f$ byłby podzielny przez $g$, co daje sprzeczność z nieredukowalnością $f$.

  %{\large\color{orange}DOWODZIK, ZE IRREDUCIBLE JEST MINIMAL}

\item Niech $n=deg(f)$, 
$$f(x)=x^n+\sum\limits_{k<n}b_kx^k$$
Z tego, że $f(a)=0$ mamy, że 
$$a^n=-\sum\limits_{k<n}b_kx^k\in Lin_K(1,a,...,a^{n-1})\subseteq L.$$
Czyli $K(a)=Lin_K(1,a,...,a^{n-1})$ i wystarczy zobaczyć, że $\{1,..., a^{n-1}\}$ jest liniowo niezależny. W przeciwnym przypadku dla pewnego $0<r<m$ $a^r\in Lim_K(1,a,...,a^{t-1})$, czyli istnieje wielomian taki, że $a$ jest jego pierwiastkiem, a stopień jest nie większy niż $r<n$ i to daje sprzeczność.

Czyli $Lim_K(1,a,...,a^n)$ jest bazą $K(a)$ nad $K$ i koniec.
\end{enumerate}
\end{proof}
\textbf{Przykład:}
\begin{enumerate}
    \item $\sqrt{2}\in\R\supseteq\Q$, wtedy $f(x)=x^2-2$ jest wielomianem minimalnym $\sqrt2$ nad $\Q$ i stopień $\sqrt{2}$ nad $\Q$ jest równy $2$.
    \item $\pi\in\R$ nie ma stopnia, bo $\pi$ nie jest liczbą algebraiczną nad $\Q$
    \item  $\sqrt[7]{7+\sqrt[3]{3}}-\sqrt[6]{6}\in\R$, czy jest to algebraiczne nad $\Q$? Tak i ma stopień $126$.
\end{enumerate}

Jeśli $K\subseteq L\ni a$ jest algebraiczny, to $deg(a/K)=n$, to 
$$K(a)=K[a]=\{\sum\limits_{i=0}^{n-1}b_ia^i\;:\;b_i\in K\}$$

\begin{fact}[$dim_K(M)=dim_L(M)\cdot dim_K(L)$]\label{fakt:4:6}
  
    Niech $K\subseteq L\subseteq M$ będą rozszerzeniami ciał. Wtedy 
    $$[M:K]=[M:L]\cdot [L:K]$$
\end{fact}

\begin{proof}

Niech $\{e_i\;:\;i\in I\}$ będzie bazą $L$ nad $K$, a $\{f_j\;:\;j\in J\}$ będzie bazą $M$ nad $L$. Stąd $|I|=[L:K]$ i $|J|=[M:L]$.

Chcemy za pomocą tych dwóch zbiorków zrobić bazę $M$ nad $K$. Rozważmy zbiór
$$X=\{e_i\cdot f_j\;:\;i\in I,j\in J\}.$$
Musimy pokazać, że 
\begin{enumerate}
    \item $X$ jest liniowo niezależny
    \item $X$ jest bazą $M$ nad $K$
    \item $|X|=|I|\cdot|J|$
\end{enumerate}

Czyli $X$ jest bazą $M$ nad $K$ (1.,2.) i ma odpowiednią moc (3.).

1. Załóżmy nie wprost, że $X$ nie jest lnz, czyli istnieją $k_{ij}\in K$ takie, że
$$\sum\limits_{j\in J}\sum\limits_{i\in I}k_{ij}e_if_j=0,$$
ale $\sum\limits_ik_ije_i=l_j$ są elementami $L$, czyli
$$\sum\limits_{j\in J}l_jf_j=0$$
więc $f_j$ są liniowo zależne, a przecież były bazowe, w takim razie
$$0=l_j=\sum\limits_{i\in I}k_{ij}e_i,$$
$e_i\neq0$, czyli $k_{ij}=0$ i koniec.

2. $X$ generuje $M$ nad $K$, bo dla $m\in M$ mam
$$m=\sum l_jf_j=\sum\left(\sum a_{ij}e_i\right)f_j=\sum\sum a_{ij}e_if_j=\sum \sum k_{ij}e_if_j$$

3. Załóżmy, nie wprost, że dla $i\neq i'$ i $j\neq j'$ i $e_if_j=e_{i'}f_{j'}$. Czyli
$$e_if_j-e_{i'}f_{j'}=0,$$
czyli $f_j,f_{j'}$ są liniowo zależne nad $L$, czyli mamy, że $f_j=f_{j'}$ i
$$0=e_if_j-e_{i'}f_{j}=(e_i-e_{i'})f_j\implies e_i-e_{i'}=0\implies i=i'$$
Z tego wynika, że $[M:K]=|X|=|I||J|=[L:K][M:L]$.
\end{proof}
\begin{conclusion}[$K_{alg}$ - podciałem]
    Niech $K\subseteq L$ będzie rozszerzeniem skończonego ciała. Niech 
    $$K_{alg}(L)=\{a\in L\;:\;a\text{ jest algebraiczny nad }K\}.$$
    Okazuje się, że $K_{alg}$ jest podciałem.
\end{conclusion}

\begin{proof}

Weźmy $a,b\in K_{alg}$. Wiemy, że $[K(a):K]$ i $[K(b):K]$ są skończone. Mamy, że
$$K\subseteq K(a)\subseteq K(a, b)$$
Z faktu \ref{fakt:4:6} wiemy, że 
$$[K(a, b):K]=[K(a,b):K(a)]\cdot[K(a):K]$$
czyli również $K(a,b)$ jest skończone. Zatem dla $x\in K(a,b)$ mamy
$$[K(x):K]\leq[K(a,b):K]$$
też jest skończone, zatem $x$ jest algebraiczny nad $K$.
\medskip

Dla $x\in K(a, b)$ mamy $[K(x):K]\leq[K(a):K]$, czyli również jest skończone. W takim razie, $x$ jest algebraiczny nad $K$ i należy do $K_{alg}$.
\end{proof}

\begin{definition}[(relatywne) algebraiczne domknięcie]{\color{pagColor}spaaaać}
    \begin{enumerate}
        \item $K_{alg}(L)$ nazywamy \important{algebraicznym domknięciem} $K$ w $L$.
        \item $K$ jest \important{relatywnie algebraicznie domknięte} w $L$ $\iff$ $K_{alg}(L)=K$.
    \end{enumerate}
\end{definition}

\textbf{Przykłady:} 
\begin{enumerate}
    \item $\Q_{alg}(\C):=\hat{\Q}=\Q^{alg}$ jest to tak zwane \acc{ciało liczb algebraicznych}. $\hat{\Q}$ jest przeliczalne, bo $\Q[x]$ jest przeliczalne, więc jest mnóstwo liczb \acc{przestępnych} (zespolonych, które nie są algebraiczne, ale nie potrafimy żadnej wskazać).
    \item $K$ jest algebraicznie domknięte w $K(X)$
    \item ${1\over\sqrt[3]{2}+\sqrt{3}}\in\Q[\sqrt{3},\sqrt[3]{2}]$, bo $\Q[\sqrt{3},\sqrt[3]{2}]$ jest ciałem
    
    $$L=\underbrace{\Q[\sqrt[3]{2},\sqrt{2}]}_{\subseteq \C}=\underbrace{\underbrace{\Q[\sqrt[3]{2}]}_{\text{ciało}}[\sqrt3]}_{\sqrt[3]{2}\text alg. w }\Q=\{a+b\sqrt[3]{2}+c\sqrt[3]{2}\;:\;a,b,c\in\Q(\sqrt{3})\}$$
    $$\sqrt[3]{2}+\sqrt3\in L\implies{1\over\sqrt[3]{2}+\sqrt{3}}\in L$$
\end{enumerate}
