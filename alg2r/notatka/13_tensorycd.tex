\section{Własności produktu tensorowego}

\textbf{Przykład:} po pierwsze zauważmy, że $R[X]$ i $R[Y]$ to $R$-moduły. Możemy powiedzieć, że $R[X]\otimes R[Y]\cong R[X, Y]$ w tym sensie, że
$$R[X]\times R[Y]\xrightarrow[]{\otimes} R[X,Y]$$
dane przez $W(X)\otimes V(Y)=W(X)\cdot V(Y)$. Odwzorowanie to spełnia warunek \hyperref[warunek:z:12.3]{{\color{purple}\PHcat}}:

\begin{center}\begin{tikzcd}[column sep=large, row sep = large]
  R[X]\times R[Y]\arrow[rr, "\otimes"]\arrow[dr, "g\text{ 2-liniowe}" below left] & & R[X, Y]\arrow[dl, "\exists!\;h\text{ R-liniowe?}" below right, dashed]\\
  & N
\end{tikzcd}\end{center}
a więc homomorfizm $h$ musi spełniać warunek
$$h(W(X)\cdot V(Y))=g(W(X), V(Y))$$
dla wszystkich $W(X)\in R[X]$ oraz $V(Y)\in R[Y]$. Ten warunek wraz z $R$-liniowością $g$ wyznacza $h$ na całym $R[X, Y]$ w sposób jednoznaczny na diagramie wyżej. Nie trudno też sprawdzić, że $h\circ \otimes=g$, co daje komutowanie diagramu.

Dlaczego jednak $h$ istnieje? Rozpocznijmy od jednomianów $\{X^nY^m\}_{n,m\geq0}$ jako bazy $R[X,Y]$. Mamy
$$h(X^nY^m)=g(X^n,Y^m)$$
jest dobrze zdefiniowane. Dla wielomianu $W(X,Y)\in R[X, Y]$ takiego, że
$$W(X,Y)=\sum_{n,m}r_{n,m}X^nY^m$$
a więc
$$h(W)=h(\sum_{n,m}r_{n,m}X^nY^m)=\sum_{n,m}r_{n,m}h(X^nY^m)=\sum_{n,m}r_{n,m}g(X^n,Y^m).$$
Czyli funkcja $h$ działa na wszystkich wielomianach z $R[X,Y]$.
\medskip

\textbf{\large\color{blue}Wniosek.} W takim razie, jeśli $M_n$ jest wolnym $R$-modułem wymiaru $n$ o bazie $\{b_1,...,b_n\}$ a $M_m$ jest wolnym $R$-modułem wymiaru $m$ o bazie $\{c_1,...,c_m\}$, to $M_n\otimes M_m$ jest $R$-modułem o bazie $\{b_i\otimes c_j\}_{\substack{i\leq n\\j\leq m}}$ i wymiarze $n\cdot m$.

\begin{proof}
  $M_n\cong R_{<n}[X]$, gdzie $R_{<n}[X]$ jest modułem wielomianów stopnia $<n$. Wtedy
  $$M_n\otimes M_m\cong R_{<n}[X]\otimes R_{<m}[Y]\cong R_{<n<m}[X, Y]$$
  gdzie $R_{<n<m}[X,Y]$ to moduł wielomianów $W(X, Y)$ takich, że $deg_X(W)<n$ oraz $deg_Y(W)<m$. Dalszy ciąg dowodu podąża za rozumowaniem z przykładu wyżej.
\end{proof}

\textbf{\large\color{green}Własności iloczynu tensorowego:}

\begin{enumerate}
  \item $M_1\otimes M_2\cong M_2\otimes M_1$
  \item $(M_1\otimes M_2)\otimes M_3\cong M_1\otimes (M_2\otimes M_3)\cong M_1\otimes M_2\otimes M_3$
  \item $R\otimes M\cong M$
\end{enumerate}

\begin{proof}$ $\newline
  \begin{enumerate}
    \item Ćwiczenie
    \item Ćwiczenie
    \item Niech $f:R\times M\to M$ będzie $R$-dwuliniowym odwzorowaniem zadanym przez
      $$f(r,m)=r\cdot m.$$
      Wystarczy teraz pokazać, że $f$ spełnia warunek \hyperref[warunek:z:12.3]{{\color{purple}\PHcat}}:

      \begin{center}\begin{tikzcd}
        R\times M\arrow[rr, "f"]\arrow[dr, "g" below left] & & M \arrow[dl, dashed, "\exists!\;h\text{ R-liniowe}"]\\
                                 & N
      \end{tikzcd}\end{center}

      Funkcja $h$ zdefiniowana przez
      $$h(m)=h(f(1,m))=g(1,m)$$
      jest jedyna ze względu na jedyność $g$. Jest też $R$-homomorfizmem, bo $g$ jest $R$-liniowe na drugiej współrzędnej.

      Diagram komutuje, bo $g$ jest $R$-liniowe na obu współrzędnych, a więc
      $$(h\circ f)(r, m)=h(rm)=g(1, rm)=r\cdot g(1, m)=g(r, m)$$
  \end{enumerate}
\end{proof}

\begin{remark}\label{uwaga:13.1}$ $\newline
  \begin{enumerate}
    \item Jeśli $A\subseteq M$ i $B\subseteq N$ są zbiorami generującymi te moduły, to $A\otimes B=\{a\otimes b\;:\;a\in A,b\in B\}$ jest generatorem $M\otimes N$.
    \item Załóżmy, że $f:M\to M'$ i $g:N\to N'$ są $R$-liniowe. Wtedy istnieje dokładnie jedno $R$-liniowe
      $$h:M\otimes N\to M'\otimes N'$$
      zdefiniowane przez
      $$h(m\otimes n)=f(m)\otimes g(n)$$
  \end{enumerate}
\end{remark}

\begin{proof}$ $\newline

  \begin{center}\begin{tikzcd}[column sep=6em, row sep=6em]
    M\times N\arrow[d, "f\times g" left]\arrow[r, "\otimes"]\arrow[dr, "\otimes\circ(f\times g)"] & M\otimes N\arrow[d, dashed, "\exists!\;h"]\\
    M'\times N'\arrow[r, "\otimes" below] & M'\otimes N'
  \end{tikzcd}\end{center}

  Gdzie $h(m\otimes n)=f(m)\otimes g(n)$.

  Funkcja $\otimes\circ(f\times g)$ jest dwuliniowa, bo
  \begin{align*}
    \otimes\circ(f\times g)(x+y, n)&=\otimes(f(x)+f(y), g(n))=(f(x)+f(y))\otimes g(n)=\\
                                   &=f(x)\otimes g(n)+f(y)\otimes g(n)=\\
                                   &=\otimes\circ(f\times g)(x, n)+\otimes\circ(f\times g)(y, n)
  \end{align*}

  $R$-liniowość $h$, jego jedyność oraz komutowanie diagramu wyżej pokazujemy tak jak w dowodach wcześniejszych uwag.
\end{proof}

\subsection{Iloczyn tensorowych funkcji}

\begin{definition} $f\otimes g=h$ jak z uwagi \ref{uwaga:13.1} nazywamy \important{iloczynem tensorowym} $f$ i $g$
\end{definition}

\begin{remark}$ $\newline
  \begin{enumerate}
    \item $ $\newline \begin{tikzcd}[column sep=2em, row sep=2em]
        M\arrow[rr, "f"]\arrow[dr, "f'\circ f" below left] & &M'\arrow[dl, "f'"] & N\arrow[rr, "g"]\arrow[dr, "g'\circ g" below left] & &N'\arrow[dl, "g'"] & M\otimes N\arrow[rr, "f\otimes g"]\arrow[dr, "X" below left]" & & M'\otimes N'\arrow[dl, "f'\otimes g'"]\\
                                                & M'' & & & N'' & & & M''\otimes N''
      \end{tikzcd}
      [dowód: wystarczy sprawdzić $X=(f'\otimes g')\circ(f\otimes g)=(f'\circ f)\otimes (g'\circ g)$]
    \item $id_M\otimes id_N=id_{M\otimes N}$
    \item $\phi:Hom(M, M')\times Hom(N, N')\to Hom(M\otimes M', N\otimes N')$ zadane przez $\phi(f, g)=f\otimes g$ jest $R$-dwuliniowe.
  \end{enumerate}
\end{remark}

\begin{remark}
  Dla $R$-modułu $M$ oraz $R$-modułów $\{N_i\}_{i\in I}$ istnieje izomorfizm
  $$\psi:M\otimes (\bigoplus_{i\in I} N_i)\isomorphism \bigoplus_{i\in I} (M\otimes N_i)$$
  zadany przez
  $$\psi(m\otimes\sum n_i)=\sum_im\otimes n_i$$
\end{remark}

\begin{proof}
  Rozważmy diagram

  \begin{center}\begin{tikzcd}[column sep=2em, row sep=3em]
    M\times\bigoplus_iN_i\arrow[rr, "\psi"]\arrow[dr, "g" below left] & &\bigoplus_i(M\otimes N_i)\arrow[dl, "\exists!\;h", dashed]\\
                                         & N
  \end{tikzcd}\end{center}
  Musimy pokazać istnienie i jedyność $h$ oraz komutowanie diagramu (ostatnie jest trywialne).

  \begin{description}
    \item[istnienie h] pokażemy najpierw dla dowolnego $i$, a następnie korzystając z własności uniwersalności $\bigoplus_i$ przeniesiemy na całą sumę prostą.

      \begin{center}\begin{tikzcd}
        M\times N_i\arrow[rr, "\otimes"]\arrow[dr, "g\restriction M\times N_i"] & & M\otimes N_i\arrow[dl, "\exists!\;h_i"]\\
                                                                                & N
      \end{tikzcd}\end{center}
      $h_i$ definiujemy wtedy jako
      $$h_i(m\otimes n_i)=g(m, n_i)$$
      a jego istnienie oraz jedyność udowadniamy tak jak wcześniej.

      Ponieważ dla każdego $i$ istnieje dokładnie jedna funkcja $h_i$, to z własności uniwersalności $\bigoplus$ możemy stwierdzić, że dla każdego $i$ istnieje $\exists!\;h:\bigoplus (M\otimes N_i)\to N$ takie, że
      
      \begin{center}\begin{tikzcd}[column sep=1em, row sep=3em]
        \bigoplus_i(M\otimes N_i)\arrow[rr, "h"] & & N\\
                                                 & M\otimes N_i\arrow[ul, sloped, "\supseteq" below left]\arrow[ur, "h_i"]
      \end{tikzcd}\end{center}

      to znaczy
      $$h(m\otimes n_i)=h_i(m\otimes n_i)=g(m, n_i),$$
      czyli
      $$h(\sum_im\otimes n_i)=\sum_ih_i(m\otimes n_i)=\sum_ig(m, n_i)=g(m, \sum n_i).$$
      Dla $N=M\otimes(\bigoplus N_i)$ dostajemy $h$-izomorfizm oraz $h(\sum m\otimes n_i)=m\otimes\sum n_i$.
    \item[jedyność h] sprawdzamy dla elementów $\sum m\otimes n_i$, gdyż generują one $\bigoplus (M\otimes N_i)$:
      $$(h\circ\psi)(m\otimes\sum n_i)=h(\sum_im\otimes n_i)=g(m, \sum_in_i)$$
      i od razu widać, że diagram jak wyżej komutuje.
  \end{description}
\end{proof}

\subsection{Iloczyny zewnętrzne}

\begin{bbox}
Niech $R=K$ będzie ciałem, a $V$ będzie przestrzenią liniową nad $K$. Zdefiniujmy
$$\color{blue}V^{\otimes n}=\underbrace{V\otimes...\otimes V}_{n}.$$

Zdefiniujmy działanie $\sigma\in S_n$ działa na $V\otimes ...\otimes V$ przez permutowanie współrzędnych w tensorach prostych, tzn.
$$\sigma(v_1\otimes...\otimes v_n)=v_{\sigma(1)}\otimes...\otimes v_{\sigma(n)}$$
\end{bbox}

\begin{definition}
  Niech $x\in V^{\otimes n}$. Mówimy, że
  
  \begin{itemize}
    \item $x$ jest \important{symetryczny}, jeśli dla każdego $\sigma\in S_n$ mamy $\sigma(x)=x$
    \item $x$ jest \important{antysymetryczny}, jeśli dla każdego $\sigma\in S_n$ $\sigma(x)=sgn(\sigma)\cdot x$.
  \end{itemize}

      Dalej, definiujemy zbiory
      $$\color{blue}\Lambda^nV=\{x\in V^{\otimes n}\;:\;x\text{ antysymetryczny}\}$$
      $$\color{blue}S^nV=\{x\in V^{\otimes n}\;:\;x\text{ symetryczny}\}$$
\end{definition}

Jeśli $char(K)=0$, to wówczas $\Lambda^nV,S^nV\leq V^{\otimes n}$.

Gdy $n=2$, to mamy $V\otimes V=\Lambda^2V\oplus S^2V$ rozumiane jako
$$V\otimes V\ni x=\frac{1}{2}(x+\sigma(x))+\frac{1}{2}(x-\sigma(x))\in \Lambda^2V\oplus S^2V,$$
gdzie $\sigma=(1, 2)\in S_2$ (bo drugi element $S_2$ to identyczność). Widzimy, że $S^2V\cap \Lambda^2V=\{0\}$.

O $n$-liniowej funkcji $f:\underbrace{V\times...\times V}_{n}\to W$ mówimy, że jest
\begin{itemize}
  \item \acc[b]{symetryczna}, gdy dla każdego $\sigma\in S_n$ mamy $f(v_{\sigma(1)},...,v_{\sigma(n)})=f(v_1,...,v_n)$
  \item \acc[b]{antysymetryczna}, gdy dla każdego $\sigma\in S_n$ mamy $f(v_{\sigma(1)},...,v_{\sigma(n)})=sgn(\sigma)\cdot f(v_1,...,v_n)$
\end{itemize}

\textbf{Przykład:} funkcja $det:K^n\times...K^n\to K$ jest antysymetryczna.

\begin{bbox}
Dla $x\in V^{\otimes n}$ definiujemy
\begin{itemize}
  \item ${\color{blue}x_s}=\frac{1}{n!}\sum_{\sigma\in S_n}\sigma(x)\in S^nV\leq V^{\otimes n}$
  \item ${\color{blue}x_a}=\frac{1}{n!}\sum_{\sigma\in S_n}sgn(\sigma)\cdot\sigma(x)\in\Lambda^nV\leq V^{\otimes n}$.
\end{itemize}

Wtedy funkcja $f$ zadana przez
$$f(v_1,..,.v_n)=v_1\land...\land v_n=(v_1\otimes...\otimes v_n)_a$$
jest $n$-liniowa antysymetryczna, a diagram
\begin{center}\begin{tikzcd}
V_1\times...\times V_n\arrow[rr, "f"]\arrow[dr, "\substack{\scriptstyle g\text{: n-liniowe}\\\text{ antysymetryczne}}" below left] & &\Lambda^nV\arrow[dl, dashed, "\exists!\;h\text{ liniowe}"]\\
                                                                                        & W
\end{tikzcd}\end{center}
komutuje.
\end{bbox}

\begin{remark} Niech $K$ będzie ciałem takim, że $char(K)\neq 2$ i niech $\{e_1,....,e_n\}\subseteq V$ będzie bazą $V$ nad $K$. Wtedy
  $$\{e_{i_1}\land...\land e_{i_n}\;:\;1\leq i_1<i_2<...<i_n\leq k\}$$
  jest bazą $\Lambda^nV$.

  Stąd też gdy $n< k$
  $$dim(\Lambda^nV)=\binom{k}{n}$$
  $$dim(\Lambda^kV)=1$$
  i dla $n:k$ $dim(\Lambda^nV)=0$.
\end{remark}

\textbf{Przykłady:}

\begin{enumerate}
  \item Mamy $v\land v=-v\land v=0$, gdyż $\{v, v\}$ i $\{v, -v\}$ nie są zbiorami liniowo niezależnymi.
  \item Dla $v_1,v_2\in V$ liniowo niezależnych nad $K$ mamy
    $$v_1\land v_2=r\cdot w_1\land w_2$$
    dla $r\neq 0$ wtedy i tylko wtedy, gdy
    $$Lin(v_1,v_2)=Lin(w_1,w_2)$$
  \item Poprzedni przykład możemy uogólnić dla $v_1,...,v_n\in V$ liniowo niezależnych. Wtedy dla $w_1,...,w_n\in V$ mamy
    $$(\exists\;r\neq 0)\;v_1\land ...\land v_n=r\cdot w_1\land...\land w_n\iff Lin(v_1,...,v_n)=Lin(w_1,...,w_n)$$
  \item $(V\otimes W)^*\cong V^*\otimes W^*$ i $\Lambda^nV^*\cong (\Lambda^nV)^*$
  \item więcej przykładów w analizie wielowymiarowej, geometrii różniczkowej, formach różniczkowych etc.
\end{enumerate}
