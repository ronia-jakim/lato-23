\section{Wykład 2: Ciała skończone i pierwiastki z jedności}

\begin{important}
Ciało $L\supseteq K$ nazywamy \deff{ciałem rozkładu nad $K$} wielomianu $f\in K[X]$, gdy spełnione są warunki:

\indent 1. $f$ rozkłada się w pierścieniu $L[X]$ na czynniki liniowe (stopnia $1$)

\indent 2. Ciało $L$ jest rozszerzeniem ciała $K$ o elementy $a_1,...,a_n$, gdzie $a_1,...,a_n$ to wszystkie pierwiastki $f$ w $L$.
\end{important}

\textbf{Przykład:} Jeżeli $deg(f)=0$, to nie istnieje ciało rozkładu $f$.

\begin{wniosek}
    Załóżmy, że $f\in K[X]$ jest wielomianem stopnia $>0$. Wtedy

\indent 1. istnieje $L$: ciało rozkładu $f$ nad $K$,

\indent 2. to ciało jest jedyne z dokładnością do izomorfizmy nad $K$.
\end{wniosek}

\textbf{Dowód:}

\indent 1. Dowód przez indukcje względem stopnia $f$

Jako przypadek bazowy rozważmy $f$ takie, że $deg(f)=1$. Wtedy $L=K$ i wszystko wniosek jest spełniony.

Załóżmy teraz, że stopień wielomianu $f$ jest $>1$ i tez zachodzi dla wszystkich wielomianów stopnia $<deg(f)$ i wszystkich ciał $K'$. Teraz z \ref{wniosek1:2:4} wiemy, że istnieje rozszerzenie ciała $L\supseteq K$ takie, że $f$ ma pierwiastek w $L$. Nazwijmy ten pierwiastek $a_0$ i niech
$$K'=K(a_0).$$
Ponieważ $K'[X]$ wielomian $f$ ma pierwiastek $a_0$, to możemy zapisać
$$f=(x-a_0)f_1$$
dla pewnego $f_1\in K'[X]$ i $deg(f_1)<deg(f)$. Z założenia indukcyjnego dla $f_a$ istnieje $L'=K'(a_1,...,a_r)$ - ciało rozkładu wielomianu $f_1$ nad $K'$. Wtedy 
$$L=K(a_0,...,a_r)$$
jest ciałem rozkładu $f$ nad $K$.

\indent 2. Udowodnimy wersję ogólniejszą: 
\phantomsection
\label{stwierdzenie:wniosek}

\emph{(\bat) Jeśli $\phi:K_1\izo{} K_2$ jest izomorfizmem nad ciałem i $f_i\in K_i[X]$ jest wielomianem stopnia $>0$, $\phi(f_1)=f_2$, to wtedy istnieje $ \psi:L_1\izo{} L_2$ izomorfizm nad ciałami rozkładu $f_i$ w $K_i$ rozszerzający izomorfizm $\phi$ (to znaczy $\phi\subseteq \psi$).}

Wykorzystamy indukcję po $deg(f)$. W przypadku bazowym mamy $deg(f)=1$, czyli $L_1=K_1,L_2=K_2$ i $\phi=\psi$.

Teraz niech $deg(f)>1$ i załóżmy, że dla wszystkich ciał $K'$ oraz wielomianów stopnia $<deg(f)$ jest to prawdą. Niech
$$f_i=f_i'\cdot g_i,$$
gdzie $f_i',g_i\in K_i[X]$ i $g_i$ jest wielomianem nierozkładalnym w $K$. Wiemy już, że istnieje $a_i\in L_i$ będące pierwiastkiem wielomianu $g_i$.

Z faktu \ref{fakt:1:2:5}:(2), wiemy, że istnieje wtedy izomorfizm
$$\psi_0:K_1(a_1)\izo{}K_2(a_2)$$
taki, że $\psi_0(a_1)=a_2$ i $\phi\subseteq\psi_0$.

\begin{illustration}
    \node (K1) at (0, 0) {$K_1(a_1)$};
    \node (K2) at (4, 0) {$K_2(a_2)$};
    \node (L1) at (0, -2.3) {$L_1$};
    \node (L2) at (4, -2.3) {$L_2$};
    \draw[->] (K1)--(K2) node [midway, above] {$\cong$} node [midway, below] {$\exists\;\psi_0$};
    \draw[->] (L1)--(L2) node [midway, above] {$\cong$} node [midway, below] {$\exists\;\psi_1$};
    \node [rotate=90] (=1) at (0, -0.5) {$=$};
    \node [rotate=90] (=1) at (4, -0.5) {$=$};
    \node (K1') at (0, -1) {$K_1'$};
    \node (K2') at (4, -1) {$K_2'$};
    \node [rotate=90] at (0, -1.65) {$\supseteq$};
    \node [rotate=90] at (4, -1.65) {$\supseteq$};
\end{illustration}

Mamy, że {\large\color{orange}$L_i$ to ciało rozkładu $f_i'$ nad $K_i$}. W takim razie z założenia indukcyjnego istnieje izomorfizm
$$\psi_1:L_1\izo{}L_2$$
taki, że $\psi\subseteq\psi_0$ i to już jest koniec.

\begin{wniosek}
    Jeśli $f_1\in K_1[X]$ i $f_2\in K_2[X]$ są nierozkładalnymi wielomianami, $\phi:K_1\izo{}K_2$ izomorfizmem i $\phi(f_1)=f_2$, a $L_1,L_2$ to ciała rozkładu $f_1,f_2$ odpowiednio nad $K_1$ i $K_2$, $a_i\in L_i$ to pierwiastek $f_i$, to wtedy istnieje $\psi:L_1\izo{}L_2$ takie, że $\psi(a_1)=a_2$.
\end{wniosek}

\textbf{Dowód:} Wynika z dowodu stwierdzenia \hyperref[stwierdzenie:wniosek]{\bat}.

\subsection{Algebraiczne domknięcie ciała}

Ciało $L$ jest \deff{algebraicznie domknięte} $\iff$ dla każdego $f\in L[X]$ o stopniu $>0$ istnieje pierwiastek $f$ w $L$, to znaczy każdy wielomian rozkłada się na czynniki liniowe nad $L$.

\textbf{Przykład:}

\indent \point $\C$ jest algebraicznie domknięte.

\indent \point $\R$ nie jest algebraicznie domknięte, gdyż $x^2+1$ nie ma pierwiastka rzeczywistego.

\indent \point $\Q[i]$ nie jest algebraicznie domknięte, bo $x^2-2$ nie ma pierwiastka.

\begin{tw}
    Każde ciało zawiera się w pewnym ciele algebraicznie domkniętym.
\end{tw}

\textbf{Dowód:}

\acc{Lemat:} Dla każdego ciała $K$ istnieje $K'\supseteq K$ takie, że $(\forall\;f\in K[X])$ stopnia $>0$ $f$ ma pierwiastek w $K'$.

Rozważmy dobry porządek na zbiorze wielomianów z $K[X]$ stopnia $>0$
$$\{f\in K[X]\;:\;deg(f)>0\}=\{f_\alpha\;:\;\alpha\subset x\}$$
Skonstruujmy rosnący ciąg ciał $\{K_\alpha\;:\;\alpha\subset x\}$ taki, że 

\indent \point $K\subseteq K_\alpha\subseteq K_\beta$ dla $\alpha<\beta< x$

\indent \point $f_\alpha$ ma pierwiastek w $K_{\alpha+1}$.

Załóżmy, że $\alpha<x$ i mamy $\{K_\beta\;:\;\beta<\alpha\}$.

\indent 1. $\alpha$ to liczba graniczna, wtedy $K_\alpha=\ubigcup\limits_{\beta<\alpha}K_\beta$

\indent 2. $\alpha=\beta+1$ to następnik, wtedy $K_\alpha=K_\beta(a)$, gdzie $a$ to pierwiastek wielomianu $f_\beta$.

Czyli lemat jest prawdziwy.

Wracamy teraz do dowodu twierdzenia i niech $(L_n,n<\omega)$ będzie rosnącym ciągiem ciał takim, że

\indent \point $L_0=K$

\indent \point $L_{n+1}\supseteq L_n$, gdzie $L_{n+1}$ dane jest przez lemat, to znaczy $(\forall\;f\in L_n[X])$ $f$ ma pierwiastek $L_{n+1}$.

Niech
$$\hat{K}=L_\infty=\ubigcup\limits_{n<\omega}L_n.$$
Jest to ciało, ponieważ suma rosnącego ciągu ciał jest ciałem. Dalej mamy, że również
$$L[X]=\ubigcup\limits_{n<\omega}L_n[X]$$
i $L[X]$ jest algebraicznie domknięte.

\begin{uwaga}
    Załóżmy, że mamy ciała $K\subseteq L$. Wtedy

\indent \point $char(K)=char(L)$

\indent \point $0_K=0_L$ oraz $1_K=1_L$

\indent \point $K^*=K\setminus\{0\}<L^*=L\setminus\{0\}$
\end{uwaga}

\subsection{Pierwiastki z jedności}

$K$ jest \deff{ciałem prostym} wtedy i tylko wtedy, gdy $K$ nie zawierza żadnego właściwego podciała. 

\textbf{Przykład:}

\indent \point $\Q$, gdzie $char(\Q)=0$ to ciało proste nieskończone.

\indent \point Ciałem prostym skończonym jest na przykład $\Z_p$ dla liczby pierwszej $p$, wtedy $char(\Z_p)=p$.
\medskip

Niech $R$ będzie pierścieniem przemiennym z $1\neq0$. Mamy następujące definicje:

\indent 1. $a\in R$ jest \deff{pierwiastkiem z $1$ }stopnia $n>0$ $\iff$ $a^n=1$

\indent 2. $\mu_n(R)=\{a\in R\;:\;a^n=1\}$ jest \deff{grupą pierwiastków z $1$} stopnia $n$

\indent 3. $\mu(R)=\bigcup\limits_{n>0}\mu_n(R)$ jest \deff{grupą pierwiastków z $1$}

\indent 4. $a$ jest \deff{pierwiastkiem pierwotnym} stopnia $n$ z $1$ $\iff$ $a\in\mu_n(R)$ oraz $(\forall\;k<n)a\notin\mu_k(R)$.

\begin{uwaga}{\color{back}dupa}

\indent 1. $\mu_n(R)\normalsubgroup R^X$ jest grupą jednostek pierścienia

\indent 2.$\mu(R)\normalsubgroup R^X$

\indent 3. $\mu(R)$ jest \acc{torsyjną grupą abelową} (każdy element jest pierwiastkiem z $1$).
\end{uwaga}

\textbf{Przykłady}

\indent 1. $\mu(\C)=\bigcup\limits_{n>0}\mu_n(\C)\lneq(\{z\in\C\;:\;|z|=1\},\cdot\})<\C^x=C\setminus\{0\}$

\indent 2. $\mu(\C)\cong(\Q,+)/(\Z,+)$, bo $f:\Q\xrightarrow[homo]{"na"}\mu(\C)$ taki, że $f(w)=\cos(w2\pi)+i\sin(w2\pi)$ ma jądro $ker(f)=\Z$.

\indent 3. $\mu(\R)=\{\pm1\}$

\indent 4. $\mu_n(K)=\{\text{zera wielomianu }w_n(x)=x^n-1\}$

\begin{uwaga}{\color{back}dupa}
    \label{uwaga:2:6}

\indent 1. Jeśli $char(K)=0$, to $w_n(x)=x^n-1$ ma tylko pierwiastki jednokrotne w $K$

\indent 2. Jeśli $char(K)=p>0$ i $n=p^ln_1$ takie, że $p\nmid n_1$, to wszystkie pierwiastki $w_n(x)=x^n-1$ mają krotność $p^l$ w $K$.
\end{uwaga}

\textbf{Dowód:}

\indent 1. Niech $a\in K$ takie, że $w_n(a)=0$. Z twierdzenia Bezouta mamy, że
$$w_n(x)=x^n-1=x^n-a^n=(x-a)(x^{n-1}+ax^{n-2}+...+a^{n-2}x+a^{n-1})=(x-a)v_n(x),$$
gdzie $v_n(x)=x^{n-1}+ax^{n-2}+...+a^{n-2}x+a^{n-1}$.

Z tego, że $char(K)=0$ wynika, że $v_n(a)=na^{n-1}+0$, skąd wynika, że $a$ jest jednokrotnym pierwiastkiem $w_n(x)$.

\begin{fakt}
    Załóżmy, że $char(K)=p>0$. Wtedy funkcja $f:K\to K$ taka, że $f(x)=x^p$ jest homomorfizmem ciał oraz monomorfizmem zwanym \deff{funkcją Frobeniusa}.
\end{fakt}

\begin{uwaga}
    $x\mapsto x^p$ nie musi być funkcją "na" (automorfizmem). Na przykład $K=\Z_p(f)$, wtedy $x\mapsto x^p$ nie jest "na".
\end{uwaga}

\indent 2. Mając powyższy fakt i uwagę z tyłu, przechodzimy do dowodu 2.

Niech $f:K[X]\to K[X]$, $f(h(x))=w(x)^p$ i 
$$f(\sum a_kx^k)=\sum a_k^px^{n\cdot p}$$
Z faktu wyżej mamy, że $f$ jest $1-1$. Ponieważ $n=p^ln_1$, to mamy 
$$w_n(x)=x^n-1=x^n-1^n=(x^{n_1})^{p^l}-(1^{n_1})^{p^l}=...l\text{ razy}...=(x^{n_1}-1)^{p^l}=\underbrace{w_{n_1}\cdot...\cdot w_{n_1}}_{p^l},$$
zatem każdy pierwiastek $w_n(x)$ ma krotność co najmniej $p^l$. Wystarczy więc pokazać, że każdy pierwiastek $w_{n_1}(x)$ jest jednokrotny.

Niech $a\in K$ takie, że $w_{n_1}(a)=0$. Wtedy 
$$w_{n_1}(x)=x^{n_1}-a^{n_1}=(x-1)(x^{n_1-1}+...+a^{n_1-1})=(x-a)v_{n_1}(x),$$
gdzie $v_{n_1}$ jest analogiczne jak w dowodzie 1. Ale przecież $v_{n_1}(a)=n_1\cdot a^{n_1-1}\neq0$, bo $p\nmid n_1$.

\begin{tw}
    Niech $G<\mu(K)$ i $G$ jest podgrupą skończoną o $|G|=n$. Wtedy

\indent 1. $G=\mu_n(K)$

\indent 2. $G$ jest cykliczna

\indent 3. Jeśli $char(K)=p>0$, to $p\nmid n$.
\end{tw}

\textbf{Dowód}

\indent 1. Jeśli $|G|=n$, to dla każdego $x\in G$ mamy $x^n=1$. Z tego wynika, że $G\subseteq \mu_n(K)$, ale $|\mu_n(K)|\leq n$, czyli $G=\mu_n(K)$.

\indent 2. Wystarczy pokazać, że istnieje $x\in G$ taki, że $rank(x)=n$.

Załóżmy nie wprost, że dla każdego $x\in G$ $rang(x)<n$. Niech 
$$k=\max\{rank(x)\;:\;x\in G\}.$$ 
Niech $x_0\in G$ takie, że $rank(x_0)=k$. Wtedy 
$$(\forall\;y\in G)\;rank(y)|k.$$ 
Czyli
$$(\forall\;y\in G)\;y^k=1,$$
co pociąga $G\subseteq \mu_k(K)$ i $|G|\leq k<n$. Sprzeczność.

\indent 3. Wiemy, że wszystkie pierwiastki $w_n=x^n-1$ są jednokrotne, bo jest ich w tym przypadku dokładnie $n$ (z poprzedniego punktu). Z uwagi \ref{uwaga:2:6}, że jeśli $n=p^ln_1$, to pierwiastki wielomianu $w_n(x)$ mają krotność $p^l$. Ale w tym przypadku pierwiastki mają krotność jeden, czyli $p^l=1$ i $n=1\cdot n_1$, gdzie $p\nmid n_1$.

\begin{wniosek}
    Jeśli $a\in \mu_n(K)$ jest pierwiastkiem pierwotnym z $1$ stopnia $n>1$, to $a$ generuje $\mu_n(K)$.
\end{wniosek}

\textbf{Dowód:}

$\mu_n(K)\supseteq\begin{pair}a\end{pair}=\mu_k(K)$ dla pewnego $k\in\N$. Ale ponieważ $a$ było pierwiastkiem pierwotnym z $1$, to musimy mieć $n=k$.

\begin{tw}
    Niech $K$ będzie ciałem skończonym. Wtedy

\indent 1. $char(K)=p\implies |K|=p^n$ dla pewnego $n\in\N$

\indent 2. Dla każdego $n>0$ istnieje dokładnie jedno ciało $K$ takie, że $|K|=p^n$ z dokładnością do izomorfizmu.

Ciało mocy $p^n$ będziemy oznaczać $\color{blue}F(p^n)$.
\end{tw}

\textbf{Dowód:}

\indent 1. Skoro $char(K)=p$, to $\Z_p\subseteq K$ jest najmniejszym podciałem $K$. W takim razie, $K$ jest przestrzenią liniową nad $\Z_p$. Jeśli $n=dim_{\Z_p}(K)$, to $K$ jest izomorficzne z $\Z_p^n$, jako przestrzenie liniowe nad $\Z_p$. W takim razie $|K|=p^n$.

\indent 2. 

\emph{Istnienie:}

Niech $n>0$. Rozważmy 
$$w_{p^{n}-1}(x)=x^{p^n-1}\in\Z_p[X].$$
Niech $L\supseteq\Z_p$ będzie ciałem rozkładu wielomianu $w_{p^n-1}$, a $K=\{0\}\cup\{\text{ pierwiastki }w_{p^n-1}\}$. Wtedy
$$|K|=1+p^n-1=p^n,$$
czyli mamy potencjalne ciało rzędu $p^n$. Wystarczy więc pokazać, że $K$ jest ciałem.

Niech $f:L\xrightarrow[]{1-1}L$ będzie funkcją Frobeniusa $x\mapsto x^p$. Teraz niech $f^n=f\circ...\circ f$, $f^n(x)=x^{p^n}$. Jest to monomorfizm, bo składamy ze sobą $n$ takich samych funkcji $1-1$. Dla $a\in L$ mamy 
$$(a^{p^n-1}=1\;\lor\;a=0)\implies a\in K.$$
Co więcej, $a^{p^n-1}=1\iff a^{p^n}=a\iff f^n(a)=a$, czyli $K=\{a\in L\;:\;f^n(a)=a\}$ jest zbiorem punktów stałych morfizmu $f^n$, czyli jest ciałem.

\emph{Jedyność $K$:}

Ciało $K$ stworzone jak wyżej jest ciałem rozkładu $w_{p^n-1}(x)$ nad $\Z_p$. Załóżmy nie wprost, że $K'$ to inne ciało mocy $p^n$. Niech $x\in K'\setminus\{0\}$. wiemy, że $x^{p^n-1}=1$, czyli $w_{p^n-1}$ rozkłada się nad $K'$ na czynniki liniowe. Zatem $K'$ jest również ciałem rozkładu $w_{p^n-1}$ nad $\Z_p$, stąd $K\cong K'$ nad $\Z_p$ i mamy sprzeczność.

\subsection{Rozszerzenia ciał}

Niech $K\subseteq L$ będą ciałami i $a\in L\setminus K$.

\indent \point Jeżeli \acc{$a$ jest algebraiczny nad $K$}, to istnieje $f\in K[X]$ stopnia $>0$ i $f(a)=0$

\indent \point $a$ jest \acc{przestępny nad $K$} $\iff$ $a$ nie jest algebraiczny.

\indent \point \deff{Rozszerzenie} $L\supseteq K$ jest \deff{algebraiczne} $\iff$ dla każdego $a\in L$ $a$ jest algebraiczny nad $K$.

\indent \point \deff{Rozszerzenie jest przestępne} $\iff$ nie jest algebraiczne.

\indent \point Niech $a\in \C$. Wtedy $a$ jest algebraiczna, gdy $a$ jest algebraiczna nad $\Q$.

\begin{uwaga}
    Niech $a$ jak wyżej. Wtedy $a$ jest algebraiczny nad $K$ $\iff$ $I(a/K)\neq\{0\}$.
\end{uwaga}

Niech $K\subseteq L$ będzie rozszerzeniem ciała $K$. Wtedy $L$ jest \deff{przestrzenią liniową nad $K$}. Definiujemy
$$[L:K]:=\dim_K(L)$$
jako \acc{wymiar przestrzeni liniowej}.

\begin{uwaga}
    Niech $a\in L\setminus K$. Następujące warunki są równoważne:

\indent 1. $a$ jest algebraiczny nad $K$

\indent 2. $K[a]=K(a)$, to znaczy $K[a]$ jest ciałem (usuwanie niewymierności z mianownika)

\indent 3. $[K(a):K]=\dim_K(a)<\infty$
\end{uwaga}

\textbf{Dowód:}

$1\implies2$

