\subsection{Domknięcia ciał}

Ciało $L$ jest \deff{algebraicznie domknięte} $\iff$ dla każdego $f\in L[X]$ o stopniu $>0$ istnieje pierwiastek $f$ w $L$, to znaczy każdy wielomian rozkłada się na czynniki liniowe nad $L$.

\textbf{Przykład:}

\indent \point $\C$ jest algebraicznie domknięte.

\indent \point $\R$ nie jest algebraicznie domknięte, gdyż $x^2+1$ nie ma pierwiastka rzeczywistego.

\indent \point $\Q[i]$ nie jest algebraicznie domknięte, bo $x^2-2$ nie ma pierwiastka.

\begin{tw}
    Każde ciało zawiera się w pewnym ciele algebraicznie domkniętym.
\end{tw}

\textbf{Dowód:}

\acc{Lemat:} Dla każdego ciała $K$ istnieje $K'\supseteq K$ takie, że $(\forall\;f\in K[X])$ stopnia $>0$ $f$ ma pierwiastek w $K'$.

Rozważmy dobry porządek na zbiorze wielomianów z $K[X]$ stopnia $>0$
$$\{f\in K[X]\;:\;deg(f)>0\}=\{f_\alpha\;:\;\alpha\subset x\}$$
KSkonstruujmy rosnący ciąg ciał $\{K_\alpha\;:\;\alpha\subset x\}$ taki, że 

\indent \point $K\subseteq K_\alpha\subseteq K_\beta$ dla $\alpha<\beta< x$

\indent \point $f_\alpha$ ma pierwiastek w $K_{\alpha+1}$.

Załóżmy, że $\alpha<x$ i mamy $\{K_\beta\;:\;\beta<\alpha\}$.

\indent 1. $\alpha$ to liczba graniczna, wtedy $K_\alpha=\ubigcup\limits_{\beta<\alpha}K_\beta$

\indent 2. $\alpha=\beta+1$ to następnik, wtedy $K_\alpha=K_\beta(a)$, gdzie $a$ to pierwiastek wielomianu $f_\beta$.

Czyli lemat jest prawdziwy.

Wracamy teraz do dowodu twierdzenia i niech $(L_n,n<\omega)$ będzie rosnącym ciągiem ciał takim, że

\indent \point $L_0=K$

\indent \point $L_{n+1}\supseteq L_n$, gdzie $L_{n+1}$ dane jest przez lemat, to znaczy $(\forall\;f\in L_n[X])$ $f$ ma pierwiastek $L_{n+1}$.

Niech
$$\hat{K}=L_\infty=\ubigcup\limits_{n<\omega}L_n.$$
Jest to ciało, ponieważ suma rosnącego ciągu ciał jest ciałem. Dalej mamy, że również
$$L[X]=\ubigcup\limits_{n<\omega}L_n[X]$$
i $L[X]$ jest algebraicznie domknięte.

\begin{uwaga}
    Załóżmy, że mamy ciała $K\subseteq L$. Wtedy

\indent \point $char(K)=char(L)$

\indent \point $0_K=0_L$ oraz $1_K=1_L$

\indent \point $K^*=K\setminus\{0\}<L^*=L\setminus\{0\}$
\end{uwaga}

\subsection{Ciała proste}

$K$ jest \deff{ciałem prostym} wtedy i tylko wtedy, gdy $K$ nie zawierza żadnego właściwego podciała. 

\textbf{Przykład:}

\indent \point $\Q$, gdzie $char(\Q)=0$ to ciało proste nieskończone.

\indent \point Ciałem prostym skończonym jest na przykład $\Z_p$ dla liczby pierwszej $p$, wtedy $char(\Z_p)=p$.
\medskip

Niech $R$ będzie pierścieniem przemiennym z $1\neq0$. Mamy następujące definicje:

\indent 1. $a\in R$ jest \deff{pierwiastkiem z $1$ }stopnia $n>0$ $\iff$ $a^n=1$

\indent 2. $\mu_n(R)=\{a\in R\;:\;a^n=1\}$ jest \deff{grupą pierwiastków z $1$} stopnia $n$

\indent 3. $\mu(R)=\bigcup\limits_{n>0}\mu_n(R)$ jest \deff{grupą pierwiastków z $1$}

\indent 4. $a$ jest \deff{pierwiastkiem pierwotnym} stopnia $n$ z $1$ $\iff$ $a\in\mu_n(R)$ oraz $(\forall\;k<n)a\notin\mu_k(R)$.

\begin{uwaga}{\color{back}dupa}

\indent 1. $\mu_n(R)\normalsubgroup R^X$ jest grupą jednostek pierścienia

\indent 2.$\mu(R)\normalsubgroup R^X$

\indent 3. $\mu(R)$ jest \acc{torsyjną grupą abelową} (każdy element jest pierwiastkiem z $1$).
\end{uwaga}

\textbf{Przykłady}

\indent 1. $\mu(\C)=\bigcup\limits_{n>0}\mu_n(\C)\lneq(\{z\in\C\;:\;|z|=1\},\cdot\})<\C^x=C\setminus\{0\}$

\indent 2. $\mu(\C)\cong(\Q,+)/(\Z,+)$, bo $f:\Q\xrightarrow[homo]{"na"}\mu(\C)$ taki, że $f(w)=\cos(w2\pi)+i\sin(w2\pi)$ ma jądro $ker(f)=\Z$.

\indent 3. $\mu(\R)=\{\pm1\}$

\indent 4. $\mu_n(K)=\{\text{zera wielomianu }w_n(x)=x^n-1\}$

\begin{uwaga}{\color{back}dupa}

\indent 1. Jeśli $char(K)=0$, to $w_n(x)=x^n-1$ ma tylko pierwiastki jednokrotne w $K$

\indent 2. Jeśli $char(K)=p>0$ i $n=p^ln_1$ takie, że $p\nmid n_1$, to wszystkie pierwiastki $w_n(x)=x^n-1$ mają krotność $p^l$ w $K$.
\end{uwaga}

\textbf{Dowód:}

\indent 1. Niech $a\in K$ takie, że $w_n(a)=0$. Z twierdzenia Bezouta mamy, że
$$w_n(x)=x^n-1=x^n-a^n=(x-a)(x^{n-1}+ax^{n-2}+...+a^{n-2}x+a^{n-1})=(x-a)v_n(x),$$
gdzie $v_n(x)=x^{n-1}+ax^{n-2}+...+a^{n-2}x+a^{n-1}$.

Z tego, że $char(K)=0$ wynika, że $v_n(a)=na^{n-1}+0$, skąd wynika, że $a$ jest jednokrotnym pierwiastkiem $w_n(x)$.

\begin{fakt}
    Załóżmy, że $char(K)=p>0$. Wtedy funkcja $f:K\to K$ taka, że $f(x)=x^p$ jest homomorfizmem ciał oraz monomorfizmem zwanym \deff{funkcją Frobeniusa}.
\end{fakt}

\begin{uwaga}
    $x\mapsto x^p$ nie musi być funkcją "na" (automorfizmem). Na przykład $K=\Z_p(f)$, wtedy $x\mapsto x^p$ nie jest "na".
\end{uwaga}

\indent 2. Mając powyższy fakt i uwagę z tyłu, przechodzimy do dowodu 2.

Niech $f:K[X]\to K[X]$, $f(h(x))=w(x)^p$ i 
$$f(\sum a_kx^k)=\sum a_k^px^{n\cdot p}$$
Z faktu wyżej mamy, że $f$ jest $1-1$.