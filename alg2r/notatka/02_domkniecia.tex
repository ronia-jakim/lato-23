\section{Ciała skończone i pierwiastki z jedności}

\begin{bbox}
Ciało $L\supseteq K$ nazywamy \important{ciałem rozkładu nad $K$} wielomianu $f\in K[X]$, gdy spełnione są warunki:

\indent 1. $f$ rozkłada się w pierścieniu $L[X]$ na czynniki liniowe (stopnia $1$)

\indent 2. Ciało $L$ jest rozszerzeniem ciała $K$ o elementy $a_1,...,a_n$, gdzie $a_1,...,a_n$ to wszystkie pierwiastki $f$ w $L$.
\end{bbox}

\textbf{Przykład:} Jeżeli $deg(f)=0$, to nie istnieje ciało rozkładu $f$.

\begin{conclusion}
    \label{wniosek:2.1}
    Załóżmy, że $f\in K[X]$ jest wielomianem stopnia $>0$. Wtedy

\indent 1. istnieje $L$: ciało rozkładu $f$ nad $K$,

\indent 2. to ciało jest jedyne z dokładnością do izomorfizmy nad $K$.
\end{conclusion}

\begin{proof}{\color{pagColor}dupa}

\begin{enumerate}
\item Dowód przez indukcje względem stopnia $f$

Jako przypadek bazowy rozważmy $f$ takie, że $deg(f)=1$. Wtedy $L=K$ i wszystko wniosek jest spełniony.

Załóżmy teraz, że stopień wielomianu $f$ jest $>1$ i tez zachodzi dla wszystkich wielomianów stopnia $<deg(f)$ i wszystkich ciał $K'$. Teraz z \ref{wniosek1:2:4} wiemy, że istnieje rozszerzenie ciała $L\supseteq K$ takie, że $f$ ma pierwiastek w $L$. Nazwijmy ten pierwiastek $a_0$ i niech
$$K'=K(a_0).$$
Ponieważ $K'[X]$ wielomian $f$ ma pierwiastek $a_0$, to możemy zapisać
$$f=(x-a_0)f_1$$
dla pewnego $f_1\in K'[X]$ i $deg(f_1)<deg(f)$. Z założenia indukcyjnego dla $f_a$ istnieje $L'=K'(a_1,...,a_r)$ - ciało rozkładu wielomianu $f_1$ nad $K'$. Wtedy 
$$L=K(a_0,...,a_r)$$
jest ciałem rozkładu $f$ nad $K$.

\item Udowodnimy wersję ogólniejszą: 
\phantomsection
\label{stwierdzenie:wniosek}

\emph{{\color{yellow}(\Bat)} Jeśli $\phi:K_1\isomorphism{} K_2$ jest izomorfizmem nad ciałem i $f_i\in K_i[X]$ jest wielomianem stopnia $>0$, $\phi(f_1)=f_2$, to wtedy istnieje $ \psi:L_1\isomorphism{} L_2$ izomorfizm nad ciałami rozkładu $f_i$ w $K_i$ rozszerzający izomorfizm $\phi$ (to znaczy $\phi\subseteq \psi$).}

Wykorzystamy indukcję po $deg(f)$. W przypadku bazowym mamy $deg(f)=1$, czyli $L_1=K_1,L_2=K_2$ i $\phi=\psi$.

Teraz niech $deg(f)>1$ i załóżmy, że dla wszystkich ciał $K'$ oraz wielomianów stopnia $<deg(f)$ jest to prawdą. Niech
$$f_i=f_i'\cdot g_i,$$
gdzie $f_i',g_i\in K_i[X]$ i $g_i$ jest wielomianem nierozkładalnym w $K$. Wiemy już, że istnieje $a_i\in L_i$ będące pierwiastkiem wielomianu $g_i$.

Z faktu \ref{fakt:1:2:5}:(2), wiemy, że istnieje wtedy izomorfizm
$$\psi_0:K_1(a_1)\isomorphism{}K_2(a_2)$$
taki, że $\psi_0(a_1)=a_2$ i $\phi\subseteq\psi_0$.
\end{enumerate}
\begin{illustration}
    \node (K1) at (0, 0) {$K_1(a_1)$};
    \node (K2) at (4, 0) {$K_2(a_2)$};
    \node (L1) at (0, -2.3) {$L_1$};
    \node (L2) at (4, -2.3) {$L_2$};
    \draw[->] (K1)--(K2) node [midway, above] {$\cong$} node [midway, below] {$\exists\;\psi_0$};
    \draw[->] (L1)--(L2) node [midway, above] {$\cong$} node [midway, below] {$\exists\;\psi_1$};
    \node [rotate=90] (=1) at (0, -0.5) {$=$};
    \node [rotate=90] (=1) at (4, -0.5) {$=$};
    \node (K1') at (0, -1) {$K_1'$};
    \node (K2') at (4, -1) {$K_2'$};
    \node [rotate=90] at (0, -1.65) {$\supseteq$};
    \node [rotate=90] at (4, -1.65) {$\supseteq$};
\end{illustration}

Z założenia wiemy, że {$L_i$ to ciało rozkładu $f_i'$ nad $K_i$}. W takim razie z założenia indukcyjnego istnieje izomorfizm
$$\psi_1:L_1\isomorphism{}L_2$$
taki, że $\psi\subseteq\psi_0$ i to już jest koniec.
\end{proof}

\begin{conclusion}
    Jeśli $f_1\in K_1[X]$ i $f_2\in K_2[X]$ są nierozkładalnymi wielomianami, $\phi:K_1\isomorphism{}K_2$ izomorfizmem i $\phi(f_1)=f_2$, a $L_1,L_2$ to ciała rozkładu $f_1,f_2$ odpowiednio nad $K_1$ i $K_2$, $a_i\in L_i$ to pierwiastek $f_i$, to wtedy istnieje $\psi:L_1\isomorphism{}L_2$ takie, że $\psi(a_1)=a_2$.
\end{conclusion}

\begin{proof}
Wynika z dowodu stwierdzenia \hyperref[stwierdzenie:wniosek]{{\color{yellow}(\Bat)}}.
\end{proof}

\subsection{Algebraiczne domknięcie ciała}

Ciało $L$ jest \important{algebraicznie domknięte} $\iff$ dla każdego $f\in L[X]$ o stopniu $>0$ istnieje pierwiastek $f$ w $L$. To znaczy każdy wielomian rozkłada się na czynniki liniowe nad $L$.

\textbf{Przykład:}
\begin{itemize}
\item$\C$ jest algebraicznie domknięte.
\item $\R$ nie jest algebraicznie domknięte, gdyż $x^2+1$ nie ma pierwiastka rzeczywistego.
\item  $\Q[i]$ nie jest algebraicznie domknięte, bo $x^2-2$ nie ma pierwiastka.
\end{itemize}
\begin{theorem}
    \label{tw:2.3}
    Każde ciało $K$ zawiera się w pewnym ciele algebraicznie domkniętym.
\end{theorem}

\begin{proof}
Jak mamy wielomian nad ciałem, to istnieje rozszerzenie ciała do tego wielomianu. I dalej leci kombinatoryka.

\acc{Lemat:} Dla każdego ciała $K$ istnieje $L\supseteq K$ takie, że $(\forall\;f\in K[X])$ stopnia $>0$, $f$ ma pierwiastek w $L$.

Rozważmy dobry porządek na zbiorze wielomianów z $K[X]$ stopnia $>0$
$$\{f\in K[X]\;:\;deg(f)>0\}=\{f_\alpha\;:\;\alpha< \kappa\}.$$
Tutaj $\alpha,\kappa$ to liczby porządkowe, niekoniecznie skończone. Skonstruujmy rosnący ciąg rozszerzeń  ciał $\{K_\alpha\;:\;\alpha< \kappa\}$ taki, że 
\begin{itemize}
\item $K\subseteq K_\alpha\subseteq K_\beta$ dla $\alpha<\beta< \kappa$
\item $f_\alpha$ ma pierwiastek w $K_{\alpha+1}$.
\end{itemize}
Dowód przez indukcję pozaskończoną. Dla $K_0=K$. 

Załóżmy, że $\alpha<\kappa$ i mamy $\{K_\beta\;:\;\beta<\alpha\}$ spełniają warunki powyżej.  Niech $K'=\abigcup\limits_{\beta<\alpha}K_\beta$. Musimy pokazać, że $K'$ jest ciałem.
\begin{enumerate}
\item 1. $\alpha$ to liczba graniczna. Definiujemy $K'=\abigcup\limits_{\beta<\alpha}K_\beta$ jako zbiór. 

Musimy określić działania w $K'$. Niech $x,y\in K'$, wtedy istnieje $\beta<\alpha$ takie, że $x,y\in K_\beta$. Czyli $x+y\in K_\beta\subseteq K'$ i $xy\in K_\beta\subseteq K'$. W takim razie $K'$ jest rozszerzeniem ciała $K_\beta$.

Teraz definiujemy $K_\alpha=K'$ i otrzymujemy pożądane rozszerzenie ciała.

\item 2. $\alpha=\beta+1$ to następnik, wtedy $K'=K_\beta$. 

Wielomian $f_\alpha$ jest wielomianem nad $K\subseteq K'$. Z wniosku \ref{wniosek1:2:4} wiemy, że istnieje rozszerzenie $K_\alpha\supseteq K$ takie, że $f_\alpha$ ma pierwiastek w $K_\alpha$.

$L$ definiujemy jako sumę po wyżej udowodnionej konstrukcji:
$$L=\abigcup\limits_{\alpha<\kappa}K_\alpha$$
i to ciało spełnia nasz lemat.
\end{enumerate}

Wracamy teraz do dowodu twierdzenia \ref{tw:2.3} i niech $(L_n,n<\omega)$ będzie rosnącym ciągiem ciał takim, że
\begin{itemize}
\item $L_0=K$
\item $L_{n+1}\supseteq L_n$, gdzie $L_{n+1}$ dane jest przez lemat, to znaczy $(\forall\;f\in L_n[X])$ $f$ ma pierwiastek w $L_{n+1}$.
\end{itemize}

Niech
$$L_\infty=\abigcup\limits_{n<\omega}L_n\supseteq K.$$
Jest to ciało, ponieważ suma rosnącego ciągu ciał jest ciałem. Dalej mamy, że jest to ciało algebraicznie domknięte, gdy dowolny $f\in L_\infty [X]$ ma stopień skończony $>0$, czyli istnieje $n$ takie, że $f\in L_n[X]$. A więc $f$ ma wszystkie pierwiastki w $L_{n+1}\subseteq L_\infty$.
\end{proof}

% $$L[X]=\ubigcup\limits_{n<\omega}L_n[X]$$
% i $L[X]$ jest algebraicznie domknięte.
