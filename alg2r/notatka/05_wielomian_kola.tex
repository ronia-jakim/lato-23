
\section{Coś}

\begin{wniosek}[algebraiczne rozszerzenia ciał]
    %\begin{enumerate}
        %\item 
        Niech $K\subseteq L\subseteq M$ będą rozszerzeniami ciał. $K\subseteq M$ jest algebraiczne $\iff$ $K\subseteq L$ i $L\subseteq M$ są algebraiczne
        %\item $K_{alg}(L)$ jest relatywnie algebraicznie domknięte w $L$, tzn. $K_{alg}(L)=[K_{alg}(L)]_{alg}(L)$ 
    %\end{enumerate}
\end{wniosek}

\textbf{Dowód:}

$\implies$ OK

$\impliedby$

Weźmy dowolny $m\in M$. $L\subseteq M$ jest algebraiczny, co oznacza $f(m)=0$, gdzie $f\in L[X]$
$$f=\sum\limits_{i=0}^na_nx^i,\quad a_n\neq 0$$
W takim razie $m$ jest algebraiczne nad ciałek $K(a_0,....,a_n)$. Ale teraz
$$[K(m)\;:\;K]\leq [K(a_0,...,a_m,m)\;:;K]\overset{fakt:4:6}{=}[K(a_0,...,a_n,m):K(a_0,...,a_n)][K(a_0,...,a_n):K]<\infty$$
bo $m$ jest algebraiczny $K(\overline a)$. Czyli
$$[K(m):K]<\infty$$
więc $m$ jest algebraiczny nad $K$ (fakt 4.3).

\begin{wniosek}[$(K_{alg}(L))_{alg}(L)=K_{alg}(L)$]
    $K_{alg}(L)$ jest relatywnie algebraicznie domknięty w $L$. To znaczy $(K_{alg}(L))_{alg}(L)=K_{alg}(L)$. 
\end{wniosek}

\textbf{Dowód:}

Ćwiczenia.

\subsection{Wielomian rozkładu koła [cyclotomic polynomials]}

Rozważamy wielomian
$$w_m(x)=x^m-1$$
dla $m\in\N$. Wiemy, że
\begin{itemize}
    \item[\point] pierwiastki $w_m$ w $\C$ są jednokrotne
    \item[\point] $\mu_m(\C)$ jest grupą cykliczną
    \item[\point] $a\in\mu_m(\C)$ jest generatorem $\mu_m(\C)=\{a^i\;:\;0\leq i\leq m\}\cong(\Z_m,+)$
    \item[\point] $a^k$ generuje $\mu_m(\C)$ $\iff NWD(k, m)=1$
\end{itemize}

\deff{Funkcja Eulera:} 
$$\phi(m)=|\{k\in\N\;:\;0\leq k<m.\;NWD(k,m)=1\}|$$
$\mu_m(\C)$ ma $\phi(m)$ generatorów.

Niech
$$\{k\in\N\;:\;0<k<m, NWD(k, m)=1\}=\{m_1,...,m_{\phi(n)}\}$$
i zdefiniujmy
$$\color{blue}F_m(x):=(x-a^{m_1})...(x-a^{m_{\phi(n)}})\in\C[X]$$
$F_m$ to \dyg{$m$-ty wielomian cyklotoniczny.}

\begin{uwaga}[$F_m\in\Z\begin{bmatrix}X\end{bmatrix}$]{\color{back}ddd}
    \begin{enumerate}
        \item $w_m(x)=x^m-1=F_m(x)\cdot v_m(x)=F_m(x)\cdot\prod\limits_{\substack{d<m\\d|m}}F_d(x)$
        \item $F_m(x)\in\Z[X]$
    \end{enumerate}
\end{uwaga}

\textbf{Dowód:}
 
1. Wiemy, że wielomian $w_m$ ma $m$ pierwiastków na płaszczyźnie Gaussa, więc jest iloczynem dwumianów $x-b,\;b\in\mu_m(\C)$, czyli
\begin{align*}
    \alpha\in\mu_m(\C)\implies\alpha^d-1\quad d=ord(\alpha),\;d|m
\end{align*}
Wtedy $\alpha$ jest pierwiastkiem pierwotnym z $1$ stopnia $d$. Wobec tego
$$F_d(x)=\prod\limits_{\substack{\alpha\in\mu_m(\C)\\ord(\alpha)=d}}(x-\alpha)\implies (\text{teza})$$
% \begin{align*}
%     w_m(x)&=\prod\limits_{b\in\mu_m(\C)}(x-b)=\prod\limits_{d|m}\prod\limits_{\substack{b\in\mu_m(\C)\\ord(b)=d}}(x-b)=\prod\limits_{d|m}F_d(x)=F_m(x)\prod\limits_{\substack{d<m\\d|m}}F_d(m)
% \end{align*}

2. Dowód przez indukcję względem $m$. Dla $m=1$ mamy $F_m(x)=x-1\in\Z[X]$. 

Teraz zakładamy, że dla wszystkich $0<d<m$ jest $F_d(x)\in\Z[X]$. Z punktu (1) wiemy, że
$$x^m-1=w_m(x)=F_m(x)v_m(x)$$
z założenia indukcyjnego $v_m(x)\in\Z[X]$, bo jest iloczynem $\prod\limits_{\substack{\alpha\in\mu_m(\C)\\ord(\alpha)=d}}(x-\alpha)$
%każdy z nich ma stopień mniejszy niż $m$ i $v_m(x)$ jest unormowany.

$w_m(x)$ w $\Z[X]$ jest podzielny przez $v_m$ i dostajemy:
$$w_m(x)=v_m(x)\cdot L(x)$$
ale w $\C[X]\supseteq\Z[X]$ było
$$w_m(x)=v_m(x)\cdot F_m(x),$$
czyli $F_m=L\in\Z[X]$.

\begin{uwaga}[lemat Gaussa: $F_m$ nierozkładalny w $\Q$]
    [\dyg{Lemat Gaussa}] $F_m(x)$ jest wielomianem nierozkładalnym w $\Q[X]$ (równoważnie w $\Z[X]$).
\end{uwaga}

\textbf{Dowód:} 

Po pierwsze zauważmy, że $F_m$ jest nierozkładalny w $\Q[X]$ $\iff$ nierozkładalny w $\Z[X]$. 

Załóżmy nie wprost, że
$$F_m(x)=G_1(x)\cdot G_2(x)$$
dla $G_1,G_2\in\Z[X]$. Możemy założyć, że $G_1(x)$ jest dalej nierozkładalny w $\Z[X]$ oraz $0<deg(G_1)<deg(F_m)=\phi(m)$
\medskip

\podz{fore}
\medskip

\acc{Lemat:} Istnieje $\varepsilon'$-pierwiastek $G_1$ oraz liczba pierwsza $p$ taka, że $p\nmid m$ i $G_1(b)=G_2(b^p)=0$.

\textbf{Dowód lematu:}

Niech $\varepsilon$ będzie jakimś pierwiastkiem $G_1$, a $\tau$ będzie jakimś pierwiastkiem $G_2$. W takim razie
$$\tau,\varepsilon\in\mu_m(\C)\implies\tau=\varepsilon^l$$
dla pewnego $l$ takiego, że $NWD(l,m)=1$.
%Zatem istnieje $k\in\N$, $NWD(k, m)=1$ takie, że $b'=b^k$, bo grupa $\mu_m(\C)$ jest cykliczna i $b$ jest jej generatorem.

% Z tego, że $0<deg(G_1)$ i $G_1|F_m$, $0<deg(G_2)$ i $G_2|F_m$ mamy, że istnieje pierwiastek $b$ stopnia $m$ taki, że $G_1(b)=0$ oraz pierwiastek pierwotny $b'$ stopnia $m$ taki, że $G_2(b')=0$. Zatem istnieje $k\in\N$, $NWD(k, m)=1$ takie, że $b'=b^k$, bo grupa $\mu_m(\C)$ jest cykliczna i $b$ jest jej generatorem.

Niech $l=p_1\cdot...p_s$ będzie rozkładem na liczby pierwsze. Wtedy mamy ciąg różnych liczb
$$\text{pierwiastem }G_1=\varepsilon, \varepsilon^{p_1},\varepsilon^{p_1p_2},...,\varepsilon^{p_1,...,p_s}=\tau\text{ pierwiastek }G_2$$
które są pierwiastkami pierwotnymi stopnia $m$. Z tego wynika, że każda z tych liczb jest pierwiastkiem $G_1$ lub $G_2$, czyli istnieje taka pozycja $i$, że
$$G_1(\varepsilon^{p_1...p_i})=0,$$
$$G_2(\varepsilon^{p_1...p_{i+1}})=0$$
wtedy $\varepsilon':=\varepsilon^{p_1...p_i}$ oraz $p=p_{i+1}$ i lemat jest spełniony.
\medskip

\podz{fore}
\medskip

Wimy już, że $G_1(\varepsilon)=0$ i $G_1\in\Z[X]$ jest wielomianem nierozkładalnym. Niech $p$ będzie liczbą pierwszą z lematu. Rozważmy
$$G_3(x)=G_2(x^p).$$
Wtedy $G_2(\varepsilon^p)=G_3(\varepsilon)=0$, ale stąd wynika, że $G_1(x)$ dzieli $G_3(x)$. Niech więc 
$$G_3(x)=G_1(x)H(x)\in\Z[X].$$

Rozważmy homomorfizm
$$f:\Z\to\Z_p\Z/p\Z=$$
i indukowany przez niego epimorfizm pierścieni
$$\overline f:\Z[X]\to\Z_p[X].$$
Z założenia $F_m=G_1G_2$ mamy, że
$$\overline f(F_m)=\overline f(G_1)\overline f(G_2)$$
a z rozumowania powyżej ($G_3=G_1H$)
$$\overline f(G_3)=\overline f(G_1)\overline f(H)$$
ale
$$\overline f(G_3(x))=\overline f(G_2(x^p))=\overline f(G_2(x))^p,$$
bo współczynniki $f(G_2(x^p))$ są w $\Z_p$, a $(\sum c_ix^i)^p=\sum c_ix^{pi}$, bo $c_i^{kp}=c_i^k$ dla $c_i\in\Z_p$.

Stąd wiemy, że
$$f(G_2(x))^p=\overline f(G_1)\overline f(H).$$
Pierścień $\Z_p[X]$ jest UFD, więc $\overline f(G_1)$ i $\overline f(G_2)$ mają wspólny dzielnik w $\Z_p[X]$, stopnia co najmniej $1$. Zatem z
$$\overline f(F_m)=\overline f(G_1)\overline f(G_2)$$
$$\overline f(F_m)|\overline f(w_m)=x^m-1.$$
Zatem w pewnym rozszerzeniu $L\supseteq\Z_p$ $w_m$ ma pierwiastek wielokrotny co daje sprzeczność.
%. Jest to sprzeczność, bo nie mogą istnieć pierwiastki wielokrotne: $\overline f(F_m)|x^m-1=w_m$, a $x^m-1$ ma pierwiastki jednokrotne w $\Q$.

{\large\color{orange}TUTAJ BYŁ KONIEC}

\begin{wniosek}[pierwiastek pierwotny a $dim_\Q(\Q(b))$]
    Jeżeli $\varepsilon\in\C$ jest pierwiastkiem pierwotnym z $1$ stopnia $m$, to $[\Q(\varepsilon):\Q]=\phi(m)$.
\end{wniosek}

\textbf{Dowód:} $F_m(x)\in\Q[X]$ jest nierozkładalny, a $\varepsilon$ jest jego pierwiastkiem. To znaczy, że $F_,(x)$ jest wielomianem minimalnym dla $\varepsilon$ nad $\Q$. Mamy, że $[\Q(b):\Q]=deg F_m=\phi(m)$.

\begin{lemat}[lemat Liouville'a o aproksymacji diofatycznej] \dyg{[lemat Liouville'a o aproksymacji diofantycznej]}: Jeżeli $a\in\R$ jest liczbą algebraiczną stopnia $N>1$, to istnieje $c=c(a)\in\R_+$ takie, że dla każdego $r=\frac pq\in\Q$ zachodzi
    $$\left|a-\frac pq\right|\geq{c\over q^N}$$
\end{lemat}

Lemat Liouville'a mówi o cesze. Jeżeli liczba nie spełnia tego lematu, to jest \acc{liczbą przestępną}.

\begin{proof}
Niech $N>1$ i $a\in\Q$. Niech $f\in\Z[X]$ taki, że $f(a)=0$ i $deg(f)=deg(a/\Q)$. Teraz zauważmy, że na $f$ patrzymy jako na funkcję wielomianową. To znaczy, dla każdego $x\in\R$ patrząc na
$$\hat{f}(x)=\hat{f}(x)-\underbrace{\hat{f}(a)}_{=0}$$
ale funkcje wielomianowe są różniczkowalne. Dlatego możemy skorzystać z twierdzenia o wartości średniej. To znaczy
$$\hat{f}(x)-\hat{f}(a)=\hat{f}'(x-a)$$
My wiemy, że $a$ jest pierwiastkiem jednokrotnym wielomianu $f(x)$. Niech $\varepsilon>0$ takie, że $a\in(a-\varepsilon,a+\varepsilon)$ jest jedynym pierwiastkiem $f(x)$ w tym przedziale. Oczywiście, 
$$deg(\hat{f}'(x))<deg(\hat{f}(x))\implies\hat{f}'(a)\neq 0.$$ 
Bez straty ogólności $\hat{f}'(a)>0$. Niech i $d=\sup\limits_{x\in I}\hat{f}'(x)$.
$$c=c(a)=\min(\varepsilon, \frac1{d}).$$

Udowodnimy, że $c$ jest dobrze określona. Niech $r=\frac{p}{q}\in\Q$ i $p,q\in\Z$, $q>0$.
$$f(x)=\sum\limits_{k=0}^Na_kx^k,\quad a_k\in\Z,a_N\neq0$$
Rozważamy przypadki:
\begin{enumerate}
    \item $f\notin I$. Wtedy $\left|a-\frac{p}{q}\right|\geq\varepsilon\geq {\varepsilon\over q^N}\geq \frac{c}{q^n}$
    \item $f\in I$. Wtedy $\left|a-\frac{p}{q}\right|$ i $\frac{p}{q}$ może być naszym $x$. Czyli 
    $$\left|a-\frac{p}{q}\right|=\frac{|f(\frac{p}{q})|}{|f(f'(t)|}\geq \frac{|f(\frac{p}{q})|}{d}\geq\frac{c}{q^N}$$
    bo $c\leq\frac1{d}$ $$0\neq|f(\frac{p}{q})|=\left|\sum\limits_{k=0}^Na_k\frac{p^k}{q^k}\right|=\frac{\left|\sum\limits_{k=0}^Na_kp^kq^{N-k}\right|}{q^N}\geq\frac{1}{q^N}$$
\end{enumerate}
\end{proof}


\begin{definicja}[algebraiczne domknięcie]
    Ciało $L\supseteq K$ jest \deff{algebraicznym domknięciem} $K$ wtedy i tylko wtedy, gdy:
    \begin{enumerate}
        \item $L$ jest algebraicznie domknięte
        \item $L\supseteq K$ jest rozszerzeniem algebraicznym, to znaczy dla każdego $a\in L$ $a$ jest pierwiastkiem algebraicznym nad $K$
    \end{enumerate}
    Takie $L$ oznaczamy przez $\color{blue}\hat{K},K^{alg}$.
\end{definicja}

\begin{wniosek}[istnieje algebraiczne domknięcie]
    Dla każdego $K$ istnieje algebraiczne domknięcie $\hat{K}$.
\end{wniosek}

\begin{proof}
Rozważmy $K_\infty\supseteq K$ - ciało algebraicznie domknięte (twierdzenie z początku wykładu). Pokażemy, że
$$\hat{K}=K_{alg}(K_\infty)=\{a\in K_\infty\;:\;a\text{ algebraiczny nad }K\}$$
\begin{enumerate}
    \item $\hat{K}$ jest algebraicznie domknięte:

    Jeżeli $f\in\hat{K}[X]$, to $f$ ma pierwiastek w $K$, ale $\hat{K}\subseteq K_\infty$, to znaczy, że $a\in\hat{K}$ jest algebraiczne nad $K$.

    \item $K\subseteq\hat{K}$ jest rozszerzeniem algebraicznym:

    $K\subseteq\widehat{K}=K_{alg}(K_\infty)$ z definicji jest rozszerzeniem algebraicznym.
\end{enumerate}
\end{proof}

\begin{tw}[jedyność domknięcia algebraicznego]\label{tw:4:15}
    $\hat{K}$ jest jedyne z dokładnością do izomorfizmu nad $K$.
\end{tw}

\begin{center}
    \begin{tikzcd}
        L_1\arrow[rr, "(\exists!\;f)\;f\restriction K=id_K" above, "\cong" below] & & L_2\\
        & K\arrow[ul, "\supseteq" lablb]\arrow[ur, "\subseteq" labl] &
    \end{tikzcd}
\end{center}

\begin{proof}
Można użyć indukcji pozaskończonej, a można też użyć lematu Zorna. My zrobimy to drugie.

Niech 
$$\mathfrak{K}=\{(k',f')\;:\;K\subseteq K'\subseteq L_1,f':K'\xrightarrow[]{1-1}L_2,\;f'\restriction K=id_k\}$$
\begin{illustration}
    \draw (0, 0) ellipse(1.2 and 2);
    \node at (0, -2.5) {$L_1$};
    \draw (5, 0) ellipse (1.2 and 2);
    \draw (0, -1) ellipse (0.8 and 1);
    \node at (-0.3, -1) {$K$};
    \draw (0, -0.55) ellipse (1 and 1.4);
    \node at (-0.3, 0.3) {$K'$}; 
    \draw (5, -1) ellipse (0.8 and 1);
    \draw (5, -0.55) ellipse (1 and 1.4); 
    \node at (5.2, 0.3) {$f'[K']$}; 
    \node at (5.3, -1) {$K$};
    \node at (5, -2.5) {$L_2$};
    \draw[->] (0.5, -1)..controls (1, 0) and (4, 0)..(4.5, -1) node [midway, above] {$id_K$};
    \draw[->] (0.5, 0.2)..controls (1, 1.2) and (4, 1.2)..(4.5, 0.2) node [midway, above] {$f'$};
\end{illustration}

Oczywiście, $\mathfrak{K}\neq\emptyset$, bo $(K,id_K)\in\mathfrak{K}$. W $\mathfrak{K}$ definiujemy relację porządku w naturalny sposób, to znaczy
$$(K', f')\leq(K'', f'')\iff K'\subseteq K''\;\land\;f''\restriction K'=f''.$$
Wtedy $(\mathfrak{K},\leq)$ jest zbiorem częściowo uporządkowanym i niepustym (bo jest $(K,id_K)\in\mathfrak{K}$). Ponadto każdy wstępujący łańcuch $(\mathfrak{K},\leq)$ ma ograniczenie górne. Na mocy lematu Kuratowskiego-Zorna w tej rodzinie istnieje element maksymalny, nazwijmy go $(K_1,f_1)$. Pokażemy, że $K_1=L_1$.

Załóżmy nie wprost, że istnieje $a\in L_1\setminus K_1$. Niech $w(x)\in K_1[X]$ będzie wielomianem minimalnym elementu $a$ nad $K_1$. Niech
$$K_2=f_1[K_1]$$
$$v(x)=f_1(a_0)+f_1(a_1)x+...+f_1(a_n)x^n\in K_2[X].$$
$v(x)$ też jest nierozkładalny nad $K_2$, bo $w(x)$ był nierozkładalny nad $K_1$. Niech $b\in L_2$ będzie pierwiastkiem wielomianu $v$.

Zauważmy, że $K_1(a)=K_1[a]$, bo $w(x)$ jest nierozkładalny nad $K_1$, ale
$$K_1[a]\simeq K_1[X]/(w)\simeq K_2[X]/(v)\simeq K_2[b]\simeq K_2(b).$$
Czyli $K_1(a)\simeq K_2(b)$ i $f_2:K_1(a)\izo{}K_2(b)$ jest izomorfizmem rozszerzającym $f_1$. Wtedy mamy $(K_1,f_1)\lneq(K_1(a),f_2)$, co daje sprzeczność z maksymalnością $(K_1,f_1)$. Zatem $L_1=K_2$.

Zrobimy sprytnie wprost: $K_1=L_1$, $K\subseteq K_2\subseteq L_2$ i $K_1\cong_K K_2$. $K_1$ jest aglebraicznie domknięte, więc $K_2$ też takie musi być. Czyli $K\subseteq K_2\subseteq L_2$ jest algebraiczne, więc $K_2=L_2$, bo założyliśmy, że $b\in L_2\setminus K_2$ i wtedy wielomina minimalny $f_b(x)\in K_2[X]$ ma pierwiastek $c\in K_2$, czyli $(x-c)|f_n(x)$ a więc $x-c=f_b(x)$ jest nierozkładalny i $b=c$.

%Niech $K_2=f[K_1]=f[L_1]$. Pokażemy nie wprost, że $K_2=L_2$. Załóżmy, że istnieje $a\in L_2\setminus K_2$. Niech $w(x)\in K_2[X]$ wielomian minimalny dla $a$ nad $K_2$. Wtedy $w(x)$ nie ma pierwiastka w $K_2$, ale $K_2=f_1[L_1]$ jest algebraicznie domknięte, bo $L_1$ jest algebraicznie domknięte, co daje sprzeczność.
\end{proof}

\begin{wniosek}[$K\cong L\implies\hat{K}\cong\hat{L}$]
    Jeśli $K\cong L$, to $\hat{K}\cong \hat{L}$. Dokładniej, jeżeli $f_0:LK\to L$ jest izomorfizmem ciał, to istnieje izomorfizm $f:\hat{K}\to\hat{L}$ taki, że $f\restriction K=f_0$.
\end{wniosek}
\begin{proof}
Ćwiczenia
\end{proof}

\begin{wniosek}[algebraiczne rozszerzenie $1-1$ $\to$ $\hat{K}$]
    Jeśli $K\subseteq L$ jest algebraicznym rozszerzeniem ciał, to istnieje monomorfizm $f:L\to \hat{K}$ taki, że $f\restriction K=id_K$.
\end{wniosek}

\begin{proof}
Ćwiczenie
\end{proof}

%\textbf{Dowód:} Mamy dane $K\subseteq L\subseteq\hat{L}$ rozszerzenia algebraiczne, zatem rozszerzenie $K\subseteq\hat{L}$ jest algebraiczne. Stąd $\hat{L}$ jest algebraicznym domknięciem $K$. Z twierdzenia \ref{tw:4:15} istnieje izomorfizm $g:\hat{L}\to\hat{K}$ taki, że $g\restriction K=id_K$. Wtedy $f=g\restriction L$ jest szukanym monomorfizmem.
