\subsection{Moduły}

\emph{Przestrzenie liniowe nad pierścieniami}

\begin{definition}
    Niech $R$ będzie pierścieniem z $1$, niekoniecznie przemienny. $(M,+,r)_r\in R$ jest modułem nad $R$, gdy spełnia aksjomaty przestrzeni liniowej nad $R$.

    Moduł może być:
    \begin{itemize}
        \item[\PHtunny] lewostronny, wtedy $M\ni x\mapsto rx$ dla każdego $r,x$ jest w $M$
        \item[\PHtunny] prawostronny (analogicznie, $xr\in M$).
    \end{itemize}
\end{definition}

Łączność mnożenia w modułach:

\begin{tabular}{ p{5cm} p{5cm} p{5cm} }
lewostronna & prawostronna & mieszana \\

$r_1(r_2m)=(r_1r_2)m$

&

$(mr_1)r_2=m(r_1r_2)$

&

jeśli jesteśmy w lewostronnym module:

$(mr_2)r_1=m(r_2r_1)$

i nie to samo co przy prawostronnym
\end{tabular}

\textbf{Przykłady:}

\begin{enumerate}
    \item $R=K$ to ciała, $K$-moduł to przestrzeń liniowa nad $K$
    \item $G$ jest grupą abelową, wtedy $G$ jest $\Z$-modułem
    \item $G$ ejst grupą abelową, wtedy $End(G)$ są pierścieniem z jednością i $G$ jest modułem nad $end(G)$
    \item Załóżmy, że $j:R\to End(G)$ jest homomorfizmem pierścieni z $1$. Wtedy $j$ wyznacza na $G$ strukturę $R$-modułu. Na odwrót: gdy $G(+r)_{r\in R}$ jest {\large\color{orange}ZMAZAŁ MI}
    \item Gdy $R_1\subseteq R$ jest podpierścieniem z $1$, to $R$ jest modułem nad $R_1$.
    \item Gdy $j:R_1\to R$ ejst homomorfizmem pierścieni z jednością i $M=(M,+,r)_{r\in R}$ jest $R$-modułem, to $M$ jest $R_1$-modułem z dzialaniem indukowanym przez $j$.
    \item $R$ jest pierścieniem z jednością i $I\subseteq R$ jest ideałem lewostronnym. Wtedy $I$ jest $R$-modułem.
\end{enumerate}

\begin{remark}
    Niech $M$ będzie $R$-modułem, wtedy
    \begin{enumerate}
    \item $0\cdot m=0\in M$
    \item $r\cdot 0=0$
    \item $(-1)m=-m$
    \end{enumerate}
\end{remark}

\begin{remark}
Przekrój dowolnej niepustej rodziny podmodułów $M$ jest podmodułem $M$.
\end{remark}

\textbf{Przykład:}

$\{0\}\subseteq M$ jest podmodułem zerowym.

\begin{conclusion}
    Niech $A\subseteq M$. Wtedy istnieje najmniejszy podmoduł (ze względu na zawieranie) $N\subseteq M$ taki, że $A\subseteq M$. Jest to \acc[b]{podmoduł generowany przez $A$}
    $$N=\{\sum r_ia_i\;:\;r_i\in R,\;a_i\in A\}\cup\{0\}$$
\end{conclusion}

\begin{enumerate}
    \item Jeśli $N_1,N_2\subseteq M$ są podmodułami, to $N_1+N_2$ też jest podmodułem. To samo, jeśli weźmiemy $n$ takich podmodułów.
    \item Prdukt $R$-modułów $M,N$, czyli $M\times N$, też jest $R$-modułem
    \item $M=N_1\oplus...\oplus N_n$ jest modułem dla $N_1,...,N_n$ podmodułów $M$
\end{enumerate}

\important{Homomorfizm modułów} $h:M\to N$ działa tak samo jak zwykle. Nazwy izo-, endo-, auto-, mono- nadal są applicable.

Niech $h:M\to N$ będzie homomorfizmem $R$-modułów. Dla $N'\subseteq N$ podmodułu $h^{-1}[N']$ jest podmodułem $M$. Dla $M'\subseteq M$ $h[M']\subseteq N$. 

$M/M'$ to \acc[b]{moduł ilorazowy}.

\begin{theorem}[zasadnicze twierdzenie o homommoorfizmie $R$-modułów] Zasadnicze twierdzenie o homomorfizmie $R$-modułów. Niech $M,N$ będą modułami i
$$M\xrightarrow[\forall\;f]N$$
Wtedy istnieje dokładnie jeden $\overline f$ taki, że

\begin{center}
\begin{tikzcd}
    M\arrow[r,"\forall\;f"] \arrow[d, "iloraz"]& N\\
    M/ker(F)\arrow[ur, dashed, "\exists!\;\ovelrine f" right]
\end{tikzcd}
\end{center}

\end{theorem}

\begin{theorem}$ $

\begin{center}\begin{tikzcd}
M\arrow[r, "f"]\arrow[d, "g"] & N\\
U\arrow[ur, dashed, "\exists\;h\iff ker(f)\supseteq ker(g)" right]
\end{tikzcd}
\end{center}

\end{theorem}

$h:M\to N$ jest $1-1$ $\iff ker(h)=\{0\}$

\subsection{Cel: zrozumieć moduły}

\begin{bbox}
Dany jest $R$-moduł $M$. Gdy $M=\bigoplus_i M_i$, gdzie $M_i\subseteq M$ jest małymi podmodułami o zrozumiałej strukturze, to struktura $M$ też jest zrozumieała.
\end{itemize}
\end{bbox}

\begin{definition}
Mówimy, że $R$-moduł $M$ jest \important{prosty}, gdy $M\neq 0$ i dla każdego $N\subseteq M$ podmodułu, $N=0$.

Pierścień enodmorfizmów $R$-modułu $M$
$$End_R(M)=\{\text{endomorfizmy }M\}$$
jest podpierścieniem $End(M, +)$.
\end{definition}

\begin{lemma}[lemat Schura] Lemat Schura: jeśl i$M$ jest $R$-modułem prostym, to $End_R(M)$ jest pierścieniem z dzieleniem (prawie ciało, poza tym, że nie musi być przemienny).
\end{lemma}
\begin{proof}
Niech $0\neq f\in End_R(M)$. Wtedy $Im(f)=M$, bo jest to niezerowy podmoduł $M$, a $M$ przecież było modułem prostym. Stąd właśnie $Im$ jest całością. $ker(f)=\{0\}$, czyli $f$ jest $1-1$ i "na".
\end{proof}

Załóżmy, że $M$ jest $R$-modułem oraz $K=End_R(M)$ jest pierścieniem z dzieleniem ("ciało nieprzemienne"). Uwaga! nie zakładamy prostości $M$ (ale możliwe że to wyniknie z $K$-pierścień z dzieleniem). Wtedy o $M$ możemy myśleć jako o $K$-module. Załóżmy, że $n=\dim_K(M)<\infty$. Wtedy $End_K(M)\cong M_{n\times n}(K)$.

Wybierzmy $r\in R$ i niech $\phi_r:M\to M$ takie, że $\phi_r(m)=r\cdot m$. Wtedy $\phi_r\in End_K(M)$ (? gdy $R$-przemienny ? - zadanie)

\begin{center}\begin{tikzcd}
r\arrow[r, mapsto] & m(\phi_r)\in M_{n\times n}(K)\\
R\arrow[u, phantom, sloped, "\in"]\arrow[r, "\text{homomorfizm pierścieni z }1"] & M_{n\times n}(K)
\end{tikzcd}\end{center}

Powyższe jest rozwinięte jako teoria reprezentacji pierścieni

\begin{bbox}
Niech $R$ będzie pierścieniem z $1\neq 0$ i $M$ będzie $R$-modułem.

\begin{itemize}%[leftmargin=*]
    \item[\PHtunny] Układ $\{m_i\}\subseteq M$ jest \important{liniowo niezależny}, gdy
    $$(\forall\;\{r_i\}\subseteq R)\;\sum r_im_i=0\implies(\forall\;i)\;r_i=0$$
    \item[\PHtunny] Liniowa zależność jest zaprzeczeniem
    \item[\PHtunny] $S\ubseteq M$ jest liniowo niezależny, gdy układ $\{m_i\}=S$ (bez powtórzeń)
    \item[\PHtunny] $B\subseteq M$ jest bazą, gdy:
    \begin{itemize}
        \item jest liniowo niezależny
        \item generuje $M$ jako $R$-moduł
        \item $Lin_R(B)=M$.
    \end{itemize}
\end{itemize}

\end{bbox}

\textbf{Przykład:}

\begin{enumerate}
    \item $\{0\}\subseteq M$ jest liniowo niezależny, natomiast układ $(m_0,m_0)$ jest liniowo zależny, bo $1\cdot m_0+(-1)\cdot m_0=0$.
    \item $\Q$ jako $\Z$-moduł $(a, b)$ jest liniowo zależny dla wszystkich $a,b\in\Q$.

    Bez straty ogólności $a,b\neq 0$ i $a\neq b$. $a=\frac{m}{n},b=\frac{p}{q}$, czyli
    $$(np)\cdot a-(qm)\cdot b=pm-mp=0$$

    W takim razie, $\Q$ nie ma bazy jako $\Z$-moduł.
\end{enumerate}

\begin{bbox}
\important{(Abstrakcyjna) suma prosta} rodziny modułów (koprodukt) to
$$\bigsqcup M_i=\obigcup M-i=\{f\in\bigsqcap M_i\;:\;\{i\in I\;:\;f(i)\neq 0\}\text{ jest skończony}\}$$
\end{bbox}

\begin{remark}
Jeśli dla każdego $i\in I$ istnieje $M_i\to M$ to istnieje dokładnie 1 $h:\bigcsqcup M_i\to M$ taki, że dla każdego $i_0$

\begin{center}\begin{tikzcd}
M_{i_0}\arrow[r, "g_{i_0}"]\arrow[d, "f_{i_0}"] & M\\
\bigsqcup M_i\arrow[ur, "h", dashed]
\end{tikzcd}\end{center}

Jest to nazywane własnością uniwersalności.
\end{remark}

\begin{remark}
$M=M_1\oplus M_2$ dla podmodułów $M_1,M_2\subseteq M$. Wtedy dla 
$$g_i=id_{M_i}:M_i\to M,$$ 
dla $h$ z uwagi wcześniej
$$h:M_1\sqcup M_2\to M$$
to jest izomorfizm modułów.
\end{remark}
\begin{proof}
Ćwiczenia
\end{proof}

\begin{definition}
$M$ jest wolnym $R$-modułem, gdy $M$ ma bazę.
\end{definition}

\textbf{Przykłady}:

\begin{enumerate}
    \item $R$ jest wolnym $R$-modułem z bazą $\{1\}$. 
    \item $\Q$ nie jest wolnym $\Z$-modułem
    \item $\{M_i\}$ są rodziną wolnych $R$-modułów, wtedy $\bigsqcup M_i$ jest wolnym $R$-modułem.

    Niech $B_i\subseteq M_i$ będą bazami. Wtedy 
    $$f_{i_0}:M_{i_0}\isomorphism f_{i_0}[M_{i_0}]\subseteq\bigsqcup M_i$$
    $$\bigcup f_i[B_i]$$
    jest bazą $\bisqcup M_i$.
\end{enumerate}
