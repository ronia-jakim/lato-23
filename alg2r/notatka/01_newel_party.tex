\section{Teoria równań algebraicznych}

Przez $R, S$ będziemy oznaczać pierścienie przemienne z $1\neq0$, natomiast $K, L$ będziemy rezerwować dla oznaczeń ciał.

\subsection{Rozwiązywanie układów równań}

Rozważmy funkcje $f_1,...,f_m\in R[X_1, ..., X_n]$. Dla wygody będziemy oznaczać krotki przez $\overline X$, czyli $R[X_1,...,X_n]=R[\overline X]$. Pojawia się problem: \emph{czy istnieje rozszerzenie pierścieni z jednością $R\subseteq S$ takie, że układ $U:f_1(\overline X)=...=f_m(\overline X)=0$ ma rozwiązanie w pierścieniu $S$?}

\begin{fact}
    $\overline a=(a_1,...,a_n)\subseteq S$, gdzie $S$ jest rozszerzeniem pierścienia $R$, jest \acc{rozwiązaniem układu równań} $U\iff g(\overline a)=0$ dla każdego wielomianu $\color{blue}g\in (f_1,...,f_m)\triangleleft R[X]$.
\end{fact}

\begin{proof}

$\impliedby$ Implikacja jest dość trywialna, jeśli każdy wielomian z $(f_1,...,f_m)$, czyli wytworzony za pomocą sumy i produktu wielomianów $f_1,...,f_m$ zeruje się na $\overline a$, to musi zerować się też na każdym z tych wielomianów.

$\implies$ Rozważamy dwa przypadki:

\begin{enumerate}
    \item $(f_1,...,f_m)\ni b\neq 0$ i $b\in R$. 

    To znaczy w $(f_1,...,f_m)$ mamy pewien niezerowy wyraz wolny. Wtedy mamy wielomian $g\in(f_1,...,f_m)$ taki, że $g(\overline a)\neq 0$. Ale przecież $g$ jest kombinacką wielomianów $f_1,...,f_m$, która na $\overline a$ przyjmują wartość $0$. W takim razie dostajemy układ sprzeczny i przypadek jest do odrzucenia.

    \item 2. $(f_1,...,f_m)\cap R=\{0\}$. (nie ma wyrazów wolnych różnych od $0$)

    Teraz wiemy, że układ $U$ jest niesprzeczny, a więc możemy skonstruować pierścień z $1$ $S$ będący rozszerzeniem $R$ [$S\supseteq R$] oraz rozwiązanie $\overline a\subseteq S$ spełniające nasz układ równań.

    Niech $S=R[\overline X]/(f_1,...,f_m)$ i rozważmy
    $$j:R[\overline X]\to S=R[\overline{X}]/(f_1,...,f_m)$$
    nazywane \acc{przekształceniem ilorazowym}. Po pierwsze, zauważmy, że  $j\restriction R$ jest $1-1$, bo
    $$\ker(j\restriction R)=\ker(j)\cap R=(f_1,...,f_m)\cap R=\{0\}$$
    i dlatego
    $$j\restriction R:R\isomorphism j[R]\subseteq S.$$
    Z uwagi na ten izomorfizm, będziemy utożsamiać $R, j[R]$. W takim razie, $S$ jest rozszerzeniem pierścienia $R$. Czyli mamy rozszerzenie pierścienia $R$.

    Niech 
    $$\overline a=(a_1,...,a_m)=(j(X_1),...,j(X_n))\subseteq S,$$
    czyli jako potencjalne rozwiązanie rozważamy zbiór obrazów wielomianów stopnia $1$ przez wcześniej zdefiniowaną funkcję $j:R[\overline X]\to S$. Tak zdefiniowane $\overline a$ jest rozwiązaniem układu $U$ w pierścieniu $S$, bo dla funkcji wielomianowej (czyli zapisywalnej jako wielomian) $\hat{f_i}\in(f_1,...,f_m)$ mamy
$$\hat{f_i}(\overline a)=\hat{f_i}(j(X_1),...,j(X_m))=j(\hat{f_i}(X_1,...,X_m))=j(f_i)=0.$$
\end{enumerate}

{\color{orange}\large TUTAJ TRZEBA POUZASADNIAĆ KILKA RÓWNOŚCI, ALE MOŻE NIE BĘDĘ TEGO ROBIŁA NA AISD}
\end{proof}
\begin{remark}
    \label{uwaga1:1:2-warunek-rozwiazanie-ogolne}
    Skonstruowane powyżej {rozwiązanie $\overline a$} układu $U$ ma następującą własność {uniwersalności}:

    ({\scriptsize\color{yellow}\PHcat}) Jeżeli $S'\supseteq R$ jest rozszerzeniem pierścienia z $1$ i $\overline a'=(a_1',...,a_m')\subseteq S$ jest rozwiązaniem $U$ w $S'$, to istnieje jedyny homomorfizm 
    $$h:R[\overline a]\to R[\overline a']$$ 
    taki, że $h\restriction R$ jest identycznością na $R$ i $h(\overline a)=\overline a'$. \acc{Wszystkie rozwiązania układów są homomorficzne.}
\end{remark}

\begin{illustration}
    \node (R) at (0, 0) {$R$};
    \node (R[a]) at (4, 0) {$R[\overline a]\subseteq S$};
    \node (R[a']) at (0, -3) {$R[\overline a']\subseteq S'$};
    \draw [ ->] (R)--(R[a]) node [midway, above] {$\subseteq$};
    \draw [->] (R)--(R[a']) node [midway, left] {$\subseteq$};
    \draw[->] (R[a])--(R[a']) node [midway, right] {$h$};
\end{illustration}

Tutaj \important{$R[\overline a]\subseteq S$ jest podpierścieniem generowanym przez $R\cup\{\overline a\}$}, czyli zbiór:
$$R[\overline a]=\{f(\overline a)\;:\;f(\overline X)\in R[\overline X]\}\subseteq S$$

\begin{proof}
Niech $I=\{g\in R[\overline X]\;:\;g(\overline a')=0\}\subseteq S'$. Oczywiście mamy, że $I\triangleleft R[\overline X]$, a więc
$$(f_1,...,f_m)\subseteq I.$$
Z twierdzenia o faktoryzacji wie
\begin{illustration}
    \node (R[x]) at (0, 0) {$R[\overline X]$};
    \node (S) at (4, 0) {$S=R[\overline X]/(f_1,...,f_m)$};
    \node (R[a']) at (0, -3) {$S'\supseteq R[\overline a']$};
    \draw[->] (R[x])--(S) node [midway, above] {$j$};
    \draw[->] (R[x])--(R[a']) node [midway, left] {$\phi$};
    \draw[->, dashed] (S)--(R[a']) node [midway, right] {$(\exists\;!h)\;h(\overline a)=\overline a'$};
\end{illustration}
Homomorfizm $\phi:R[\overline X]\to R[\overline a']$ określamy wzorem
$$\phi(w)=w(\overline a),$$
a homomorfizm $j$ jest jak wyżej odwzorowaniem ilorazowym. Widzimy, że 
$$I=\ker(\phi)$$
$$\ker(j)=(f_1,...,f_m).$$
Z twierdzenia o homomorfizmie pierścieni dostajemy jedyny homomorfizm 
$$h:R[X]/(f_1,...,f_m)\to R[\overline a]$$
taki, że $h(\overline a)=\overline a'$.
\end{proof}

\begin{remark}
    Jeśli $I=(f_1,...,f_m)$, to $h:R[\overline a]\isomorphism R[\overline a']$.
\end{remark}

Wtedy mamy $\ker\phi=\ker j$, czyli $\ker(h\circ j)=\ker\phi=\ker j$, no a z tego wynika, że $\ker h$ jest trywialne, czyli $h$ jest apimorfizmem (1-1). Z drugiej strony, $Im \phi=Im(h\circ j)$, a $\phi$ jest epimorfizmem ("na"), więc również $h$ musi być "na".
\medskip

\begin{bbox}
Załóżmy, że $S\supseteq R$ jest rozszerzeniem pierścienia oraz $\overline a\in S^n$. Wtedy:
\begin{enumerate}
\item ideał $\overline a$ nad $R$ definiujemy jako 
$$\color{blue}I(\overline a/R)=\{g\in R[\overline X]\;:\;g(\overline a)=0\}$$
\item $\overline a$ nazywamy \important{rozwiązaniem ogólnym} układu $U$, jeśli ideał 
$$I(\overline a/R)=(f_1,...,f_m).$$
\end{enumerate}
\end{bbox}

\begin{remark}
    \label{uwaga:1.4}
    W sytuacji jak z definicji wyżej, gdy $U$ jest układem niesprzecznym, wtedy 
    
    $\overline a$ jest rozwiązaniem ogólnym układu $U$ $\iff$ zachodzi warunek {\hyperref[uwaga1:1:2-warunek-rozwiazanie-ogolne]{ (\scriptsize\color{yellow}\PHcat }) } .
\end{remark}

\begin{proof}Ćwiczenia.\end{proof}

\subsection{Rozszerzanie ciał}

Dla $K\subseteq L$ ciał i $\overline a\subseteq L$ definiujemy \important{ideał $\overline a$ nad $K$} jako:
$$\color{blue}I(\overline a /L):=\{f(X_1,...,X_n)\in K[\overline X]:f(\overline a)=0\},$$
to znaczy generujemy ideał w wielomianach nad $K$ zawierający wszystkie wielomiany (niekoniecznie tylko jednej zmiennej) zerujące się w $\overline a$. 
\medskip

\textbf{Przykład:}
\smallskip

Dla $K=\Q,L=\R,n=1,a_1=\sqrt{2}$ mamy
$$I(\sqrt{2}/\Q)=\{f(x^2-2)\;:\;f\in\Q[X]\}=(x^2-2)\triangleleft \Q[X]$$

Dalej, definiujemy
$$\color{blue}K[\overline a]:=\{f(\overline a)\;:\;f\in K[X]\}$$
czyli \important{podpierścień $L$ generowany przez $K\cup\{\overline a\}$} oraz \important{$K(\overline a)$, czyli podciało $L$} generowane przez $K\cup\{\overline a\}$:
$$\color{blue}K(\overline a):=\{f(\overline a)\;:\;f\in K(X_1,...,X_n)\;i\;f(\overline a)\text{ dobrze określone}\}.$$
Tutaj $K(X_1,...,X_n)$ to \acc{ciało ułamków pierścienia} $K[\overline a]$ w ciele $L$ (czyli najmniejsze ciało, że pierścień może być w nim zanurzony). Czasami oznaczamy to przez $K[\overline a]_0$.
\medskip

% \textbf{Przykład:}
% \smallskip

% Dla $K=\Q,L=\R$ zachodzi:
% $$K[\sqrt2]=\Q[\sqrt2]=\{q+p\sqrt2\;:\;q,p\in\Q\}$$
% $$K[\sqrt2,\sqrt3]=\Q[\sqrt2,\sqrt3]$$
% $$K(\sqrt2)=\Q[\sqrt2]$$
% to ostatnie to usuwanie niewymierności z mianownika.
% \medskip

\begin{remark}
    \label{uwga:1:1:5}
    Niech $K\subseteq L_1,K\subseteq L_2$ będą ciałami. Wybieramy $\overline a_1\in L_1$ i $\overline a_2\in L_2$, $|\overline a_1|=|\overline a_2|=n$. Wtedy następujące warunki są równoważne:

\indent 1. istnieje izomorfizm $\phi:K[\overline a_1]\to K[\overline a_2]$ taki, że $\phi\restriction K=id_K$ oraz $\phi(\overline a_1)=\overline a_2$.

\indent 2. $I(\overline a_1/K)=I(\overline a_2/K)$.
\end{remark}

\begin{proof} $ $\newline

%\begin{center}
% $K[\overline a]\isomorphism K[\overline b]$ \acc{$\implies$} $I(\overline a/K)=I(\overline b/K)$

% Niech $\omega\in K[\overline X]$. Wtedy $\omega\in I(\overline a/K)$ wtedy i tylko wtedy, gdy $\omega(\overline a)=0$, to mamy z definicji $I(\overline a/K)$. Wiemy też, że $\phi(a)\in K[\overline X]$ wtedy, gdy $\omega(\phi(\overline a))=0$, a ponieważ $\phi(\overline a)=\overline b$, to również $\omega(\overline b)=0$ i mamy, że $\omega\in I(\overline b/K)$. Czyli izomorfizm między $K[\overline a]=K[\overline b]$ implikuje, że $I(\overline a/K)=I(\overline b/K)$.
% \smallskip

% $K[\overline a]\isomorphism K[\overline b]$ \acc{$\impliedby$} $I(\overline a/K)=I(\overline b/K)$

% Spróbujmy zdefiniować izomorfizm $\phi$ tak, że dla $\omega\in K[\overline X]$ mamy $\phi(\omega(\overline a))=\omega(\overline b)$

% 1. $\phi$ jest homomorfizmem: 
% $$\phi(\omega(\overline a)\cdot v(\overline a))=f((\omega\cdot v)(\overline a))=(\omega\cdot v)(\overline b)=\omega(\overline b)\cdot v(\overline b)=\phi(\omega(\overline a))\cdot \phi(v(\overline a))$$

% 2. $\phi$ jest różnowartościowe:
% $$\phi(\omega(\overline a))=\phi(v(\overline a))\iff \omega(\overline b)=v(\overline b)\iff (\omega-v)(\overline b)=0\iff \omega-v\in I(\overline b/K)=I(\overline a/K)\iff (\omega-v)(\overline a)=0\iff \omega(\overline a)=v(\overline a)$$

% 3. $\phi$ jest dobrze zdefiniowane (czyli przyjmuje tylko jedną wartość dla jednego argumentu):
% $$\omega(\overline a)-v(\overline a)=0\iff (\omega-v)(\overline a)=0\iff \omega -v\in I(\overline a/K)\iff \omega-v\in I(\overline b/K)\iff (\omega-v)(\overline b)=0\iff \omega(\overline b)-v(\overline b)=0$$

% \podz{dark-green}
% \bigskip

% Możemy teraz zapytać, czy każdy ideał w pierścieniu wielomianów $K[X]$ jest postaci $I(\overline a/K)$ dla pewnego $\overline a\in L\supset K$? Albo ogólniej, czy dla pierścienia przemiennego $R$ z $1_R\neq0_R$ oraz ideału $I=(f_1,...,f_m)=I(\overline a/R)\normalsubgroup R[X]$, czy istnieje nadpierścień $S$ taki, że $1_S=1_R$ i $0_S=0_R$ oraz układ
% $$f_1(\overline x)=...=f_m(\overline m)=0$$
% ma rozwiązanie w $S$? Takie rozwiązanie spełniałoby $\overline a\in S\iff (\forall\;g\in(f_1,...,f_m))\;g(\overline a)=0$.
%\end{center}

$1\implies 2$

Implikacja jest jasna, bo dla $g(\overline X)\in K[\overline X]$ mamy $g(\overline a_1)=0$ w $K[\overline a_1]$ $\iff$ $g(\phi(\overline a_1))=0$, a $\phi(\overline a_1)=\overline a_2$, czyli $g(\overline{a}_2)=0$. Stąd $g\in I(\overline{a}_1/K)\iff g\in I(\overline{a}_2/K)$.

$1\impliedby 2$

Zwróćmy uwagę na odwzorowanie ewaluacji $\overline a_1$
$$\phi_{\overline a_1}:K[\overline X]\xrightarrow[]{"na"}K[a_1]$$
zadane wzorem
$$\phi(w(\overline X))=w(\overline a_1).$$
Mamy
$$\ker(\phi_{\overline a_1})=I(\overline a_1/K).$$

Tak samo dla $\overline a_2$ możemy określić analogicznie odwzorowanie ewaluacyjne $\phi_{\overline a_2}:K[\overline X]\to K[\overline a_2]$. Wtedy
$$I(\overline a_2/K)=\ker(\phi_{\overline a_2}),$$
ale ponieważ $I(\overline a_1/K)=I(\overline a_2/K)$, to $\ker(\phi_{\overline a_1})=\ker(\phi_{\overline a_2})$. Oznaczmy $I=I(\overline a_1/K)=I(\overline a_2/K)$. Widzimy, że $\phi_{\overline a_i}\restriction K=id_k$ (wielomiany mające tylko wyraz stały nie zmieniają wartości po podstawieniu $x$).

Z twierdzenia o izomorfizmie wiemy, że istnieją izomorfizmy
$$f_i:K[\overline{X}]/ker(\phi_{\overline{a}_i})= K[\overline{X}]/I(\overline{a}_i/K)=K[\overline{X}]/I\;\isomorphism\; Im(\phi_{\overline{a}_i})=K[\overline{a}_i]$$

\begin{illustration}
    \node (K) at (4, 0) {$K[\overline X]$};
    \node (K[a1]) at (0, -2) {$K[\overline a_1]$};
    \node (K[a2]) at (8, -2) {$K[\overline a_2]$};
    \node (K[X]/I) at (4, -2.5) {$K[X]/I$};
    \draw[->] (K)--(K[a1]) node [midway, left] {$\phi_{\overline a_1}$};
    \draw[->] (K)--(K[a2]) node [midway, right] {$\phi_{\overline a_2}$};
    \draw[->] (K)--(K[X]/I) node [midway, right] {$j${\scriptsize - ilorazowe}};
    \draw[->, dashed] (K[X]/I)--(K[a1]) node [midway, below] {$f_1$} node [midway, above] {$\cong$};
    \draw[->, dashed] (K[X]/I)--(K[a2]) node [midway, below] {$f_2$} node [midway, above] {$\cong$};
\end{illustration}

Niech $f=f_2f_1^{-1}:K[\overline a_1]\to K[\overline a_2]$. Jako złożenie dwóch izomorfizmów $f$ również jest izomorfizmem. Pozostaje sprawdzić, czy $f(\overline{a}_1)=\overline{a}_2$.

$f(\overline{a}_1)=f_2(f_1^{-1}(\overline{a}_1))$ i zauważmy, że $f_1^{-1}(\overline{a}_1)=w(\overline{X})\in K[\overline{X}]/I$, gdzie $w(\overline{X})=\overline{X}$.
Idąc po kolei wynika to z tego, że $f_1\circ j=\phi_{\overline{a}_1}$. 

Gdy włożymy w lewą stronę $w(\overline{X})=\overline{X}$ dostajemy $f_1\circ j(w)=f_1(\overline{X})$ (gdy oczywiście $\overline{a}_i\neq 0$), a z kolei po włożeniu tego do prawej strony wychodzi $\phi_{\overline{a}_1}(w)=w(\overline{a}_1)=\overline{a}_1$ i mamy, że $f_1(w)=\overline{a}_1\implies f_1^{-1}(\overline{a}_1)=w$.
$$f(\overline{a}_1)=f_2(f_1^{-1}(\overline{a}_1))=f_2(w)=w(\overline{a}_2)=\overline{a}_2$$
\end{proof}

%{\large\color{orange}MOŻE TUTAJ ŁADNIE SPRAWDZIĆ ŻE NAPRAWDĘ JEST TO DOBRZE SPEŁNIAJĄCA WARUNKI FUNKCJA?}

\textbf{\large\color{blue}Uwaga.}
    \emph{Niech $I\triangleleft K[\overline X]$ \acc{noetherowskiego} pierścienia $K[\overline X]$. Niech $I=(f_1,...,f_m)$ dla pewnych $f_i\in K[\overline X]$. Wtedy istnieje rozszerzenie pierścienia $S\supseteq K$ oraz $\overline a\subseteq S$ - rozwiązanie ogólne układu $f_1(\overline X)=...=f_m(\overline X)=0$ takie, że $\color{blue}I(\overline a/K)=I$.}

\begin{proof}
Uwaga \ref{uwaga:1.4}.
\end{proof}

\begin{theorem}
    Niech $I\triangleleft K[\overline X]$. Wtedy istnieje ciało $L\supseteq K$ oraz $\overline a=(a_1,...,a_n)\subseteq L$ takie, że $f(\overline a)=0$ dla każdego $f\in I$.
\end{theorem}

\begin{proof} 
Niech $I\subseteq M\triangleleft K[\overline X]$ będzie ideałem maksymalnym. Niech $L=K[\overline X]/M$ i określmy przekształcenie ilorazowe
$$j:K[\overline X]/M\to L=K[\overline X]/M.$$
Ponieważ $M\cap K=\{0\}$ (bo inaczej w ideale byłby wielomian odwracalny), to $j\restriction K:K\to L$ jest funkcją $1-1$, czyli
$$j\restriction K:K\xrightarrow[]{1-1}j[K]\subseteq L.$$
Możemy utożsamić $K$ z $j[K]$, czyli $K\subseteq L$. Niech $\overline a=(a_1,..., a_n)$ takie, że dla każdego $i\in[n]$ 
$$a_i=j(X_i)\in L.$$
Wtedy $g(\overline a)=0$ dla każdego $g(\overline X)\in M\supseteq I$ (bo inaczej mielibyśmy wyrazy wolne).
\end{proof}

\begin{conclusion}
    \label{wniosek1:2:4}
    Niech $f\in K[X]$ stopnia $>0$. Wtedy istnieje ciało $L\supseteq K$ rozszerzające ciało $K$ takie, że $f$ ma pierwiastek w ciele $L$.
\end{conclusion}

\textbf{Przykłady:}

\begin{enumerate}
\item 1. Rozpatrzmy ciało $K=\Q$ i $f(X)=X-2$. Wtedy $I=(f)\triangleleft\Q[X]$ jest ideałem maksymalnym, bo jest on pierwszy (w tym wypadku nierozkładalny). Równanie $f=0$ ma rozwiązanie ogólne w pierścieniu ilorazowym
$$\Q[X]/I\cong \Q.$$
Czyli nie zawsze musimy rozszerzać ciało do czegoś nowego.

\item 2. $\C=\R[i]=\R(i)=\R[z]$ dla każdego $z\in\C\setminus\R$, co jest na liście zadań.
\end{enumerate}

Załóżmy, że $K\subseteq L_1, K\subseteq L_2$ są rozszerzeniami ciała. Wtedy mówimy, że \important{$L_1$ jest izomorficzne z $L_2$ nad $K$} [\acc{$L_1\cong_KL_2$}] $\iff$ istnieje izomorfizm $f:L_1\to L_2$ taki, że $f\restriction K=id_K$.

\begin{fact}{\color{pagColor}dupa }
\label{fakt:1:2:5}
\begin{enumerate}
\item Załóżmy, że $f(X)\in K[X]$ jest nierozkładalny. Niech $L_1=K(a_1)$, $L_2=K(a_2)$ i $f(a_i)=0$ w $L_i$. Wtedy $L_1\cong_KL_2$.

\item Ogółniej: załóżmy, że $\phi:K_1\to K_2$ jest izomorfizmem i $f_1\in K_1[X],f_2\in K_2[X]$, $\phi(f_1)=f_2$, $f_i$ jest nierozkładalne. Dodatkowo załóżmy, że $L_1=K_1(a_1)$ i $L_2=K_2(a_2)$, gdzie $f_i(a_i)=0$ w $L_i$. Wtedy istnieje \acc{izomorfizm $\phi\in\psi:L_1\to L_2$ taki, że $\psi(a_1)=a_2$.}
\end{enumerate}
\end{fact}

\begin{proof}{\color{pagColor}dupa}

\begin{enumerate}
\item 1. $I(a_1/K)=(f)=I(a_2/K)$, stąd na mocy \ref{uwga:1:1:5} mamy $K(a_1)\cong_KK(a_2)$. Po dowodzie przypadku 2. możemy uzasadniać, że jest to szczególny przypadek tego ogólniejszego stwierdzenia właśnie.

\item 2. Zacznijmy od rozrysowania tej sytuacji:

\begin{illustration}
    \draw (-2, 0) ellipse (1.5 and 0.7) node [below] {$K_1$};
    \draw (2, 0) ellipse (1.5 and 0.7) node [below] {$K_2$};
    \draw (-3.5, 0)..controls (-3, 3) and (-1, 3)..(-0.5, 0) node [midway, above] {$L_1$};
    \draw (3.5, 0)..controls (3, 3) and (1, 3)..(0.5, 0) node [midway, above] {$L_2$};
    \draw[->] (-2, 1)..controls (-1, 2) and (1, 2)..(2, 1) node [midway, below] {$\psi$} node [midway, above] {$\cong$};
    \draw[->] (-1.2, -0.6)..controls (-0.8, -1) and (0.8, -1)..(1.2, -0.6) node [midway, below] {$\phi$} node [midway, above] {$\cong$};
\end{illustration}

Izomorfizm $\phi:K_1[X]\isomorphism K_2[X]$ indukuje nam przekształcenie
$$K_1[X]/(f_1)\isomorphism{\phi} K_2[X]/(f_2),$$
bo $\phi(f_1)=f_2$. Wiemy, że $f_i$ jest nierozkładalne, czyli
$$I(a_i/K_i)=(f_i)\triangleleft K_i[X]$$
jest ideałem maksymalnym. Mamy
$$L_i=K_i(a_i)=K_i[a_i]\cong K[X]/I(a_i/K_i).$$

\begin{illustration}
    \node (K1) at (0, 0) {$K_1[X]$};
    \node (K2) at (4, 0) {$K_2[X]$};
    \draw[->] (K1)--(K2) node [midway, above] {$\cong$} node [midway, below] {$\phi$};
    \node (K1f) at (0, -2) {$K_1[X]/(f_1)$};
    \node (K2f) at (4, -2) {$K_2[X]/(f_2)$};
    \draw[->] (K1f)--(K2f) node [midway, below] {$\phi$} node [midway, above] {$\cong$};
    \draw[->] (2, -.7)--(2, -1.3);
    \node (L1) at (0, -4) {$L_1=K_1(a_1)$};
    \node (L2) at (4, -4) {$L_2=K_2(a_2)$};
    \draw[->] (L1)--(L2) node [midway, above] {$\cong$} node [midway, below] {$\psi$};
    \node (K1) at (0, -5) {$K_1$};
    \node (K2) at (4, -5) {$K_2$};
    \draw[->] (K1)--(K2) node [midway, below] {$\phi$};
    \node[rotate=90] at (0, -4.5) {$\subseteq$};
    \node[rotate=90] at (4, -4.5) {$\subseteq$};
    \draw[->] (K1f)--(L1) node [midway, left] {$\cong$} node [midway, right] {$h_1$};
    \draw[->] (K2f)--(L2) node [midway, left] {$\cong$} node [midway, right] {$h_2$};
\end{illustration}
\end{enumerate}
\end{proof}
