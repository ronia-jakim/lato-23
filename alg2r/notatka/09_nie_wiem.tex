\section{DUPAA}
\subsection{Rozszerzenia abelowe}
\setcounter{theorem}{2}

\begin{bbox}
Załóżmy, że $K\subseteq L$ jest skończonym rozszerzeniem Galois. Wtedy rozszerzenie $K\subseteq L$ jest abelowe (cykliczne) gdy $Gal(L/K)$ jest abelowe (cykliczne).
\end{bbox}

\begin{theorem}
    Założmy, że $K\subseteq L_1\subseteq L$ to rozszerzenia ciał. Jeśli $K\subseteq L$ jest abelowe (cykliczne), to $K\subseteq L_1$ i $L_1\subseteq L$ też takie są.
\end{theorem}
\begin{proof}
Z tego, że $Gal(L/L_1)\triangleleft Gal(L/K)$ wynika, że $K\subseteq L_1\;i\;L_1\subseteq L$ jest rozszerzeniem Galois i $Gal(L_1/K)\cong Gal(L/K)/Gal(L/L_1)$. Dlatego mamy $Gal(L/L_1)$ i $Gal(L_1/K)$ są abelowe (cykliczne).
\end{proof}

\textbf{Przykłady:}

\begin{enumerate}[leftmargin=*]
    \item Niech $K\subseteq\hat{K}$ i $\zeta\in\hat{K}$ będzie pierwiastkiem pierwotnym stopnia $n$ z $1$.
    
    \begin{center}\begin{tikzcd}
    Gal(K(\zeta)/K)\arrow[r, hook] & \Z_n^*\\
    f\arrow[r, mapsto]\arrow[u, phantom, sloped, "\in"] & l_f\arrow[u,sloped,phantom,"\in"]
    \end{tikzcd}\end{center}
    $l_f$ wybieramy tak, żeby $f(\zeta)=\zeta^{l_f}\;0<l_f<n$. Gdy $char(K)=0$, to homomorfizm wyżej jest izomofrizmem, wpp nie musi być to prawdą. Natomiast mamy pewność, że $K(\zeta)\supseteq K$ jest rozszerzeniem abelowym.
    \item Niech $char(K)=p$ i $p\nmid n$. Wybierzmy $a\in K$ takie, że $\sqrt[n]{a}\notin K$. Załóżmy, że $\zeta\in K$ jest pierwiastkiem pierwotnym z $1$ stopnia $n$.

    W takim przypadku, $L=K(\sqrt[n]{a})\supseteq K$ jest rozszerzeniem Galois i niech $w(x)=x^n-a$ (niekoniecznie nierozkładalny). Pierwiastki $w(a)$ w $L$ mają postać $\zeta^i\sqrt[n]{a}$ dla $i=0,...,n-1$.

    Niech $f\in Gal(L/K)$ będzie wyznaczony przez $f(\sqrt[n]{a})=\zeta^{l_f}\sqrt[n]{a}$ dla $0\leq l_f<n$. Wtedy funkcja jak wyżej, tzn.
    $$Fal(L/K)\ni f\mapsto l_f\in \Z_n^*$$
    jest monomorfizmem, ponieważ
    $$Gal(L/K)\ni f\mapsto l_f$$
    $$Fal(L/K)\ni g\mapsto l_g$$
    $$(g\circ f)(\sqrt[n]{a})=g(\zeta^{l_f}\sqrt[n]{a})=\zeta^{l_f}g(\sqrt[n]{a})=\zeta^{l_f}\zeta^{l_g}\sqrt[n]{a}=\zeta^{l_f+l_g}\sqrt[n]{a},$$
    więc $l_{g\circ f}=l_g+_n l_f$. Z tego powodu, $Gal(L/K)$ jest grupą cykliczną.
\end{enumerate}

\begin{theorem}\label{tw:9.4}
Załóżmy, że $K\subseteq L$ jest rozszerzeniem cykliczny takim, że $[L:K]=n$. Niech $\zeta\in K$ będzie pierwiastkiem pierwotnym z $1$ stopnia $n$ (czyli $p\nmid n$ gdy $char(K)=p$). Wtedy $(\exists\;a\in K)\;L=K(\sqrt[n]{a})$.
\end{theorem}
\begin{proof}
Niech $\gamma\in Gal(L/K)$ będzie generatorem rozszerzenia $L$ rzędu $n$. Dla $b\in L$ niech 
$$c(b)=b+\zeta\gamma(b)+...+\zeta^{n-1}\gamma^{n-1}(b)$$
$$\gamma(c(b))=\gamma(b)+\zeta\gamma^2(b)+...+\zeta^{n-1}\underbrace{\gamma^n(b)}_{=b}=\zeta^{-1}c(b)$$
$$\gamma^i(c(b))=\zeta^{-a}c(b),\;i=0,1,2,...$$
Jeżeli $c(b)\neq 0$ [założenie ad hoc], to
$$\{\gamma^0(c(b)),\gamma(c(b)), ...,\gamma^{n-1}(c(b)\}$$
jest $n$-elementowym zbiorem pierwiastków wielomianu $w_{c(b)}(x)\in K[X]$, czyli
$$[K(c(b):K]\geq n\implies K(c(b))=L,$$
bo $K(c(b))\subseteq L$. 

Mamy $c(b)^n\in K$, bo 
$$\gamma^i(c(b)^n)=\left[\gamma^i(c(b))\right]^n=[\zeta^{-i}c(b)]^n=\zeta^{-in}c(b)^n=c(b)^n$$
dla wszystkich $i=0,1,...,n-1$. Dlatego $c(b)=\sqrt[n]{a}$ dla $a=c(b)^n\in K$ i $L=K(\sqrt[n]{a})$.

Wszystko to zachodzi pod warunkiem, że $c(b)\neq 0$, ale wiemy, że istnieje $b\in L$ takie, że $c(b)\neq 0$, bo:
\end{proof}

\begin{theorem}[twierdzenie Dedekinda o liniowej niezależności charakterów]\label{tw:9.5}
Załóżmy, że $\alpha_1,...,\alpha_n\in Aut(L)$, $a_1,...,a_n\in L$ i każdy jest $\neq0$. Wtedy 
$$(\exists\;c\in L)\;(\sum a_i\alpha_i)(c)\neq 0$$
Innymi słowy: $\alpha_1,...,\alpha_n$ są liniowo niezależne w przestrzeni $L^L$ nad $L$.
\end{theorem}
\begin{proof}
Indukcja względem $n$. Dla $n=1$ jest to oczywiste. $c=1:a_1\alpha_1(1)=a_1\neq 0$.

Krok indukcyjny:

Załóżmy nie prosty, że $(\forall\;x\in L)\;\sum^{n+1}a_i\alpha_i(x)=0$. Niech $a\in L$ dowolne różne od zera. Wtedy
\begin{align*}
(\forall\;x\in L)\;&\sum^{n+1}a_i\alpha_i(ax)=0\\
&\sum^{n+1}(a_i\alpha_i(a))\alpha_i(x)=0\\
&\sum^{n+1}a_i\alpha_i(a)[\alpha_{n+1}(a)]^{-1}\alpha_i(x)=0\\
&\underbrace{\sum^{n+1}a_i\alpha_i(x)}_{=0}-\sum^{n+1}a_i\alpha_i(a)\alpha_{n+1}(a)^{-1}\alpha_i(x)=0\\
&\sum^{n+1}\underbrace{\left[a_i-a_i\alpha_i(a)\alpha_{n+1}(a)^{-1}\right]}_{=0\text{ gdy }i=n+1}\cdot\alpha_i(x)=0\\
&\sum^n\left[a_i-a_i\alpha_i(a)\alpha_{n+1}(a)^{-1}\right]\alpha_i(x)=0\\
(1-\alpha_{n+1}(a)^{-1})\sum^n a_i\alpha_i(a)=0
\end{align*}
Z założenia indukcyjnego wiemy, że cała ta suma nie jest zerem, więc zerem musi być $1-\alpha_{n+1}$, czyli każdy poziom sumy po wymnożeniu jest zerem i:
$$a_i-a_i\alpha_i(a)\alpha_{n+1}(a)^{-1}=0,$$
czyli $\alpha_i(a)=\alpha_{n+1}(a)$ gdy $a_i\neq 0$. Z tego wynika, że dla każdego $a\in L$ jest $\alpha_i(a)=\alpha_{n+1}(a)$ i w takim razie $\alpha_i=\alpha_n$, co daje sprzeczność, bo $\alpha_i$ były parami różne.
\end{proof}

\subsection{Rozwiązywalne rozszerzenia ciał i rozszerzenia przez pierwiastki}

\begin{bbox}
Załóżmy, że $K\subseteq L$ jest skończonym rozszerzeniem ciał.
\begin{enumerate}[leftmargin=*]
    \item $K\subseteq L$ jest \important{rozszerzeniem rozwiązywalnym}, gdy $K\subseteq L$ jest Galois i $Gal(L/K)$ jest grupą rozwiązywalną.
    \item $K\subseteq L$ jest \important{rozszerzeniem ciała przez pierwiastki} [radicals], gdy istnieje $k$ oraz 
    $$L\subseteq L_0\supseteq L_1\supseteq...\supseteq L_k=K$$
    takie, że dla każdego $i<k$ $L_i$ jest ciałem rozkładu wielomianu 
    \begin{itemize}
        \item $x^{n_i}-b_i$, $b_i\in L_{i+1}$ nad $L_{i+1}$ ($p\nmid n_i$ jeśli $char(K)=p$  
        \item lub $x^p-x-b_i$ dla $L_{i+1}$ nad $L_{i+1}$
    \end{itemize}
\end{enumerate}
\end{bbox}

\begin{theorem}
Załóżmy, że $K\subseteq L$ jest rozszerzeniem skończonym ciał. Wtedy $K\subseteq L$ jest rozszerzeniem przez pierwiastki $\iff$ istnieje $L'\supseteq L$ takie, że $K\subseteq L'$ jest rozwiązalne.
\end{theorem}
\begin{proof}
$\implies$

Możemy założyć, że $K\subseteq L_0$ jest rozszerzeniem Galois (przez rozszerzenie ciąg), wtedy mamy ciąg normalny grup [ćwiczenie].

$$Gal(L_0/L_k)\triangleright Gal(L_0/L_{k-1}\triangleright Gal(L_0/L_{k-1})\triangleright...\triangleright Gal(L_0/L_1)\triangleright\{e\}$$
faktorami tego ciągu są $Gal(L_i/L_{i+1})$. Wystarczy pokazać, że $L_i\supseteq L_{i+1}$ jest rozwiązywalna [wtedy można rozdrobić ciąg wyżej tak, by miał faktory abelowe].

Alternatywnie: $H\triangleleft G$, jeśli $H$ jest rozwiązywalna i $G/H$ jest rozwiązywalna, to $G$ jest rozwiązywalna [ćwiczenie].

Rozważamy przypadki wielomianów z definicji wyżej:
\begin{itemize}
    \item $x^{n_i}-b_i$

    Niech $a_i=\sqrt[n_i]{b_i}\in L_i$. Wtedy $L_i=L_{i+1}(\zeta_{n_i},a_i)$, $\zeta_{n_i}$ jest pierwiastkiem pierwotnym z $1$ stopnia $n_i$.
    $$L_i=L_{i+1}(\zeta_{n_i},a_i)\overset{\text{(\Radioactivity)}}{\supseteq} L_{i+1}(\zeta_{n_i})\supseteq L_{i+1}$$
    Ponieważ $L_{i+1}\supseteq L_i$ jest rozszerzeniem Galois, to takie jest również rozszerzenie $\text{(\Radioactivity)}$ i $Gal(L_{i+1}\overbrace{(\zeta_{n_i},a_i)}^{L_i}/L_{i+1}(\zeta_{n_i}))\cong\Z_{n_i}^*$ jest cykliczna i abelowa.

    Również rozszerzenie $L_{i+1}\subseteq L_{i+1}(\zeta_{n_i})$ jest Galois i grupa $Gal(L_{i+1}(\zeta_{n_i})/L_{i+1})$ jest abelowa.

    Stąd 
    $$Gal(L_i/L_{i+1})\overset{\text{(\Moon)}}\triangleright Gal(L_i/L_{i+1}(\zeta_{n_i})\triangleright\{e\}$$
    i faktor w $\text{(\Moon)}$ jest izomorficzny do abelowej grupy $Gal(L_i(\zeta_{n_i})/L_{i+1})$. Czy $Gal(L_i/L_{i+1})$ jest rozwiązywalna stopnia $\leq 2$.

    \item $x^p-x-b_i$

    Niech $a\in L_i$ będzie peirwiastkiem wielomianu wyżej. Wtedy $a+1$ jest również pierwiastkiem, bo 
    $$(a+1)^p-(a+1)-b_i=a^p+1^p-a-1-b_i=a^p-a-b_i=0$$
    Dlatego $a, a+1,...,a_(p-1)\in L_i$ i wszystkie są pierwiastkami wielomianu wyżej. Stąd $L_i=L_{i+1}(a)$.

    Niech $f\in Gal(L_i/L_{i+1})$ będzie wielomianem wyznaczanym przez $f(a)=a+l_f$. Przekształcanie
    $$Fal(L_i/L_{i+1})\ni f\mapsto l_f\in \Z_p^*$$
    daje $Gal(L_i/L_{i+1})\hookrightarrow\Z_p^*$ (w istocie jest tutaj $\cong$). Więc $l_i\supseteq L_{i+1}$ jest rozszerzeniem cyklicznym, czyli rozwiązywalny,
\end{itemize}

$\impliedby$

Niech $K\subseteq L$ będzie rozszerzeniem rozwiązywalnym. Pokażemy, że jest też rozszerzeniem pierwiastkowym.

Niech
$$Gal(L/K)\triangleright G_{k-1}\triangleright G_{k-2}\triangleright...\triangleright G_0=\{e\}$$
będzie ciągiem normalnym podgrup o faktorach abelowych i bez straty ogólności cyklicznych, prostych, tzn. $\cong\Z_q$, $q$ - liczba pierwsza. Wtedy
\begin{center}\begin{tikzcd}[column sep=tiny]
    L\arrow[r, phantom, sloped, "="] & L^{G_0}\arrow[r,phantom,sloped,"\supseteq"] & L^{G_1}\arrow[r,phantom,sloped,"\supseteq"]&...\arrow[r,phantom, sloped, "\supseteq"] & K\\
    & L_0\arrow[u,phantom,sloped, "="]\arrow[r,phantom,sloped,"\supseteq"] & L_1\arrow[u,phantom,sloped,"="]\arrow[r,phantom,sloped,"\supseteq"] & ...\arrow[r, phantom,sloped,"\supseteq"] & L_k\arrow[u,phantom,sloped,"="]
\end{tikzcd}\end{center}
jest ciągiem rozszerzeń cyklicznych, prostych.

\textbf{Claim:} Wystarczy teraz pokazać, że jeśli $K\subseteq L$ jest cykliczne, $L\subseteq\hat{K}$ i $Gal(L/K)$ jest prosta, to $K\subseteq L$ jest pierwiastkowe.

\textbf{Dowód na boczku:} Niech $[L:K]=n$, $Gal(L/K)\cong\Z_n^*$, a $n$ jest liczbą pierwszą. Rozważamy przypadki charakterystyk ciał:
\begin{itemize}
    \item $carh(K)=p\neq n$ lub $char(K)=0$

    Niech $\zeta\in\hat{K}$ będzie pierwiastkiem pierwotnym z $1$ stopnia $n$. Mamy, że $K\subseteq K(\zeta)$ i $K(\zeta)\subseteq L(\zeta)$ jest rozszerzeniem Galois. Dalej, $[L(\zeta):K(\zeta)]|[L:K]$, bo $Gal(L(\zeta)/K(\zeta))\hookrightarrow Gal(L/K)\cong\Z_n^*$. Niech $m=[L(\zeta):K(\zeta_]$, czyli $m=1$ lub $m=n$. Z twierdzenia \ref{tw:9.4} dostajemy
    $$L(\zeta)=K(\zeta)(\sqrt[n]{a}),\;a\in K(\zeta)$$
    gdy $m=n$. Gdy $m=1$ jest trywialne.
    \item $char(K)=p=n$

    Niech $\gamma\in Gal(L/K)$ będzie generatorem. Z twierdzenia Dedekinda (\ref{tw:9.5}) wiemy, że istneiej $b\in L$ takie, że
    $$K\in Tr_{L/K}(b)=\sum\limits_{i=0}^{p-1}\gamma^{i}(b)\neq 0$$
    Dla $b'=\frac{1}{t}b$ mamy $Tr_{L/K}(b')=1$.

    Niech $a=\gamma(b')+2\gamma^2(b')+...+(p-1)\gamma^{p-1}(b')$. Wtedy 
    $$\gamma(a)=\gamma^2(b')+2\gamma^3(b')+...+\underbrace{(p-1)\gamma^p(b')}_{=b'}=a-Tr_{L/K}(b')=a-1,$$
    ale 
    $$\gamma(a^p-a)=\gamma(a)^p-\gamma(a)=(a-1)^p-(a-1)=a^p-a$$
    więc $a^p-a\in Fix(\gamma)=K$. Niech $c=a^p-a$. Stąd $a$ jest pierwiastkiem $x^p-x-v$ oraz $L$ to ciało rozkładu $x^p-x-c$ nad $K$, więc $K\subseteq L$ jest pierwiastkowe.
\end{itemize}
\end{proof}

\textbf{Przykłady:}
\begin{enumerate}[leftmargin=*]
    \item Niech $S_n:= Sym(\{x_1,...,x_n\})$ będzie grupą funkcji symetrycznych o $n$ zmiennych, $L=K(x_1,...,x_n)$ i $M=K(x_1,...,x_n)^{S_n}$. Wiemy, że $S_n<Aut(L)$. Z twierdzenia Artina wiemy, że $K\subseteq L$ jest rozszerzeniem Galois oraz $S_n=Gal(L/M)$.

    W przypadku, gdy $n\geq5$ $S_n$ nie jest rozwiązalna, więc $M\subseteq L$ też takie nie jest. $L$ jest ciałem rozkładu wielomianu
    \begin{align*}
    M[T]\ni w(T)=&(T-x_1)(T-x_2)...(T-x_n)=\\
    &=T^n-\sigma_1(\overline{x})T^{n-1}+\sigma_2(\overline{x})T^{n-2}+...+(-1)^{n-1}\sigma_{n-1}(\overline{x})T+(-1)^n\sigma_n(\overline{x})
    \end{align*}
    gdzie $\sigma_i(\overline{x})=\sum_{1\leq j_1<...<j_i\leq n}x_{j_1}x_{j_2}...x_{j_n}$ to bazowe funkcje symetryczne (wzory Viete'a). Mamy $\sigma_i(\overline{x})\in M=L^{S_n}$.
    \item Gdy $K\subseteq L$ jest rozszerzeniem ciał oraz $L$ jest ciałem rozkładu nad $K$ wielomianu $w(x)$ stopnia co najwyżej $4$, to $Gal(L/K)$ wkłada się w $S_4$, a $S_4$ jest grupą rozwiązywalną. Podgrupa grupy rozwiązywalnej jest nadal rozwiązywalna, więc równanie
    $$w(x)=0$$
    jest rozwiązywalne przez pierwiastki.

    Niech $M=L^{Gal(L/K)}$. Wtedy z twierdzenia Artina wiemy, że $K\subseteq M$ jest radykalne, a $M\subseteq L$ jest Galois (fakt 7.4.). $Gal(L/M)=Gal(L/K)\implies M\subseteq L$ jest rozszerzeniem pierwiastkowym, tzn:
    $$L\subseteq L_0\supseteq L_1\supseteq ...\supseteq L_k=M,$$
    wszystkie rozszerzenia $L_i\supseteq L_{i+1}$ są rozszerzeniami o pierwiastki, więc wszystkie pierwiastki $w(x)$ dają się wyrazić nad $K$ poprzez stosowanie działań ciała (włączając dzielenie, odejmowanie) oraz "pierwiastkowanie" tj. branie rozwiązań wielomianów $x^n-a$ lub $x^p-x-a$.

    Gdy z kolei wielomian $w(x)$ jest stopnia $5$ to nie musi być to prawdą [ćwiczenie: czy dla $6,7$ powyższe zachodzi?]
\end{enumerate}

\textbf{\large\color{blue}Fakt}{\slshape %
    $K(\sigma_1,...,\sigma_n)=K(x_1,...,x_n)^{S_n}$
}
\begin{proof}
$\subseteq$ jasne

$\supseteq$

$$K(\vec{\sigma})\subseteq K(\overline{x})^{S_n}\subseteq K(\overline{x})$$
$$n!=[K(\overline{x}):K(\overline{x})^{S_n}]\leq[K(\overline{x}):K(\vec{\sigma})]\leq n!,$$
z czego ostatnia nierówność zachodzi, bo $K(\overline{x})$ jest ciałem rozkładu wielomianu
$$w(T)=(T-x_1)...(T-x_n)$$
nad $K(\sigma)$. Czyli mamy 
$$[K(\overline{x}):K(\overline{x})^{S_n}]=[K(\overline{x}):K(\vec{\sigma})]]$$
i zawieranie $K(\vec{\sigma})\subseteq K(\overline{x})^{S_n}$ jest tak naprawdę równością.
\end{proof}

Można też pokazać, że $K[\sigma_1,...,\sigma_n]=K[x_1,...,x_n]^{S_n}$, co jest \important{podstawowym twierdzeniem o wielomianach symetrycznych}.

\textbf{Zastosowania:} czyli konstrukcje przy pomocy cyrkla i linijki. Dane są punkty $A\neq B\in \R^2$.
\begin{itemize}
    \item \acc[b]{cyrkiel}

    Mamy okrąg $\{\begin{pmatrix}x\\y\end{pmatrix}\;:\;(x-a)^2+(y-b)^2=r^2\}$:
    \begin{illustration}
        \node[circle, draw, minimum size=5cm] (c) at (0, 0) {};
        \filldraw (0, 0) circle (2pt) node [below] {$A=\begin{pmatrix}a\\b\end{pmatrix}$};
        \filldraw (c.north east) circle (2pt) node [above] {$B=\begin{pmatrix}a'\\b'\end{pmatrix}$};
        \draw (0,0)--(c.north east) node [midway, below] {r};
    \end{illustration}
    czyli $r=\sqrt{(a'-a)^2+(b'-b)^2}$

    \item \acc[b]{linijka}
    
    Rozważamy prostą $L$ przechodzącą przez punkty $A$ i $B$, czyli o równaniu
    $$\left|\begin{matrix}x-a&a'-a\\y-b&b'-b\end{matrix}\right|=0$$
    \begin{illustration}
        \filldraw (0, 0) circle (1pt) node [below] {$A$};
        \filldraw (4, 2) circle (1pt) node [below] {$B$};
        \draw (-1, -0.5)--(0, 0)--(4, 2)--(5, 2.5);
        \filldraw (2, 1) circle (1pt) node [above] {$X=\begin{pmatrix}x\\y\end{pmatrix}$};
    \end{illustration}
\end{itemize}

Niech $(a_1,b_1),...,(a_n,b_n)\in \R^2$. {\slshape Punkt $(a, b)\in \R^2$ jest konstruowany przy pomocy cyrkla i linijki na płaszczyźnie $\R^2$ z punktów $(a_1,b_1),...,(a_n,b_n)$ i punktów $(0, 1),(1,0)$ $\iff$ rozszerzenie ciał $K\subseteq K(a,b)$ jest rozszerzeniem przez pierwiastki stopnia $\leq 2$.} Tutaj oczywiście $K=\Q(a_1,b_1,...,a_n,b_n)$.
\begin{itemize}[leftmargin=*]
    \item Kwadratura koła:

    Dane jest koło o promieniu $1$ i punkt $(0, 1)$. Szukamy kwadratu o polu $\pi$. Równoważnie problem można wyrazić jako szukanie punktu $(0, \sqrt{\pi})$. Ale $\pi$ jest liczbą przestępną, więc $\sqrt{\pi}$ też takie jest i rozwiązanie jest niemożliwe.

    \item Trysekcja kąta:
    
    Dany jest kąt $0<\theta<\pi$ i naszym celem jest skonstruować kąt $\frac{1}{3}\theta$.

    \begin{illustration}
        \draw[->] (0, 0)--(5, 0);
        \draw [->] (0,0)--(0, 4);
        \draw (0, 0)--(3, 3) node [midway, above] {$1$};
        \draw (0, 0)--(4.5, 1) node [midway, above] {$1$};
        \draw[->] (2, 0) arc (0:45:2) node [midway, right] {$\theta$};
        \draw[dashed](3, 3)--(3, 0);
        \filldraw (3, 0) circle (2pt) node [below] {$b=\cos\theta$};
        \draw[dashed] (4.5, 1)--(4.5, 0);
        \filldraw (4.5, 0) circle (2pt);
        \draw[->] (4.5, -1) node [below] {$a=\cos\frac{1}{3}\theta$}--(4.5, -0.3);
        \draw[->] (4, 0) arc (0:12.5:4) node [midway, right] {$\frac{1}{3}\theta$};
    \end{illustration}

    $a$ jest algebraiczne nad $b$, bo
    $$4a^3-3a-b=0.$$
    Cel jest niemożliwy, gdyż $[\Q(a,b):\Q(b)]=3$.

    \item Podwojenie sześcianu o krawędzi jednostkowej, równoważnie skonstruowanie $(0, a)$, gdzie $s^3=2$. Również jest to niemożliwe.
\end{itemize}
