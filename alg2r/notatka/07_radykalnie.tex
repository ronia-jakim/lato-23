\section{Rozszerzenia radykalne (czysty Bangladesz)}

\begin{bbox}
    Niech $K\subseteq L\subseteq \hat{K}$ jak zwykle. Wtedy
    \begin{itemize}
        \item[\PHtunny] $a\in L$ jest \important{czysto nierozdzielczy} nad $K$, czyli \acc[b]{radykalny}, gdy wielomian minimalny $a$ nad $K$, $w_a(x)\in K[X]$, ma tylko jeden pierwiastek w $\hat{K}$.
        \item[\PHtunny] $K\subseteq L$ jest \important{rozszerzeniem radykalnym} (czysto nierozdzielczym), gdy dla każdego $a\in L$ $a$ jest radykalne nad $K$.
    \end{itemize}
\end{bbox}

\begin{remark}
    $ $\newline
    \begin{enumerate}
    \item Jeśli $char(K)=0$, to $a$ nad $K$ jest czysto nierozdzielczy $\iff$ $a\in K$.
    \item $a$ jest radykalne nad $K$ $\iff$ dla każdego $f\in Gal(\hat{K}/K)$ $f(a)=a$
    \item Jeśli $char(K)=p$, to $a$ jest radykalne nad $K$ $\iff$ istnieje $n\geq 0$ $a^{p^n}\in K$.
    \end{enumerate}
\end{remark}
\begin{proof}
$ $\newline
\begin{enumerate}[leftmargin=*]
    \item $w_a(x)$ ma tylko pierwiastki jednokrotne, gdy $char(K)=0$
    \item Oczywiste $\star$
    \item $\impliedby$ oczywiste: $w_a(x)\in K[X]$ dzieli $x^{p^n}-a^{p^n}=(x-a)^{p^n}\in K[X]$

    $\implies$ Dowodzimy indukcją po $n=deg(a/K)$. Niech $w_a(x)=(x-a)^n\in K[X]$i $w_a'(x)=n(x-a)^{n-1}\in K[X]$ i $w_a'\in I(a/K)$ gdy $n>1$, czyli $w_a'(x)=0,$ więc $p|n$. Niech więc $n=p\cdot n_1$ i wtedy $w_a(x)=(x^p-a^p)^{n_1}$ i $a^p$ jest radykalny nad $K$, bo $deg(a^p/K)\leq n_1<n$. Z założenia indukcyjnego istnieje $k\geq 0$ takie, że $(a^p)^{p^k}=a^{p^{k+1}}\in K$ i to jest to, czego szukaliśmy. 
\end{enumerate}
\end{proof}

\begin{bbox}
Niech $K\subseteq L$ będzie rozszerzeniem algebraicznym. Definiujemy
\begin{enumerate}
    \item \acc{rozdzielcze domknięcie} $K$ w $L$: $sep_L(K)=\{a\in L\;:\;a\text{ rozdzielcze nad } K\}$
    \item \acc{radykalne domknięcie} (czysto nierozdzielcze) $K$ w $L$: $rad_L(K)=\{a\in L\;:\;a\text{ radykalny nad }K\}$
\end{enumerate}
\end{bbox}

\begin{conclusion}[przekrój $sep_L$ i $rad_L$]
$K\subseteq sep_L(K)$ i $rad_L(K)\subseteq L\subseteq \hat{K}$ to ciała takie, że $sep_L(K)\cap rad_L(K)=K$.
\end{conclusion}
\begin{proof}
Fakt, że $sep_L(K)$ jest ciałem wynika z \ref{wniosek:6:9}. Natomiast to, że $rad_L(K)$ jest ciałem wynika z tego, że
$$rad_L(K)=L\cap\bigcap\limits_{f\in Gal(\hat{K}/K)}Fix(f)=\{a\in\hat{K}\;:\;f(a)=a\}$$

Dalej, dla $a\in sep_L(K)\cap rad_L(K)$ mamy $w_a(x)=x-a$ jest wielomianem minimalnym $a$ nad $K$.
\end{proof}

\begin{bbox}
\begin{itemize}%[leftmargin=*]
    \item[\PHrosette] $\hat{K}^s=sep_{\hat{K}}(K)$ jest rozdzielczym domknięciem $K$
    \item[\PHrosette] $\hat{K}^r=rad_{\hat{K}}(K)$ jest radykalnym domknięciem $K$.
\end{itemize}
\end{bbox}

\begin{remark}
$ $\newline
\begin{enumerate}
    \item Gdy $K\subseteq L\subseteq\hat{K}$, to $sep_L(K)=\hat{K}^s\cap L$, $rad_L(K)=\hat{K}^r\cap L$
    \item Załóżmy, że $K\subseteq L\subseteq M\subseteq\hat{K}$, wtedy $K\underset{rad}{\subseteq}L\underset{rad}{\subseteq}M\iff K\underset{rad}{\subseteq} M$
    \item Jeśli $char(K)=0$, to $sep_L(K)=K^{alg}(L)$ i $rad_L(K)=K$, oraz $\hat{K}^s=\hat{K}$, $\hat{K}^r=K$.
\end{enumerate}
\end{remark}

\begin{fact}
Załóżmy, że $K\subseteq L\subseteq\hat{K}$, $K_s= sep_L(K)$, $K_r=rad_L(K)$, $L'=K_s\cdot K_r$ i niech $L'=K_s\cdot K_r$ będzie złożeniem ciał $K_s$ i $K_r$ w $L$ (tzn. ciało generowane w $L$ przez $K_s\cup K_r$: $L'=K_s(K_r)=K_r(K_s)$). Wtedy:
\begin{enumerate}
    \item $[L':K]=[K_s:K]\cdot[K_r:K]$
    \item Gdy $K\subseteq L$ jest rozszerzeniem normalnym, to $K_s\cdot K_r=L$
    \item $K_s\subseteq L$ jest radykalne, a $K_r\subseteq L'$ rozdzielcze 
\end{enumerate}
\end{fact}
\begin{proof}
Jeśli $chark(K)=0$, to problem jest trywialny, bo $K_r=K$, $K_s=L$ i $L'=L$.

Załóżmy więc, że $char(K)=p>0$.

\begin{illustration}
    \draw (0,0) circle (0.7) node {$K$};
    \draw (-0.53, -0.53)..controls(-3.5, -0.4)and(-3.5, 2.4)..(-0.53,0.53);
    \draw (0.53, -0.53)..controls(3.5, -0.4)and(3.5, 2.4)..(0.53,0.53);
    
    \draw(2.6, 1.05)..controls(1.5, 3)and(-1.5, 3)..(-2.6, 1.05);
    \draw(2.6, 1.05)..controls(1.5, 4.5)and(-1.5, 4.5)..(-2.6, 1.05);
    \node at (1.7, 0.3) {$K_r$};
    \node at (-1.7, 0.3) {$K_s$};
    \node at (0, 1.5) {$L'$};
    \node at (0, 3) {$L$};
\end{illustration}

\begin{enumerate}[leftmargin=*]
    \item $L'=K_r(K_s)\supseteq K_r\supseteq K$, więc:
    $$[L':K]=[K_r(K_s):K_r][K_r:K]$$
    Wystarczy pokazać, że $[K_s:K]=[K_r(K_s):K_r]$. 

    Zadanie z listy $4$: Załóżmy, że $K\subseteq L, M\subseteq\hat{K}$ są rozszerzeniami ciała takie, że $L\cap M=K$. Jeśli dla wszystkich $L_0,M_0$ takich, że $K\subseteq L_0\subseteq L$ i $K\subseteq M_0\subseteq M$ są skończone i $[L_0(M_0):L_0]=[M_0:K]$, to $[L(M):L]=[M:K]$.
    
    W takim razie wystarczy, że pokażemy
    $$[K_r(K_s):K]=[K_s:K]$$
    korzystając z zadania 4 (wyżej). Niech $K\subseteq K_r^0\subseteq K_r$ i $K\subseteq K_s^0\subseteq K_s$, pierwsze rozszerzenia są skończone. Na mocy twierdzenia Abela możemy wybrać $a\in K_s^0$ takie, że $K_s^0=K(a)$. Wtedy również
    $$K_r^0(K_s^0)=K_r^0(a)$$
    i $[K_s^0:K]=$ stopień $a$ nad $K$, $[K_r^0(a):K_r^0]=$ stopień $a$ nad $K_r^0$. Wystarczy pokazać, że oba te stopnie się zgadzają.

    Niech $n=[K(a):K]=$ stopień $a$ nad $K$. Wtedy
    $$1,a,...,a^{n-1}$$
    to baza liniowa $K(a)$ nad $K$. Przez to, że $a$ jest rozdzielczy nad $K$ i $p=char(K)$, to $K(a)=K(a^p)$ [zad. 7 lista 4], czyli dla każdego $l>0$
    $$1,a^{p^l},...,a^{(n-1)p^l}$$
    też jest bazą $K(a)$ nad $K$.

    Pokażemy, że $1,a,...,a^{n-1}$ jest bazą liniową $K_r^0(a)$ nad $K_r^0$:
    \begin{itemize}
        \item liniowa niezależność:
        $$\sum k_ia^i=0,\;k_i\in K_r^0$$
        Niech $l$ będzie takie, że $k_i^{p^l}\in K$ dla wszystkich $i$, wtedy
        $$\sum k_i^{p^l}a^{ip^l}=0\implies (\forall\;i)\;k_i=0$$
        Czyli $[K_r^0(a):K_r^0]\leq[K(a):K]=n$ i $1,a,...,a^{n-1}$ jest bazą $K_r^0(a)/K_r^0$.
    \end{itemize}
    \item Bez straty ogólności załóżmy, że $[L:K]<\infty$, bo 
    $$L=\bigcup\{L_0:K\underset{skon,norm}{\subseteq} L_0\subseteq L\}$$
    \begin{enumerate}
        \item Niech $a\in L\supseteq K_r$, postulujemy, że $a$ jest rozdzielczy nad $K_r$. Niech $a=a_1,a_2,...,a_n$ będą wszystkimi pierwiastkami wielomianu $w_a(X)\in K[X]$ i niech 
        $$v(x)=\prod\limits_{i=1}^n(x-a_i).$$
        Wtedy dla $f\in Gal(\hat{K}/K)$ mamy $f[L]=L$, więc $f$ permutuje $\{a_1,...,a_n\}$. Stąd $f(v(x))=v(x)$, czyli $f$ zachowuje współczynniki $v(x)$. To onzacza, że $v(x)\in K_r[X]$ i mamy, że $a$ jest rozdzielczy nad $K_r$.
        \item $L\supseteq K_s$ jest radykalne: z uwagi 6.6(3) wiemy, że jeśli $a\in L$ to dla pewnego $l$ mamy $a^{p^l}$ jest rozdzielcze nad $K$. Czyli $a^{p^l}\in K_s$, więc $a$ jest radykalny nad $K_s$.
    \end{enumerate}

    Z podpunktów wyżej  wiemy, że $L\subseteq K_r\cdot K_s$ jest rozszerzeniem rozdzielczym i radykalnym, więc $L=K_r\cdot K_s$.
    \item $L\supseteq K_s$ jest radykalne w sposób analogiczny do rozumowania wyżej. $L'\supseteq K_r$ jest rozdzielcze, bo $L'=K_r[K_s]$.
\end{enumerate}
\end{proof}

\subsection{Stopień rozdzielczy, radykalny ciała}

\begin{bbox}
$K\subseteq L\subseteq\hat{K}$

Definiujemy \acc{$[L:K]_s=[sep_L(K):K]$} jako \important{stopień rozdzielczy} ciała $L$ nad $K$ oraz \acc{$[L:K]_r=[L:rad_L(K)]$} jako \important{stopień radykalny} $L$ nad $K$. 
\end{bbox}

Z wyników wyżej dostajemy
$$[L:K]=[L:K]_s\cdot[L:K]_r,$$
bo $K\subseteq sep_L(K)$ jest rozdzielcze, a $sep_L(K)\subseteq L$ jest radykalne.
\begin{remark}
$K\subseteq L\subseteq\hat{K}$
\begin{enumerate}
    \item Jeśli $K\subseteq L$ jest rozdzielcze, to $[L:K]=|\{f\restriction L\;:\;f\in Gal(\hat{K}/K)\}|=|\{f:L\to\hat{K}\;:\;f\restriction K=id\}|$
    \item Ogólnie, $[L:K]_s=|\{f\restriction L\;:\;f\in Gal(\hat{K}/K)\}|$ (jak wyżej)
\end{enumerate}
\end{remark}
\begin{proof}
Rozważamy $[L:K]<\infty$. Przypadek ogólny $[L:K]$ można zredukować do przypadku skończonego, co jest ćwiczeniem na liście [wskazówka: \emph{rozważyć odpowiednią bazę liniową $L$ nad $K$}]
\begin{enumerate}[leftmargin=*]
    \item Z twierdzenia Abela $L=K(a)$ i dla $f\in Gal(\hat{K}/K)$, $f\restriction L$ jest wyznaczone jednoznacznie przez $f(a)$. Wiemy, że $f(a)\in\{\text{pierwiastki }w_a(x)\}$, których jest $n=[L:K]$.
    \item $l\supseteq K_s$ to rozszerzenie radykalne, więc $f\restriction L$ jest wyznaczone przez $f\restriction K_s$. Dlatego:
    $$|\{f\restriction L\;:\;f\in Gal(\hat{K}/K)\}|=|\{f\restriction K_s\;:\;f\in Gal(\hat{K}/K)\}|=[K_s:K]=[L\underset{\underset{\scriptscriptstyle sep_L(K)}{\rotatebox{90}{=}}}{:K_s]}$$
    %\rotatebox[origin=l]{-90}{$\scriptscriptstyle=sep_L(K)$}}{K_s}]
\end{enumerate}
\end{proof}

\textbf{\large\color{blue}Uwaga. }{\slshape Jeśli $char(K)=p$ i $[L:K]_r<\infty$, to $[L:K]_r$ jest potęgą $p$.}

\begin{proof}
Indukcja względem $[L:K]_r=[L:K_s]$. Bez starty ogólności załóżmy, że $K=K_s$. Niech $a\in L\setminus K$, wtedy $a$ jest radykalne nad $K$, czyli istnieje minimalne $l$ takie, że  $a^{p^l}\in K$.

Niech $a'=a^{p^{l-1}}$, wtedy $a'\in L\setminus K$ i $(a')^p\in K$, dlatego $w_{a'}(x)=x^p-(a')^p$ i $K\subseteq K(a')\subseteq L$, pierwsze rozszerzenie ma stopień $p$, a drugie jest radykalne.

Mamy $[L:K(a')]<[L:K]$, więc z założenia indukcyjnego $[L:K(a')]=p^r\implies [L:K]=p^{r+1}$
\end{proof}





















