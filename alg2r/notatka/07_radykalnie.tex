\section{Rozszerzenia radykalne (czysty Bangladesz)}

\begin{bbox}
    Niech $K\subseteq L\subseteq \hat{K}$ jak zwykle. Wtedy
    \begin{itemize}
        \item[\PHtunny] $a\in L$ jest \important{czysto nierozdzielczy} nad $K$, czyli \acc[b]{radykalny}, gdy wielomian minimalny $a$ nad $K$, $w_a(x)\in K[X]$, ma tylko jeden pierwiastek w $\hat{K}$.
        \item[\PHtunny] $K\subseteq L$ jest \important{rozszerzeniem radykalnym} (czysto nierozdzielczym), gdy dla każdego $a\in L$ $a$ jest radykalne nad $K$.
    \end{itemize}
\end{bbox}

\begin{remark}
    $ $\newline
    \begin{enumerate}
    \item Jeśli $char(K)=0$, to $a$ nad $K$ jest czysto nierozdzielczy $\iff$ $a\in K$.
    \item $a$ jest radykalne nad $K$ $\iff$ dla każdego $f\in Gal(\hat{K}/K)$ $f(a)=a$
    \item Jeśli $char(K)=p$, to $a$ jest radykalne nad $K$ $\iff$ istnieje $n\geq 0$ $a^{p^n}\in K$.
    \end{enumerate}
\end{remark}
\begin{proof}
$ $\newline
\begin{enumerate}[leftmargin=*]
    \item $w_a(x)$ ma tylko pierwiastki jednokrotne, gdy $char(K)=0$
    \item Oczywiste $\star$
    \item $\impliedby$ oczywiste: $w_a(x)\in K[X]$ dzieli $x^{p^n}-a^{p^n}=(x-a)^{p^n}\in K[X]$

    $\implies$ Dowodzimy indukcją po $n=deg(a/K)$. Niech $w_a(x)=(x-a)^n\in K[X]$i $w_a'(x)=n(x-a)^{n-1}\in K[X]$ i $w_a'\in I(a/K)$ gdy $n>1$, czyli $w_a'(x)=0,$ więc $p|n$. Niech więc $n=p\cdot n_1$ i wtedy $w_a(x)=(x^p-a^p)^{n_1}$ i $a^p$ jest radykalny nad $K$, bo $deg(a^p/K)\leq n_1<n$. Z założenia indukcyjnego istnieje $k\geq 0$ takie, że $(a^p)^{p^k}=a^{p^{k+1}}\in K$ i to jest to, czego szukaliśmy. 
\end{enumerate}
\end{proof}









