\section{Rozszerzenia radykalne (czysty Bangladesz)}

\begin{bbox}
    Niech $K\subseteq L\subseteq \hat{K}$ jak zwykle. Wtedy
    \begin{itemize}
        \item[\PHtunny] $a\in L$ jest \important{czysto nierozdzielczy} nad $K$, czyli \acc[b]{radykalny}, gdy wielomian minimalny $a$ nad $K$, $w_a(x)\in K[X]$, ma tylko jeden pierwiastek w $\hat{K}$.
        \item[\PHtunny] $K\subseteq L$ jest \important{rozszerzeniem radykalnym} (czysto nierozdzielczym), gdy dla każdego $a\in L$ $a$ jest radykalne nad $K$.
    \end{itemize}
\end{bbox}

\begin{remark}
    $ $\newline
    \begin{enumerate}
    \item Jeśli $char(K)=0$, to $a$ nad $K$ jest czysto nierozdzielczy $\iff$ $a\in K$.
    \item $a$ jest radykalne nad $K$ $\iff$ dla każdego $f\in Gal(\hat{K}/K)$ $f(a)=a$
    \item Jeśli $char(K)=p$, to $a$ jest radykalne nad $K$ $\iff$ istnieje $n\geq 0$ $a^{p^n}\in K$.
    \end{enumerate}
\end{remark}
\begin{proof}
$ $\newline
\begin{enumerate}[leftmargin=*]
    \item $w_a(x)$ ma tylko pierwiastki jednokrotne, gdy $char(K)=0$
    \item Oczywiste $\star$
    \item $\impliedby$ oczywiste: $w_a(x)\in K[X]$ dzieli $x^{p^n}-a^{p^n}=(x-a)^{p^n}\in K[X]$

    $\implies$ Dowodzimy indukcją po $n=deg(a/K)$. Niech $w_a(x)=(x-a)^n\in K[X]$i $w_a'(x)=n(x-a)^{n-1}\in K[X]$ i $w_a'\in I(a/K)$ gdy $n>1$, czyli $w_a'(x)=0,$ więc $p|n$. Niech więc $n=p\cdot n_1$ i wtedy $w_a(x)=(x^p-a^p)^{n_1}$ i $a^p$ jest radykalny nad $K$, bo $deg(a^p/K)\leq n_1<n$. Z założenia indukcyjnego istnieje $k\geq 0$ takie, że $(a^p)^{p^k}=a^{p^{k+1}}\in K$ i to jest to, czego szukaliśmy. 
\end{enumerate}
\end{proof}

\begin{bbox}
Niech $K\subseteq L$ będzie rozszerzeniem algebraicznym. Definiujemy
\begin{enumerate}
    \item \acc{rozdzielcze domknięcie} $K$ w $L$: $sep_L(K)=\{a\in L\;:\;a\text{ radykalne nad } K\}$
    \item \acc{radykalne domknięcie} (czysto nierozdzielcze) $K$ w $L$: $rad_L(K)=\{a\in L\;:\;a\text{ radykalny nad }K\}$
\end{enumerate}
\end{bbox}

\begin{conclusion}[przekrój $sep_L$ i $rad_L$]
$K\subseteq sep_L(K)$ i $rad_L(K)\subseteq L\subseteq \hat{K}$ to ciała takie, że $sep_L(K)\cap rad_L(K)=K$.
\end{conclusion}
\begin{proof}
Fakt, że $sep_L(K)$ jest ciałem wynika z \ref{wniosek:6:9}. Natomiast to, że $rad_L(K)$ jest ciałem wynika z tego, że
$$rad_L(K)=L\cap\bigcap\limits_{f\in Gal(\hat{K}/K)}Fix(f)=\{a\in\hat{K}\;:\;f(a)=a\}$$

Dalej, dla $a\in sep_L(K)\cap rad_L(K)$ mamy $w_a(x)=x-a$ jest wielomianem minimalnym $a$ nad $K$.
\end{proof}

\begin{bbox}
\begin{itemize}%[leftmargin=*]
    \item[\PHrosette] $\hat{K}^s=sep_{\hat{K}}(K)$ jest rozdzielczym domknięciem $K$
    \item[\PHrosette] $\hat{K}^r=rad_{\hat{K}}(K)$ jest radykalnym domknięciem $K$.
\end{itemize}
\end{bbox}

\begin{remark}
$ $\newline
\begin{enumerate}
    \item Gdy $K\subseteq L\subseteq\hat{K}$, to $sep_L(K)=\hat{K}^s\cap L$, $rad_L(K_=\hat{K}^r\cap L$
    \item Załóżmy, że $K\subseteq L\subseteq M\subseteq\hat{K}$, wtedy $K\underset{rad}{\subseteq}L\underset{rad}{\subseteq}M\iff K\underset{rad}{\subseteq} M$
    \item Jeśli $char(K)=0$, to $sep_L(K)=K^{alg}(L)$ i $rad_L(K)=K$, oraz $\hat{K}^s=\hat{K}$, $\hat{K}^r=K$.
\end{enumerate}
\end{remark}

\begin{fact}
Załóżmy, że $K\subseteq L\subseteq\hat{K}$, $K_s= sep_L(K)$, $K_r=rad_L(K)$, $L'=K_s\cdot K_r$ i niech $L'=K_s\cdot K_r$ będzie złożeniem ciał $K_s$ i $K_r$ w $L$ (tzn. ciało generowane w $L$ przez $K_s\cup K_r$: $L'=K_s(K_r)=K_r(K_s)$). Wtedy:
\begin{enumerate}
    \item $[L':K]=[K_s:K]\cdot[K_r:K]$
    \item Gdy $K\subseteq L$ jest rozszerzeniem normalnym, to $K_s\circ K_r=L$
    \item $K_s\subseteq L$ jest radykalne, a $K_r\subseteq L'$ rozdzielcze 
\end{enumerate}
\end{fact}
\begin{proof}
Jeśli $chark(K)=0$, to problem jest trywialny, bo $K_r=K$, $K_s=L$ i $L'=L$.

Załóżmy więc, że $char(K)=p>0$.

\begin{illustration}
\node at (0,0) {kiedyś mi się zechce to rysować.};
\end{illustration}

\begin{enumerate}[leftmargin=*]
    \item $L'=K_r(K_s)\supseteq K_r\supseteq K$, więc:
    $$[L':K]=[K_r(K_s):K_r][K_r:K]$$
    Wystarczy pokazać, że $[K_s:K]=[K_r(K_s):K_r]$. To można zrobić pokazując, że dla wszystkich $K_r^0$ i $K_s^0$ takich, że $\subseteq K_r^0\subseteq K_r$ i $K\subseteq K_s^0\subseteq K_s$ są skończone mamy $[K_s^0:K]=[K_r^0(K_s^0):K_r^0]$.

    Zadanie z listy $4$: Załóżmy, że $K\subseteq L, M\subseteq\hat{K}$ są rozszerzeniami ciała takie, że $L\cap M=K$. Jeśli dla wszystkich $L_0,M_0$ takich, że $K\subseteq L_0\subseteq L$ i $K\subseteq M_0\subseteq M$ są skończone i $[L_0(M_0):L_0]=[M_0:K]$, to $[L(M):L]=[M:K]$.
    {\large\color{orange}DOKOŃCZYĆ DOWODZIK}
\end{enumerate}

\end{proof}























