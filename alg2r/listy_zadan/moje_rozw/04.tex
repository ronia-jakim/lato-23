\documentclass{article}

\usepackage{../../../lecture_notes}

\title{Problem List 4}
\author{Weronika Jakimowicz}
\date{sometime in the future}

\begin{document}
\begin{problem}[1]{}
Calculate cyclotomic polynomials
$$F_1(X),F_2(X),F_4(X),F_8(X),F_{16}(X),F_{15}(X)$$
and then calculate their images in the ring $\Z_3[X]$, under the homomorphism $\Z[X]\to\Z_3[X]$ induced by the quotient homomorphism $\Z\mapsto\Z_3$. Which of them are irreducible over $\Z_3$?
\end{problem}

$F_1(X)=X-1$ is easy, then $F_2(X)=X^2-1$ is also self-explanatory. 

With $F_4(X)$ I know that it cannot have degree $4$ because $2$ divides $4$ and cannot be counted in $\phi(4)$. I use the definition of $F_m$ from the lecture and write:
\begin{align*}
    F_4(x)&=(x-e^{\frac{\pi i}{2}})(x-e^{\frac{3\pi i}{2}})=x^2-x(e^{\frac{3\pi i}{2}}+e^{\frac{\pi i}{2}})+e^{2\pi i}=\\
    &=x^2+1
\end{align*}
However, I think I could get it from the fact that the roots of a cyclotomic polynomial $F_m$ are all the primitive roots of $1$ of order $m$. So
$$x^4-1=(x^2-1)(x^2+1)$$
and every root that comes from $x^2-1$ is not primitive, so only $x^2+1$ has primitive roots of order $4$.

A similar story is with $F_8:$
$$x^8-1=(x^4-1)(x^4+1)\implies F_8(x)=x^4+1$$

$F_{15}(x)$ should have degree $8$ and so here is a lot of computation to avoid multiplying $\prod\limits_{\substack{1\leq k<15\\gcd(k,15)=1}}(x-e^{k\frac{2\pi i}{15}})$ because why not
\begin{align*}
    x^{15}-1&=(x-1)(x^{14}+x^{13}+...+x+1)=\\
    &=(x-1)(x^{12}(x^2+x+1)+x^{9}(x^2+x+1)+...+x^2+x+1)=\\
    &=(x-1)(x^2+x+1)(x^{12}+x^9+x^6+x^3+1)=\\
    &=(x-1)(x^2+x+1)(x^{12}+x^{11}-x^{11}+x^{10}-x^{10}+...+x^3+x^2-x^2+x-x+1)=\\
    &=(x-1)(x^2+x+1)(x^8(x^4+x^3+x^2+x+1)-x^7(x^4+1)+x^6(x^4+...+1)-...+(x^4+x^3+x^2+x+1))=\\
    &=\underbrace{(x-1)}_{=F_1(x)}\underbrace{(x^2+x+1)}_{div.\;F_3(x)}\underbrace{(x^4+x^3+x^2+x+1)}_{div.\;F_5(x)}(x^8-x^7+x^6-x^5+x^4-x^3+x^2-x+1)
\end{align*}
$$\rotatebox{-90}{$\implies$}$$
$$F_{15}(x)=x^8-x^7+x^6-x^5+x^4-x^3+x^2-x+1$$

And now for the final boss because I messed up the order in which they should appear and am too lazy to change it: $F_{16}(x)$!!! I expect it to have order 8
$$x^{16}-1=(x^8-1)(x^8+1)\implies F_{16}(x)=x^8+1$$

\end{document}
