\documentclass{article}

\usepackage{../../../lecture_notes}

\title{Problem List 4}
\author{Weronika Jakimowicz}
\date{\scriptsize sometime in the future}

\begin{document}
\maketitle\thispagestyle{empty}

\begin{problem}[1]{}
Calculate cyclotomic polynomials
$$F_1(X),F_2(X),F_4(X),F_8(X),F_{16}(X),F_{15}(X)$$
and then calculate their images in the ring $\Z_3[X]$, under the homomorphism $\Z[X]\to\Z_3[X]$ induced by the quotient homomorphism $\Z\mapsto\Z_3$. Which of them are irreducible over $\Z_3$?
\end{problem}

$F_1(X)=X-1$ is easy, then $X^2-1=(X-1)(X+1)$, so $F_2(x)=x+1$ because $x=1$ is not a primitive root of order $2$. 

With $F_4(X)$ I know that it cannot have degree $4$ because $2$ divides $4$ and cannot be counted in $\phi(4)$. I use the definition of $F_m$ from the lecture and write:
\begin{align*}
    F_4(x)&=(x-e^{\frac{\pi i}{2}})(x-e^{\frac{3\pi i}{2}})=x^2-x(e^{\frac{3\pi i}{2}}+e^{\frac{\pi i}{2}})+e^{2\pi i}=\\
    &=x^2+1
\end{align*}
However, I think I could get it from the fact that the roots of a cyclotomic polynomial $F_m$ are all the primitive roots of $1$ of order $m$. So
$$x^4-1=(x^2-1)(x^2+1)$$
and every root that comes from $x^2-1$ is not primitive, so only $x^2+1$ has primitive roots of order $4$.

A similar story is with $F_8:$
$$x^8-1=(x^4-1)(x^4+1)\implies F_8(x)=x^4+1$$

$F_{15}(x)$ should have degree $8$ and so here is a lot of computation to avoid multiplying $\prod\limits_{\substack{1\leq k<15\\gcd(k,15)=1}}(x-e^{k\frac{2\pi i}{15}})$ because why not
\begin{align*}
    x^{15}-1&=(x-1)(x^{14}+x^{13}+...+x+1)=\\
    &=(x-1)(x^{12}(x^2+x+1)+x^{9}(x^2+x+1)+...+x^2+x+1)=\\
    &=(x-1)(x^2+x+1)(x^{12}+x^9+x^6+x^3+1)=\\
    &=(x-1)(x^2+x+1)(x^{12}+x^{11}-x^{11}+x^{10}-x^{10}+...+x^3+x^2-x^2+x-x+1)=\\
    &=(x-1)(x^2+x+1)(x^8(x^4+x^3+x^2+x+1)-x^7(x^4+1)+x^6(x^4+...+1)-...+(x^4+x^3+x^2+x+1))=\\
    &=\underbrace{(x-1)}_{=F_1(x)}\underbrace{(x^2+x+1)}_{div.\;F_3(x)}\underbrace{(x^4+x^3+x^2+x+1)}_{div.\;F_5(x)}(x^8-x^7+x^6-x^5+x^4-x^3+x^2-x+1)
\end{align*}
$$\rotatebox{-90}{$\implies$}$$
$$F_{15}(x)=x^8-x^7+x^6-x^5+x^4-x^3+x^2-x+1$$

And now for the final boss because I messed up the order in which they should appear and am too lazy to change it: $F_{16}(x)$!!! I expect it to have order 8
$$x^{16}-1=(x^8-1)(x^8+1)\implies F_{16}(x)=x^8+1$$
\medskip

Images in $\Z_3[X]$:
$$F_1(x)=x-1\mapsto x+2$$
$$F_2(x)=x+1\mapsto x+1$$
$$F_4(x)=x^2+1\mapsto x^2+1$$
$$F_8(x)=x^4+1\mapsto x^4+1$$
$$F_{16}(x)=x^8+1\mapsto x^8+1$$
$$F_{15}(x)=x^8-x^7+x^6-x^5+x^4-x^3+x^2-x+1\mapsto x^8+2x^7+x^6+2x^5+x^4+2x^3+x^2+2x+1$$

Let me start from $F_{15}(x)$. I see that $2$ divides $F_{15}(x)$ and it is easy to check that $(x+1)^8=F_{15}(x)$ in $\Z_3$.

Now, $F_4(x)$, it has no roots in $\Z_3$ and because it is a quadratic polynomial, it cannot be divided by any other polynomial than one of degree $1$. Hence, it is irreducible.

$F_8(x)$ also has no roots in $\Z_3$ so we surely cannot split it into a linear polynomial and a polynomial of degree $3$. The only hope is in two polynomials of degree $2$. Let us check
$$(x^2+x+2)(x^2+2x+2)=x^4+{\color{yellow}2x^3}+{\color{blue}2x^2}+{\color{yellow}x^3}+{\color{blue}2x^2}+{\color{green}2x}+{\color{blue}2x^2}+{\color{green}x}+1=x^4+1$$

$F_{16}(x)$ is the worst because I cannot find a decomposition using simple tricks but showing that it is irreducible can be a little painful. I will leave it for now and most probably forget to return to it later. I apologize.

%Suppose that
%$$x^4+1=(x^2+ax+b)(x^2+cx+d)$$
%then
%$$\begin{cases}bd=1\\ad+bc=0\\c+a=0\\d+b+ac=0\end{cases}$$
%we have that $d=b^{-1}$ and $c=-a$. Hence, if we put this into the second equation, we get:
%$$0=ab^{-1}-ab=a(b^{-1}-b)$$
%\begin{enumerate}
%    \item if $a=0$, then $c=0$ and from the third equation: $b+b^{-1}=0\implies b^2+1=0\implies b^2=2$ which cannot happen if $b\in\{0,1\}$ and even powers of $2$ are $1$ modulo $3$ so this gives a contradiction
%    \item if $b^{-1}-b=0$, then $b=b^{-1}\implies b^2=1$ and the arguments is the same as above.
%\end{enumerate}
%This shows that $x^4+2$ is irreducible over $\Z_3$.


%Most of them are irreducible as they have no roots in $\Z_3$ but $F_{15}(2)=681\mod 3=0$

\begin{problem}[2]{}
Describe the normal closures of the following field extensions:
\begin{enumerate}[label=(\alph*)]
    \item $\Q[\sqrt[n]{2}]\supseteq\Q$
    \item $\Q(\sqrt[n]{X})\supseteq\Q(X)$
    \item $\C(\sqrt[n]{X})\supseteq\C(X)$
    \item $\Q[\zeta]\supseteq\Q$, where $\zeta$ is a primitive root of $1$ of degree $n>1$.
\end{enumerate}
(hint: in $(a)-(c)$ find the minimal polynomial, in $(c)$ use the fact that $\C$ is algebraically closed, in (b) notice that $X$ may be replaced by any transcendental number, this is not necessary, but it helps.)
\end{problem}

\begin{enumerate}[label=(\alph*), leftmargin=*]
    \item $\Q[\sqrt[n]{2}]\supseteq\Q$

    The minimal polynomial for $\sqrt[n]{2}$ over $\Q$ is $w(x)=x^n-2$ and its roots are of form
    $$a_k=\sqrt[n]{2}e^{\frac{2\pi i}{n}k}$$
    Now, I know that an extension of a field is normal if for any polynomial, if it has one root, then it has all the roots. So I need to find the minimal field that contains all those roots and $\Q[\sqrt[n]{2}]$ and it is
    $$L=\Q(a_1,...,a_n=\sqrt[n]{2})$$
    because we have already showed that it is the smallest field such that $a_1,...,a_n$ are contained within it.
\end{enumerate}

\begin{problem}[3]{}
Prove that every field extension of degree $2$ is normal.
\end{problem}

Let $K$ be a field and $f\in K[X]$ be a polynomial of degree $2$, WLOG $f$ is monic. We consider $K(a)$, where $f(a)=0$. Let us assume that
$$f(x)=\alpha_0+\alpha_1x+x^2$$
for $\alpha_0,\alpha_1\in K$. We know that if $a,b$ are solutions of $f$, then $a+b=-\alpha_1\implies b=-\alpha_1-a\in K$, hence both roots of $f$ are in our extension $K(a)$ and $K(a)$ is normal.
%$$\begin{cases}a+b=-{\alpha_1}\\ab={\alpha_0}\end{cases}$$

%Any other $\beta\in K(a)$ is either in $K$ or is of form $\beta=a^kb^l\gamma,k>0\;or\;l>0,\gamma\in K$. Then $x-\gamma\in K[X]\subseteq L[X]$ and 
%Any other $\beta\in K(a)$ is of form $\beta=a^kb^l\gamma,\gamma\in K$. We have the minimal polynomial of $\beta$ is $x-\beta\in L[X]$.

\begin{problem}[4]{}
Assume that the field extension $K\subseteq L$ is algebraic and $f:L\to L$ is a monomorphism, $f\restriction K=id$. Prove that $f$ is "onto".
\end{problem}

Let us take any $\alpha\in L$ such that $\alpha\neq 0$. Then, since $K\subseteq L$ is algebraic, we know that there exists a minimal polynomial $w\in K[X]$ such that $w(\alpha)=0$. Let
$$w(x)=\sum\limits_{i=0}^na_ix^i$$
and since $w$ is minimal, then it is irreducible and $a_0\neq 0$. Now, consider
$$f(w(\alpha))=f(\sum a_i\alpha^i)=\sum f(a_i\alpha^i)=\sum f(a_i)f(\alpha^i)=\sum a_i\cdot f(\alpha)^i$$
Hence, $f(\alpha)$ must be another root of $w$. Since $f$ is a monomorphism, we cannot have that two roots go to the same roots but we still need all of them to permute. Hence, every element of $L$ is represented in $Im(f)$.

\begin{problem}[5]{d}
Show that if $K\subseteq L\subseteq\hat{K}$ and $K\subseteq L$ is radical, then $Gal(\hat{K}/K)=Gal(\hat{K}/L)$.
\end{problem}

$K\subseteq L$ is radical means that if $a\in L$ and $w_a\in K[X]$ is the minimal polynomial of $a$, then $w_a$ has only one root in $\hat{K}$

$$Gal(\hat{K}/K)=Gal(\hat{K}/L)$$

$\supseteq$ is obvious because $f\restriction L=id_L$ and $id_L\restriction K=id_K$ so $f\restriction K=id_K$.

$\subseteq$

Take any $f\in Gal(\hat{K}/K)$ and any $a\in L$. I know that $w_a\in K[X]$ has only one root in $\hat{K}$ and that this root is $a$. Let $w_a=\sum b_ix^i$ and see that
$$f(w_a(a))=f(\sum b_ia^i)=\sum f(b_i)f(a^i)=\sum b_if(a)^i=0$$
so $f(a)$ must also be a root of $w_a$ and because this root is unique, then $f(a)=a$.

%Take any $f\in Gal(\hat{K}/K)$ and any $a\in L$. I want to show that $f(a)=a$.

%Let $w_a=\sum b_ix^i$ be the minimal polynomial of $a$. Consider
%$$0=f(w_a(a))=f(\sum b_ixa^i)=\sum f(b_i)f(a)^i=b_if(a)^i$$
%ok, so $f(a)$ must be yet another root of $w_a$.

%How about I assume that $f(a)\neq a$? Then
%\begin{align*}
%    0&=w_a(a)-f(w_a(a))=\sum b_ia^i-\sum b_if(a)^i=\sum b_i[a^i-f(a)^i]
%\end{align*}
%Then
%$$0=b_1[a-f(a)]+b_2[a^2-f(a)^2]+...+b_n[a^n-f(a)^n]$$
%$$0=[a-f(a)][b_1+b_2(a+f(a))+b_3(a^2+af(a)+f(a)^2)+...+b_n(a^{n-1}+a^{n-2}f(a)+...+af(a)^{n-2}+f(a)^{n-1})]$$
%
%Now I only need to show that $b_1+b_2(a+f(a))+...+b_n(a^{n-1}+a^{n-2}f(a)+...+af(a)^{n-2}+f(a)^{n-1})$ cannot be equal to $0$.
%

\begin{problem}[7]{u}
Assume that $char(K)=p>0$ and $a\in \hat{K}$ is separable over $K$. Prove that $K(a)=K(a^p)$. (Hint: consider the minimal polynomial of $a$ over $K$.)
\end{problem}

Let $w_a\in K[X]$ be the minimal polynomial of $a$ and because $a$ is separable, then $w_a(x)$ has only simple roots in $\hat{K}$. Furthermore, we cannot have $w_a(x)\in K[X^p]$.

Frobenius function $F(x)=x^p$ goes brrrr? I know that $a^p$ is a root of $F(w_a(x))$ and that there exists a minimal $n$ such that $[a^p]^n=a$. Hecne, $x^{pn}-x$ is a polynomial with derivative equal to $-1$ that assumes $0$ at $x=a$. So, if I plug in $a^p$ it also is zero and the derivative does not change. So the minimal polynomial of $a^p$ must divide this badboy and because of this $w_{a^p}\notin K[X^p]$?

Bullshit $\uparrow$

$\supseteq$ is obvious

$\subseteq$

%my plan:
%\begin{enumerate}
    %\item $(\exists\;m)\;a^{p^m}=a$

    Let $w_a(x)$ be the minimal polynomial of $a$ over $K$ with $deg(w_a)=m$. Then $K(a)$ is a field with $p^m$ elements and $K(a)^*$ is a group of order $p^{m-1}$. So I have that $p^{m-1}=1$ and $p^{m-1}\cdot a=1\cdot a=a$. Noice.

    %$$[K(a):K]=[K(a):K(a^p)][K(a^p)$$

    %I know that $x^{p}-a^p=(x-a)^p$ has a root $a$. What if $a^p$ is not separable?

    %\item $a^p$ separable

    %I know that
    %$$v(x)=x^{p^m}-x$$
    %is zero at both $a$ and $a^p$ and its derivative is a constant $-1$. Hence, both $w_a|v$ and $w_{a^p}|v$.

    $$[K(a):K]=[K(a, a^p):K(a^p)][K(a^p):K]$$
    Since I already know that $a\in K(a^p)$, then I know that the minimal polynomial of $a$ over $K(a^p)$ has degree $1$. So $[K(a, a^p):K(a^p)]=[K(a):K(a^p)]=1$?
    %What if $a^p$ was not separable? Then $w_{a^p}\in K[X^p]$ and so if I take the Frokajsdhf function I have that $F(w_a(x))=w_{a^p}(x^p)$. So I would say that $w_a(a^p)=0$. But this would also mean that $w_a$

%\end{enumerate}



%I know that $a^p$ is a root of $F(w_a(x))$ and that there exists a polynomial $v(x)\in K[X]$ such that $v(a^p)=a$ (which can be changed to $q(x^p)=v(x^p)$ such that $q(a)=a$ and $q\in K[X^p]$). Ok, then I know that $q(x)-x$ has a root in $a$ and it is a single root since the derivative of 

%Oh, I must show that $a^p$ is separable.

%$$K(a)=K(a^p)$$

%$\supseteq$ is trivial

%$\subseteq$ is more difficult? Is it enough to show that $a^p$ is separable?

\begin{problem}[8]{}
\begin{enumerate}[label=(\alph*)]
    \item Prove that if $a\in L$ is radical over $K$, then $deg(a/K)=\min\{p^n\;:\;a^{p^n}\in K\}$
    \item Conclude that if a finite extension $K\subseteq L$ is radical, then its degree is a power of $p$ (here $p=char(K)$).
\end{enumerate}
\end{problem}

\begin{enumerate}[label=(\alph*), leftmargin=*]
    \item Ok, so $w_a(x)$, the minimal polynomial of $a$, has only one root in $\hat{K}$.

    I know that there exists aminimal $n$ such that $a^{p^n}\in K$ and that $w_a(x)$ divides $x^{p^n}-a^{p^n}$. From this I get that $deg(a/K)\leq p^n$.

    Now, let $k=deg(a/K)$, then $w_a=(x-a)^k$ and using binomial something something
    \begin{align*}
        (x-a)^k&=x^k-\binom{k}{1}x^{k-1}a+...+\binom{k}{k-1}xa^{k-1}+a^k\in K[X]
    \end{align*}
    so firstly, $k$ must be divisible by $p$ for $\binom{k}{m}x^{k-m}a^m$ to disappear if $a\notin K$. Secondly, $a^k$ must be the lowest power of $a$ to be inside of $K$.
\end{enumerate}

\begin{problem}[9]{}
Assume that $K\subseteq L,M\subseteq\hat{K}$ are field extenstions such that $L\cap M=K$. Prove that if 
$$(\forall\;K\subseteq_{fin}L_0\subset L)(\forall\;K\subseteq_{fin}M_0\subseteq M)\;[L_0(M_0):L_0]=[M_0:K],$$
then $[L(M):L]=[M:K]$.
\end{problem}

%Well, I think I have
$$[M_0(L_0):M_0][M_0:K]=[M_0(L_0):K]={\color{blue}[L_0(M_0):K]}=[L_0(M_0):L_0][L_0:K]$$

From the previous list I know that
$$[L(M):L][L:K]=[L(M):K]\leq[L:K][M:K]$$
and so
$$[L(M):L]\leq[M:K].$$
So now I need the $\geq$ inequality.


\begin{align*}
    [L(M):K]&=[L(M):L_0(M_0)][L_0(M_0):K]=[L(M):L_0(M_0)][L_0(M_0):L_0][L_0:K]=\\
    &=[L(M):L_0(M_0)][M_0:K][L_0:K]\geq[LM:L_0M_0][L_0M_0:K]=[LM:K]
\end{align*}
which implies that in this case
$$[LM:K]=[L:K][M:K]$$
and so
$$[LM:K]=[LM:L][L:K]$$

$$[L(M):K]=[L(M):M][M:K]$$
$$[L(M):K]=[L(M):L_0(M_0)][M_0:K]$$

\begin{align*}
    [LM:K]=[LM:M][M:K]=[LM:M][M:M_0][M_0:K]=[LM:M][M:M_0][L_0M_0:L_0]\\
    [LM:K]=[LM:L][L:L_0][L_0:K]
\end{align*}



%and so if this did not work, then I would have
%$$[M(L):M]\neq[L:K]$$


\begin{problem}[11]{d}
Assume that the numbers $m,n>1$ are relatively prime and $\zeta_n,\zeta_m\in\C$ are primitive roots of $1$ of degree $n,m$ respectively. Prove that $\Q(\zeta_n)\cap\Q(\zeta_m)=\Q$. (Hint: notice that $\Q(\zeta_n,\zeta_m)=\Q(\zeta_{mn})$. Rely on the fact that $\phi(mn)=\phi(m)\phi(n)$ for any co-prme $m,n$ (without proof).)
\end{problem}

Ok, so first I show that $\Q(\zeta_n,\zeta_m)=\Q(\zeta_{nm})$.

$\subseteq$

$$\zeta_{nm}=e^{\frac{2\pi}{nm}}$$
$$\zeta_{nm}^n=e^{\frac{2n\pi}{nm}}=e^{\frac{2\pi}{m}}=\zeta_m$$
$$\zeta_{nm}^m=e^{\frac{2m\pi}{nm}}=e^{\frac{2\pi}{n}}=\zeta_n$$

$\supseteq$
$$\zeta_{nm}=e^{\frac{2\pi}{nm}}=e^{\frac{2a\pi}{m}}e^{\frac{2b\pi}{n}}=e^{\frac{2\pi(an+bm)}{mn}}$$



Now I have that
$$\phi(nm)=[\Q(\zeta_{nm}):\Q]=[\Q(\zeta_n,\zeta_m):\Q(\zeta_n)][\Q(\zeta_n):\Q]=a\cdot \phi(n)$$
and if $\Q(\zeta_n)\cap\Q(\zeta_m)\neq\Q$, then there is some power of $\zeta_n$ that is shared with $\zeta_m$ and so it would mean that $[\Q(\zeta_n,\zeta_m):\Q(\zeta_n)]\neq\phi(m)$ which gives me a nice little contradiction.




\end{document}
