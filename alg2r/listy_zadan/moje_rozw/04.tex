\documentclass{article}

\usepackage{../../../lecture_notes}

\title{Problem List 4}
\author{Weronika Jakimowicz}
\date{\scriptsize sometime in the future}

\begin{document}
\maketitle\thispagestyle{empty}

\begin{problem}[1]{}
Calculate cyclotomic polynomials
$$F_1(X),F_2(X),F_4(X),F_8(X),F_{16}(X),F_{15}(X)$$
and then calculate their images in the ring $\Z_3[X]$, under the homomorphism $\Z[X]\to\Z_3[X]$ induced by the quotient homomorphism $\Z\mapsto\Z_3$. Which of them are irreducible over $\Z_3$?
\end{problem}

$F_1(X)=X-1$ is easy, then $X^2-1=(X-1)(X+1)$, so $F_2(x)=x+1$ because $x=1$ is not a primitive root of order $2$. 

With $F_4(X)$ I know that it cannot have degree $4$ because $2$ divides $4$ and cannot be counted in $\phi(4)$. I use the definition of $F_m$ from the lecture and write:
\begin{align*}
    F_4(x)&=(x-e^{\frac{\pi i}{2}})(x-e^{\frac{3\pi i}{2}})=x^2-x(e^{\frac{3\pi i}{2}}+e^{\frac{\pi i}{2}})+e^{2\pi i}=\\
    &=x^2+1
\end{align*}
However, I think I could get it from the fact that the roots of a cyclotomic polynomial $F_m$ are all the primitive roots of $1$ of order $m$. So
$$x^4-1=(x^2-1)(x^2+1)$$
and every root that comes from $x^2-1$ is not primitive, so only $x^2+1$ has primitive roots of order $4$.

A similar story is with $F_8:$
$$x^8-1=(x^4-1)(x^4+1)\implies F_8(x)=x^4+1$$

$F_{15}(x)$ should have degree $8$ and so here is a lot of computation to avoid multiplying $\prod\limits_{\substack{1\leq k<15\\gcd(k,15)=1}}(x-e^{k\frac{2\pi i}{15}})$ because why not
\begin{align*}
    x^{15}-1&=(x-1)(x^{14}+x^{13}+...+x+1)=\\
    &=(x-1)(x^{12}(x^2+x+1)+x^{9}(x^2+x+1)+...+x^2+x+1)=\\
    &=(x-1)(x^2+x+1)(x^{12}+x^9+x^6+x^3+1)=\\
    &=(x-1)(x^2+x+1)(x^{12}+x^{11}-x^{11}+x^{10}-x^{10}+...+x^3+x^2-x^2+x-x+1)=\\
    &=(x-1)(x^2+x+1)(x^8(x^4+x^3+x^2+x+1)-x^7(x^4+1)+x^6(x^4+...+1)-...+(x^4+x^3+x^2+x+1))=\\
    &=\underbrace{(x-1)}_{=F_1(x)}\underbrace{(x^2+x+1)}_{div.\;F_3(x)}\underbrace{(x^4+x^3+x^2+x+1)}_{div.\;F_5(x)}(x^8-x^7+x^6-x^5+x^4-x^3+x^2-x+1)
\end{align*}
$$\rotatebox{-90}{$\implies$}$$
$$F_{15}(x)=x^8-x^7+x^6-x^5+x^4-x^3+x^2-x+1$$

And now for the final boss because I messed up the order in which they should appear and am too lazy to change it: $F_{16}(x)$!!! I expect it to have order 8
$$x^{16}-1=(x^8-1)(x^8+1)\implies F_{16}(x)=x^8+1$$
\medskip

Images in $\Z_3[X]$:
$$F_1(x)=x-1\mapsto x+2$$
$$F_2(x)=x+1\mapsto x+1$$
$$F_4(x)=x^2+1\mapsto x^2+1$$
$$F_8(x)=x^4+1\mapsto x^4+1$$
$$F_{16}(x)=x^8+1\mapsto x^8+1$$
$$F_{15}(x)=x^8-x^7+x^6-x^5+x^4-x^3+x^2-x+1\mapsto x^8+2x^7+x^6+2x^5+x^4+2x^3+x^2+2x+1$$

Let me start from $F_{15}(x)$. I see that $2$ divides $F_{15}(x)$ and it is easy to check that $(x+1)^8=F_{15}(x)$ in $\Z_3$.

Now, $F_4(x)$, it has no roots in $\Z_3$ and because it is a quadratic polynomial, it cannot be divided by any other polynomial than one of degree $1$. Hence, it is irreducible.

$F_8(x)$ also has no roots in $\Z_3$ so we surely cannot split it into a linear polynomial and a polynomial of degree $3$. The only hope is in two polynomials of degree $2$. Let us check
$$(x^2+x+2)(x^2+2x+2)=x^4+{\color{yellow}2x^3}+{\color{blue}2x^2}+{\color{yellow}x^3}+{\color{blue}2x^2}+{\color{green}2x}+{\color{blue}2x^2}+{\color{green}x}+1=x^4+1$$

$F_{16}(x)$ is the worst because I cannot find a decomposition using simple tricks but showing that it is irreducible can be a little painful. I will leave it for now and most probably forget to return to it later. I apologize.

%Suppose that
%$$x^4+1=(x^2+ax+b)(x^2+cx+d)$$
%then
%$$\begin{cases}bd=1\\ad+bc=0\\c+a=0\\d+b+ac=0\end{cases}$$
%we have that $d=b^{-1}$ and $c=-a$. Hence, if we put this into the second equation, we get:
%$$0=ab^{-1}-ab=a(b^{-1}-b)$$
%\begin{enumerate}
%    \item if $a=0$, then $c=0$ and from the third equation: $b+b^{-1}=0\implies b^2+1=0\implies b^2=2$ which cannot happen if $b\in\{0,1\}$ and even powers of $2$ are $1$ modulo $3$ so this gives a contradiction
%    \item if $b^{-1}-b=0$, then $b=b^{-1}\implies b^2=1$ and the arguments is the same as above.
%\end{enumerate}
%This shows that $x^4+2$ is irreducible over $\Z_3$.


%Most of them are irreducible as they have no roots in $\Z_3$ but $F_{15}(2)=681\mod 3=0$

\begin{problem}[2]{}
Describe the normal closures of the following field extensions:
\begin{enumerate}[label=(\alph*)]
    \item $\Q[\sqrt[n]{2}]\supseteq\Q$
    \item $\Q(\sqrt[n]{X})\supseteq\Q(X)$
    \item $\C(\sqrt[n]{X})\supseteq\C(X)$
    \item $\Q[\zeta]\supseteq\Q$, where $\zeta$ is a primitive root of $1$ of degree $n>1$.
\end{enumerate}
(hint: in $(a)-(c)$ find the minimal polynomial, in $(c)$ use the fact that $\C$ is algebraically closed, in (b) notice that $X$ may be replaced by any transcendental number, this is not necessary, but it helps.)
\end{problem}

\begin{enumerate}[label=(\alph*), leftmargin=*]
    \item $\Q[\sqrt[n]{2}]\supseteq\Q$

    The minimal polynomial for $\sqrt[n]{2}$ over $\Q$ is $w(x)=x^n-2$ and its roots are of form
    $$a_k=\sqrt[n]{2}e^{\frac{2\pi i}{n}k}$$
    Now, I know that an extension of a field is normal if for any polynomial, if it has one root, then it has all the roots. So I need to find the minimal field that contains all those roots and $\Q[\sqrt[n]{2}]$ and it is
    $$L=\Q(a_1,...,a_n=\sqrt[n]{2})$$
    because we have already showed that it is the smallest field such that $a_1,...,a_n$ are contained within it.
\end{enumerate}

\begin{problem}[3]{}
Prove that every field extension of degree $2$ is normal.
\end{problem}

Let $K$ be a field and $f\in K[X]$ be a polynomial of degree $2$, WLOG $f$ is monic. We consider $K(a)$, where $f(a)=0$. Let us assume that
$$f(x)=\alpha_0+\alpha_1x+x^2$$
for $\alpha_0,\alpha_1\in K$. We know that if $a,b$ are solutions of $f$, then $a+b=-\alpha_1\implies b=-\alpha_1-a\in K$, hence both roots of $f$ are in our extension $K(a)$ and $K(a)$ is normal.
%$$\begin{cases}a+b=-{\alpha_1}\\ab={\alpha_0}\end{cases}$$

%Any other $\beta\in K(a)$ is either in $K$ or is of form $\beta=a^kb^l\gamma,k>0\;or\;l>0,\gamma\in K$. Then $x-\gamma\in K[X]\subseteq L[X]$ and 
%Any other $\beta\in K(a)$ is of form $\beta=a^kb^l\gamma,\gamma\in K$. We have the minimal polynomial of $\beta$ is $x-\beta\in L[X]$.




\end{document}
