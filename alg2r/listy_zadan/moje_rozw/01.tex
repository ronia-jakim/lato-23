\documentclass{article}

\usepackage{../../../notatki}

\title{\large Algebra 2R\smallskip\\ \textbf{Problem List 1}}
\author{\normalsize Weronika Jakimowicz}
\date{~~~}

\begin{document}
\maketitle
\thispagestyle{empty}

\subsection*{EXERCISE 1.}
\emph{\color{pink}Proof that $\C=\R[z]$ for every complex number $z\in \C\setminus\R$.}
\smallskip

To begin with, let us take any $z\in\C\setminus\R$ such that $z=ai$ for some $a\in\R$. We have that
$$\R[z]=\{f(z)\;:\;f\in\R[X]\}.$$

Let $I=(X^2+a^2)\normalsubgroup\R[X]$ be an ideal of $\R[X]$ generated by a polynomial with no real roots. We know that $\R[X]/I\cong \C$. 

This is because $\R$ is a field and so $\R[X]$ is an euclidean domain: if we take any $f\in\R[X]$ then we can write it as $f=v(X^2+a^2)+w$, where $w$ is of degree $0$ or $1$ ($<def(X^2+a^2)$) and so $f$ in $\R[X]/I$ is represented only by $w$. Now it is quite easy to map polynomials with real coefficients and maximal degree $1$ to $\C$, for example $f:\R[X]/I\to\C$ such that $f(aX+b)=ai+b$. Therefore $\R[X]/I\cong \C$.

Consider the evaluation homomorphism $\phi_z$ which maps $\R[X]\ni w\mapsto w(z)\in\R[z]$. We can see that $\ker(\phi_z)=(X^2+a^2)=I$. Therefore, by the fundamental theorem on ring homomorphism we have an isomorphism
$$f:Im(\phi_z)=\R[z]\to \R[X]/ker(\phi_z)=\R[X]/I$$
and as mentioned above, $\R[X]/I\cong\C$. Hence, $\R[z]\cong\C$.

\proofend

\subsection*{EXERCISE 2.}
\emph{\color{pink}Assume that $K\subset L$ are fields and $a,b\in L$. For a rational function $f(X)\in K(X)$ define $f(a)$ as ${g(a)\over h(a)}$, where $g,h\in K[X]$, $f=\frac gh$ and $h(a)\neq 0$, provided such $g,h$ exist. If not, $f(a)$ is undetermined. Prove that}
\smallskip

\emph{\color{pink}(a) if $f(X)\in K(X)$ and $f(a)$ is defined, then $f(a)$ is determined uniquely (does not depend on the choice of $g, h$)}
\smallskip

% Let $g,h\in K[X], h(a)\neq0$ be polynomials that have no common roots and $f={g\over h}$. This means that the quotient ${g\over h}$ cannot be simplified. Now, let us take another pair of polynomials $g',h'\in K[X], h'(a)\neq 0$ such that $f={g'\over h'}$. I claim that 
% $${g(a)\over h(a)}={g'(a)\over h'(a)}.$$

% We know that $f={g\over h}={g'\over h'}$ and hence
% $$gh'=hg'$$

%Because $K$ is a field, I know that $K[X]$ is an euclidean domain. 

%such that $g,h$ have no common divisor. This means that the quotient ${g\over h}$ cannot be simplified further. Because $K[X]$ is an euclidean domain, I can write $g=p\cdot h+r$, where $deg(r)<deg(h)$ and $r\neq 0$.

Suppose by contradiction that $f(a)$ depends on which $g,h$ we choose. That means that there exist $g,h, g',h'\in K[X], h(a)\neq 0, h'(a)$ such that $f={g\over h}={g'\over h'}$ but ${g(a)\over h(a)}+c= {g'(a)\over h'(a)}$, where $c\in L\setminus\{0\}$. 

From $f={g\over h}={g'\over h'}$ we get that $g\cdot h'=g'\cdot h$ and in particular
\begin{align*}
    (gh')(a)&=(g'h)(a)\\
    g(a)h'(a)&=g'(a)h(a)\\
    g(a)h'(a)-g'(a)h(a)=0
\end{align*}

From the assumption that $f(a)$ depends on the choice of polynomials we get that 
\begin{align*}
    {g'(a)\over h'(a)}&={g(a)\over h(a)}+c\\
    g'(a)h(a)&=g(a)h'(a)+ch'(a)\\
    g'(a)h(a)-g(a)h'(a)&=ch'(a)\neq 0
\end{align*}
Which is a contradiction because $c\neq 0$, $h'(a)\neq 0$ and we have no zero divisors.
\smallskip

%$$f(a)={g(a)\over h(a)}={p(a)\cdot h(a)+r(a)\over h(a)}=p(a)+{r(a)\over h(a)}$$

\emph{\color{pink}(b) $K(a)=\{f(a)\;:\;f\in K(X)\text{ and }f(a)\text{ is defined}\}$}

We know that $K(a)$ is a subfield of $L$ that is generated by $K\cup\{a\}$. Let us label this field as $L'$. We will show that $L'=K(a)$.

$L'\subseteq K(a)$

Let us take any $x\in L'$. Then $x$ is a finite linear combination of elements from $K$ and $\{a. a^{-1}\}$:
$$x=\sum\limits_{0\leq k\leq n}\alpha_k a^{i_kk},\quad i_k\in\{1,-1\},\;\alpha_k\in K.$$
We need to change this into a rational function. Take $p_k\in K[X]$ such that $p_k(X)=\alpha_kX^k$. We have that
$$x=\sum\limits_{0\leq k\leq n}p_k(a^{i_k}).$$
It is clear that when working with rational functions we may say that $p_k(a^{-1})={1\over p_k'(a)}$ where $p_k(X)=\alpha_k^{-1}X^k$.
\begin{align*}
    x&=\sum\limits_{0\leq k\leq n}p_k(a^{i_k})={\sum\limits_{0\leq k\leq n}p_k(a)\prod\limits_{\substack{0\leq l\leq n,\\ i_l=-1}}p_k'(a)\over\prod\limits_{\substack{0\leq k\leq n\\i_k=-1}}p_k'(a)}\in K(a)
\end{align*}

% a + b^-1 + c + d^-1 = a + 1/b + c + 1/d = ab/b + 1/b + cd/d + 1/d = abd/bd + d/bd + cdb/bd + b/bd = (abd+d+cdb+b)/bd

$K(a)\subseteq L'$

Let us take any $f\in K(X)$ such that $f(a)$ is defined. We may write $f={g\over h}$ for $g,h\in K[X]$ and $h(a)\neq 0$. We have that $g(a)\in L'$ and $h(a)\in L'$. Therefore, ${g(a)\over h(a)}=g(a)\cdot[h(a)]^{-1}\in L'$.
\smallskip

\emph{\color{pink}(c) $K(a, b)=(K(a))(b)$}
\smallskip

Let 
$$I_{ab}=I((a, b)/K[x, y])$$
$$I_a=I(a/(K[y])[x])$$
$$I_b=I(b/(K(a))(y))$$
and $j_a,j_b, j_{ab}$ are quotient functions defined as below. We know that $ker(j_a)=I_a$, $ker(j_b)=I_b$ and $ker(j_{ab})=I_{ab}$. Let $\phi_a$ be an evaluation function that substitutes only one variable:
$$\phi_a:K(x, y)\to (K(a))(y)$$
$$\phi_a(f(x, y))=f(a, y)$$
that is $\phi_a$ returns a rational function with changed coefficients. $\phi_b, \phi_{ab}$ are defined as evaluation functions without such modifications.

\begin{center}\begin{tikzcd}
    K(x, y)/I_{a, b} \arrow[d, "\cong"]  &  K(x, y) \arrow[l, "j_{ab}" above] \arrow[r, "j_a"] \arrow[dl, "\phi_{ab}"] \arrow[to=2-3, "\phi_a"] \arrow[to=4-3, "\psi", bend right=20] &  K(x, y)/I_{a} \arrow[d, "\cong"]\\
    K(a, b)        &                & (K(a))(y) \arrow[d, "j_b"] \arrow[to=4-3, bend left=60, "\phi_b"]\\
    & & (K(a))(y)/I_b \arrow[d, "\cong"]\\
    & & (K(a))(b) \arrow[to=2-1, leftrightarrow, dashed, bend left=20, "f"]
\end{tikzcd}\end{center}

Function $\psi$ is a ring homomorphism defined as composition of $\phi_a$ and $\phi_b$:
$$\psi:K(x, y)\to(K(a))(y)$$
$$\psi=\phi_b\circ\phi_a$$

For $f$ to be an isomorphism
$$f:(K(a))(b)\to K(a, b)$$
we need to show that $ker(\phi_{ab})=\ker(\psi)$ because then
\begin{center}
    \begin{tikzcd}[row sep=large]
        & K(x, y)/ker(\phi_{ab}) = K(x, y)/ker(\psi) \arrow[dl, "\cong" above] \arrow[dr, "\cong"] &\\
        K(a, b) \arrow[rr, leftrightarrow, dashed, "\cong" above] & & (K(a))(b)
    \end{tikzcd}
\end{center}

$ker(\phi_{ab})=ker(\psi)$

$\subseteq$

$f\in ker(\phi_{ab})$ means that $f(a, b)=0$. That is, either of the following is true for any $x, y\in K$

\indent $f(a, b)=0$ this directly implies that $f\in ker(\psi)$.

\indent $f(a, y)=0$ the same as above.

\indent $f(x, b)=0$ we know that for any $x\in K$ $f(x, b)=0$ then for $x=a$ this is also true and so $f(a, b)=0$ and $f\in ker(\psi)$.

$\supseteq$

$f\in ker(\psi)$ means that $f(a, b)=0$ or $f(a, y)=0$. This means that $f\in ker(\phi_{ab})$.

Therefore, there exists an isomorphism $K(a, b)\cong (K(a))(b)$.

\subsection*{EXERCISE 3.}
\emph{\color{pink}Assume that $K\subseteq L$ are fields and $f_1,...,f_m\in K[X_1,...,X_n]$ have degree $1$.}

\emph{\color{pink}(a) Prove that if the system of equations $f_1=...=f_m=0$ has a solution in $L$ then it has a solution in $K$. (hint: use linear algebra).}
\smallskip

Let
$$f_i=\sum\limits_{1\leq k\leq n}b_{i, k}X_k$$
for $i=1,...,m$.

%\begin{center}
% Take $\overline a=(a_1,..., a_n)$ be a solution from $L$. We have
% $$0=f_i(\overline a)=\sum\limits_{1\leq k\leq n}b_{i, k}a_k.$$
% This is a linear combination of elements from $L$ and therefore we have three possibilities:

% \indent 1. $(\forall\;k=1,...,n)\;b_{i,k}=0$ and so this equation does not influence the remaining $(m-1)$ polynomials. From those remaining polynomials either one has non-zero coefficients $b_{j, k}$ (in this case we jump to case 2 or 3) or all polynomials from the set of equations are trivial and any sequence from $K$ is a solution.

% \indent 2. $(\forall\;k=1,...,n)\;a_k=0$ and hence $\overline a=(0,...,0)\in K^n$ is a solution.

% \indent 3. $a_k$ and $b_{i, k}$ are linearly dependent and
% \begin{align*}
%     0&=\sum\limits_{1\leq k\leq n}a_kb_{i, k}\\
%     0&=a_1b_{i, 1}+\sum\limits_{2\leq k\leq n}a_kb_{i, k}\\
%     -a_1b_{i, 1}&=\sum\limits_{2\leq k\leq n}a_kb_{i, k}\\
%     b_{i, 1}&=\sum\limits_{2\leq k\leq n}[a_k(-a_1)^{-1}]b_{i, k}
% \end{align*}
% The last operation is permitted because we are inside a field and $-a_1$ is non-zero, therefore it has a multiplicative inverse. We have that 
% $$\sum\limits_{2\leq k\leq n}[a_k(-a_1)^{-1}]b_{i, k}\in K$$
% and so $a_k(-a_1)^{-1}\in K\implies a_1,a_k\in K$.

% \proofend
%\end{center}

We are working on linear equations, therefore we can construct a matrix that stores the same information as the system of equations $f_1=...=f_m$. Let
$$f_i=\sum\limits_{1\leq k\leq n}b_{i, k}X_k$$
for $i=1,...,m$. The matrix representation of this system of equations is:
$$\begin{bmatrix}
    b_{1,1} & b_{1, 2} & b_{1, 3} &... &b_{1, n-1} & b_{1, n}\\
    b_{2,1} & b_{2, 2} & b_{2, 3} &... &b_{2, n-1} & b_{2, n}\\
    ...     &   ...    & ...      &... & ...       & ...\\
    b_{m,1} & b_{m, 2} & b_{m, 3} &... &b_{m, n-1} & b_{1, n}
\end{bmatrix}X=0.$$
Using Gaussian algorithm, we can create an upper triangular matrix with coefficients from $K$. The solution would be found by backwards substitution. That is, $a_n$ would be in the bottom right corner of the matrix and it is an element of $K$ because such are the coefficients within my matrix. Then $a_{n-1}$ would be a combination of $a_n$ with two elements of $K$, hence it would still be in $K$ and so on. Each $a_i$ would be a linear combination of elements from $K$ and $a_k$, $k<i$, which we know are in $K$.

\proofend

\emph{\color{pink}(b) Does $K$ contain a generic solution of this system (over $K$)?}
\smallskip

From Remark 1.4. we know that $\overline{a}$ is a generic solution $\iff$ for any other solution $\overline a'\in K^n$ we have only one homomorphism $h:K[\overline a]\to K[\overline a]$ such that $h(\overline{a})=h(\overline{a}')$ and $h\obciete K=id_K$. It is suffice to notice that because $K[\overline a]$ and $K[\overline a']$ are evaluations of polynomials with coefficients from $K$, then they are finite combinations of elements from $K$ and therefore $K[\overline{a}]\subseteq K$ and $K[\overline a']\subseteq K$. Therefore $h\subseteq id_K$ and thus is unique.

\subsection*{ZADANIE 5.}
\emph{\color{pink}Which of the following solutions of the equation $X_1^2-X_2^3=0$ in the field of rational functions $\C(X)$ are generic over the field $\Q$?}

\emph{\color{pink}(a) $(1, 1)$}
\smallskip

Ok, so a solution is generic if $\{g\in K[X]\;:\;g(a)=0\}=(f_1,...,f_m)$. So maybe I should look for all polynomials that $g(1, 1)=0$?
$$(x-1),(y-1)$$
And it is quite difficult to make a linear polynomial from a polynomial of order $3$.

\emph{\color{pink} $(\sqrt[6]{8}, \sqrt[6]{4})$}
\smallskip

$$(x^6-8)=(x^2-2)(x^4+2x^2+4)$$
$$(y^6-4)=(y^3-2)(y^3+2)$$

So the 
$$I(\overline a/K=\Q)=((x^2-2), (y^3-2))?$$ 
Suppose that $((x^2-2), (y^3-2))=((x^2-y^3))$. Then we would have that for some $p, q\in \Q[X]$
$$\begin{cases}
    x^2-2=p(x^2-y^3)\\
    y^3-2=q(x^2-y^3)
\end{cases}$$
But like, $p$ cannot reduce the degree of $y$ ans $q$ cannot reduce the degree of $x$ so this is absurd.

\emph{\color{pink} $(1, \cos\frac23\pi+i\sin\frac23\pi)$}

Osz to jest w zespo.

\subsection*{ZADANIE 6.}
\emph{\color{pink}Assume that $f\in K[X]$ is irreducible, $deg(f)=n>0$, $char(K)=0$ and $L$ is the splitting field of polynomial $f$ over $K$. Prove that the field $L$ has at least $n$ distinct automorphisms.}
\smallskip

First of all, I need $f$ to have $n$ distinct roots in $L$. 

% Suppose that $f$ has a root $a$ that is repeated. Then we know from analysis that $f'(x)$, the derivative of $f$, is also equal to zero. 

% $f,f'\in K[X]$ which is an euclidean domain because $K$ is a field. Therefore there exist $p,r\in K[X]$, $deg(r)<n$ such that
% $$f(x)=f'(x)p(x)+r(x)$$

% Let
% $$f(x)=(x-a)^2f_1(x)$$
% where $deg(f_1)=n-2$. Then, from the product rule,
% $$f'(x)=2(x-a)f_1(x)+(x-a)^2f_1'(x)$$
% $$(x-a)f'(x)=2(x-a)^2f_1(x)+(x-a)^3f_1'(x)$$
% $$(x-a)f'(x)=2f(x)+(x-a)^3f_1'(x)$$
% $$(x-a)f'(x)-(x-a)^3f_1'(x)=2f(x)$$
% $$2^{-1}(x-a)f'(x)-2^{-1}(x-a)^3f_1'(x)=f(x)$$
% That would mean that $p(x)=(x-a)$ and $(x-a)^3f_1'(x)=r(x)$. We know that the remainder of Euclidean division is uniquely defined. Bu let us look at the degree of $(x-a)^3f_1'(x)$. Since $deg(f_1)=n-2$, we have that $deg(f_1)=n-3$. Since the leading coefficient of $f_1$ and of $(x-a)^3$ are nonzero, we have that 
% $$deg((x-a)^3f_1'(x))=deg((x-a)^3)+ deg(f_1'(x))=3+(n-3)=n$$
% which is a contradiction with $deg(r(x))<n$. Hence $f$ has $n$ different roots.

If $a$ is at least a double root of $f$ then $f'(a)=0$. Let
$$f(x)=\alpha_nx^n+\alpha_{n-1}x^{n-1}+...+\alpha_1x+\alpha_0$$
where $\alpha_n\neq 0$. Then, the derivative is
$$f'(x)=n\alpha_nx^{n-1}+(n-1)\alpha_{n-1}x^{n-1}+...+\alpha_1$$
and because we $char(K)=0$, then $n\alpha_n=\alpha_n+...+\alpha_n\neq 0$. Thus, $f'(x)\not\equiv 0$.

We know that $f\in K[X]$ is irreducible and $f'$ has lower degree, hence $f'$ does not divide $f$. From Bezout's identity I get that there exist $p, q\in K[X]\setminus\{0\}$ such that
$$fp+f'q=1.$$
If $f'(a)=0$, then
$$0=f(a)p(a)+f'(a)q(a)=1$$
which is a contradiction, hence $f'(a)\neq 0$ and $f$ has only simple roots.

Let $\phi\in Aut(L)$ such that $\phi_{\obciete K}=id_K$. Let $a_1,...,a_n\in L$ be roots of $f$. Then for $i=1,...,n$ we have
\begin{align*}
    0=\phi(f(a_i))&=\phi\Big(\sum\limits_{k=0}^n\alpha_ka_i^k\Big)=\sum\limits_{k=0}^n\phi(\alpha_ka^k_i)=\\
    &=\sum\limits_{k=0}^n\phi(\alpha_k)\phi(a_i^k)=\sum\limits_{k=0}^n\alpha_k\phi(a_i)^k=f(\phi(a_i))
\end{align*}
which implies that we can define an automorphism on $L$ by simply mapping $a_i$ to any of the roots of $f$ and keeping the coefficients from $K$ in place. This gives us with at least $n$ such permutations of roots.

\subsection*{ZADANIE 7.}
\emph{\color{pink}Assume that $K_1\subseteq K_2\subseteq K_3\subseteq...$ is an ascending sequence of fields. Verify in detail that $\bigcup K_n$ is also a field, containing $K_1,K_2,...$ as subfields.}
\smallskip

\indent 1. Group under addition:

Let $x, y\in K$, then $x,y\in K_i$ and so $x+y\in K_i,-x\in K_i, -y\in K_i$ because $K_i$ is a field. Therefore, $x+y,-x\in K$.

\indent 2. Group under multiplication:

Let $x,y\in K$ then $x,y\in K_i$ and so $xy\in K_i$ and $x^{-1}\in K_i$. Therefore, $xy\in K$ and $x^{-1}\in K$.

\indent 3. Identity element \& the zero:

I guess it boils down to showing that if $K\subseteq L$ are fields, then $1_K=1_L$. Consider $x^2-x\in K[X]$. We know that $1_K, 0_K$ are the two solutions in $K$ and $1_L, 0_L$ are the two solutions in $L$. Therefore $\{1_K,0_K\}=\{1_L, 0_L\}$. If $1_K=0_L$, then for any $x\in K\subseteq L$ we would have that $x=1_Kx=0_Lx=0_L$ and so $x$ would be a zero divisor in $L$ which cannot be true.

\indent 4. No zero divisors?

Suppose that $x,y\in K\setminus\{0\}$ such that $xy=0$. But $x,y\in K_i$ and so $xy\in K_i$ and because $K_i$ is a field, then $xy\neq 0$

\subsection*{EXERCISE 8.}
\emph{\color{pink}Prove that the set $\{\sqrt{p}\;:\;p\text{ is a prime number}\}$ is linearly independent over the field $\Q$.}
\smallskip

Consider a polynomial $a_1x_1+a_2x_2+...+a_nx_n\in\Q[x_1,...,x_n]$

\end{document}