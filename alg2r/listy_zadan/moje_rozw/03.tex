\documentclass{article}

\usepackage{../../../lecture_notes}

\title{Problem List 3\smallskip\\{\normalsize Algebra 2r}}
\author{Weronika Jakimowicz}
\date{}

\begin{document}
\maketitle
\thispagestyle{empty}

\begin{problem}[1]{}
Let $K$ be a field.
\begin{enumerate}[label=(\alph*)]
    \item Prove that the field extension $L\supseteq K$ is transcendental, where $L=K(X)$ is the field of rational functions in varriable $X$ over $K$.
    \item Let $M=L[\sqrt{X}]$ be an algebraic extension of the field $L$ by an element $Y=\sqrt{X}$ such that $Y^2-X=0$ in the field $M$. Prove that $M$ and $L$ are isomorphic over $K$.
\end{enumerate}
\end{problem}

\emph{(b)}

We kave $L=K(X)$ and $M=L[Y]$ and $Y^2-x=0$. We claim that $L\cong_KM$.
$$f_1:L\to M$$
$$f_1(p)=p(Y)$$
$$f_2:M\to L$$
$$y\mapsto x$$
$$x\mapsto x^2$$

So take a function $h\in L$, then $f_2(f_1(h))=f_2(h(y))=h(x)$

\begin{problem}[2]{}
Let $K$ be a field.
\begin{enumerate}[label=(\alph*)]
    \item Let $g\in K(X)\setminus K$. Prove that $X$ is algebraic over the field $K(g)$. In particular $[K(X):K(g)]<\infty$. What is the degree of this extreme?
    \item For $g$ as in $(c)$ prove that $K(g)$ is isomorphic with $K(X)$ over $K$.
\end{enumerate}
\end{problem}

\emph{(a)}

First of all, we know that there exist $p,q\in K[Y]$ such that
$$g=\frac{p}{q}$$
$$gq=p$$
and so
$$g(x)q(y)-p(y)=w(y)\in K(g)[Y]$$
Now consider $w(x)$
$$w(x)=g(x)q(x)-p(x)=p(x)-p(x)=0$$
hence, $X$ is algebraic over $K(g)$.

%$deg(w)=max(deg(p),deg(q))$.
$[K(X):K(g)]=max(deg(p),deg(q))$.Because $\frac{1}{g}$ and $g$ generate the same extension, then we can assume that $deg(p)\geq deg(q)=k$. It is obvious that $deg(w)\leq k$, we need to show that $deg(w)\geq k$.

Take $(1,...,x^{k-1})$ which is linearly independent. We take some coefficients $a_0,...,a_{k-1}\in K(g)$  such that
$$a_0+a_1x+...+a_{k-1}x^{k-1}=0$$
Now, multiply by all denominators of $a_i$ to obtain
$$a_0'+a_1'x+...+a_{k-1}'x^{k-1}=0$$
Therefore, $a_i'$ are all polynomials and we have:
$$a_i'=b_i+\frac{p}{q}R_i(\frac{p}{q}),$$
where $b_i\in K$: we just take a constant term and remove $x$ from it.

Notice that there exists $b_i\neq0$, otherwise we could just divide the whole thing by $\frac{p}{q}$ and repeat the process one more time.

AAAAAAAAAAAAAAAAAAAAAAAAAAAAAAAAAAAAAAAAAAAAAAAAAAAAAAAAAAA



\begin{problem}[3]{d}
Let $v_1,...,v_n$ be vertices of a regular $n$-gon inscribed in a circle on the plane $\R^2$ with equation $x^2+y^2=1$. What is the linear dimension over $\Q$ of the system of vectors $v_1,...,v_n$.
\end{problem}

Without the loss of generality, I will consider polygons with one vertex in $(1, 0)$. Then, the remaining vertices are in $(\cos\frac{2\pi k}{n},\sin\frac{2\pi k}{n})$, for $k=1,...,n-1$. Now, let me switch where I live and let us consider roots of
$$x^n-1.$$
We have $n$ roots $z_1,...,z_n$ in $\C$. Notice, that $z_k=\cos\frac{2\pi k}{n}+i\sin\frac{2\pi k}{n}$ and adding complex numbers works almost like adding vectors in $2D$. The minimal polynomial over $\Q$ of each of $z_k$ is $F_n(x)$. Therefore, $dim(v_1,...,v_n)=dim(z_1,...,z_n)=\phi(n)$, where $\phi$ is Euler's function.

%\begin{problem}[4]{u}
%    Assume that $K\supseteq F(p)$ is a finite field extension of $F(p)$. Assume that $a\in K$ is a primitive root of $1$ of degree $m$. Let $n$ be the smallest natural number $>0$ such that $m|p^n-1$.
%    \begin{enumerate}[label=(\alph*)]
%        \item Prove that $n$ equals the degree of $a$ over $F(p)$
%        \item Prove that $n|\phi(m)$. Give an example where $n<\phi(m)$.
%    \end{enumerate}
%\end{problem}

Well, I think I kinda showed it before XD



\begin{problem}[6]{d}
    Find the minimal polynomials over $\Q$ fot the following numbers:
    
    (a) $\sqrt{2}+\sqrt{3}$
\end{problem}
\begin{align*}
    x-(\sqrt{2}+\sqrt{3})=0\\
    x-\sqrt{2}=\sqrt{3}\\
    (x-\sqrt{2})^2=3\\
    x^2-x2\sqrt{2}+2=3\\
    x^2-1=x2\sqrt{2}\\
    (x^2-1)^2=8x^2\\
    x^4-2x^2+1=8x^2\\
    {\color{blue}x^4-10x^2+1=0}
\end{align*}


\begin{problem}[7]{d}
Prove (using Liouville Lemma) that the number
$$\sum\limits_{n=1}^\infty \frac{1}{2^{n!}}$$
is transcendental. (the real numbers, whose transcendence follows from Liouville Lemma are called Liouville numbers).
\end{problem}

Liouville Lemma states that if $a\in\R$ is an algebraic number of degree $N>1$, then there exists $c\in\R_+$ such that for all $\frac{p}{q}\in\Q$ the following is true:
$$\left|a-\frac{p}{q}\right|\geq\frac{c}{q^N}$$
If a number fails to meet this criterion, then it is called transcendental.

Ok, so I have no clue what the degree of my number is, but let me assume that it is some $N\in\N$. Now, let
$$p=\sum\limits_{n=1}^{N+k}2^{(N+k)!\cdot n!}.$$
Then, we have that
$$\sum\limits_{n=1}^\infty\frac{1}{2^{n!}}=\frac{p}{2^{(N+k)!}}+\sum\limits_{n=N+k}^\infty \frac{1}{2^{n!}}$$
with $q=2^{(N+k)!}$. From this we get
\begin{align*}
    \left|\sum\limits_{n=1}^\infty\frac{1}{2^{n!}}-\frac{p}{2^{(N+k)!}}\right|&=\left|\sum\limits_{n=N+k+1}^\infty\frac{1}{2^{n!}}\right|\leq\text{(\Coffeecup)}
\end{align*}
and notice that
\begin{align*}
    \sum\limits_{n=N+k+1}^\infty\frac{1}{2^{n!}}\leq \sum\limits_{n=N+k+1}^\infty\frac{1}{2^n}=\frac{1}{2^{(N+k+1)!}}\frac{1}{1-\frac12}=\frac{2}{2^{(N+k+1)!}}
\end{align*}
\begin{align*}
    \text{(\Coffeecup)}\leq\frac{2}{2^{(N+k+1)!}}=\frac{2}{q^{N+k+1}}<\frac{1}{q^{N+k}}  
\end{align*}
for any $k\in\N$ and so we cannot choose one universal $c$ such that this inequality changes to $\geq$ for all. Thus, the number from the problem is a Liouville number.















\end{document}
