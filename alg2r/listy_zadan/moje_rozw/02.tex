\documentclass{article}

\usepackage{../../../notatki}

\title{\large Algebra 2R\smallskip\\ \textbf{Problem List 2}}
\author{\normalsize Weronika Jakimowicz}
\date{~~~}

\begin{document}
\maketitle\thispagestyle{empty}

\subsection*{EXERCISE 3.}
{\color{pink}Assume that $f:K\to K$ is a non-zero endomorphism (e.g. the Frobenius function)/ Prove that $Fix(f)=\{x\in K\;:\;f(x)=x\}$ is a subfield of the field $K$}
\smallskip



\subsection*{EXERCISE 4.}
{\color{pink}Assume that $K$ is a finite field, characteristic $p$.

(a) Prove that every irreducible polynomial $f\in K[x]$ divides the polynomial $w_n(x)=x^n-1$ for some $n$ not divisible by $p$. (hint: prove that the splitting field of $f$ is finite.)}

Let $f$ be an irreducible polynomial $f\in K[x]$ of degree $n=deg(f)>0$. Without loss of generality assume that $f$ is monic. Let $a\in L\supseteq K$ be one of its roots, where $L$ is the splitting field of $f$ over $K$. Because $K$ is finite, i can say that $|K|=p^k$.
\smallskip

"Proof graph"
\begin{illustration}
    \node (1) at (0,0) {irreducible $\implies$ minimal};
    \node (2) at (0, -1.5) {$[L:K]=n<\infty$};
    \node (3) at (0, -3) {$w_m(a)=0$};
    \node (4) at (0, -4.5) {$f|w_m$};
    \draw[->] (1)--(2);
    \draw[->] (2)--(3);
    \draw[->] (3)--(4);
\end{illustration}
\smallskip

\phantomsection
\label{lemaczysko}
\acc{Lemaczysko:} \emph{An irreducible monic polynomial $f\in K[X]$ is the minimal polynomial for some root $a$, $f(a)=0$}

$$w_m(x)=x^m-1=(x-1)\underbrace{(x^{m-1}+x^{m-2}+...+x+1)}_{v_m(x)}$$

As $K$ is a field, the ring $K[X]$ is an euclidean domain. Let us suppose that $h\in K[X]$ is the minimal polynomial of $a$ in $K$ such that $deg(h)<deg(f)$. We have that there exists $p,r\in K[X]$ such that
$$f=hp+r$$
but notice that $f(a)=0$ and $h(a)=0$, so $r=0$ and we would have $f=hp$ but $f$ was irreducible.
\medskip

\acc{Lemat:}\emph{ The splitting field of $f$ is finite.}

The ideal
$$I(a/K)=\{w\in K[X]\;:\;w(a)=0\}=(f)$$
because $f$ is irreducible. We showed that $f$ is minimal in \hyperref[lemaczysko]{\acc{Lemaczysko}} and so from Remark 4.5. (\hyperref[remark:4:5]{below}) we have that $[L:K]=deg(f)=n$.

\acc{Lemacik:} \emph{This is not really a lemma but the third step in the diagram: $w_m(a)=0$ for $m=p^{kn}-1$.}

Now let us look at $L^*$, which is the multiplicative group of $L$. Because $L$ was a field, we know that 
$$|L|=p^{kn}=p^l$$ 
($[L:K]=n$ and there were $p^k$ elements in $K$) and that 
$$|L^*|=|L\setminus\{0\}|=p^l-1.$$ 
Furthermore, we know that every finite group is isomorphic to the field $\Z_p$ so we must have that $L^*$ is a cyclic group with $a\in L^*$ as one of its generators. We know that $a^{p^l}=a$ will "loop back" inside of $L^*$ and so $a^{p^l-1}=1$ inside of $L^*$. This gives us the following equality:
$$w_{p^l-1}(a)a^{p^l-1}-1=1-1=0$$
with $p\nmid p^{l}-1$.

\acc{Lemaciuś:} \emph{Once again not a lemma but showing that $f$ divides $w_m$, $m$ as above.}

What remains now is to show that $f|w_m$. Suppose that this is untrue and that their "gcd" is equal to 1. Then by Bezout's identity we have that there exist $c,d\in K[X]$ such that
$$f(x)c(x)+w_m(x)d(x)=1$$
but for $x=a$ we would have $0=1$ which is a contradiction. Hence, one has to divide the other. It is quite logical that the minimal polynomial cannot have degree higher than the number of elements in a field and so $n\leq p^k<p^{kn}-1$ and so $deg(f)<deg(w_m)$ implying that $f|w_m$.

\phantomsection
\label{remark:4:5}
{\color{orange}Remark 4.5.}
\emph{ Suppose that $I(a/K)=(f)$ and $f$ is monic. Then:
\begin{enumerate}
    \item $f$ is the minimal monic polynomial such that $f(a)=0$
    \item $deg(f)=[K(a):K]$, thus the degree of the minimal polynomial is equal to the dimension of the linear space $K(a)$ over $K$.
\end{enumerate}}
\medskip

\end{document}