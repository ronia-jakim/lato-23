\documentclass{article}

\usepackage{../../../notatki}

\title{\large Algebra 2R\smallskip\\ \textbf{Problem List 2}}
\author{\normalsize Weronika Jakimowicz}
\date{~~~}

\color{black}
\pagecolor{white}
\definecolor{pink}{HTML}{000000}
\definecolor{blue}{HTML}{1520A6}
\definecolor{orange}{HTML}{CC5801}

\begin{document}
\maketitle\thispagestyle{empty}

\subsection*{EXERCISE 3.}
{\color{pink}Assume that $f:K\to K$ is a non-zero endomorphism (e.g. the Frobenius function). Prove that $Fix(f)=\{x\in K\;:\;f(x)=x\}$ is a subfield of the field $K$}
\smallskip

\begin{enumerate}
    \item Closure under addition:

    Let $x,y\in Fix(f)$. Then
    $$f(x+y)=f(x)+f(y)=x+y$$
    and so $x+y\in Fix(f)$

    \item Closure under multiplication:

    Let $x,y\in Fix(f)$. Then
    $$f(xy)=f(x)f(y)=xy$$
    and $xy\in Fix(f)$.

    \item Identity element, zero: in every homomorphism $0\mapsto 0$ and $1\mapsto 1$ and
    so $0,1\in Fix(f)$.

    \item Multiplicative inverse:
    
    Let $x\in Fix(f)$. Then
    $$f(x^{-1})=f(x)^{-1}=x^{-1}$$
    and so $x^{-1}\in Fix(f)$.
\end{enumerate}

I think that closure under addition could also be tackled with using a function $\phi:K\to
K$ $\phi(x)=f(x)-x$ as $Fix(f)=ker(\phi)$, kernel of any homomorphism is an ideal, thus
$Fix(f)$ is closed under addition and multiplication.

\subsection*{EXERCISE 4.}
{\color{pink}Assume that $K$ is a finite field, characteristic $p$.

(a) Prove that every irreducible polynomial $f\in K[x]$ divides the polynomial $w_n(x)=x^n-1$ for some $n$ not divisible by $p$. (hint: prove that the splitting field of $f$ is finite.)}

Let $f$ be an irreducible polynomial $f\in K[x]$ of degree $n=deg(f)>0$. Without loss of generality assume that $f$ is monic. Let $a\in L\supseteq K$ be one of its roots, where $L$ is the splitting field of $f$ over $K$. Because $K$ is finite, i can say that $|K|=p^k$.
\smallskip

"Proof graph"
\begin{illustration}
    \node (1) at (0,0) {irreducible $\implies$ minimal};
    \node (2) at (0, -1.5) {$[L:K]=n<\infty$};
    \node (3) at (0, -3) {$w_m(a)=0$};
    \node (4) at (0, -4.5) {$f|w_m$};
    \draw[->] (1)--(2);
    \draw[->] (2)--(3);
    \draw[->] (3)--(4);
\end{illustration}
\smallskip

\phantomsection
\label{lemaczysko}
\acc{Lemaczysko:} \emph{An irreducible monic polynomial $f\in K[X]$ is the minimal polynomial for some root $a$, $f(a)=0$}

%$$w_m(x)=x^m-1=(x-1)\underbrace{(x^{m-1}+x^{m-2}+...+x+1)}_{v_m(x)}$$

As $K$ is a field, the ring $K[X]$ is an euclidean domain. Let us suppose that $h\in K[X]$ is the minimal polynomial of $a$ in $K$ such that $deg(h)<deg(f)$. We have that there exists $p,r\in K[X]$ such that
$$f=hp+r$$
but notice that $f(a)=0$ and $h(a)=0$, so $r=0$ and we would have $f=hp$ but $f$ was irreducible.
\medskip

\acc{Lemat:}\emph{ The splitting field of $f$ is finite.}

The ideal
$$I(a/K)=\{w\in K[X]\;:\;w(a)=0\}=(f)$$
because $f$ is irreducible. We showed that $f$ is minimal in \hyperref[lemaczysko]{\acc{Lemaczysko}} and so from Remark 4.5. (\hyperref[remark:4:5]{below}) we have that $[L:K]=deg(f)=n$.

\acc{Lemacik:} \emph{This is not really a lemma but the third step in the diagram: $w_m(a)=0$ for $m=p^{kn}-1$.}

Now let us look at $L^*$, which is the multiplicative group of $L$. Because $L$ was a field, we know that 
$$|L|=p^{kn}=p^l$$ 
($[L:K]=n$ and there were $p^k$ elements in $K$) and that 
$$|L^*|=|L\setminus\{0\}|=p^l-1.$$ 
Furthermore, we know that every finite group is isomorphic to the field $\Z_p$ so we must have that $L^*$ is a cyclic group with $a\in L^*$ as one of its generators. We know that $a^{p^l}=a$ will "loop back" inside of $L^*$ and so $a^{p^l-1}=1$ inside of $L^*$. This gives us the following equality:
$$w_{p^l-1}(a)a^{p^l-1}-1=1-1=0$$
with $p\nmid p^{l}-1$.

\acc{Lemaciuś:} \emph{Once again not a lemma but showing that $f$ divides $w_m$, $m$ as above.}

What remains now is to show that $f|w_m$. Suppose that this is untrue and that their "gcd" is equal to 1. Then by Bezout's identity we have that there exist $c,d\in K[X]$ such that
$$f(x)c(x)+w_m(x)d(x)=1$$
but for $x=a$ we would have $0=1$ which is a contradiction. Hence, one has to divide the other. $f$ is irreducible so it cannot be divided by anything but itself and so $f|w_m$.
\bigskip

\phantomsection
\label{remark:4:5}
{\color{orange}Remark 4.5.}
\emph{ Suppose that $I(a/K)=(f)$ and $f$ is monic. Then:
\begin{enumerate}
    \item $f$ is the minimal monic polynomial such that $f(a)=0$
    \item $deg(f)=[K(a):K]$, thus the degree of the minimal polynomial is equal to the dimension of the linear space $K(a)$ over $K$.
\end{enumerate}}
\medskip

\subsection*{EXERCISE 5.}
\emph{\color{pink}(a) Prove that if $K\subseteq L$ are finite fields, $|K|=p^m,|L|=p^n$, then $m|n$.}
\smallskip

Let $[L:K]=d$. Then we have that the basis of $L$ over $K$ has $d$ elements. Every element of $L$ can be expressed as a linear combination of elements from the basis with coefficients from $K$. There are
$$|K|^d=p^{md}$$
such combinations. Hence $|L|=p^{md}=p^n\implies n=md\implies m|n$.
\medskip

\emph{\color{pink}(b) Prove that every field with $p^n$ elements contains a unique subfield with $p^m$ elements, where $m|n$.}
\smallskip

"Proof graph" of existence
\begin{illustration}
    \node (1) at (0, 0) {$x\in\mu_{p^n-1}(L)\implies x\in\mu_{p^m-1}(L)$};
    \node (2) at (0, -1.5) {$x^{p^n-1}=1\implies x^{p^m-1}=1\implies x^{p^m}=x$};
    \node (3) at (0, -3) {$x\in Fix(x^{p^m})\subseteq L$};
    \node (4) at (0, -4.5) {$|Fix(x^{p^m})^*|=|\mu_{p^m-1}|=p^m-1\implies|Fix(x^{p^m})|=p^m$};
    \draw[->] (1)--(2);
    \draw[->] (2)--(3);
    \draw[->] (3)--(4);
\end{illustration}

Let $n=md$ for some $m,d\in\N$. Notice that $\mu_{p^m-1}(L)\subseteq\mu_{p^n-1}(L)$ because if $x\in\mu_{p^m-1}$ then
$$x^{p^n-1}-1=(x^{p^m-1}-1)(x^{p^{n-m}}+x^{p^{n-m-1}}+...+1)$$
and so $x^{p^m-1}-1$ must be equal to zero. Setting an $x\in\mu_{p^m-1}(L)$ allows us to do the following computation:
$$x^{p^m-1}-1=0$$
$$x^{p^m-1}=1$$
$$x^{p^m}=x$$
which gives us an endomorphism $f(x)=x^{p^m}$. From ex. 3. we know that $Fix(f)$ is a subfield of $L$ and from the reasoning above we know that $Fix(L)$ contains the elements from $\mu_{p}(L)$ (which according to Theorem 3.4. has cardinality $p^{m}-1$) and $\{0\}$. Thus, $|Fix(f)|=p^m$.
\smallskip

"Proof graph" of uniqueness:
\begin{illustration}
    \node (1) at (0,0) {suppose that $K_1,K_2\subseteq L, |K_1|=|K_2|=p^m$};
    \node (2) at (0, -1.5) {$|K_1^*|=p^m-1=|K_2^*|$};
    \node (3) at (0, -3) {$K_1^*=\mu_{p^m}(L)=K_2^*$};
    \draw[->] (1)--(2);
    \draw[->](2)--(3);
    \node at (1.8, -3.5) {{\large\lightning}};
\end{illustration}

Suppose that there exist two subfields $K_1, K_2\subseteq L$ with $|K_1|=p^m=|K_2|$. Then $|K_1^*|=p^m-1$ and $|K_2^*|=p^m-1$, which from Theorem 3.4. means that
$$K_1^*=\mu_{p^m-1}(L)$$
$$K_2^*=\mu_{p^m-1}(L).$$
From the fact that $K_1^*=K_2^*$ follows that $K_1=K_2$, which is a contradiction.
\smallskip

{\color{orange}Theorem 3.4.} Let $G<\mu(K)$ and $G$ is finite with $|G|=n$. Then:
\begin{enumerate}
    \item $G=\mu_n(K)$
    \item $G$ is cyclic
    \item if $char(K)=p>0$ then $p\nmid n$.
\end{enumerate}


\newpage

\subsection*{EXERCISE 6.}
{\color{pink}\emph{Let $F(p^n)$ be a field with $p^n$ elements. From Problem $5$ it follows from that}
$$F(p)\subseteq F(p^2)\subseteq F(p^{3!})\subseteq...\subseteq F(p^{n!})\subseteq...$$
\emph{(after suitable identifications of isomorphic fields). Let}
$$F=\bigcup\limits_{n>0}F(p^{n!})$$
\emph{Prove that the field $F$ is algebraically closed. (hint: use Problem 4.)}
}
\smallskip

A field is algebraically closed if every non-constant polynomial $f\in F[X]$ has a root in $F$.


%From exercise 4 we know that every irreducible polynomial must divide $x^n-1$ for some
%$n$. Thus, it has to have a common root with $x^n-1$. If we show that for every $n$, the
%field $F$ contains a root of $w_m(x)$ this would imply that it contains the root of every
%irreducible polynomial. Hence, every polynomial has a root in $F$ and $F$ is algebraically
%closed.

"Proof graph"
\begin{illustration}
    \node (1) at (0, 0) {Ex. 4: $(\forall\;f\in F[X])$ $f$-irreducible $\implies f|w_m$};
    \node (2) at (0, -1.5) {$(\forall\;n\in\N)(\exists\;a_1,...,a_n\in F)\;w_n(a_i)=0$.
    $i=1,...,n$};
    \node (3) at (0, -3) {$w_n(a_i)=0\implies f(a_i)=0$ for some $i\in\{1,...,n\}$};
    \draw[->](1)--(2);
    \draw[->](2)--(3);
\end{illustration}

Because all polynomials in $F[X]$ are either irreducible or a product of irreducible
polynomials, it is sufficient to show that every irreducible polynomial in $F[X]$ has a
root in $F$. Let $f\in F[X]$ be irreducible and $n=deg(f)$. From Ex. 4 we know that
$f|w_{p^{nk}-1}$ for some $k\in\N$ and so they must have a common root. Thus, it will be
sufficient to show that all roots of $w_{p^{nk}-1}$ are within $F$.

Take $n\in\N$ and consider $w_n(x)\in F[X]$. The field $F(p^k)$ such that
$n<p^k$ will have all roots of $w_n(x)$. But $F(p^k)\subseteq F$ so we have that $F$ also
contains all roots of $w_n(x)$.

The above reasoning was conducted for arbitrary chosen $n$, so it will be true for
$p^{nk}-1$ and so $F$ contains all roots of $p^{nk}-1$, meaning that $f$ contains at least
one root in $F$ and so $F$ is algebraically closed.


%Let us take an irreducible polynomial (because other polynomials are products of irreducible
%polynomials) $f\in F[X]$ with $deg(f)=n$. Since $F[X]$ is also a sum of $F(p^k)[X]$, then
%$f\in F(p^k)[X]$, with $k$ being the least number for which this is true. From Ex. 4 we
%know that $f|w_{p^{kn}-1}$. 
%
%Let $m\in\N$ be any natural number. We will show that all roots of $w_m$ are within
%$F$. As above, because $w_m\in F[X]$ then there exists $k$ for which $w_m\in F(p^k)[X]$. 
%
\end{document}
