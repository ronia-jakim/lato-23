\documentclass{article}

\usepackage{../../../notatki}

\title{\large Algebra 2R\smallskip\\ \textbf{Problem List 2}}
\author{\normalsize Weronika Jakimowicz}
\date{~~~}

\begin{document}
\maketitle\thispagestyle{empty}

\subsection*{EXERCISE 4.}
{\color{pink}Assume that $K$ is a finite field, characteristic $p$.

(a) Prove that every irreducible polynomial $f\in K[x]$ divides the polynomial $w_n(x)=x^n-1$ for some $n$ not divisible by $p$. (hint: prove that the splitting field of $f$ is finite.)}

Let $f$ be an irreducible polynomial $f\in K[x]$ and $n=deg(f)>0$ and let $a_1,...,a_r\in L\supseteq K$ be its roots, where $L$ is the splitting field of $f$ over $K$. Because $K$ is finite, i can say that $|K|=q$.

For my convenience, I will consider $g=b_n^{-1}f$, where $b_n$ is the leading coefficient in $f$. So now $g$ is a monic polynomial and considering the splitting field of $f$ is the same as considering the splitting field of $g$ - I just multiplied a polynomial by a nonzero constant.
\smallskip

\phantomsection
\label{lemaczysko}
\acc{Lemaczysko:} \emph{An irreducible polynomial $g\in K[X]$ is the minimal polynomial for some root $a$, $f(a)=0$}

As $K$ is a field, the ring $K[X]$ is an euclidean domain. Let us suppose that $h\in K[X]$ is the minimal polynomial of $a$ in $K$ such that $deg(h)<deg(g)$. We have that there exists $p,r\in K[X]$ such that
$$f=hp+r$$
but notice that $f(a)=0$ and $h(a)=0$, so $r=0$ and we would have $f=hp$ but $f$ was irreducible.
\medskip

\acc{Lemat:}\emph{ The splitting field of $g$ (equivalently, of $f$) is finite.}

We will construct the splitting field of $K$ as such:
$$L_1=K(a_1)$$
$$L_2=L_1(a_2)$$
$$L_i=L_{i-1}(a_i)$$
and then $L=L_r$.

1. $[L_1:K] = n$. The ideal
$$I(a_1/K)=\{w\in K[X]\;:\;w(a_1)=0\}=(g)$$
because $g$ is irreducible. We showed that $g$ is minimal in \hyperref[lemaczysko]{\acc{Lemaczysko}} and so from Remark 4.5. (\hyperref[remark:4:5]{below}) we have that $[L_1:K]=deg(g)=n$.

2. $[L_{i+1}:L_i] = n$. Once again, $g$ is irreducible over $L_{i}$ (because not all roots of $g$ are in $L_i$)
$$I(a_{i+1}/K)=\{w\in K[X]\subseteq L_i[X]\;:\;w(a_{i+1})=0\}=(g)$$
and it follows from Remark 4.5. (\hyperref[remark:4:5]{once again}) that $[L_{i+1},L_i]=deg(g)=n$.
\medskip

Now, using Fact 4.6. (\hyperref[fact:4:6]{even belower}) We have that
\begin{align*}
    [L:K]&=[L_r:L_{r-1}][L_{r-1}:L_{r-2}]=...=\prod\limits_{i=1}^r[L_i:L_{i-1}]=n^r<\infty.
\end{align*}

If the original field $K$ had $p^k$ elements, then the new field would have $p^l$ elements, where $l=k\cdot [L:K]$. Therefore, we have $p^l$ elements in the base of $L$ over $K$.

Now we want to show that $v_n(a_1)=0$ and from this and the fact that $K[X]$ is euclidean conclude that "gcd" of those two polynomials cannot be 1, hence $g$ divides $v_n$.



We know that $v_n(a_1)=0$. Suppose that $g\nmid v_n$, then we would be able to find $c, b\in K[X]$ such that
$$g\cdot c+v_n\cdot b=1$$
but then
$$g(a_1)\cdot c(a_1)+v_n(a_1)\cdot b(a_1)=1$$
which gives a contradiction. Hence, $g|v_n$ and because $v_n|w_n$ we have that $g|w_n$.


\phantomsection
\label{remark:4:5}
{\color{orange}Remark 4.5.}or some $i$ 
\emph{ Suppose that $I(a/K)=(f)$ and $f$ is monic. Then:
\begin{enumerate}
    \item $f$ is the minimal monic polynomial such that $f(a)=0$
    \item $deg(f)=[K(a):K]$, thus the degree of the minimal polynomial is equal to the dimension of the linear space $K(a)$ over $K$.
\end{enumerate}}
\medskip

\phantomsection
\label{fact:4:6}
{\color{orange}Fact 4.6}
\emph{ Let $K\subseteq L\subseteq M$ be extensions of fields. Then
$$[M:K]=[M:L][L:K]$$}

\end{document}