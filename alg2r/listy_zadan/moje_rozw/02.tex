\documentclass{article}

\usepackage{../../../notatki}

\title{\large Algebra 2R\smallskip\\ \textbf{Problem List 2}}
\author{\normalsize Weronika Jakimowicz}
\date{~~~}

\begin{document}
\maketitle\thispagestyle{empty}

\subsection*{EXERCISE 4.}
{\color{pink}Assume that $K$ is a finite field, characteristic $p$.

(a) Prove that every irreducible polynomial $f\in K[x]$ divides the polynomial $w_n(x)=x^n-1$ for some $n$ not divisible by $p$. (hint: prove that the splitting field of $f$ is finite.)}

Let $f$ be an irreducible polynomial $f\in K[x]$ and $n=deg(f)>0$ and let $a_1,...,a_r\in L\supseteq K$ be its roots, where $L$ is the splitting field of $f$ over $K$. Because $K$ is finite, i can say that $|K|=q$.

For my convenience, I will consider $g=b_n^{-1}f$, where $b_n$ is the leading coefficient in $f$. So now $g$ is a monic polynomial and considering the splitting field of $f$ is the same as considering the splitting field of $g$ - I just multiplied a polynomial by a nonzero constant.
\smallskip

\acc{Lemacik:}\emph{ The splitting field of $g$ (equivalently, of $f$) is finite.}

We will construct the splitting field of $K$ as such:
$$L_1=K(a_1)$$
$$L_2=L_1(a_2)$$
$$L_i=L_{i-1}(a_i)$$
and then $L=L_r$.

Let
$$f(x)=\prod\limits_{i=1}^r(x-a_i)^{k_i}$$
and notice that $\sum k_i=n$.

I will show that $[L_r:K]=\prod\limits_{i=1}^rk_i<\infty$ using finite induction.

1. $[L_1:K]$. We know that $g$ in $L_1$ can be written as
$$g=(x-a_1)^{k_1}u_1,$$
where $u_1\in L_1[x]$ is an irreducible polynomial such that $u_1(a)\neq 0$. Then, the ideal
$$I(a_1/K)=\{w\in K[x]\;:\;w(a_1)=0\}=((x-a_1)^{k_1})$$
because even though I write $(x-a_1)^{k_1}$, I know that a polynomial of this form is irreducible in $K[x]$ if $a_1\notin K$. Which must be the case because we started from a polynomial that is not a product of linear polynomials.

From Remark 4.4. (\hyperref[remark:4:4]{below}) I know that $[L_1:K]=deg((x-a_1)^{k_1})=k_1$.
\smallskip

2. $[L_{i+1}:L_i]=q^{k_{i+1}}$

Right now we have that
$$g=(x-a_{i+1})^{k_{i+1}}u_{i+1}(x)\prod\limits_{j=1}^i(x-a_j)^{k_j}$$
which looks horrific but I am trying to be formal and so on. I want $u_{i+1}(a_{i+1})\neq 0$. The product part is a linear combination of polynomials from $L_i$ and $(x-a_{i+1})^{k_{i+1}}$ is a linear combination of polynomials from $L_{i+1}$ but not from $L_i$. And so is irreducible over $L_{i}$. So now we have that
$$I(a_{i+1}/K)=((x-a_{i+1}))^{k_{i+1}}$$
by the same argument as above. Thus, from remark 4.4. (\hyperref[remark:4:4]{still there}) we have that $[L_{i+1}\;:\;L_i]=k_{i+1}$.
\medskip

Now that we are done, we will use Fact 4.5. (\hyperref[fact:4:5]{even belower}). We see, from condition, that
$$K\subseteq L_1\subseteq L_2\subseteq...\subseteq L_r=L$$
and now:
\begin{align*}
    [L_r:K]=[L_r:L_{r-1}][L]
\end{align*}

\phantomsection
\label{remark:4:4}
{\color{orange}Remark 4.4.}
\emph{ Suppose that $I(a/K)=(f)$ and $f$ is monic. Then:
\begin{enumerate}
    \item $f$ is the minimal monic polynomial such that $f(a)=0$
    \item $deg(f)=[K(a):K]$, thus the degree of the minimal polynomial is equal to the dimension of the linear space $K(a)$ over $K$.
\end{enumerate}}
\medskip

\phantomsection
\label{fact:4:5}
{\color{orange}Fact 4.5}
\emph{ Let $K\subseteq L\subseteq M$ be extensions of fields. Then
$$[M:K]=[M:L][L:K]$$}

% The ideal $I(a_1,...,a_2/K)$ satisfies the conditions from the remark and thus
% $$deg(g)=[K(a_1,...,a_n)\;:\;K]$$
% and so $[K(a_1,..,a_n)\;:\;K]=n$. We know that $(1, a_1, )$

\end{document}