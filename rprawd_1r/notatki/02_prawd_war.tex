\section{Prawdopodobieństwo warunkowe}

Dalsza część wykładu będzie raczej oderwana od tego co się dzieje tutaj.

Niech $(\Omega,\set{F},\mathbb{P})$ będzie przestrzenią probabilistyczną i niech $A, B$ będą zdarzeniami takimi, że $\prawdo{B}>0$. Wówczas \deff{prawdopodobieństwem warunkowym} zdarzenia $A$ względem $B$ nazywamy wartość
$$\prawdo{A|B}={\prawdo{A\cap B}\over\prawdo{B}}$$
Jeśli ustalimy zbiór $B$, to miara $\color{blue}\prawdo{\bullet|B}$ jest miarą probabilistyczną.

\textbf{Przykład:}

\indent 1. Wybieramy losową rodzinę z dwójką dzieci. Oblicz prawdopodobieństwo, że jest to dwóch chłopców, jeżeli

\indent a). starsze dziecko jest chłopcem.

Mamy $\Omega=\{(d, d), (d, c), (c, c), (c, d)\}$ przypadki, kiedy starsze dziecko to chłopiec:
$$B=\{(d, c), (c, c)\}$$
i podzbiór tego, gdy oboje są chłopcami to $A=\{(c, c)\}$. Czyli
$$\prawdo{A|B}={1\over 2}$$

\indent b). jedno z tych dzieci to chłopak.

Omega jest taka sama jak wcześniej, zmienia nam się definicja zbioru $B$:
$$B=\{(c, d), (d, c), (c, c)\}$$
$A$ jest nadal singletonem. Ogółem mamy
$$\prawdo{A|B}=\frac13$$

Mówimy, że rodzina zbiorów $\{B_k\}_{k=1}^n$ (dopuszczamy $n=\infty$) jest \deff{rozbiciem zbioru $\Omega$}, jeśli $\Omega=\bigsqcup_{k=1}^nB_k$ (suma rozłączna).

\subsection{Prawdopodobieństwo całkowite}

\deff{\large Twierdzenie:} [wzór na \dyg{prawdopodobieństwo całkowite}] Niech $\{B_k\}_{k=1}^n$ będzie rozbiciem $\Omega$ takim, że $\prawdo{B_k}>0$ dla każdego $k$. Wówczas dla każdego $A\in\set{F}$ zachodzi:
$$\prawdo{A}=\sum\limits_{k=1}^n\prawdo{A|B_k}\prawdo{B_k}$$

\textbf{Dowód:}
$$\prawdo{A}=\prawdo{A\cap\left[\bigsqcup\limits_{k=1}^nB_k\right]}=\prawdo{\bigsqcup\limits_{k\leq n}[A\cap B_k]}=\sum\limits_{k\leq n}\prawdo{A\cap B_k}=\sum\limits_{k\leq n}\prawdo{A|B_k}\cdot\prawdo{B_k}$$

\proofend

\textbf{Przykład:} W loterii fantowej mamy $3$ rodzaje losów:

\indent \point W - wygrana, wyciągany z prawdopodobieństwem $p$

\indent \point P - przegrana - z prawdopodobieństwem $q$

\indent \point D - graj dalej - z prawdopodobieństwem r

gdzie $p+q+r=1$. Oblicz prawdopodobieństwo wygranej. 

Niech $Z$ będzie zdarzeniem, które mówi, że wygraliśmy. Chcemy obliczyć $\prawdo{Z}$. $W, P, D$ to rozbicie przestrzeni $\Omega$. Ze wzoru na prawdopodobieństwo całkowite dostajemy
\begin{align*}
    \prawdo{Z}&=\prawdo{Z|W}\cdot\prawdo{W}+\prawdo{Z|P}\cdot\prawdo{P}+\prawdo{Z|D}\cdot\prawdo{D}=\\
    &=1\cdot p+0\cdot q+\prawdo{Z}\cdot r\\
    \prawdo{Z}&={p\over 1-r}={p\over p+q}
\end{align*}

\subsection{Wzór Bayesa}

\deff{\large Twierdzenie:} [\dyg{wzór Bayesa}] załóżmy, że mamy rozbicie $\Omega$ $\{B_k\}_{k=1}^n$ to znaczy, $\prawdo{B_k}>0$. Weźmy dowolne zdarzenie $A$ takie, że $\prawdo{A}>0$. Wówczas dla każdego $j$ zachodzi
$$\prawdo{B_j|A}={\prawdo{A|B_j}\prawdo{B_j}\over \sum_{k\leq n}\prawdo{A|B_k}\prawdo{B_k}}$$

\textbf{Dowód:} Użycie wzoru na prawdopodobieństwo całkowite:
\begin{align*}
    {\prawdo{A|B_j}\prawdo{B_j}\over \sum_{k\leq n}\prawdo{A|B_k}\prawdo{B_k}}&={\prawdo{A|B_j}\cdot\prawdo{B_j}\over \prawdo{A}}={\prawdo{A\cap B_j}\over\prawdo{B_j}}\cdot{\prawdo{B_j}\over\prawdo{A}}=\prawdo{B_j|A}
\end{align*}

\proofend

\textbf{Przykład:} 

\indent 1. Mamy $100$ monet i $99$ z nich jest uczciwych, a jedna jest fałszywa (orły po dwóch stronach). Losujemy monetę i wypadło $10$ orłów. Oblicz prawdopodobieństwo, że wylosowaliśmy fałszywą monetę.

Niech $B_1$ oznacza, że wylosowaliśmy monetę uczciwą, a $B_2$ - że fałszywą. Wtedy zdarzeniem $A$ będzie wyrzucenie $10$ orłów. Mamy
\begin{align*}
    \prawdo{B_2|A}&={\prawdo{A|B_2}\prawdo{B_2}\over\prawdo{A|B_1}\prawdo{B_1}+\prawdo{A|B_2}\prawdo{B_2}}={1\cdot\frac1{100}\over{1\over2^{10}}\cdot{99\over100}+\frac1{100}}=\\
    &={1024\over1123}\approx91\%
\end{align*}

\indent 2. U pacjenta przeprowadzono test na rzadką chorobę. Wiadomo, że na tę chorobę choruje $1$ osoba na $1000$. Test jest "mocny", to znaczy jeżeli osoba jest chora, to test wskazuje na chorobę z prawdopodobieństwem ${99\over 100}$. Jeżeli natomiast osoba jest zdrowa, to test nie wskazuje na chorobę z prawdopodobieństwem ${95\over100}$. Test wskazał na chorobę. Oblicz prawdopodobieństwo, że pacjent jest chory.

Mamy trzy zdarzenia:

$Z$ - pacjent jest zdrowy,

$C$ - pacjent jest chory,

$T$ - test wyszedł pozytywny.

Używamy wzoru Bayesa, żeby obliczyć
\begin{align*}
    \prawdo{C|T}&={\prawdo{T|C}\prawdo{C}\over\prawdo{T|Z}\prawdo{Z}+\prawdo{T|C}\prawdo{C}}={0.99\cdot0.001\over0.95\cdot0.999+0.99\cdot0.001}={99\over 5094}\approx 2\%
\end{align*}