\section{Zmienne losowe}

Niech $(\Omega,\set{F},\mathbb{P})$ będzie przestrzenią probabilistyczną. \deff{Zmienna losowa} jest to funkcja mierzalna $X:\Omega\to\R$. ($\R$ rozważamy z $\sigma$-ciałem zbiorów borelowskich). To znaczy $X^{-1}(B)\in \set{F}$.

\textbf{Przykład:}

\indent \point suma wyników $5$ rzutów kostką

\indent \point aktualny limit akcji

\deff{\large Uwaga:}

\indent 1. Jeżeli zbiór $\Omega$ jest przeliczalny i $\set{F}=2^{\Omega}$, to każda funkcja $X:\Omega\to\R$ jest mierzalna.

\indent 2. $X$ jest zmienną losową
\medskip

\deff{\large Twierdzenie:} Jeżeli $X_1, X_2...$ są zmiennymi losowymi, to

\indent 1. $X_1+X_2$, $X_1-X_2$... są zmiennymi losowymi

\indent 2. $f:\R^n\to \R$, to $f(X_1,...,X_n)$ jest zmienną losową

\indent 3. $\inf X_n$, $\sup X_n$, $\lim\inf X_m$, $\lim\sup X_n$ są zmiennymi losowymi.
\medskip

Mówimy, że miara $\mu$ na $(\R,Bor(\R))$ zdefiniowana wzorem
$$\mu(B)=\prawdo{X^{-1}(B)}=\prawdo{\{\omega\;:\;X(\omega)\in B\}}={\color{blue}\prawdo{X\in B}}$$
jest \deff{rozkładem zmiennej losowej $X$.} Zauważmy, że $(\R,Bor(\R),\mu)$ jest przestrzenią probabilistyczną.

\deff{Dystrybuantą} zmiennej losowej $X$ nazywamy funkcję $F:\R\to[0,1]$ zdefiniowaną następująco:
$$F(t)={\color{blue}\prawdo{X\leq t}}=\mu{(-\infty,t]}$$

\textbf{Przykład}

\indent 1. Rzut monetą, $\Omega=(O, R)$, $X(O)=1$ i $X(R)=0$. Jeżeli $t<0$, to $F(t)=0$. Jeżeli $t=0$, to $F(t)=\frac12$:
$$F(t)=\prawdo{X\leq0}=\prawdo{X\leq 0}=\frac12$$
dla $t\in (0, 1)$ mamy $F(t)=\frac12$ tak jak wyżej, a da $t\geq 1$ jest $F(t)=1$, bo wyżej niż $1$ już nie wejdziemy.

\indent 2. Rzut kostką

{\large\color{orange}ZDJĘCIA!!!}

\indent 3. Odcinek $([0,1], Bor(\R), Leb)$, $X(\omega)=\omega$.
$$F(t)=\begin{cases}
    0\quad t<0\\
    1\quad t>1\\
    \lambda([0,t])=t \quad t\in[0, 1]
\end{cases}$$

Dystrybuanty nie są ciągłe, ale są prawostronnie ciągłe, a z lewej strony istnieją granice.

\deff{\large Twierdzenie:} Niech $F$ będzie dystrybuantą pewnej zmiennej losowej. Wtedy

\indent 1. $F$ jest niemalejąca

\indent 2. $\lim\limits_{t\to-\infty}F(t)=0$ i $\lim\limits_{t\to\infty}F(t)=1$

\indent 3. $F$ jest prawostronnie ciągła.

\indent 4. Dla dowolnego $t$ istnieje lewa granica w $t$ taka, że
$${\color{blue}F(t-)}=\lim\limits_{s\to t^-}F(s)=\prawdo{X<t}$$
\indent 5. $F$ jest nieciągła w punkcie $t$ tylko wtedy, gdy $\prawdo{X=t}>0$. Wówczas $\prawdo{X=t}=F(t)-F(t-)$

\textbf{Dowód:}

1. Jeżeli $s<t$, to $(-\infty, s]\subseteq(-\infty t]$, czyli $F(s)=\mu((-\infty,s])\leq\mu((-\infty,t])=F(t)$

2. Weźmy dowolny ciąg $t_n\to-\infty$ malejący, wtedy
$$(-\infty, t_{n+1})\subseteq(-\infty,t_n],$$
w szczególności
$$B_k=\bigcap\limits_{n=k}^\infty(-\infty,t_n]$$
jest zstępująca i $\bigcap B_k=\emptyset$. Zatem z lematu o ciągłości miary:
$$\lim\limits_{n\to\infty}F(t_n)=\lim\limits_{n\to\infty}\mu(B_n)=\mu(\bigcap B_n)=0$$

3. Niech $t_n\downarrow t$, wtedy rodzina zbiorów $(-\infty, t_n]$ jest zstępująca. 
$$\bigcap (-\infty, t_n]=(-\infty, t]$$
Z lematu o ciągłości miary:
$$\lim F(t_n)=\lim \mu((-\infty, t_n])=\mu(-\infty, t]=F(t)$$

Pozostałe podpunkty pozostawione jako ćwiczenie.
\medskip

\deff{\large Twierdzenie}: Jeżeli funkcja $F:\R\to[0,1]$ spełnia własności $(1), (2), (3)$, to jest dystrybuantą pewnej zmiennej losowej.

\textbf{Dowód:} Dowód pokarzemy dla $F$, które są odwracalne.

Musimy znaleźć $(\Omega,\set{F},\mathbb{P})$ oraz zmienną losową $X_n$ z $\Omega$ taką, że $F$ jest dystrybuantą $X_n$.

Weźmy $([0,1],Bor([0,1]), Leb)$. Niech 
$$X(\omega)=F^{-1}(\omega).$$

1. Trzeba pokazać, że to jest faktycznie zmienną losową, to znaczy, że $F$ jest mierzalne [ćwiczenie].

2. $F$ jest dystrybuantą, czyli warunki (4) i (5).
\begin{align*}
    \prawdo{X\leq t}=\prawdo{X(\omega)\leq t}=\prawdo{F^{-1}(\omega)\leq t}=\prawdo{\omega\leq F(t)}=F(t)
\end{align*}

\begin{definicja}[$\pi$-układ]
Niech $\set{K}$ będzie niepustą rodziną podzbiorów zbioru $\Omega$. Wtedy $\set{K}$ nazywamy \deff{$\pi$-układem}, jeżeli $(\forall\;A,B\in\set{K})\;A\cap B\in\set{K}$.
\end{definicja}

\begin{definicja}
Niepustą rodzinę $\set{L}$ nazywamy \deff{$\lambda$-układem}, jeżeli
\begin{itemize}
    \item $\Omega\in\set{L}$
    \item $A,B\in\set{L}, A\subseteq B$ to $B\setminus A\in\set{L}$
    \item $\{A_n\}_{n\in\N}$ jest zstępującą rodziną zbiorów z $\set{L}$, to $\bigcup A_n\in\set{L}$
\end{itemize}
\end{definicja}

\begin{lemat}[lemat o $\pi-\lambda$-układach, tw. Dynkina]
Jeżeli $\set{L}$ jest $\lambda$-układem zawierającym $\pi$-układ $\set{K}$, to $\set{L}$ zawiera $\sigma$-ciało generowane przez $\set{K}$.
\end{lemat}

\textbf{Dowód:}

\begin{enumerate}
    \item Jeżeli $\set{L}$ jest jednocześnie $\lambda$-układem i $\pi$-układem, to $\set{L}$ jest $\sigma$-ciałem.
        \begin{itemize}
            \item Jeżeli $A,B\in\set{L}$, to $A\cup B\in\set{L}$, bo
            $$A\cup B=
            A\cup(B\setminus A\cap B)=
            \left[A^c\setminus
            (\underbrace{B\setminus A\cap B}_{\substack{\in\set{L}\\bo\;A\cap B\subseteq B}})\right]^c$$
            \item Przez indukcję pokazujemy, że $A_1,...,A_n\in\set{L}$, to $\bigcup\limits_{i\leq n} A_i\in\set{L}$
            \item Pozostaje pokazać, że $\bigcup\limits_{n\in\N}A_n\in\set{L}$ jeżeli $A_1,...,A_n,...\in\set{L}$. Nazywamy $B_k=\bigcup\limits_{i\leq k}A_k$, ciąg $B_k$ jest wstępujący więc z definicji $\lambda$-układu jest w $\set{L}$.
        \end{itemize}
    \item Niech $\set{L}_0$ będzie przekrojem wszystkich $\lambda$-układów zawierających $\set{K}$. Łatwo zobaczyć, że $\set{L}_0$ jest $\lambda$-układem. Chcemy pokazać, że $\set{L}_0$ jest $\pi$-układem, bo wtedy $\lambda_0$ jest $\sigma$-ciałem zawierającym $\set{K}$.
    
    W szczególności musi zawierać $\sigma$-ciało generowane przez $\set{K}$.

        \begin{itemize}
            \item Ustalmy $A\in K$ i niech
            $$K^A_1=\{B\subseteq\Omega\;:\;A\cap B\in\set{L}_0\}$$
            \item Zauważmy, że $K\subseteq K_1^A$, bo $K\subseteq\set{L}_0$. Ponadto $K_1^A$ jest $\lambda$-układem.
            \item $\Omega\in K_1^A$, bo $\Omega\cap A=A\in K\subseteq\set{L}_0$
            \item Niech $B_1,B_2\in K_1^A$ i $B_1\subseteq B_2$. Wtedy $(B_2\setminus B_1)\cap A=(B_2\cap A)\setminus(B_1\cap A)\in\set{L}_0$, bo $\set{L}_0$ jest $\lambda$-układem.
            \item Stąd $B_2\setminus B_1\in K_1^A$
            \item Niech $\{B_n\}_{n\in\N}$ będzie wstępującym ciągiem elementów z $K_1^A$.
            $$A\cap \bigcup B_n=\bigcup(B_n\cap A)\in\set{L}_0$$
            bo przekroje też są wstępujące, stąd $B_n\cap A$ jest wstępującym ciągiem z $\set{L}_0$ 
        \end{itemize}
    Mamy więc, że $K_1^A$ jest $\lambda$-układem, zatem $\set{L}_0\subseteq K_1^A$. Pokazaliśmy, że $(\forall\;B\in\set{L}_0)\;A\cap B\in\set{L}_0$.
    \item $A$ wyżej był dowolny. W takim razie pokazaliśmy, że
    $$(\forall\;A\in K)(\forall\;B\in\set{L}_0)\;A\cap B\in\set{L}_0$$
    Ustalmy $B\in \set{L}_0$ taki, że
\end{enumerate}

{\large\color{orange}UKRAŚĆ NOTATKI ZE SKRYPTU, BO NIE CHCE MI SIĘ TERAZ PISAĆ}


\subsection{KONIEC TEORII MIARY NA DZISIAJ}

\begin{tw}[twierdzenie o jednoznaczności]
Dystrybuanta zmiennej losowej $X$ jednoznacznie wyznacza jej rozkład. To znaczy, jeśli dwie zmienne losowe mają różny rozkład, to mają też różne dystrybuanty.
\end{tw}

\textbf{Dowód:} Załóżmy nie wprost, że mamy dwie zmienne losowe: $X$ o rozkładzie $\mu_X$ i dystrybuantę $F$ oraz $Y$ o rozkładzie $\mu_Y$ i tej samej dystrybuancie. 

Skorzystamy z twierdzenia Dynkina, czyli musimy wskazać $\pi$-układ i $\lambda$-układ. Niech $\set{K}$ będzie rodziną zbiorów postaci $(-\infty, t]$. Łatwo zobaczyć, że tak zdefiniowane $\set{K}$ jest $\pi$-układem.

Niech $\set{L}$ będzie rodziną zbiorów $A$ takich, że $\mu_X(A)=\mu_Y(A)$. Czyli $\set{L}$ jest $\lambda$-układem.

Dla każdego $t\in\R$ mamy, że $\mu_X(-\infty,t]=F(t)$.
Zatem $\set{K}\subseteq\set{L}$, czyli z twierdzenia Dynkina $\sigma(\set{K})=Bor(\R)\subseteq\R$, czyli ta równość $\mu_X(A)=\mu_Y(A)$ zachodzi dla wszystkich zbiorów borelowskich $A$.


\subsection{PRZYKŁADY DYSKRETNE}
\begin{definicja}[rozkład dyskretny]
Zmienna losowa $X$ o rozkładzie $\mu$ ma \deff{rozkład dyskretny}, jeżeli istnieje przeliczalny zbiór $S$ taki, że $\mu(S)=1$. Wtedy
$$S=\{x\;:\;\mu(\{x\})>0\}$$
nazywamy \acc{zbiorem atomów}
\end{definicja}

\textbf{PRZYKŁAD:}
\begin{enumerate}
    \item $\mu=\delta_a,a\in\R$ i $\mu(\{a\})=1$
    \item Rzut kostką
    \item Rozkład dwumianowy (rozkład Bernoulliego, liczba sukcesów w $n$ doświadzczeniach). Ten rozkład ma dwa parametry: liczbę doświadczeń i prawdopodobieństwo sukcesu. Oznaczamy $\color{blue}Bin(n,p)$. Zmienna losowa ma rozkład $Bin(n,p)$ [ozn. $X\sim Bin(n,p)$], jeżeli
    $$\prawdo{X=k}={n\choose k}p^k(1-p)^{n-k}$$
    \item Rozkład Poissona z parametrem $\lambda>0$ [ozn. $\color{blue} Poiss(\lambda)$]. $X$ ma taki rozkład, jeżeli
    $$\prawdo{X=k}=\lambda^kk!e^{-\lambda}$$
    \item Ogólnie to mamy pewien rozkład $S=\{x_1,...,x_n,...\}$ i wagi $\{p_1,...,p_n,...,\}$ takie, że $\sum p_i=1$ i możemy patrzeć na to tak, że $x_i$ to jest atom, a $p_i$ to jest jego waga.
\end{enumerate}

\subsection{Rozkłady absolutnie ciągłe}

\begin{definicja}[rozkład absolutnie ciągły]
Zmienna losowa $X$ o rozkłądzie $\mu$ ma rozkład \deff{absolutnie ciągły} względem miary Lesbegue'a, jeżeli istnieje \acc{gęstość}, t.j. funkcja borelowska $f:\R\to[0,\infty)$ taka, że dla każdego zbioru Borelowskiego $B\in Bor(\R)$ $\mu(B)=\int_Bf(x)dx$
\end{definicja}

\begin{itemize}
    \item $\int_\R f(x)dx=1$
    \item $F(t)=\mu(-\infty,t]=\int_{-\infty}^tf(x)dx$
    \item $f$ ciągła $\implies$ $F'(t)=f(t)$
    \item każda funkcja $f$, która jest nieujemna, mierzalna i $\int_\R f(x)dx=1$ definiuje rozkład pewnej zmiennej losowej.
\end{itemize}

\textbf{Przykład:}
\begin{itemize}
    \item Rozkład jednostajny na $[0,1]$, $X\sim U([0,1])$
    \item Rozkład wykłądniczy z parametrem $\lambda>0$. $X\sim Exp(\lambda)$. 
    $$f(x)=\lambda e^{-\lambda x}\mathbb{1}_{[0,\infty)}$$
    $$F(t)=\begin{cases}0\quad t<0\\1-e^{-\lambda t}\end{cases}$$
    \item Rozkład Gaussa (normalny) $X\sim N(0, 1)$
    $$f(x)={1\over\sqrt{2\pi}}e^{-\frac{x^2}{2}}$$
    $$F(t)=\int_{-\infty}^tf(x)dx$$
\end{itemize}

Ogólnie: Każdą miarę probabilistyczną $\mu$ na $\R$ można przedstawić wpostaci
$$\mu=\mu_{sing}+\mu_{abs},$$
gdize $\mu_{abs}$ jest absolutnie ciągła względem miary Lesbegue'a, a $\mu_{sing}$ jest singularna (tzn. żyje tam, gdzie miara Lesbegue'a jest zerem).

\section{Wielowymiarowe zmienne losowe}
\begin{definicja}
Mamy przestrzeń probabilistyczną $(\Omega,\set{F},\mathbb{P})$. Zmienna losowa o wartościach w $\R^d$ jest to funkcja mierzalna $X:\Omega\to(\R^d,Bor(\R^d)$.
\end{definicja}

\textbf{Przykład:}
Losujemy z talii $5$ kart.


AAAAAAAAAAA


\begin{definicja}
Rozkładem $d$-wymiarowej zmiennej losowej $X$ nazywamy miarę probabilistyczną $\mu$ na $\R^d$
$$\mu(B)=\prawdo{X\in B}=\prawdo{\{\omega\;:\;X(\omega)\in B\}}=\prawdo{X^d(B)}$$
Wtedy $(\R^d,Bor(\R^d),\mu)$ jest przestrzenią probabilistyczną.
\end{definicja}

\begin{definicja}
Dystrybuanta $d$-wymiarowej zmiennej losowej $X=(x_1,...,x_n)$ to funkcja $F:\R^d\to[0,1]$ taka, że
$$F(t_1,...,t_n)=\prawdo{X_1\leq t_1,X_2\leq t_2,...,X_d\leq t_d}$$















