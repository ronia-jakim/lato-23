\section{Zmienne losowe}

Niech $(\Omega,\set{F},\mathbb{P})$ będzie przestrzenią probabilistyczną. \deff{Zmienna losowa} jest to funkcja mierzalna $X:\Omega\to\R$. ($\R$ rozważamy z $\sigma$-ciałem zbiorów borelowskich). To znaczy $X^{-1}(B)\in \set{F}$.

\textbf{Przykład:}

\indent \point suma wyników $5$ rzutów kostką

\indent \point aktualny limit akcji

\deff{\large Uwaga:}

\indent 1. Jeżeli zbiór $\Omega$ jest przeliczalny i $\set{F}=2^{\Omega}$, to każda funkcja $X:\Omega\to\R$ jest mierzalna.

\indent 2. $X$ jest zmienną losową
\medskip

\deff{\large Twierdzenie:} Jeżeli $X_1, X_2...$ są zmiennymi losowymi, to

\indent 1. $X_1+X_2$, $X_1-X_2$... są zmiennymi losowymi

\indent 2. $f:\R^n\to \R$, to $f(X_1,...,X_n)$ jest zmienną losową

\indent 3. $\inf X_n$, $\sup X_n$, $\lim\inf X_m$, $\lim\sup X_n$ są zmiennymi losowymi.
\medskip

Mówimy, że miara $\mu$ na $(\R,Bor(\R))$ zdefiniowana wzorem
$$\mu(B)=\prawdo{X^{-1}(B)}=\prawdo{\{\omega\;:\;X(\omega)\in B\}}={\color{blue}\prawdo{X\in B}}$$
jest \deff{rozkładem zmiennej losowej $X$.} Zauważmy, że $(\R,Bor(\R),\mu)$ jest przestrzenią probabilistyczną.

\deff{Dystrybuantą} zmiennej losowej $X$ nazywamy funkcję $F:\R\to[0,1]$ zdefiniowaną następująco:
$$F(t)={\color{blue}\prawdo{X\leq t}}=\mu{(-\infty,t]}$$

\textbf{Przykład}

\indent 1. Rzut monetą, $\Omega=(O, R)$, $X(O)=1$ i $X(R)=0$. Jeżeli $t<0$, to $F(t)=0$. Jeżeli $t=0$, to $F(t)=\frac12$:
$$F(t)=\prawdo{X\leq0}=\prawdo{X\leq 0}=\frac12$$
dla $t\in (0, 1)$ mamy $F(t)=\frac12$ tak jak wyżej, a da $t\geq 1$ jest $F(t)=1$, bo wyżej niż $1$ już nie wejdziemy.

\indent 2. Rzut kostką

{\large\color{orange}ZDJĘCIA!!!}

\indent 3. Odcinek $([0,1], Bor(\R), Leb)$, $X(\omega)=\omega$.
$$F(t)=\begin{cases}
    0\quad t<0\\
    1\quad t>1\\
    \lambda([0,t])=t \quad t\in[0, 1]
\end{cases}$$

Dystrybuanty nie są ciągłe, ale są prawostronnie ciągłe, a z lewej strony istnieją granice.

\deff{\large Twierdzenie:} Niech $F$ będzie dystrybuantą pewnej zmiennej losowej. Wtedy

\indent 1. $F$ jest niemalejąca

\indent 2. $\lim\limits_{t\to-\infty}F(t)=0$ i $\lim\limits_{t\to\infty}F(t)=1$

\indent 3. $F$ jest prawostronnie ciągła.

\indent 4. Dla dowolnego $t$ istnieje lewa granica w $t$ taka, że
$${\color{blue}F(t-)}=\lim\limits_{s\to t^-}F(s)=\prawdo{X<t}$$
\indent 5. $F$ jest nieciągła w punkcie $t$ tylko wtedy, gdy $\prawdo{X=t}>0$. Wówczas $\prawdo{X=t}=F(t)-F(t-)$

\textbf{Dowód:}

1. Jeżeli $s<t$, to $(-\infty, s]\subseteq(-\infty t]$, czyli $F(s)=\mu((-\infty,s])\leq\mu((-\infty,t])=F(t)$

2. Weźmy dowolny ciąg $t_n\to-\infty$ malejący, wtedy
$$(-\infty, t_{n+1})\subseteq(-\infty,t_n],$$
w szczególności
$$B_k=\bigcap\limits_{n=k}^\infty(-\infty,t_n]$$
jest zstępująca i $\bigcap B_k=\emptyset$. Zatem z lematu o ciągłości miary:
$$\lim\limits_{n\to\infty}F(t_n)=\lim\limits_{n\to\infty}\mu(B_n)=\mu(\bigcap B_n)=0$$

3. Niech $t_n\downarrow t$, wtedy rodzina zbiorów $(-\infty, t_n]$ jest zstępująca. 
$$\bigcap (-\infty, t_n]=(-\infty, t]$$
Z lematu o ciągłości miary:
$$\lim F(t_n)=\lim \mu((-\infty, t_n])=\mu(-\infty, t]=F(t)$$

Pozostałe podpunkty pozostawione jako ćwiczenie.
\medskip

\deff{\large Twierdzenie}: Jeżeli funkcja $F:\R\to[0,1]$ spełnia własności $(1), (2), (3)$, to jest dystrybuantą pewnej zmiennej losowej.

\textbf{Dowód:} Dowód pokarzemy dla $F$, które są odwracalne.

Musimy znaleźć $(\Omega,\set{F},\mathbb{P})$ oraz zmienną losową $X_n$ z $\Omega$ taką, że $F$ jest dystrybuantą $X_n$.

Weźmy $([0,1],Bor([0,1]), Leb)$. Niech 
$$X(\omega)=F^{-1}(\omega).$$

1. Trzeba pokazać, że to jest faktycznie zmienną losową, to znaczy, że $F$ jest mierzalne [ćwiczenie].

2. $F$ jest dystrybuantą, czyli warunki (4) i (5).
\begin{align*}
    \prawdo{X\leq t}=\prawdo{X(\omega)\leq t}=\prawdo{F^{-1}(\omega)\leq t}=\prawdo{\omega\leq F(t)}=F(t)
\end{align*}