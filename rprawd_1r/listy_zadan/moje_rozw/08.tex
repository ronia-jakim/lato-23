\documentclass{article}

\usepackage{../../../lecture_notes}

\title{Rachunek Prawdopodobieństwa 1R\\{\normalsize Lista 8}}
\author{}
\date{}

\begin{document}
\maketitle
\thispagestyle{empty}

\begin{problem}{}
Zmienne losowe $X,Y$ spełniają warunki: $Var(X)=3,Cov(X,Y)=-1,Var(Y)=2$. Oblicz $Var(4X-3Y)$ oraz $Cov(2X-Y,2x+Y)$.
\end{problem}

Twierdzenie 5.16: Jeżeli $\e X^2<\infty$ dla $i=1,...,n$, to $Var(X_1+...+X_n)=\sum_{k=1}^nVar(X_k)+2\sum_{k<l}Cov(X_k,X_l)$
$$Var(4X-3Y)=Var(4X)+Var(-3Y)+2Cov(4X,-3Y)$$
Twierdzenie 5.15(4): kowariacja jest operatorem dwuliniowym
%$Cov(aX+bY,Z)=aCov(X,Z)+bCov(Y,Z)$:
$$16\cdot Var(X)+9\cdot Var(Y)-2\cdot 4\cdot3Cov(X,Y)=48+9+24=81$$

\begin{align*}
    Cov(2X-Y,2X+Y)&=Cov(2X,2X+Y)-Cov(Y,2X+Y)=\\
    &=Cov(2X,2X)+Cov(2X,Y)-Cov(Y,2X)-Cov(Y,Y)=\\
    &=4\cdot Var(X)+2\cdot Cov(X,Y)-2\cdot Cov(Y,X)-Var(Y)=\\
    &=12-2=10
\end{align*}

\begin{problem}{}
Wyznacz dystrybuantę wektora losowego $(X,Y)$ o rozkładzie jednostajnym na przekątnej kwadratu jednostkowego $[0,1]^2$ łączącej punkty $(0,0)$ i $(1,1)$. Wyznacz rozkłady brzegowe, oblicz $\e X,\e Y, Var(X), Var(Y), Cov(X, Y), Var(X+Y)$ oraz sprawdź, czy zmienne $X$ i $Y$ są niezależne. 
\end{problem}

{\rightskip=7cm %
Długość tej prostej wynosi $\sqrt{2}$, więc gęstość to $\frac{1}{\sqrt{2}}$. Dalej, wiem, że wartości, jakie może ten wektor przyjmować są postaci $\begin{pmatrix}a\\a\end{pmatrix}$, czyli
\begin{multicols}{2}$$F(t, k)=\prob{X\leq t,Y\leq k}=\begin{cases}0&a<0\\1&a>1\\\frac{a\sqrt{2}}{\sqrt{2}}&wpp\end{cases}$$
\end{multicols}}
gdzie $a=\min(t,k)$.

\begin{tikzpicture}[remember picture, overlay, shift={(current page.east)}, xshift=-7cm, yshift=-6cm]
    \filldraw[color=yellow!20!white] (0,0) rectangle (2.5,2);
    \draw[->](0,-0.3)--(0,4);
    \draw[->](-0.3,0)--(4,0);
    \draw (0,0)--(3,3);
    \draw[dashed](0, 3)--(3, 3)--(3, 0);
    \draw[green, thick, dashed](0,2)--(2.5,2)--(2.5,0);
    \draw[green,thick](0,0)--(2,2);
    \node at (2.5,2) [above right] {\scalebox{0.65}{$\begin{pmatrix} t\\k\end{pmatrix}$}};
    \draw[orange](1, -0.2)--(1, 3.5);
    \filldraw[orange] (1, 1) circle (1.5pt);
    \node at (1, 0) [below right] {x};
    \draw[orange](1, 0)--(0,0);
\end{tikzpicture}

Rozkłady brzegowe:
$$F_X(x)=\prob{X\leq x, Y\in\R}=\begin{cases}x&x\in[0,1]\\0&wpp\end{cases}$$
$$f_X(x)=1\cdot\mathds{1}_{[0,1]}$$
jak na rysunku. Analogicznie dla $Y$
$$F_Y(y)=\begin{cases}y&y\in[0,1]\\0&wpp\end{cases}$$
$$f_Y(y)=1\cdot\mathds{1}_{[0,1]}$$

$$\e X=\int_\R\prob{X=x}\cdot x\;dx=\int_0^1xdx=\frac{1}{2}$$
$$\e Y=\int_\R\prob{Y=y}\cdot y\;dy=\int_0^1ydy=\frac{1}{2}$$

$$Var(X)=\e[X-\e X]^2=\int_0^1\left[x-\frac{1}{2}\right]^2dx=\frac{1}{12}$$
$$Var(Y)=\frac{1}{12}$$

\begin{align*}
    Cov(X,Y)&=\e XY-\e X\e Y=\e X^2-\frac{1}{4}=\\
    &=\int_0^1\prob{X=x,Y\in\R}xdx-\frac{1}{4}=\\
    &=\int_0^1\prob{X=x,Y=x}xdx-\frac{1}{4}=\\
    &=\int_0^1x^2-\frac{1}{4}=\frac{1}{3}-\frac{1}{4}=\frac{1}{12}
\end{align*}
%\frac{1}{3}-\frac{1}{4}=\frac{1}{12}$$
%\e[(X-\e X)(Y-\e Y)]=\e(XY-\frac{1}{2}(X+Y)+\frac{1}{4})=$$
$$Var(X+Y)=Var(X)+Var(Y)+2\cdot Cov(X, Y)=\frac{2}{12}+2\cdot{1}{12}=\frac{1}{3}$$

%$$\mathfrak{Z}$$

\begin{problem}{}
$d$-wymiarowa zmienna losowa $X$ ma rozkład normalny $N(m,A^{-1})$ o gęstości
$$g(x)=\frac{\sqrt{det(A)}}{(2\pi)^{\frac{d}{2}}}\exp\left[-\frac{1}{2}\langle A(x-m),(x-m)\rangle\right].$$
Udowodnij, że $\e X=m$ oraz $\Lambda=A^{-1}$ jest macierzą kowariancji $X$. 
\end{problem}

$X$ jest $d$-wymiarowe, czyli $X=(X_1,...,X_d)$. Czyli
$$\e X=(\e X_1, \e X_2...,\e X_d),$$
więc wystarczy pokazać, że dla każdego $i$ $\e X_i=m_i$.

Zaczynamy od tego, że $A$ musi być symetryczna i nieujemna, żeby potem mogła być macierzą kowariancji. Czyli w szczególności, $A$ jest diagonalizowalna o wartościach własnych rzeczywistych. Niech więc
$$A=Q^TDQ,$$
$Q$ jest ortogonormalna i $D=(\lambda_1,...,\lambda_d)$.

Policzmy najpierw rozkłady brzegowy tego zła:
\begin{align*}
    \e X_k&=\int_\R...\int_\R x_kg(x_1,...,x_d)\;dx_{1,...,k-1,k+1,...d}=\\
    &=\int...\int\frac{\sqrt{det(Q^TDQ)}}{(2\pi)^{d/2}}x_kexp\left[-\frac{1}{2}\langle Q^TDQ(x-m),(x-m)\rangle\right]=\\
    &=\int...\int\frac{\sqrt{det(Q)det(D)det(Q)}}{(2\pi)^{d/2}}exp\left[-\frac{1}{2}\langle DQ(x-m),Q(x-m)\rangle\right]=\\
    &=\begin{bmatrix}z=Q(x-m)\\dz=detQdx\end{bmatrix}=\\
    &=\int \frac{det(Q)\sqrt{det(D)}}{(2\pi)^{d/2}}\exp\left[-\frac{1}{2}\langle Dz,z\rangle\right]\cdot(det(Q))^{-1}dz=\\
    &=\int\frac{\sqrt{det(D)}}{(2\pi)^{d/2}}\exp\left[-\frac{1}{2}\sum\lambda_iz_i^2\right]dz=\\
    &=\int\frac{\sqrt{\prod\lambda_i}}{(2\pi)^{d/2}}\prod\exp\left[-\frac{1}{2}\lambda_iz_i^2\right]dz=\\
    &=\frac{\sqrt{\prod\lambda_i}}{(2\pi)^{d/2}}e^{-1/2\lambda_k(x_k-m_k)^2}\int e^{-1/2\lambda_1z_1^2}\int...\int e^{-1/2\lambda_dz_d^2}\;dz_{1,...,k-1,k+1,d}=\\
    &=\frac{\sqrt{\lambda_k}}{(2\pi)^{\frac{1}{2}}}e^{-1/2\lambda_k(x_k-m_k)^2}%\frac{\sqrt{\lambda_k}}{\sqrt{2\pi}}e^{-1/2\lambda_k(
\end{align*}
bo całki się zwijają do $\int e^{-1/2\lambda_iz_i^2}dz_i=\frac{\sqrt{2\pi}}{\sqrt{\lambda_i}}$ i tak jakby przy $k$-tej współrzędnej nie podstawiam

\begin{align*}
    \e X_k&=\int_\R xg_k(x)dx=\int_\R x\cdot\frac{\sqrt{\lambda_k}}{\sqrt{2\pi}}e^{-1/2\lambda_k(x-m_k)^2}dx=
\end{align*}

\begin{problem}{}
Niech $X_1$ i $X_2$ będą niezależnymi zmiennymi losowymi o rozkładzie $N(0,1)$. Wykaż, że zmienne losowe $\frac{X_1+X_2}{\sqrt{2}}$ i $\frac{X_1-X_2}{\sqrt{2}}$ są niezależne i obie mają rozkład $N(0,1)$.
\end{problem}

\begin{problem}{}
Niech $X=(X_1,X_2,...,X_n)$ będzie wektorem losowym o standardowym rozkładzie normalnym $N(0,I)$, gdzie $I$ jest macierzą identyczności. Sprawdź, że $X_1,X_2,...,X_n$ są niezależnymi zmiennymi losowymi o jednakowym standardowym rozkładzie normalnym $N(0,1)$.
\end{problem}

\begin{problem}{}
Niech $X_1,X_2,...,X_n$ będą wzajemnie nieskorelowanymi zmiennymi losowymi takimi, że ich łączny rozkład jest normalny. Wykazać, że $X_1,X_2,...,X_n$ są niezależne.
\end{problem}

\begin{problem}{}
Niech $X_1,...,X_n$ będą niezależnymi zmiennymi losowymi o rozkładzie normalnym $N(0,1)$ oraz niech $a=(a_1,...,a_n)$ i $b=(b_1,...,b_n)$ będą ustalonymi wektorami. Pokaż, że zmienne losowe
$$W=\sum_{j=1}^na_jX_j,\quad Z=\sum_{j=1}^n b_jX_j$$
są niezależne $\iff$ wektory $a$ i $b$ są prostopadłe. Opisz rozkłady $W$ i $Z$.
\end{problem}

\begin{problem}{}
Podaj przykład nieskorelowanych zmiennych losowych o rozkładzie normalnym, które nie są niezależne.
\end{problem}

\begin{problem}{}
(\textbf{Transformata Boza=M\"ullera}) Pokaż, że jeśli zmienne losowe $X,Y$ są niezależne o rozkładzie jednostajnym na $(0,1)$, to
$$U=\sqrt{-2\log X}\cos(2\pi Y) \quad i\quad V=\sqrt{-2\log X}\sin(2\pi Y)$$
są niezależne i mają rozkład $N(0,1)$.
\end{problem}

\begin{problem}{}
Niech $A_1,...,A_{2021}\in\set{F}$ będą zbiorami o własności $\prob{A_i}\geq\frac{1}{2}$. Wykaż, że istnieje $\omega\in\Omega$ taka, że $\omega\in A_i$ dla przynajmniej $1011$ wartości $i$.
\end{problem}

\begin{problem}{}
Dane są dwa ciągi $\{X_n\}_{n\geq1},\{Y_n\}_{n\geq 1}$ zbieżne prawie wszędzie do zmiennych $X,Y$. Pokaż, że jeśli dla każdego $n$ zmienne $X_n$ i $Y_n$ mają ten sam rozkład, to $X$ i $Y$ też mają ten sam rozkład.
\end{problem}

\end{document}
