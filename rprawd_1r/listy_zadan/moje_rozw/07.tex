\documentclass{article}

\usepackage[polish]{../../../lecture_notes}

\title{Lista 7}
\author{}
\date{}

\begin{document}
\maketitle
\thispagestyle{empty}

\begin{problem}{u}
  Monika wybrała się do kasyna w Las Vegas mając przy sobie 255\$. Jako cel postawiła sobie wygranie $1$ dolara i wyjście z kasyna z kwotą 256\$. Podczas tej wizyty obstawiała kolory. Wszystkie pola poza 0 i 00 są czerwone lub czarne (po 18 pól). Poprawne wskazanie koloru (z prawdopodobieństwem $\frac{18}{38}$ podwaja zaryzykowaną kwotę. Monika zastosowała następującą strategię: postanowiła, że będzie grać kolejno o 1\$, 2\$, 4\$, 8\$, 16\$, 32\$, 64\$, 128\$. Jeżeli w jednej z gier wygra, zabiera nagrodę i opuszcza kasyno z 256\$ dolarami. Obliczyć prawdopodobieństwo, że jej się powiodło. Obliczyć wartość oczekiwaną wygranej.
\end{problem}

Chyba nie rozumiem tej gry.

\begin{problem}{d}
  Oblicz $\e X$ jeżeli $X$ jest zmienną o rozkładzie
  \begin{enumerate}[label=(\alph*)]
    \item $Poiss(\lambda)$
    \item $Exp(\lambda)$
    \item $Geom (p)$
  \end{enumerate}
\end{problem}

To jest po prostu liczenie całki lub sumy?

\begin{enumerate}[label=(\alph*)]
  \item $Poiss(\lambda)$

  \begin{align*}
    \sum_{k> 0}k\cdot \frac{\lambda^ke^{-\lambda}}{k!}=\sum_{k>0}\frac{\lambda^ke^{-\lambda}}{(k-1)!}=\lambda e^{-\lambda}\sum_{k>0}\frac{\lambda ^{k-1}}{(k-1)!}=\lambda e^{-\lambda}\cdot e^{\lambda}=\lambda
  \end{align*}
  bo ta suma to wzór Taylora.

  \item $Exp(\lambda)$

  \begin{align*}
    \int_0^\infty x\lambda e^{-\lambda x}\;dx=\begin{bmatrix}u=x & du = dx\\
    v=-e^{-\lambda x} & dv=\lambda e^{-\lambda x}\end{bmatrix}=\int_0^\infty e^{-\lambda x}\;dx=\frac{1}{\lambda}
  \end{align*}

  \item $Geom(p)$

    $$\sum x^k=\frac{1}{1-x}$$
    $$\sum k\cdot x^{k-1}=\frac{1}{(1-x)^2}$$

    \begin{align*}
      \sum_{k>0}k\cdot (1-p)^{k-1}\cdot p=p\cdot\sum_{k>0}k\cdot (1-p)^{k-1}=\frac{p}{(1-(1-p))^2}=\frac{1}{p}
    \end{align*}
\end{enumerate}

\begin{problem}{}
  Zmienna losowa $X$ ma rozkład jednostajny $U[0, 1]$. Obliczyc $\e Y$ jeżeli

  \begin{enumerate}[label=\alph*)]
    \item $Y=e^X$
    \item $Y=\cos^2(\pi X)$
  \end{enumerate}
\end{problem}

\begin{enumerate}[label=\alph*)]
  \item $Y=e^X$
    
  Chcę poznać $f_Y$, czyli liczę dystrybuantę i różniczkuję:
  \begin{align*}
    F_y(t)=\prob{Y\leq t}=\prob{e^X\leq t}=\prob{X\leq \ln t}=\int_0^{\ln t}1\;dx=\ln t
  \end{align*}
  czyli $f_Y(t)=\frac{1}{t}$. Wartości $Y$ są z przedziału $[e^0, e^1]$:
  $$\e Y=\int_1^ey\cdot\frac{1}{y}\;dy=e-1$$
  
  \item $Y=\cos^2(\pi X)$

  Zgaduję, że to tak samo jak wcześniej:
  \begin{align*}
    F_y(t)&=\prob{Y\leq t}=\prob{\cos^2(\pi X)\leq t}=\prob{\cos(\pi X)\in[-|t|, |t|]}=\\
          &=\prob{X\in[-\frac{}{}}.
  \end{align*}

  {\large\color{red}TUTAJ MI SIĘ ODECHCIAŁO}

\end{enumerate}

\begin{problem}{}
  Zmienna losowa $X$ ma rozkład jednostajny $U[0, 1]$. Obliczyć $\e X, \e[1/(1+X^5)]$.
\end{problem}
$$\e X=\int_0^1x\;dx=\frac{1}{2}$$

\begin{align*}
  \e[1/(1+X^5)]=\int_0^1
\end{align*}

\end{document}
