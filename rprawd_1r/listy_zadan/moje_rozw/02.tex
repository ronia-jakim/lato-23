\documentclass{article}

\usepackage{../../../notatki}
%\usetikzlibrary{patterns}

\usetikzlibrary{patterns, patterns.meta}

\begin{document}

\subsection*{ZADANIE 1.}
\emph{Udowodnić wzór włączeń i wyłączeń}
$$\prawdo{\bigcup\limits_{i=1}^nA_i}=\sum\limits_{i=1}^n\prawdo{A_i}-\sum\limits_{1\leq i<j\leq n}\prawdo{A_i\cap A_j}+...+(-1)^{n+1}\prawdo{\bigcap\limits_{i=1}^n A_i}$$

Indukcja po $n$. Jeżeli mamy tylko $A_1, A_2$, to
$$\prawdo{A_1\cup A_2}=\prawdo{A_1\cup (A_2\setminus(A_1\cap A_2))}=\prawdo{A_1}+\prawdo{A_2\setminus(A_1\cap A_2)}=\prawdo{A_1}+\prawdo{A_2}-\prawdo{A_1\cap A_2}$$

Teraz załóżmy, że wzorek działa dla dowolnego ciągu długości $n$ i weźmy ciąg długości $(n+1)$. Mamy
\begin{align*}
    \prawdo{A_1\cup...\cup A_n\cup A_{n+1}}&=\prawdo{(A_1\cup...\cup A_n)\cup(A_{n+1}\setminus(A_1\cup...\cup A_n))}=\\
    &=\prawdo{A_1\cup...\cup A_n}+\prawdo{A_{n+1}\setminus(({A_1\cup...\cup}A_n)\cap A_{n+1})}=\\
    &=\prawdo{A_1\cup...\cup A_n}+\prawdo{A_{n+1}}-\prawdo{A_{n+1}\cap(A_1\cup...\cup A_n)}=\\
    &=\prawdo{A_1\cup...\cup A_n}+\prawdo{A_{n+1}}-\prawdo{(A_{n+1}\cap A_1)\cup...\cup(A_{n+1}\cap A_n)}=\\
    &=\sum\limits_{i\leq n+1}\prawdo{A_i}-\sum\limits_{i<j\leq n}\prawdo{A_i\cap A_j}+...(-1)^{n+1}\prawdo{\bigcap\limits_{i=1}^n A_i}+\\
    -&\left(\sum\limits_{i\leq n}\prawdo{A_{n+1}\cap A_i}-\sum\limits_{i<j\leq n}\prawdo{(A_{n+1}\cap A_i)\cap (A_{n+1}\cap A_i)}+...+(-1)^{n+1}\prawdo{\bigcap\limits_{i\leq n}(A_{n+1}\cap A_i)}\right)=\\
    &=\sum\limits_{i\leq n+1}\prawdo{A_i}-\sum\limits_{i<j\leq n+1}\prawdo{A_i\cap A_j}+...+(-1)^{n+2}\prawdo{\bigcap\limits_{i\leq n+1}A_i}
\end{align*}

\subsection*{ZADANIE 2.}
(Nierówności Boole'a) \emph{Udowodnij nierówności} [te na {\color{blue}niebiesko}]
$$\color{blue}\prawdo{\bigcup\limits_{i\leq n}A_i}\leq\sum\limits_{i\leq n}\prawdo{A_i}$$

Może tutaj też indukcją? Dla $A_1, A_2$ jest to dość oczywiste, bo
$$\prawdo{A_1\cup A_2}=\prawdo{A_1}+\prawdo{A_2}-\prawdo{A_1\cap A_2}\leq \prawdo{A_1}+\prawdo{A_2}$$
gdyż $\prawdo{A_1\cap A_2}\geq0$.

To teraz załóżmy, że śmiga dla dowolnego ciągu $n$ zbiorów i weźmy ciąg $(n+1)$-elementowy. 

\begin{align*}
    \prawdo{\bigcup\limits_{i\leq n+1} A_i}&=\prawdo{\bigcup\limits_{i\leq n}A_i\cup A_{n+1}}=\\
    &=\prawdo{\bigcup\limits_{i\leq n}A_i}+\prawdo{A_{n+1}}-\prawdo{A_{n+1}\cap\bigcap\limits_{i\leq n}A_i}\leq \\
    &\leq\prawdo{A_{n+1}}+\prawdo{\bigcup\limits_{i\leq n}A_i}\leq\\
    &\leq \prawdo{A_{n+1}}+\sum\limits_{i\leq n}\prawdo{A_i}=\\
    &=\sum\limits_{i\leq n+1}\prawdo{A_i}
\end{align*}
Pierwsza nierówność tak samo jak wcześniej, a druga nierówność z założenia indukcyjnego.

$$\color{blue}\prawdo{\bigcap\limits_{i\leq n}A_i}\geq 1-\sum\limits_{i\leq n}\prawdo{A_i^c}$$

To śmiga na podstawie poprzedniej nierówności:
\begin{align*}
    \prawdo{\bigcap A_i}=\prawdo{\left(\bigcup A_i\right)^c}=1-\prawdo{\bigcup A_i}\geq1-\sum\prawdo{A_i}
\end{align*}

\subsection*{ZADANIE 3.}
\emph{Pokaż, że jeżeli $\prawdo{A_i}=1$ dla $i\geq 1$, to $\prawdo{\bigcap\limits_{i=1}^\infty A_i}=1$}
\medskip

Rozważmy ciąg $B_n$ taki, że $B_n=\bigcap\limits_{i\leq n}A_i$. Jak wygląda prawdopodobieństwo takiego osła?
\begin{align*}
    \prawdo{B_n}=\prawdo{\bigcap\limits_{i\leq n}A_i}\geq 1-\sum\limits_{i\leq n}\prawdo{A_i^c}=1-\sum\limits_{i\leq n}[1-\prawdo{A_i}]=1-\sum\limits_{i\leq n}0=1
\end{align*}
Skoro $\prawdo{B_n}\geq 1$, to musi być równe $1$.

Skorzystamy teraz z twierdzenia o ciągłości, które mówi, że dla zstępującego ciągu $B_n$ (jakim on jest, bo to widać) mamy
$$\prawdo{\bigcap B_n}=\lim\prawdo{B_n}=1$$
a ponieważ
$$\bigcap B_n=\bigcap \bigcap\limits_{i\leq n}A_i=\bigcap A_n$$
to śmiga.

\subsection*{ZADANIE 4.}
\emph{Rzucamy symetryczną kostką do gry do chwili otrzymania szóstki. Zdefiniuj odpowiednią przestrzeń probabilistyczną. Jaka jest szansa, że liczba rzutów będzie podzielna przez $6$?}
\medskip

Dziwne to zadanko. $\Omega$ to ciągi liczb $1,2,...,5$ które na końcu mają $6$. Nas interesuje ich długość. Może zróbmy tak, że rzucamy $6n$ razy kostką i zapisujemy kolejne wyniki. Zdarzenia sprzyjające to rzuty, w których $6$ pojawia się po raz pierwszy na pozycji podzielnej przez $6$? 

To będzie ciąg rzeczy wstępujących. Dla $n=1$ prawdopodobieństwo to po prostu 
$$\left(\frac56\right)^5\cdot\frac16={5^5\over 6^6}.$$
Dka $n=2$ jest już troszkę ciężej, ale prawdopodobieństwo to
$$\left(\frac56\right)^5\cdot\frac16+\left(\frac56\right)^{11}\cdot\frac16,$$
czyli wyrzuci $6$ w $6$ ruchu lub wyrzuci $6$ w $12$ ruchu. Uogólnić to można do
\begin{align*}
    p_n&=\sum\limits_{i\leq n}\left(\frac56\right)^{6i-1}\frac16=\frac16\cdot\frac65\sum\limits_{i\leq n}\left(\frac56\right)^{6i}=\frac15\cdot{5^6\over 6^6}\cdot{(5/6)^{6n}-1\over(5/6)^6-1}={5^5\cdot((5/6)^{6n}-1)\over5^6-1}
\end{align*}
Nie chce mi się liczyć, ale na oko to jest jakieś
$${5^5\over 5^6-1}$$

\subsection*{ZADANIE 5.}
\emph{Na odcinku $[0,1]$ umieszczono losowo punkty $L$ i $M$. Obliczyć prawdopodobieństwo, że:}

\indent \emph{\color{blue}a) środek odcinka $LM$ należy do $[0, \frac13]$}
\smallskip

Żeby ich środek był w $[0, \frac13]$, to ich średnia arytmetyczna musi być mniejsza niż $\frac13$, czyli
$${x+y\over 2}\leq \frac13$$
$$x+y\leq \frac23$$
No i coś takiego na płaszczyźnie to jest trójkącik
\begin{illustration}
    \filldraw[color=green, very thick] (0, 0)--(2, 0)--(0, 2)--cycle;
    \draw [very thick] (0, 0)--(3.5, 0);
    \draw [very thick] (0, 0)--(0, 3.4);
    \draw [very thick] (-0.05, 3)--(0.05, 3);
    \draw [very thick] (3, 0.05)--(3, -0.05);
    \node at (-0.4, 3) {1};
    \node at (3, -0.4) {1};
    \node at (-0.4, 2) {$\frac23$};
    \node at (2, -0.4) {$\frac23$};
\end{illustration}
Czyli szukane prawdopodobieństwo to pole tego trójkącika, a więc $\frac29$.
\smallskip

\indent \emph{\color{blue}b) z $L$ jest bliżej do $M$ niż do zera.}
\smallskip

Czyli $|L-M|>L$, znowu ładnie to się zaprezentuje w dwóch wymiarach. Po co oni dali zadanie o prostej, które się robi płaszczyzną?

Dla $L\geq M$ mam $|L-M|=L-M>L$, czyli $0>M$ co tak średnio śmiga. Dla $L<M$ mam z kolei $|L-M|=M-L>L$, czyli $M>2L$ 
\begin{illustration}
    \filldraw[color=green] (0, 0)--(0, 3)--(1.5, 3)--cycle;
    \draw [very thick] (0, 0)--(4, 0);
    \draw [very thick] (0, 0)--(0, 4);
    \draw [very thick] (-0.05, 3)--(0.05, 3);
    \draw [very thick] (3, 0.05)--(3, -0.05);
    \node at (-0.4, 3) {1};
    \node at (3, -0.4) {1};
    \node at (4, -0.4) {L};
    \node at (-0.4, 4) {M};
    \draw [dashed] (0, 3)--(3, 3);
    \draw [dashed] (3, 0)--(3, 3);
    \draw [dashed] (1.5, 3)--(1.5, 0);
    \node at (1.5, -0.4) {$\frac12$};
\end{illustration}
Czyli tutaj prawdopodobieństwo to $\frac14$.
\newpage

\subsection*{ZADANIE 6.}
\emph{Z przedziału $[0, 1]$ wybrano losowo dwa punkty, które podzieliły go na trzy odcinki. Obliczyć prawdopodobieństwo, że z tych odcinków można skonstruować trójkąt.}
\smallskip

Zamiast rozważać położenie punktów od $0$, rozważę długości odcinków przez nich tworzonych. Moje odcinki mają długość $x, y, 1-y-x$, gdzie $x<\frac12$, bo nie ważne jaka będzie kolejność rzucania punktów, zawsze mogę ten krótszy odcinek wyróżnić bez tracenia niczego. Czyli działam na $\Omega=[0, \frac12]\times[0, 1]$.

Potrzebuję, żeby $x,y$ spełniały następujące warunki na raz:

\indent 1. $x+y>1-y-x$, czyli $2x+2y>1$, $\color{orange}y>\frac12-x$

\indent 2. $x+1-x-y>y$, czyli $1>2y$, $\color{green}\frac12>y$

\indent 3. $y+1-x-y>x$, czyli $1>2x$, $\frac12>x$ <-- ten jest już spełniony.

ROZRYSUJMY TO!
\begin{illustration}
    \draw [very thick] (0, 0)--(4, 0);
    \draw [very thick] (0, 0)--(0, 4);
    \draw [very thick] (-0.05, 3)--(0.05, 3);
    \draw [very thick] (3, 0.05)--(3, -0.05);
    \node at (-0.4, 3) {1};
    \node at (3, -0.4) {1};
    \node at (1.5, -0.4) {$\frac12$};
    \node at (4, -0.4) {x};
    \node at (-0.4, 4) {y};
    \draw [dashed] (0, 3)--(3, 3);
    \draw [dashed] (3, 0)--(3, 3);
    \filldraw [color=back2, fill=back2] (0, 1.5)--(1.5, 1.5)--(1.5, 0)--cycle;
    %\filldraw [color=blue, pattern={Dots[distance=3mm, radius=0.7mm]}, pattern color=blue] (0, 0)--(1.5, 0)--(1.5, 3)--(0, 3)--cycle;
    \filldraw [very thick, color=green, pattern={Dots[radius=0.7mm, distance=3mm, yshift=1.5mm, xshift=1.5mm]}, pattern color=green] (0, 0)--(1.5, 0)--(1.5, 1.5)--(0, 1.5)--cycle;
    \filldraw [very thick, color=orange, pattern={Dots[distance=3mm, radius=0.7mm]}, pattern color=orange] (1.5, 0)--(0, 1.5)--(0, 3)--(1.5, 3)--(1.5, 0)--cycle;
    \draw [very thick, dashed] (1.5, 0)--(1.5, 3);
\end{illustration}

Czyli to gdzie oba warunki są spełnione to ten trójkącik w środku o polu $\frac18$, ale ponieważ pole całości wynosi $\frac12$, to dostajemy
$$\prawdo{A}={\frac18\over\frac12}=\frac14.$$

\subsection*{ZADANIE 7.}
\emph{Wybrano losowy punkt $(x,y)$ z kwadratu $[0,1]^2$. Oblicz prawdopodobieństwo, że}

\indent \emph{\color{blue}a) $x$ jest liczbą wymierną.}
\smallskip

To po prostu miara $\lambda(\Q)=0$
\smallskip

\indent \emph{\color{blue}b) obie liczby $x$ i $y$ są niewymierne.}
\smallskip

$A^c$ to co najmniej jedna liczba jest wymierna. Czyli mam $A^c=$ x wymierny + y wymierny + oba wymierne. Oba wymierne mają prawdopodobieństwo 0, tak samo to, że $x$ jest wymierny i że $y$ jest wymierny też ma prawdopodobieństwo $0$, czyli $A^c$ wydaje się mieć prawdopodobieństwo $0$, a więc $\prawdo A=1$?
\smallskip

\indent \emph{\color{blue}c) spełniona jest nierówność $x^2+y^2<1$}
\smallskip

Czyli punkt ma wylądować w środku ćwiartki koła o promieniu $1$, a więc $\prawdo{A}=\frac\pi4$.
\smallskip

\indent \emph{\color{blue}d) spełniona jest równość $x^2+y^2=1$.}
\smallskip

No to też jest miary zero? Bo ograniczam od góry przez koło o promieniu $1+\varepsilon$ i od dołu przez $1-\varepsilon$. Miara to ich różnica i jest dowolnie mała.

\subsection*{ZADANIE 8.}
\emph{W kwadracie $[0,1]^2$ wybrano losowo dwa punkty $A$ i $B$. Zdefiniuj odpowiednią przestrzeń probabilistyczną. Oblicz prawdopodobieństwo, że}

\indent \emph{\color{blue}a) odcinek $AB$ przecina przekątną łączącą wierzchołki $(0,0)$ i $(1,1)$}
\smallskip

To ten, wybieram $A$ w dolnym trójkącie, to będzie $\frac12$ i każę $B$ być w górnym, to też jest $\frac12$. Jest symetryczne, więc całość to dwa razy $2\cdot \frac12\cdot\frac12=\frac12$.

\indent \emph{\color{blue}b) odległość punktu $A$ od $(1, 1)$ jest mniejsza niż $1$, a odległość punktu $B$ od $(1, 1)$ jest większa niż $1$}
\smallskip

Czy to będzie wybieram $A$, jakie jest prawdopodobieństwo, że $A$ będzie w środku ćwiartki koła jednostkowego. Potem wybieram $B$, jakie jest prawdopodobieństwo, że $B$ nie będzie w środku tej ćwiartki. Mnożę i ta da? Czyli
$$\frac\pi4\cdot(1-\frac\pi4)$$

\indent \emph{\color{blue}c) oba punkty leżą pod parabolą $y=-x(x-1)$}
\smallskip

To akurat to jest chyba pole pod parabolą do kwadratu?
$$\int_0^1x(1-x)dx=\int_0^1x-\int_0^1x^2dx=1-\frac12=\frac12$$
czyli całość to $\frac14$.

\subsection*{ZADANIE 9.}
\emph{Igłę o długości $l$ rzucono na podłogę z desek o szerokości $a\geq l$. Znajdź prawdopodobieństwo, że igła przetnie krawędź deski.}
\smallskip

Zgaduję, że tutaj będą zdarzenia wstępujące i rozważam $n$ desek, ale mi się bardzo nie chce.

\subsection*{ZADANIE 10.}
\emph{Niech $(\Omega,\set{F})$ będzie przestrzenią mierzalna. Uzasadnij, że $\sigma$-ciało $\set{F}$ nie może być nieskończoną przeliczalną rodziną zbiorów.}
\smallskip

A to akurat robiłam jako pracę domową na MiC XD.
\smallskip

Mogę wziąć sobie dowolny $A_1\in \set{F}$. Zdefiniujmy teraz $\set{F}_1$ jako tylko te zbiory z $\sigma$-algebry, które są zawarte w $A_1^c$. Wyciągnijmy nowy zbiór $A_2\in\set{F}_1$. Od razu widzimy, że zawsze $A_2\cap A_1=\emptyset$. Możemy tak lecieć dalej, zwężając za każdym razem sigma algebrę do dopełnienia $A_n$ i brać $A_{n+1}$ z tego zwężenia, zawsze zbiory będą parami rozłączne, bo schodzimy coraz to niżej. Ładnie to można narysować.
\smallskip

MÓJ NA SZYBKO DOWODZIK:

Weźmy $\Omega$ o mocy $\omega$.

Cały dowód to skonstruowanie sobie ciągu rozłącznych zbiorów, ich różne sumy zawsze będą różne, a tych sum możemy wybrać na $2^\omega$, czyli $\set{F}$ jest nieprzeliczalne.

To lecimy. Weźmy sobie dowolny ciąg $A_1\subsetneq A_2\subsetneq A_3\subsetneq...\in\Omega$ oraz $\prawdo{A_1}<1$. Możemy tak zrobić, choćby dlatego, że biorą kolejno sumę coraz to większej liczby singletonów dostaję nowego pyśka. Zdefiniujmy teraz ciąg $B_1\subseteq A_1$, $B_2\subseteq A_2\setminus A_1$ i ogólniej
$$B_n\subseteq A_n\setminus\bigcup\limits_{i< n} A_{i}$$
No i teraz $B_n$ są rozłączne.

\subsection*{ZADANIE 11.}
\emph{Oznaczmy przez $\set{B}_0$ ciało składające sie ze skończonych sum rozłącznych przedziałów $(a, b]$ zawartych w odcinku $(0,1]$. Określmy na $\set{B}_0$ funkcję $P$ taką, że $P(A)=1$ lub $0$, w zależności od tego, czy zbiór $A$ zawiera przedział postaci $(\frac12,\frac12+\varepsilon]$ dla pewnego $\varepsilon>0$, czy też nie. Pokaż, że $P$ jest miarą addytywną, ale nie przeliczalnie addytywną.}
\smallskip

Skończoną addytywność śmignie się za chwilę, najpierw uwalmy przeliczalną addytywność.

Rozważmy ciąg zbiorów zdefiniowany:
$$A_n=(\frac12+\frac1{2^{n+1}}, \frac12+\frac1{2^{n}}]$$
Oczywiście $A=\bigcup A_n=(\frac12, 1)$, czyli $P(A)=1$. Czy one są już rozłączne? Ej no są XD
$$A_i\cap A_{i+1}=(\frac12+\frac1{2^{i+1}}, \frac12+\frac1{2^i}]\cap(\frac12+\frac1{2^{i+2}},\frac12+\frac1{2^{i+1}}]=\emptyset$$
Dla dowolnego $n$ $P(A_n)=0$, bo nie zawiera odcinka $(\frac12, \frac12+\varepsilon]$, ale już suma go zabiera, więc nie jest to funkcja przeliczalnie addytywna.

W sumie skończona addytywność jest widoczna od razu. Weźmy dowolny skończony ciąg rozłącznych pyśków.

\end{document}