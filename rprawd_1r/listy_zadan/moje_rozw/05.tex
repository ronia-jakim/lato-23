\documentclass{article}

\usepackage{../../../lecture_notes}

\title{Rachunek Prawdopodobieństwa 1R\\{\normalsize Lista 5}}
\author{Weronika Jakimowicz}
\date{}

\begin{document}
\maketitle\thispagestyle{empty}

\begin{problem}[1]{d}
Czy $\lambda$-układ jest zawsze $\sigma$-ciałem?
\end{problem}

NIE, ale $\sigma$-ciało jest zawsze $\lambda$-układem.

Popatrzmy sobie na przestrzeń rzucania dwa razy monetą. Niech $A=\{(O, O), (O, R)\}$, a $\mathfrak{L}$ będzie zbiorem zdarzeń niezależnych od $A$ (zamkniętość na sumy i różnice już troszkę była na poprzednich listach, więc nie rozpisuję). Poniższe zbiory są na przykład w takim $\mathfrak{L}$:
$$\begin{matrix}\{(O, O), (R, R)\} & \{(O, O), (R, O)\}\end{matrix}.$$
Gdyby $\mathfrak{L}$ było $\sigma$-ciałem, to suma powyższych zdarzeń, czyli $\{(O, O), (R, R), (R, O)\}$, należałaby do $\mathfrak{L}$. Tak ewidentnie nie jest, bo $\prob$ przekroju wynosi $\frac{1}{4}$, a iloczyn $\prob$ to $\frac{1}{2}\cdot\frac{3}{4}$.

\begin{problem}[2]{}
Niech $X$ i $Y$ będą zmiennymi losowymi. Oznaczmy przez $\mu_X$ i $\mu_Y$ ich rozkłady. Pokaż, że rodzina
$$\mathfrak{L}=\{A\in Bor(\R)\;:\;\mu_X(A)=\mu_Y(A)\}$$
jest $\lambda$-układem.
\end{problem}

\begin{itemize}
    \item $\R\in\mathfrak{L}$ jest dość oczywista, bo $\mu_X(\R)=\prob{X\in\R}=1=\prob{Y\in\R}=\mu_Y(\R)$.
    \item $A\subseteq B\implies B\setminus A\in\mathfrak{L}$

    Teraz bierzemy $A,B\in\mathfrak{L}$, czyli $\mu_X(A)=\mu_Y(A)$ i $\mu_X(B)=\mu_Y(B)$ i BSO $A\subseteq B$. Wtedy
    \begin{align*}
        \mu_X(B\setminus A)&=\prob{X\in B\setminus A}=\prob{X\in B}-\prob{X\in A}=\\
        &=\prob{Y\in B}-\prob{Y\in A}=\prob{Y\in B\setminus A}=\mu_Y(B\setminus A)
    \end{align*}
    \item $A_1\subseteq A_2\subseteq...\implies\bigcup A_i\in\mathfrak{L}$
    \begin{align*}
        \mu_X(\bigcup A_i)&=\prob{X\in\bigcup A_i}=\lim_N\prob{X\in\bigcup_1^NA_i}=\lim_N\prob{X\in A_N}=\\
        &\lim_N\prob{Y\in A_N}=\lim_N\prob{Y\in\bigcup_1^NA_i}=\prob{Y\in\bigcup A_i}=\mu_Y(\bigcup A_i)
    \end{align*}
\end{itemize}


\end{document}
