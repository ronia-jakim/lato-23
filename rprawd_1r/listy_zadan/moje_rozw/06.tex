\documentclass{article}

\usepackage[polish]{../../../lecture_notes}

\title{Lista 6\\{\normalsize Rachunek Prawdopodobieństwa 1R}}
\author{Weronika Jakimowicz}
\date{}

\usepackage{wrapfig}

\begin{document}
\maketitle
\thispagestyle{empty}

\begin{problem}{d}
Niech $(X,Y)$ będzie $2$-wymiarową zmienną losową o rozkładzie zadanym gęstością $f(x,y)=C(x+y)$ dla $0\leq y\leq x\leq 1$ i $f(x,y)=0$ poza tym zbiorem. Znajdź wartość $C$. Znajdź rozkłady brzegowe. Czy zmienne losowe $X$ i $Y$ są niezależne?
\end{problem}

\begin{wrapfigure}{r}{0.25\textwidth}
\begin{tikzpicture}
    \filldraw[color=green, pattern=dots, pattern color=green](0, 0)--(4.8, 0)--(4.8, 4.8)--cycle;
    \draw[->](-0.3, 0)--(5.3, 0) node [above] {x};
    \draw[->](0, -0.3)--(0, 5.3) node [left] {y};
\end{tikzpicture}
\end{wrapfigure}
Mamy dane
$$f(x,y)=\begin{cases}C(x+y)&0\leq y\leq x\leq1\\0&wpp\end{cases}$$
i w pierwszej kolejności pytamy o wartość zmiennej $C$. Wiemy, że 
$$\int_{\R^2}f(x,y)=1,$$ 
a ponieważ my żyjemy w świecie trójkąta pod $y=x$, to mamy:
$$1=\int\limits_0^1\int\limits_0^xC(x+y)dydx=C\int_0^1(x^2+\frac{x^2}{2})dx=C(\frac{1}{3}+\frac{1}{6})$$
czyli z moich bardzo precyzyjnych kalkulacji wynika, że $C=2$.

Teraz pora na rozkłady brzegowe. 
\begin{align*}
    \prob{X\in A}&=\prob{(X,Y)\in A\times\R}=\int_{A\times\R}f(x,y)dydx=\int_A\int_\R2(x+y)dydx=\\
    &=\int_A\int_0^x2(x+y)dydx=\int_A{3x^2}dx
\end{align*}
\begin{align*}
    \prob{Y\in B}&=\prob{(X,Y)\in\R\times B}=\int_{\R\times B}f(x,y)dydx=\int_\R\int_Bf(x,y)dydx=\int_B\int_\R f(x,y)dxdy=\\
    &=\int_B\int_y^12(x+y)dxdy=\int_B[1+2y-3y^2]dy
\end{align*}

Na pytanie, czy są to zmienne niezależne odpowiadamy patrząc na gęstości tych dwóch zmiennych losowych. Żeby były niezależne, musiałoby zachodzić
$$f(x,y)=f_X(x)f_Y(y)$$
Tutaj mamy
$$f_X(x)f_Y(y)=3x^2(1+2y-3y^2)\neq2(x+y)$$
więc są bardzo zależne.

\begin{problem}[4]{}
Niech $X_1,...,X_n$ będą niezależnymi zmiennymi losowymi o rozkładzie wykładniczym z parametrem $1$. Znajdź rozkład $Y=\min\limits_{1\leq i\leq n}X_i$. Czy $X_n$ i $Y$ są niezależne?
\end{problem}

Mają rozkład wykładniczy, więc funkcja gęstości $X_i$ to:
$$f_i(t)=\lambda_ie^{-\lambda_i t}$$
jeśli $t\geq 0$, wpp. mamy $0$.


\begin{problem}[5]{}
Zmienne losowe $X$ i $Y$ są niezależne. Pokaż, że jeżeli $X$ nie ma atomów, to $P(X=Y)=0$.
\end{problem}

Ponieważ $X$ nie ma atomów, to zbiór $\{x\;:\;\prob{X=x}>0\}=\emptyset$.

Szukamy tak naprawdę $\prob{X=t, Y=t}$ po wszystkich $t$. Ponieważ zdarzenia są niezależne, to:
\begin{align*}
    \prob{X=Y}&=i\int_{-\infty}^\infty\prob{X=t, Y=t}dt=\int\prob{X=t}\prob{Y=t}dt=\int0dt=0,
\end{align*}
gdyż $X$ jest bezatomowa.

\begin{problem}[7]{d}
Pokaż, że zmienne losowe $X_1,...,X_n$ o gęstościach $f_1,...,f_n$ są niezależne $\iff$ zmienna $X=(X_1,...,X_n)$ ma gęstość
$$f(x_1,...,x_n)=f_1(x_1)...f_n(x_n)$$
\end{problem}

\begin{center}
Niezależne $\iff$ $f(x_1,...,x_n)=f_1(x_1)...f_n(x_n)$
\end{center}

$\implies$

Nirch $T_i\subseteq\R$, wtedy z niezależności zmiennych mamy:
\begin{align*}
    \prob{X_1\in T_1,...,X_n\in T_n}&=\int_{T_1}...\int_{T_n}f(x_1,...,x_n)dx_1...dx_n=\\
    &=\int_{T_1}f_1(x_1)dx_1...\int_{T_n}f_n(x_n)dx_n=\prob{X_1\in T_1}...\prob{X_n\in T_n}
\end{align*}
rozpisując krok po kroku:
\begin{align*}
    \prob{X_1\in T_1}...\prob{X_n\in T_n}&=\int_{T_1}f_1(x_1)dx_1...\int_{T_n}f_n(x_n)dx_n=\int_{T_1}f_1(x_1)\int_{T_2}f_2(x_2)dx_2dx_1...\int_{T_n}f_n(x_n)dx=\\
    &=\int_{T_1}\int_{T_2}f_1(x_1)f_2(x_2)dx_2dx_1...\int_{T_n}f_n(x_n)dx_n=...=\int_{T_1}...\int_{T_n}f_1(x_1)...f_n(x_n)dx_1...dx_n=\\
    &=\int_{T_1}...\int_{T_n}f(x_1,...x_n)dx_1...dx_n=\prob{X_1\int T_1,...,X_n\in T_n}
\end{align*}
Ponieważ dzieje się tak dla dowolnych $T_i$, to funkcje pod całką muszą się równać (prawie wszędzie?). Czyli
$$f(x_1,...,x_n)=f_1(x_1)...f_n(x_n).$$

$\impliedby$

Wychodzimy z tego, że
$$f(x_1,...,x_n)=f_1(x_1)...f_n(x_n).$$
Wybierając dowolne $T_i\subseteq\R$ i całkując obie strony dostajemy:
\begin{align*}
\int_{T_1}...\int_{T_n}f(x_1,...,x_n)dx_1...dx_n&=\int_{T_1}...\int_{T_n}f_1(x_1)...f_n(x_n)dx_1...dx_n=\\
&=\int_{T_1}...\int_{T_{n-1}}f_1(x_1)...f_{n-1}(x_{n-1})dx_1...dx_n\int_{T_n}f_n(x_n)dx_n=...\\
&=\int_{T_1}f_1(x_1)dx_1...\int_{T_n}f_n(x_n)dx_n
\end{align*}
Prawa strona równania to iloczyn $\prob{X_1\in T_1}...\prob{X_n\in T_n}$, a lewa to $\prob{X_1\in T_1,...,X_n\in T_n}$ i znowu dzieje się tak bez względu na wybór $T_i$, czyli mamy równość i zmienne są niezależne.

\begin{problem}[8]{d}
Z odcinka $[0,1]$ losujemy niezależnie w sposób jednostajny liczby $X_1,X_2,...$. Uzasadnij, że z prawdopodobieństwem $1$ ciąg $\{X_n\}$ jest gęsty w odcinku $[0,1]$.
\end{problem}

Weźmy sobie dowolną kulę na odcinku $[0,1]$ o promieniu $r$ i środku $x$: $B_r(x)$. Prawdopodobieństwo, że ani jedna ze zmiennych w nią trafi wynosi $1-2r$ (tutaj $r\leq\frac12$). Losujemy niezależnie, więc zmienne są niezależne. Jeśli rozważymy pierwsze $n$ zmiennych, to prawdopodobieństwo, że ani jedna z nich wpadnie w $B_r(x)$ wynosi:
\begin{align*}
    \prob{X_1\in B_r(x)^c,X_2\in B_r(x)^c,...X_n\in B_r(x)^c}&=\prob{X_1\in B_r(x)^c}...\prob{X_n\in B_r(x)^c}=(1-2r)^n
\end{align*}
Chcemy użyć lematu Borela-Cantelliego, więc sprawdzamy sumę:
\begin{align*}
    \sum_{n\in\N}(1-2r)^n=\frac{1}{1-(1-2r)}=\frac{1}{2r}<\infty
\end{align*}
czyli z prawdopodobieństwem $1$ skończenie wiele zmiennych nie trafi do $B_r(x)$, czyli nieskończenie wiele z nich do $B_r(x)$ trafi. Tak się dzieje dla każdej kuli, więc z prawdopodobieństwem $1$ przetniemy dowolną kulę - tworzy się gęsty podzbiór $[0,1]$.

\begin{problem}[9]{d}
Zmienne losowe $X$ i $Y$ są niezależne i mają rozkłady wykładnicze z parametrami $\lambda$ i $\mu$ odpowiednio. Znajdź rozkład zmiennej losowej $X+Y$.
\end{problem}

Co wiemy? Że gęstość $X$ to $f_X(t)=\lambda e^{-\lambda t}$ gdy $t\geq 0$, a gęstość $Y$ to $f_Y(t)=\mu e^{-\mu t}$ gdy $t\geq 0$. Dalej, wiem że
$$f(t_x, t_y)=f_X(t_x)f_Y(t_y)$$
a poszukuję $\prob{X+Y=t}$

Skrypt mówi, że sploty is the way (ale miałam nawet ten sam pomysł!). Nie mogę puścić całki aż do $\infty$, bo wtedy mi się zeruje $e^{t-s}$ dla $s>t$. Czyli:
$$\prob{X+Y=t}=\int_0^t\prob{X=a, Y=t-a}da,$$
a w mowie skryptowej:
\begin{align*}
    \prob{X+Y=t}&=f_x\star f_y(t)=\int_0^t f_x(t-s)f_y(s)ds=\lambda\mu\int_0^t e^{-\lambda(t-s)}e^{-\mu s}ds=\\
    &=\lambda\mu e^{-\lambda t}\int_0^t e^{\lambda s-\mu s}ds=\lambda\mu e^{-\lambda t}\int_0^t e^{s(\lambda-\mu)}ds=\frac{\lambda\mu}{\lambda-\mu}[e^{-\lambda t}-e^{-\mu t}]
\end{align*}


























\end{document}
