\documentclass{article}

\usepackage[polish]{../../../lecture_notes}

\title{Lista 6\\{\normalsize Rachunek Prawdopodobieństwa 1R}}
\author{Weronika Jakimowicz}
\date{}

%\usepackage{unicode-math}
%\newcommand{\d}[1]{\symbf{d#1}}

\usepackage{wrapfig}

\begin{document}
\maketitle
\thispagestyle{empty}

\begin{problem}[1]{d}
Niech $(X,Y)$ będzie $2$-wymiarową zmienną losową o rozkładzie zadanym gęstością $f(x,y)=C(x+y)$ dla $0\leq y\leq x\leq 1$ i $f(x,y)=0$ poza tym zbiorem. Znajdź wartość $C$. Znajdź rozkłady brzegowe. Czy zmienne losowe $X$ i $Y$ są niezależne?
\end{problem}

\begin{wrapfigure}{r}{0.25\textwidth}
\begin{tikzpicture}
    \filldraw[color=green, pattern=dots, pattern color=green](0, 0)--(4.8, 0)--(4.8, 4.8)--cycle;
    \draw[->](-0.3, 0)--(5.3, 0) node [above] {x};
    \draw[->](0, -0.3)--(0, 5.3) node [left] {y};
\end{tikzpicture}
\end{wrapfigure}
Mamy dane
$$f(x,y)=\begin{cases}C(x+y)&0\leq y\leq x\leq1\\0&wpp\end{cases}$$
i w pierwszej kolejności pytamy o wartość zmiennej $C$. Wiemy, że 
$$\int_{\R^2}f(x,y)=1,$$ 
a ponieważ my żyjemy w świecie trójkąta pod $y=x$, to mamy:
$$1=\int\limits_0^1\int\limits_0^xC(x+y)dydx=C\int_0^1(x^2+\frac{x^2}{2})dx=C(\frac{1}{3}+\frac{1}{6})$$
czyli z moich bardzo precyzyjnych kalkulacji wynika, że $C=2$.

Teraz pora na rozkłady brzegowe. 
\begin{align*}
    \prob{X\in A}&=\prob{(X,Y)\in A\times\R}=\int_{A\times\R}f(x,y)dydx=\int_A\int_\R2(x+y)dydx=\\
    &=\int_A\int_0^x2(x+y)dydx=\int_A{3x^2}dx
\end{align*}
\begin{align*}
    \prob{Y\in B}&=\prob{(X,Y)\in\R\times B}=\int_{\R\times B}f(x,y)dydx=\int_\R\int_Bf(x,y)dydx=\int_B\int_\R f(x,y)dxdy=\\
    &=\int_B\int_y^12(x+y)dxdy=\int_B[1+2y-3y^2]dy
\end{align*}

Na pytanie, czy są to zmienne niezależne odpowiadamy patrząc na gęstości tych dwóch zmiennych losowych. Żeby były niezależne, musiałoby zachodzić
$$f(x,y)=f_X(x)f_Y(y)$$
Tutaj mamy
$$f_X(x)f_Y(y)=3x^2(1+2y-3y^2)\neq2(x+y)$$
więc są bardzo zależne.

\begin{problem}[2]{u}
Zmienna losowa $(X, Y)$ ma rozkład z gęstością $g(x,y)=C\cdot xy\cdot\mathds{1}_{[0,1]^2}$.
\begin{enumerate}[label=(\alph*)]
    \item Wyznaczyć $C$.
    \item Obliczyć $\prob{X+Y\leq1}$
    \item Wyznaczyć rozkład zmiennej losowej $\frac{X}{Y}$.
    \item Czy zmienne $X$ i $Y$ są niezależne?
    \item Czy $\frac{X}{Y}$ i $Y$ są niezależne?
\end{enumerate}
\end{problem}

\begin{enumerate}[label=(\alph*), leftmargin=*]
    \item To będzie całeczką $\star$
    \begin{align*}
        1=\int\int_{\R^2}g(x,y)\;dxdy=C\int_0^1\int_0^1xydxdy=C\int_0^1y\frac{1}{2}\;dy=\frac{C}{4}\implies C=4
    \end{align*}
    %\item 
    %Możemy to zrobić na dwa sposoby: pamiętać, że w skrypcie są sploty, albo dojść do tego mozolnym rozumowaniem. Zrobimy to drugie bo tak.
%
    \item Ustalmy sobie najpierw $s$ i niech $X=s$. Wtedy $X+Y\leq 1\iff Y\leq1-s$. Takie prawdopodobieństwo liczymy w następujący sposób:
    $$\prob{X=s, Y\leq 1-s}=\int_0^{1-s}g(s,t)\;dt.$$
    Super, ale $s$ może być dowolnym punktem z przedziału $[0, 1]$, więc my chcemy zliczyć wartości dla każdego takiego $s$. W tym pomaga nam kolejna całka:
    $$\prob{X+Y\leq 1}=\int_0^1\prob{X=s, Y\leq 1-s}=\int_0^1\int_0^{1-s}4\cdot st \;{dtds}$$
   % \item Ustalmy sobie $t$ i zastanówmy się, kiedy mamy $\frac{X}{Y}\leq t$, $X\leq tY$. Wartości $X$ chodzą od $0$ do $1$, czyli wybierzmy sobie $s\in[0,1]$. Wtedy
   % $$\prob{\frac{X}{Y}\leq t}\geq\prob{X\leq s, \frac{s}{t}\leq Y}$$
   % my liczymy po wszystkich takich $s$, czyli
   % \begin{align*}
   %     \prob{\frac{X}{Y}\leq t}&=\int_0^1\prob{X\leq s,\frac{s}{t}\leq Y}\;ds=\int_0^1\int_{\frac{s}{t}}^1g(s, p)\;dpds=\\
   %     &=\begin{bmatrix}u=\frac{s}{t}\\du=\frac{1}{t}ds\end{bmatrix}=\int_0^1t\int_u^1g(u, p)\;dpdu
   % \end{align*}
   % Chcemy znaleźć rozkład zmiennej $\frac{X}{Y}$, co możemy zrobić szukając dystrybuanty. Wszystko i tak jest izomorficzne $\star$
   % \begin{align*}
   %     \prob{\frac{X}{Y}\leq t}&=\prob{X\leq tY}=\int_\R\prob{X\leq ts}\;ds=(\Coffeecup)
   % \end{align*}
   % tutaj szybko zauważamy, że wartości $Y=[0,1]$, czyli ta całka nie jest po $\R$ tylko po $[0,1]$.
   % \begin{align*}
   %     (\Coffeecup)=\int_0^1\prob{X\leq ts}\;ds
   % \end{align*}
   \item Policzymy dystrybuantę. Do tego potrzebuję się zastanowić, kiedy $\frac{X}{Y}\leq t$? Ustalmy sobie $Y=s$, wtedy $\frac{X}{Y}=\frac{X}{s}\leq t\implies X\leq ts$
   \begin{align*}
        \prob{\frac{X}{Y}\leq t}&=\int_0^1\prob{X\leq ts, Y=s}\;ds=\int_0^1\int_0^{ts}\prob{X=p, Y=s}\;dpds=\\
        &=\int_0^1\int_0^{ts}4ps\;dpds=2\int_0^1t^2s^2\;ds=\frac{2}{3}t^2
   \end{align*}
   \item Do sprawdzania, czy zmienne $X$ i $Y$ są niezależne potrzebujemy znać rozkłady brzegowe, czyli rzuty na $X$ i na $Y$. Będę liczyć dystrybuantę tych rzutów:
   \begin{align*}
        \prob{X\leq t}&=\prob{X\leq t, Y\in[0,1]}=\\
        &=\int_0^t\int_0^1g(p, s)\;dsdp=\\
        &=2\int_0^tp\;dp=t^2
    \end{align*}
   %=tC\int_0^1s\;ds=2t$$
   $$\prob{Y\leq t}=\int_0^t\int_0^1g(p, s)\;dsdp=t^2$$
   Teraz sprawdzamy, czy $\prob{X\leq x, Y\leq y}=\prob{X\leq x}\prob{Y\leq y}$
   \begin{align*}
        \prob{X\leq x, Y\leq y}&=\int_0^x\int_0^yg(s, t)\;dtds=\int_0^x4s\int_0^yt\;dtds=\\
        &=\int_0^x2sy^2\;ds=x^2y^2
   \end{align*}
   Czyli zmienne są niezależne.
   \item Zmienne są niezależne, jeśli $\prob{A}\prob{B}=\prob{A\cap B}$. U nas niech
   $$A=\{\omega\;:\;\frac{X(\omega)}{Y(\omega)}\leq t\}\quad\left[\prob{A}=\prob{\frac{X}{Y}\leq t}\right]$$
   $$B=\{\omega\;:\;Y(\omega)\leq s\}\quad\left[\prob{B}=\prob{Y\leq s}\right]$$
   $$A\cap B=\{\omega\;:\;\frac{X(\omega)}{Y(\omega)}\leq t\;i\;Y(\omega)\leq s\}=\{\omega\;:\;X(\omega)\leq tY(\omega)\;i\;Y(\omega)\leq s\}$$
   Pierwsze dwa prawdopodobieństwa już mamy, zostaje nam obliczyć prawdopodobieństwo przekroju.
   \begin{align*}
        \prob{A\cap B}&=\prob{X\leq tY, Y\leq s}=\int_0^s\prob{X\leq ty, Y=y}\;dy=\\
        &=\int_0^s\int_0^{ty}\prob{X=x, Y=y}\;dxdy=\int_0^s\int_0^{ts}4xy\;dxdy=\\
        &=2\int_0^syt^2y^2\;dy=\frac{1}{2}t^2{s^4}
   \end{align*}
   No i wyszło, że są zależne [co jest dość rozsądnym wynikiem].
\end{enumerate}

\begin{problem}[3]{d}
Niech $X$ i $Y$ będą niezależnymi zmiennymi losowymi, których rozkład jest zadany gęstością $2x\cdot\mathds{1}_[0,1]$. Znaleźć prawdopodobieństwo, że
\begin{enumerate}[label=(\alph*)]
    \item $X+Y<\frac{1}{2}$
    \item $XY<\frac{1}{2}$
    \item $|X-Y|<\frac{1}{2}$
    \item $X^2+y^2\leq\frac{1}{2}$
\end{enumerate}
\end{problem}

Zmienne są niezależne, więc gęstość $f_{(X, Y)}(x, y)=f_X(x)\cdot f_Y(y)$.

\begin{enumerate}[label=(\alph*), leftmargin=*]
    \item Robimy tym samym trikiem co wcześniej, tzn. $X+Y<\frac{1}{2}$, ustalamy $x$ takie, że $X=t$ i mamy $Y<\frac{1}{2}-t$. Zauważamy jeszcze, że $t\in[0,\frac{1}{2}]$, Czyli dostajemy
    \begin{align*}
        \prob{X+Y<\frac{1}{2}}&=\int_0^\frac{1}{2}\prob{X=x, Y<\frac{1}{2}-x}\;dx=\int_0^\frac{1}{2}\int_0^{\frac{1}{2}-x}\prob{X=x, Y=y}\;dydx=\\
        &=\int_0^\frac{1}{2}\int_0^{\frac{1}{2}-x}4xy\;dydx
    \end{align*}
    \item \begin{align*}
        \prob{XY<\frac{1}{2}}&=\int_0^1\prob{X=x, Y<\frac{1}{2x}}\;dx=\int_0^1\int_0^{\frac{1}{2x}}4xy\;dydx
    \end{align*}
    \item \begin{align*}
        \prob{|X-Y|<\frac{1}{2}}&=\prob{-\frac{1}{2}<X-Y<\frac{1}{2}}=\int_0^1\prob{X=x,x-\frac{1}{2}<Y<x+\frac{1}{2}}\;dx
    \end{align*}
    a to już można sobie scałkować jak ma się ochotę.
    \item Tutaj zauważmy, że $X\leq\frac{1}{\sqrt{2}}=a$, bo inaczej wyjdziemy poza zakres. 
    \begin{align*}
        \prob{X^2+Y^2\leq \frac{1}{2}}&=\int_0^a\prob{X=x, Y^2\leq \frac{1}{2}-x^2}\;dx=\int_0^a\prob{X=x, Y\leq\sqrt{\frac{1}{2}-x^2}}\;dx=\\
        &=\int_0^a\int_0^{\sqrt{\frac{1}{2}-x^2}}4xy\;dydx
    \end{align*}
\end{enumerate}

\begin{problem}[4]{u}
Niech $X_1,...,X_n$ będą niezależnymi zmiennymi losowymi o rozkładzie wykładniczym z parametrem $1$. Znajdź rozkład $Y=\min\limits_{1\leq i\leq n}X_i$. Czy $X_n$ i $Y$ są niezależne?
\end{problem}

%Mają rozkład wykładniczy, więc funkcja gęstości $X_i$ to:
%$$f_i(t)=\lambda_ie^{-\lambda_i t}$$
%jeśli $t\geq 0$, wpp. mamy $0$.

Mają rozkład wykładniczy, więc funkcja gęstości $X_i$ to
$$f_i(x)=e^{-x}\mathds{1}_{[0,\infty)}$$

%Dobra, zacznijmy od wersji, że każda z osobna ma ten sam rozkład, tj.
%$$f_X(x)=e^{-x}\cdot\mathds{1}_{[0,\infty)}$$

Aby zmienne były niezależne, musimy mieć
$$\prob{X_n= x}\prob{Y = y}=\prob{X_n = x, Y = y}$$

Policzmy najpierw funkcję gęstości $Y$. Niech $k$ takie, że $X_k=Y$, wtedy:
\begin{align*}
    \prob{Y=y}&=\prob{X_k=y}\prod_{i\neq k}\prob{X_i\geq y}=e^{-y}\prod_{i\neq k}\int_y^\infty e^{-x}\;dx=e^{-y}\prod_{i\neq k}e^y=e^{(n-2)y}
\end{align*}
$k$ jest tutaj nieważne tak naprawdę.

%\begin{enumerate}[leftmargin=*]
%    \item Jeśli $x<y$ to mamy absurdalną sytuację, bo $X_n=x<Y$, ale $Y$ przyjmowało minimum. Więc wtedy $\prob{X_n=x, Y=y}=0$.
%
%    \item To teraz, jeśli $x>y$ mamy
%\begin{align*}
%    \prob{X_n=x, Y=y}=\prob{X_k=y}\prob{X_n=x}\prod_{\subsetack{i\neq k\\i\neq n}}\prob{X_i\geq y}=e^{-y}\cdot e^{-x}e^{(n-2)y}=e^{(n-3)y-x}
%\end{align*}
%natomiast
%$$\prob{X_n=x}\prob{Y=y}=e^{-x}\cdot e^{(n-2)y}$$
%czyli w tym przypadk
Jeśli $x=y$, wtedy
    \begin{align*}
        \prob{X_n=x, Y=x}&=\prob{X_n=x}\prod_{i<n}\prob{X_i\geq x}=e^{-x}e^{(n-1)x}=e^{(n-2)x}
    \end{align*}
    ale 
    $$\prob{X_n=x}\prob{Y=x}=e^{-x}e^{(n-2)x}=e^{(n-3)x}$$
%\end{enumerate}






\begin{problem}[5]{d}
Zmienne losowe $X$ i $Y$ są niezależne. Pokaż, że jeżeli $X$ nie ma atomów, to $P(X=Y)=0$.
\end{problem}

Ponieważ $X$ nie ma atomów, to zbiór $\{x\;:\;\prob{X=x}>0\}=\emptyset$.

Szukamy tak naprawdę $\prob{X=t, Y=t}$ po wszystkich $t$. Ponieważ zdarzenia są niezależne, to:
\begin{align*}
    \prob{X=Y}&=\int_{-\infty}^\infty\prob{X=t, Y=t}dt=\int\prob{X=t}\prob{Y=t}dt=\int0dt=0,
\end{align*}
gdyż $X$ jest bezatomowa.

\begin{problem}[6]{d}
Zmienne $X$ i $Y$ są niezależne. $X$ ma rzokład jednostajny na przedziale $[0,1]$, a $Y$ ma rozkład zadany przez $\prob{Y=-1}=\frac{1}{3}$, $\prob{Y=2}=\frac{2}{3}$.
\begin{enumerate}[label=(\alph*)]
    \item Oblicz $\prob{3X<Y}$
    \item Wyzacz rozkład zmiennej $XY$
\end{enumerate}
\end{problem}

\begin{enumerate}[label=(\alph*), leftmargin=*]
    \item W dyskretnym rozkładzie (jaki ma $Y$) dodajemy wartości, więc zrobimy to najpierw, a potem policzymy po $X$.
    \begin{align*}
        \prob{3X<Y}&=\frac{1}{3}\prob{3X<-1}+\frac{2}{3}\prob{3X<2}=\int_0^\frac{2}{3}\frac{2}{3}dx=\frac{4}{9}
    \end{align*}
    \item \begin{align*}
        \prob{XY<t}&=\frac{1}{3}\prob{X\cdot(-1)<t}+\frac{2}{3}\prob{X\cdot 2<t}=\frac{t}{3}
    \end{align*}
    i obcinam nadmiar ponad $1$.
\end{enumerate}



\begin{problem}[7]{d}
Pokaż, że zmienne losowe $X_1,...,X_n$ o gęstościach $f_1,...,f_n$ są niezależne $\iff$ zmienna $X=(X_1,...,X_n)$ ma gęstość
$$f(x_1,...,x_n)=f_1(x_1)...f_n(x_n)$$
\end{problem}

\begin{center}
Niezależne $\iff$ $f(x_1,...,x_n)=f_1(x_1)...f_n(x_n)$
\end{center}

$\implies$

Nirch $T_i\subseteq\R$, wtedy z niezależności zmiennych mamy:
\begin{align*}
    \prob{X_1\in T_1,...,X_n\in T_n}&=\int_{T_1}...\int_{T_n}f(x_1,...,x_n)dx_1...dx_n=\\
    &=\int_{T_1}f_1(x_1)dx_1...\int_{T_n}f_n(x_n)dx_n=\prob{X_1\in T_1}...\prob{X_n\in T_n}
\end{align*}
rozpisując krok po kroku:
\begin{align*}
    \prob{X_1\in T_1}...\prob{X_n\in T_n}&=\int_{T_1}f_1(x_1)dx_1...\int_{T_n}f_n(x_n)dx_n=\int_{T_1}f_1(x_1)\int_{T_2}f_2(x_2)dx_2dx_1...\int_{T_n}f_n(x_n)dx=\\
    &=\int_{T_1}\int_{T_2}f_1(x_1)f_2(x_2)dx_2dx_1...\int_{T_n}f_n(x_n)dx_n=...=\int_{T_1}...\int_{T_n}f_1(x_1)...f_n(x_n)dx_1...dx_n=\\
    &=\int_{T_1}...\int_{T_n}f(x_1,...x_n)dx_1...dx_n=\prob{X_1\int T_1,...,X_n\in T_n}
\end{align*}
Ponieważ dzieje się tak dla dowolnych $T_i$, to funkcje pod całką muszą się równać (prawie wszędzie?). Czyli
$$f(x_1,...,x_n)=f_1(x_1)...f_n(x_n).$$

$\impliedby$

Wychodzimy z tego, że
$$f(x_1,...,x_n)=f_1(x_1)...f_n(x_n).$$
Wybierając dowolne $T_i\subseteq\R$ i całkując obie strony dostajemy:
\begin{align*}
\int_{T_1}...\int_{T_n}f(x_1,...,x_n)dx_1...dx_n&=\int_{T_1}...\int_{T_n}f_1(x_1)...f_n(x_n)dx_1...dx_n=\\
&=\int_{T_1}...\int_{T_{n-1}}f_1(x_1)...f_{n-1}(x_{n-1})dx_1...dx_n\int_{T_n}f_n(x_n)dx_n=...\\
&=\int_{T_1}f_1(x_1)dx_1...\int_{T_n}f_n(x_n)dx_n
\end{align*}
Prawa strona równania to iloczyn $\prob{X_1\in T_1}...\prob{X_n\in T_n}$, a lewa to $\prob{X_1\in T_1,...,X_n\in T_n}$ i znowu dzieje się tak bez względu na wybór $T_i$, czyli mamy równość i zmienne są niezależne.

\begin{problem}[8]{d}
Z odcinka $[0,1]$ losujemy niezależnie w sposób jednostajny liczby $X_1,X_2,...$. Uzasadnij, że z prawdopodobieństwem $1$ ciąg $\{X_n\}$ jest gęsty w odcinku $[0,1]$.
\end{problem}

Weźmy sobie dowolną kulę na odcinku $[0,1]$ o promieniu $r$ i środku $x$: $B_r(x)$. Prawdopodobieństwo, że ani jedna ze zmiennych w nią trafi wynosi $1-2r$ (tutaj $r\leq\frac12$). Losujemy niezależnie, więc zmienne są niezależne. Jeśli rozważymy pierwsze $n$ zmiennych, to prawdopodobieństwo, że ani jedna z nich wpadnie w $B_r(x)$ wynosi:
\begin{align*}
    \prob{X_1\in B_r(x)^c,X_2\in B_r(x)^c,...X_n\in B_r(x)^c}&=\prob{X_1\in B_r(x)^c}...\prob{X_n\in B_r(x)^c}=(1-2r)^n
\end{align*}
Chcemy użyć lematu Borela-Cantelliego, więc sprawdzamy sumę:
\begin{align*}
    \sum_{n\in\N}(1-2r)^n=\frac{1}{1-(1-2r)}=\frac{1}{2r}<\infty
\end{align*}
czyli z prawdopodobieństwem $1$ skończenie wiele zmiennych nie trafi do $B_r(x)$, czyli nieskończenie wiele z nich do $B_r(x)$ trafi. Tak się dzieje dla każdej kuli, więc z prawdopodobieństwem $1$ przetniemy dowolną kulę - tworzy się gęsty podzbiór $[0,1]$.

\begin{problem}[9]{d}
Zmienne losowe $X$ i $Y$ są niezależne i mają rozkłady wykładnicze z parametrami $\lambda$ i $\mu$ odpowiednio. Znajdź rozkład zmiennej losowej $X+Y$.
\end{problem}

Co wiemy? Że gęstość $X$ to $f_X(t)=\lambda e^{-\lambda t}$ gdy $t\geq 0$, a gęstość $Y$ to $f_Y(t)=\mu e^{-\mu t}$ gdy $t\geq 0$. Dalej, wiem że
$$f(t_x, t_y)=f_X(t_x)f_Y(t_y)$$
a poszukuję $\prob{X+Y=t}$

Skrypt mówi, że sploty is the way (ale miałam nawet ten sam pomysł!). Nie mogę puścić całki aż do $\infty$, bo wtedy mi się zeruje $e^{t-s}$ dla $s>t$. Czyli:
$$\prob{X+Y=t}=\int_0^t\prob{X=a, Y=t-a}da,$$
a w mowie skryptowej:
\begin{align*}
    \prob{X+Y=t}&=f_x\star f_y(t)=\int_0^t f_x(t-s)f_y(s)ds=\lambda\mu\int_0^t e^{-\lambda(t-s)}e^{-\mu s}ds=\\
    &=\lambda\mu e^{-\lambda t}\int_0^t e^{\lambda s-\mu s}ds=\lambda\mu e^{-\lambda t}\int_0^t e^{s(\lambda-\mu)}ds=\frac{\lambda\mu}{\lambda-\mu}[e^{-\lambda t}-e^{-\mu t}]
\end{align*}

\begin{problem}[10]{d}
Zmienne losowe $X$ i $Y$ są niezależne i mają rozkład wykładniczy z parametrem $1$. Udowodnić, że zmienne $\frac{X}{Y}$ oraz $X+Y$ są niezależne.
\end{problem}

To lecimy od znalezienia rozkładu $\frac{X}{Y}$, potem $X+Y$ i na końcu rozkład $(\frac{X}{Y}, X+Y)$.

\begin{align*}
    \prob{\frac{X}{Y}\leq t}&=\prob{X\leq tY}=\int_0^\infty\prob{X\leq ty, Y=y}\;dy=\int_0^\infty\int_0^{ty}\prob{X=x, Y=y}\;dxdy=\\
    &=\int_0^\infty\int_0^{ty}e^{-x}e^{-y}\;dxdy=\int_0^\infty e^{-y}[1-e^{-ty}]\;dy=1-\frac{1}{1+t}=\frac{t}{1+t}
\end{align*}

%$$\frac{X}{Y}=t\iff X=x\;\land\;Y=\frac{t}{x}$$

%\begin{align*}
%    \prob{XY\leq t}&=\int_0^\infty\prob{X=x, Y\leq \frac{t}{x}}\;dx=\int_0^\infty\int_0^{\frac{t}{x}}e^{-x}e^{-y}\;dydx=\\
%    &=\int_0^\infty e^{-x}[1-e^{-\frac{t}{x}}]\;dx=\int_0^\infty [e^{-x}-e^{-x-\frac{t}{x}}]\;dx=BRZYDKIE\; COS
%\end{align*}

\begin{align*}
    \prob{X+Y\leq t}&=\int_0^t\prob{X+Y=s}\;ds=\int_0^t\int_0^s\prob{X=x, Y=s-x}\;dxds=\int_0^t\int_0^se^{-x}e^{x-s}\;dxds=\\
    &=\int_0^t\int_0^se^{-s}\;dxds=\int_0^tse^{-s}ds=1-e^{-t}(t+1)
    %\int_0^t\prob{X=x, Y\leq t-x}\;dx=\int_0^t\int_0^{t-x}\prob{X=x}\prob{Y=y}\;dydx=\\
    %&=\int_0^te^{-x}[1-e^{x-t}]\;dx=1-e^{-t}-te^{-t}
\end{align*}

%$$\prob{X+Y\leq t}=\int_0^t\int_0^\infty e^{-x+y}e^{-y}\;dydx$$

$$\prob{\frac{X}{Y}\leq t}\prob{X+Y\leq s}=\frac{t}{1+t}[1-e^{-s}(1+s)]$$

%$$ty\leq s-y\iff y(1+t)\leq s\iff y\leq \frac{s}{1+t}$$
%\begin{align*}
%    \prob{\frac{X}{Y}\leq t, X+Y\leq s}&=\prob{X\leq tY, X\leq s-Y}=\int_0^s\prob{X\leq ty, X\leq s-y}\;dy=\\
%    &=\int_0^\frac{s}{1+t}\prob{X\leq ty}\;dy+\int_{\frac{s}{1+t}}^s\prob{X\leq s-y}\;dy=\\
%    &=\int_0^a\int_0^{ty}e^{-x}\;dxdy+\int_a^s\int_0^{s-y}e^{-x}\;dxdy=\\
%    &=\int_0^a[1-e^{-ty}]\;dy+\int_a^s[1-e^{y-s}]\;dy=\\
%    &=\frac{e^{-at}-1}{t}+a+e^{a-s}-a+s-1=\\
%    &=\frac{e^{-\frac{st}{1+t}}-1+te^{\frac{-st}{1+t}}-t}{t}=[e^{-\frac{st}{1+t}}-1]\frac{1+t}{t}
%    %&=\prob{\frac{X}{Y}\leq t, \frac{X}{Y}\leq\frac{s}{Y}-1}
%    %=\\
%    %&=\int_0^s\prob{X\leq min(ty, s-y)}\;dy=\int_0^t\int_0^{min(ty, s-y)} e^{-x}\;dxdy
%\end{align*}

$$\frac{X}{t}\leq Y$$
$$Y\leq s-X$$

\begin{illustration}
    \draw[->] (0, -0.3)--(0, 5) node [left] {$Y$};
    \draw[->] (-0.3, 0)--(7, 0) node [below] {$X$};
    \filldraw[pattern=bricks, pattern color=blue!40!black!80] (0, 0)--(0, 2.8)--(1.8, 1.04)--cycle;
    \draw[thick, pink] (-.2, 3) node[left] {$Y=s-X$} --(3, -0.2);
    \draw[thick, yellow] (-0.33, -0.2)--(0, 0)--(5, 3) node [right] {$\frac{X}{t}=Y$};
\end{illustration}

\begin{align*}
    \prob{\frac{X}{Y}\leq t,X+Y\leq s}&=\int_0^{\frac{st}{s+1}}e^{-x}\prob{\frac{x}{t}\leq Y, Y\leq s-x}\;dx=\int_0^{\frac{st}{s+1}}e^{-x}\int_{\frac{x}{t}}^{s-x}e^{-y}\;dydx=\frac{t}{t+1}[1-e^{-s}(1+s)]
    %\frac{t}{t+1}[1-e^{-\frac{s(t+1)}{t}}]-se^{-s}
    %[\prob{Y\leq s-x}-\prob{Y\leq\frac{x}{t}}]\;dx=\\
    %&=\int_0^s[e^
\end{align*}

\begin{problem}[11]{d}
Zmienne losowe $X_1,...X_n$ są niezależne i mają rozkłady Poissona z parametrami $\lambda_i$. Pokaż, że $X_1+...+X_n$ ma rozkład Poissona z parametrem $\lambda_1+...+\lambda_n$.
\end{problem}

To jest rozkład dyskretny. Gęstość $X_i$ to
$$f_i(k)=\lambda_i^k\frac{e^{-\lambda}}{k!}$$

%\begin{align*}
%    \prob{X_1+...+X_n=t}&=\sum_{x_1=0}^t\prob{X_1=x_1,X_2+...+X_n=t-x_1}=\\
%    &=\sum_{x_1=0}^t\sum_{x_2=0}^{t-x_1}...\sum_{x_n=0}^{t-\sum x_i}\prob{X_1=x_1,..., X_n=x_n}=\sum_{x_1=0}^t\sum_{x_2=0}^{t-x_1}...\sum_{x_n=0}^{t-\sum x_i}e^{-x_1-x_2-...-x_n}=\\
%    &=\sum_{x_1}^t...\sum_{x_{n-1}=0}^{t-\sum_i}e^{-x_1-...-x_{n-1}}
%\end{align*}

%Chyba indukcja mi prześmignie? Bo jeśli $X_1,...,X_n$ były niezależne, to tym bardziej $X_1+...+X_{n-1}$ i $X_n$ będą niezależne. 

Dowód przez indukcję:

Dla $n=1$ nie ma co robić, jeśli $n=2$, to
\begin{align*}
    \prob{X_1+X_2=t}&=\sum_{x_1=0}^t\prob{X_1=x_1,X_2=t-x_1}=\sum_{x_1=0}^t\lambda_1^{x_1}\frac{e^{-\lambda_1}}{x_1!}\lambda_2^{t-x_1}\frac{e^{-\lambda_2}}{(t-x_1)!}=\\
    &=\frac{e^{-(\lambda_1+\lambda_2)}}{t!}\sum_{x_1=0}^t{t\choose x_1}\lambda_1^{x_1}\lambda_2^{t-x_1}=(\lambda_1+\lambda_2)^t\frac{e^{-(\lambda_1+\lambda_2)}}{t!}
\end{align*}
czyli to co chcemy.

Niech $\lambda=\sum\lambda_i$.
\begin{align*}
    \prob{X_1+...+X_n=t}&=\sum_{x_1+...+x_n=t}\lambda_1^{x_1}...\lambda_n^{x_n}\frac{e^{-\lambda}}{x_1!...x_n!}=\frac{e^{-\lambda}}{t!}\sum_{x_1+...+x_n}{t\choose x_1,...,x_n}\lambda_1^{x_1}...\lambda_n^{x_n}=\frac{e^{-\lambda}}{t!}\lambda^t
    %\sum_{x_1=0}^t\sum_{x_2=0}^{t-x_1}...\sum_{x_{n-1}=0}^{t-\sum_0^{n-1}x_i}\lambda_1^{x_1}\lambda_2^{x_2}...\lambda_{n}^{t-\sum_0^{n-1}x_i}\frac{e^{-(\lambda_1+...+\lambda_{n})}}{x_1!(t-x_1)!...(t-\sum_0^{n-1}x_i)!}=\\
    %&=\frac{e^{-\lambda}}{t!}\sum\lambda_1^{x_1}...\lambda_n^{t-\sum_0^{n-
\end{align*}

%\begin{lemma}
%Jeśli $X_1,...,X_n$ są zmiennymi niezależnymi, to $X_n$ i $X_1+...+X_{n-1}$ też jest są niezależne
%\end{lemma}

%\begin{align*}
%    \prob{X_1+...+X_{n-1}=t, X_n=s}&=
%\end{align*}

\begin{problem}[12]{w}
Załóżmy, że $X_1$ i $X_2$ są niezależnymi zmiennymi losowymi o rozkładach odpowiednio $N(m_1\sigma_1)$ i $N(m_2,\sigma_2)$. Oblicz rozkład zmiennej losowej $X_1+X_2$.
\end{problem}

Funkcja gęstości w rozkładzie normalnym:
$$f_i(x)=\frac{1}{\sigma_i\sqrt{2\pi}}e^{-\frac{(x-m_i)^2}{2\sigma_i^2}}$$

\begin{align*}
    \prob{X_1+X_2= t}&=\int_\R\prob{X_1=x, X_2=t-x}\;dx=\int_\R\prob{X_1=x}\prob{X_2=t-x}\;dx=\\
    &=\int_\R\frac{1}{\sigma_1\sqrt{2\pi}}e^{-\frac{(x-m_1)^2}{2\sigma_1^2}}\frac{1}{\sigma_2\sqrt{2\pi}}e^{-\frac{(t-x_1-m_1)^2}{2\sigma_2^2}}\;dx=\\
    &=\frac{1}{2\pi\sigma_1\sigma_2}\int_\R exp\left[-\frac{\sigma_2^2(x-m_1)^2+\sigma_1^2(t-x-m_2)^2}{2\sigma_1^2\sigma_2^2}\right]\;dx
    %\int_\R\prob{X_1=x_1,X_2\leq t-x_1}\;dx_1=\\
    %&=\int_\R\int_{-\infty}^{t-x_1}\frac{1}{\sigma_1\sqrt{2\pi}\sigma_2\sqrt{2\pi}}e^{-\frac{(x_1-m_1)^2}{2\sigma_1^2}}e^{-\frac{(x_2-m_2)^2}{2\sigma_2^2}}\;dx_1dx_2=\\
    %&=\frac{1}{\sigma_1\sigma_22\pi}\int_\R e^{\frac{(x_1-m_1)^2}{2\sigma_1^2}}\int_{-\infty}
\end{align*}

O JEZU NIE?

\begin{problem}[13]{d}
Niech $\{X_n\}_{n\in\N}$ będzie ciągiem niezależnych zmiennych losowych o rozkładzie $N(0,1)$. Uzasadnij, że z prawdopodobieństwem $1$ istnieje nieskończenie wiele indeksów $n$ takich, że
$$|X_{2n}-X_{2n+1}|\leq\frac{1}{n}.$$
\end{problem}

Szacujemy prawdopodobieństwo z zadania:
\begin{align*}
    \prob{|X_{2n}-X_{2n+1}|\leq\frac{1}{n}}&\geq\prob{X_1=0,-\frac{1}{n}\leq X_{2n}\leq \frac{1}{n}}=\int_{-\frac{1}{n}}^{\frac{1}{n}}\frac{1}{2\pi}e^{-\frac{y^2}{2}}\;dydx
    %=\frac{1}{n\pi}\int_\R e^{\frac{1}{2}(x-\frac{1}{n}-x)(x-\frac{1}{n}+x)}\;dx=\\
    %&=\frac{1}{n\pi}\int_\R e^{\frac{1}{2n}(2x-\frac{1}{n})}\;dx
    %\prob{X_{2n}-X_{2n-1}\leq\frac{1}{n}}+\prob{X_{2n+1}-X_{2n}\leq\frac{1}{n}}
\end{align*}
Wykres funkcji $e^{-\frac{x^2}{2}}$ jest dla ujemnych $x$. W dodatku, funkcja ta jest parzysta. 

\begin{illustration}
\begin{axis}[axis lines=left, yticklabels={,,}, xtick={0}, xticklabels={0}]
    \addplot [domain=-2:2,samples=200,color=blue, very thick]{e^(-x^2)};
    \draw[dashed] (0, 0)--(0, 1);
\end{axis}
\end{illustration}

Czyli możemy oszacować:
\begin{align*}
    \int_{-\frac{1}{n}}^{\frac{1}{n}}\frac{1}{2\pi}e^{-\frac{y^2}{2}}\;dy&\geq \frac{1}{2\pi}\cdot\frac{2}{n}\cdot e^{-\frac{1}{2n^2}}=\frac{1}{n\pi}e^{-\frac{1}{2n^2}}
\end{align*}

Zmienne zawierające informacje o tym, kiedy $X_{2n}=0\;\land\;|X_{2n+1}|\leq\frac{1}{n}$ są niezależne (to widać z rozpisania wyżej), więc chcę użyć lematu Borela-Cantellego.
\begin{align*}
    \sum\prob{\left|X_{2n}-X_{2n+1}\right|\leq\frac{1}{n}}&\geq\sum\prob{X_{2n}=0,|X_{2n+1}|\leq\frac{1}{n}}\geq\sum\frac{1}{n\pi}e^{-\frac{1}{2n^2}}\geq\\
    &\geq\sum\frac{1}{n\pi}\geq\sum\frac{1}{n}=\infty
\end{align*}
No to super! Z prawdopodobieństwem $1$ nieskończenie wiele razy zdarzy się, że $X_{2n}=0$ i $|X_{2n+1}|\leq\frac{1}{n}$, czyli tym bardziej $|X_{2n}-X_{2n+1}|\leq\frac{1}{n}$ zdarzy się nieskończenie wiele razy z prawdopodobieństwem $1$.

%\begin{align*}
%    \prob{|X_{2n}-X_{2n+1}|\leq\frac{1}{n}}&=\int_\R\prob{-\frac{1}{n}\leq X_{2n}-x\leq\frac{1}{n}}\;dx=\\
%    &=\int_\R\prob{x-\frac{1}{n}\leq X_{2n}\leq\frac{1}{n}+x}\;dx=\int_\R\int_{x-\frac{1}{n}}^{x+\frac{1}{n}} \frac{1}{\sigma\sqrt{2\pi}}e^{-\frac{y^2}{2}}\;dydx\geq\frac{1}{\sigma\sqrt{2\pi}}
%\end{align*}
%Czyli mam
%$$\sum_{n\in\N}\prob{|X_{2n}-X_{2n+1}}|\leq\frac{1}{n}\geq\sum\frac{1}{\sigma\sqrt{2\pi}}=\infty$$

\begin{problem}[14]{u}
Niech $\{X_i\}_{i=1,...,5}$ będzie ciągiem niezależnych zmiennych losowych.
\begin{enumerate}[label=(\alph*)]
    \item Czy zmienne losowe $X_1+X_2$ oraz $X_3+X_4X_5$ są niezależne?
    \item Czy zmienne losowe $X_1$, $X_1X_2$ są niezależne?
\end{enumerate}
\end{problem}

\begin{enumerate}[label=(\alph*), leftmargin=*]
    \item TAK, SĄ NIEZALEŻNE
    \begin{align*}
        \prob{X_1+X_2=t,X_3+X_4X_5=s}&=\int_\R\prob{X_1=a,X_2=t-a,X_3+X_4X_5}\;da=\\
        &=\int_\R\int_\R\prob{X_1=a,X_2=t-a,X_3=b, X_4X_5=s-b}\;dadb=\\
        &=\int\int\int\prob{X_1=a,X_2=t-a,X_3=b,X_4=c,X_5=\frac{s-b}{c}}\;dadbdc=\\
        &=\int\int\int\prob{X_1=a}\prob{X_2=t-a}\prob{X_3=b}\prob{X_4=c}\prob{X_5=\frac{s-b}{c}}\;dadbdc=\\
        %&=\int\int\prob{X_3=b}\prob{X_4=c}\prob{X_5=\frac{s-b}{c}}\int\prob{X_1=a}\prob{X_2=t-a}\;dadbdc=\\
        &=\int\int\prob{X_3=b,X_4=c,X_5=\frac{s-b}{c}}dbdc\int\prob{X_1=a,X_2=t-a}\;da=\\
        &=\int\prob{X_3=b,X_4X_5=s-b}\;db\prob{X_1+X_2=t}=\\
        &=\prob{X_3+X_4X_5=s}\prob{X_1+X_2=t}
    \end{align*}
    \item Odp.: WELES JEST CZARNY

    Jeśli popatrzymy na zmienne losowe takie, że $\prob{X_i=1}=\frac{1}{2},\prob{X_i=-1}=\frac{1}{2}$, to zmienne $X_1$ i $X_1X_2$ są faktycznie niezależne:
    \begin{align*}
        \prob{X_1=1,X_1X_2=1}&=\prob{X_1=1,X_2=1}=\frac{1}{4}=\frac{1}{2}\left(\frac{1}{4}+\frac{1}{4}\right)=\prob{X_1=1}\prob{X_1X_2=1}\\
        \prob{X_1=-1,X_1X_2=1}&=\prob{X_1=-1,X_2=-1}=\frac{1}{4}=\frac{1}{2}\left(\frac{1}{4}+\frac{1}{4}\right)=\prob{X_1=-1}\prob{X_1X_2=1}
    \end{align*}
    i tak dalej.

    Jeśli natomiast popatrzymy na zmienne o rozkładzie wykładniczym z parametrem $1$, to mamy
    %, tj. $\prob{X_1=x}=e^{-x}$ i $\prob{X_2=x}=e^{-x}$, to mamy
    \begin{align*}
        \prob{X_1X_2=y}&=\int_0^\infty\prob{X_1=t,X_2=\frac{y}{t}}\;dt=\int_0^\infty e^{-t}e^{-b\frac{y}{t}}\;dt=\\
        &=\int_0^\infty e^{-t-\frac{y}{t}}\;dt
    \end{align*}
    \begin{align*}
        \prob{X_1=x,X_1X_2=y}&=\prob{X_1=x,X_2=\frac{y}{x}}=\prob{X_1=x}\prob{X_2=\frac{y}{x}}=\\
        &=e^{-x}e^{-\frac{y}{x}}
    \end{align*}
    Czyli, żeby zmienne były niezależne, musielibyśmy mieć
    $$\prob{X_1X_2=y}=e^{-\frac{y}{x}},$$
    a my mamy, że
    $$\prob{X_1X_2=y}=\int_0^\infty e^{-\frac{x^2+y}{x}}\;dx\neq e^{-\frac{y}{x}}$$
    Czyli tutaj nie są niezależne.
\end{enumerate}













\end{document}
