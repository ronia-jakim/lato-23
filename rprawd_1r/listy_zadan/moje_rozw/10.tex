\documentclass{article}

\usepackage{../../../lecture_notes}
%\NeedsTeXFormat{LaTeX2e}
%\ProvidesPackage{kaczuszka}

\RequirePackage{tikz}
\RequirePackage{xcolor}

\definecolor{y}{HTML}{FFCC00}
\definecolor{bb}{HTML}{FFCC00}
%\definecolor{bb}{HTML}{323232}

\def\dupa{fill=y, draw=bb, line width=3mm}

\def\kaczka{
  \begin{scope}
    \filldraw[fill=y, draw=bb, line width=3mm] (0, 0) ellipse (2.1cm and 2cm);
    \filldraw[fill=y, draw=bb, line width=3mm] (1.8, -4) ellipse (4cm and 3.2cm);
  \end{scope}
}

%\endinput


\title{Rachunek Prawdopodobieństwa 1R\\{\normalsize Lista 10}}
\author{}
\date{}

\begin{document}
\maketitle\thispagestyle{empty}

\begin{problem}{d}
  Pokaż, że jeśli $0<p<q$, to
  $$\left(\e|X|^p\right)^\frac{1}{p}\leq\left(\e|X|^q\right)^\frac{1}{q}$$
\end{problem}

Dowód taki, jak wiele dowodów na analizie funkcjonalnej.

\begin{align*}
  \e|X|^p&=\e|X\cdot 1|^p\overset{\star}{\leq} \left[\e(|X|^p)^\frac{q}{p}\right]^\frac{p}{q}\left[\e1^{\frac{q}{q-p}}\right]^\frac{q-p}{q}=\left[\e|X|^q\right]^\frac{p}{q}
\end{align*}
  $$\left[\e|X|^p\right]^\frac{1}{p}\leq \left[\e|X|^q\right]^\frac{1}{q}$$

  $\star$ wynika z nierówności H\"oldera dla $\frac{q}{p}$ i $\frac{q}{q-p}$. Wszystko śmiga, bo $\e$ to tak naprawdę całkowanie względem miary $\prob{|X|^p}$, więc np. $\e1=\int 1\;d\prob{|X|^p}=1$ bo prawdopodobieństwo całości to dokładnie $1$.

\begin{problem}{d}
  (\emph{Reguła $n$ sigm}) Pokaż, że jeśli $Var(X)=\sigma^2<\infty$, to
  $$\prob{|X-\e X|>n\sigma}\leq\frac{1}{n^2}$$
\end{problem}

Nierówność Czebyszewa:
$$\prob{|X-\mu|>\lambda}\leq\frac{\e(f(|X-\mu|))}{f(\lambda)}.$$
Niech $\lambda=n\sigma$ i $f(x)=x^2$. Podstawiając do wzoru wyżej:
$$\prob{|X-\mu|>n\sigma}\leq\frac{\e|X-\mu|^2}{n^2\sigma^2}=\frac{Var(X)}{n^2\sigma^2}=\frac{1}{n^2}$$

\begin{problem}{u}
  Sprawdzić, że zdarzenie $\{\lim_{n\to\infty}X_n=a\}$ należy do $F_\infty$.
\end{problem}

Chyba wystarczy zauważyć, że dla dowolnego $i$ oraz ciągu $Y_j=X_{i+j-1}$ zachodzi $\lim Y_j=\lim X_j=a$ oraz ciąg $Y_j\subseteq F_{i, \infty}$. Stąd $(\forall\;i)\;\{\lim X_n=a\}\in F_{i,\infty}\implies\{\lim X_n=a\}\in \bigcap F_{i,\infty}=F_\infty$.

\begin{problem}[5]{u}
  Zbadaj zbieżność szeregu $\sum X_n$, jeśli $\{X_n\}$ jest ciągiem niezależnych zmiennych losowych o rozkładach:

  \begin{enumerate}[label=(\alph*)]
    \item $\prob{X_n =2^{-n}}=\prob{X_n=0}=\frac12$
    \item $\prob{X_n=\frac{1}{n}}=1-\prob{X_n=0}=\frac{1}{n\log n}$
  \end{enumerate}
\end{problem}

\begin{enumerate}[label=(\alph*)]
  \item $\prob{X_n=2^{-n}}=\prob{X_n=0}=\frac12$

      Tutaj kolejne wyrazy to albo $\frac{1}{2^n}$, albo $0$, więc
      $$\sum X_n\leq\sum2^{-n}=2$$
      co jest bardzo skończone.

  \item $\prob{X_n=\frac{1}{n}}=1-\prob{X_n=0}=\frac{1}{n\log n}$

    Zakładam, że dla $n=1$ zachodzi $\prob{X_n=\frac{1}{n}}=1$, bo $\log n=\infty$.

    Ponieważ $\sum_{n=1}^{2137}X_n<\infty$ (sumujemy skończenie wiele skończonych wyrazów), to wystarczy pokazać, że $\sum_{n=2137}^\infty X_n<\infty$. Czyli dla twierdzenia Kołmogorowa liczę:
    $$\sum_{n=2137}^\infty\e X_i=\sum \frac{1}{n^2\log n}\leq\sum\frac{1}{n^2}<\infty$$
    $$\sum_{n=2137}^\infty Var X_i=\sum[\e(X^2)-(\e X)^2]=\sum\e(X^2)-\sum(\e X)^2\leq \sum \frac{1}{n^2}\cdot\frac{1}{n\log n}-\sum\frac{1}{n^4}<\infty$$

    W takim razie na mocy twierdzenia Kołmogorowa, $\sum X_n$ zbiega prawie na pewno.
\end{enumerate}

\begin{problem}[6]{d}
  Niech $\{X_n\}$ będzie ciągiem niezależnych zmiennych losowych takich, że $X_n$ ma rozkład jednostajny $U[-n,n]$. Dla jakich wartości parametru $p>0$ szereg $\sum\frac{X_n}{n^p}$ jest zbieżny prawie wszędzie?
\end{problem}

$$\e X_n=\int_{-n}^n x\cdot \frac{1}{2n}\;dx=\frac{1}{2n}\cdot 0$$
czyli $\sum \e X_n=0<\infty$
$$Var X_n=\e X^2-(\e X)^2=\e X^2=\int x\cdot\prob{X_n^2=x}\;dx=\int x\cdot\prob{X_n=\sqrt{x}}\;dx=\int_0^{n^2}x\cdot\frac{1}{2n}\;dx=\frac{n^3}{4}$$

Czyli normalnie ten szereg jest rozbieżny, ale możemy rozważać za to zmienne $Y_n=\frac{X_n}{n^p}$ o rozkładzie jednostajnym na $[-n^{1-p},n^{1-p}]$. Wtedy $\e Y_n$ pozostaje zerem, bo nadal mamy całkę po symetrycznym przedziale z nieparzystej funkcji, ale wariacja się zmienia:
$$Var\; Y_n=Var(n^{-p}X_n)=n^{-2p}Var(X_n)=n^{-2p}\cdot \frac{n^3}{4}=\frac{n^3}{4n^{2p}}$$
Chcemy, aby
$$\sum \frac{n^3}{4n^{2p}}<\infty,$$
co jest prawdą na pewno dla $p>2$ [$2p-3>1$]. Dla $0<p<2$ mamy szereg ograniczony od dołu przez szereg harmoniczny, a on jest bardzo niezbieżny.

\begin{problem}[7]{u}
  Niech $\prob{X_n=n}=\prob{X_n=-n}=\frac{1}{n^3},\;\prob{X_n=0}=1-\frac{2}{n^3}$. Pokaż, że $\sum X_n$ jest zbieżny p.p. chociaż $\sum Var(X_n)=\infty$.
\end{problem}

Ustalmy dowolne $c>0$ jak w twierdzeniu Kołmogorowa o 3 szeregach. Zauważmy, że jeśli $n>c$, to $\e X_n^{(c)}=0$, bo jedyną wartością jaką może $X_n^{(c)}$ przyjąć jest $0$ (przez obcięcie). W takim razie, wariacja $X_n^{(c)}$ też jest równa zerem. Czyli seregi $\sum \e X_n^{(c)}$ i $\sum Var(X_n^{(c)})$ mają skończenie wiele niezerowych wyrazów, więc są zbieżne.

Pozostaje roztłumaczyć $\sum\prob{|X_n|>c}$, który dla odmiany ma niezerowe wyrazy tylko dla $n>c$. Weźmy takie $n$ i zauważmy, że $\prob{|X_n|>c}=\frac{2}{n^3}$, czyli
$$\sum \prob{|X_n|>c}=\sum\frac{2}{n^3}<\infty,$$
więc na mocy twierdzenia Kołmogorowa o 3 szeregach mamy $\sum X_n<\infty$ p.n.

\begin{problem}[10]{}
  Zbadaj zbieżność szeregu $\sum X_n$ jeśli $\{X_n\}$ jest ciągiem niezależnych zmiennych losowych o rozkładzie 
  $$\prob{X_n=a_n}=\prob{X_n=-a_n}=\frac12$$
  dla pewnego ciągu $\{a_n\}$.
\end{problem}

Jeśli $\sum |a_n|<\infty$ to $\sum X_n\leq\sum|X_n|=\sum|a_n|<\infty$.

\end{document}
