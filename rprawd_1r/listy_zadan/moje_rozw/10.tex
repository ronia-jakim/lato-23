\documentclass{article}

\usepackage{../../../lecture_notes}

\title{Rachunek Prawdopodobieństwa 1R\\{\normalsize Lista 10}}
\author{}
\date{}

\begin{document}
\maketitle\thispagestyle{empty}

\begin{problem}{d}
  Pokaż, że jeśli $0<p<q$, to
  $$\left(\e|X|^p\right)^\frac{1}{p}\leq\left(\e|X|^q\right)^\frac{1}{q}$$
\end{problem}

  Dowód taki, jak wiele dowodów na analizie funkcjonalnej.

\begin{align*}
  \e|X|^p&=\e|X\cdot 1|^p\overset{\star}{\leq} \left[\e(|X|^p)^\frac{q}{p}\right]^\frac{p}{q}\left[\e1^{\frac{q}{q-p}}\right]^\frac{q-p}{q}=\left[\e|X|^q\right]^\frac{p}{q}
\end{align*}
  $$\left[\e|X|^p\right]^\frac{1}{p}\leq \left[\e|X|^q\right]^\frac{1}{q}$$

  $\star$ wynika z nierówności H\"oldera dla $\frac{q}{p}$ i $\frac{q}{q-p}$. Wszystko śmiga, bo $\e$ to tak naprawdę całkowanie względem miary $\prob{|X|^p}$, więc np. $\e1=\int 1\;d\prob{|X|^p}=1$ bo prawdopodobieństwo całości to dokładnie $1$.

\begin{problem}{d}
  (\emph{Reguła $n$ sigm}) Pokaż, że jeśli $Var(X)=\sigma^2<\infty$, to
  $$\prob{|X-\e X|>n\sigma}\leq\frac{1}{n^2}$$
\end{problem}

Nierówność Czebyszewa:
$$\prob{|X-\mu|>\lambda}\leq\frac{\e(f(|X-\mu|))}{f(\lambda)}.$$
Niech $\lambda=n\sigma$ i $f(x)=x^2$. Podstawiając do wzoru wyżej:
$$\prob{|X-\mu|>n\sigma}\leq\frac{\e|X-\mu|^2}{n^2\sigma^2}=\frac{Var(X)}{n^2\sigma^2}=\frac{1}{n^2}$$

\begin{problem}{}
  Sprawdzić, że zdarzenie $\{\lim_{n\to\infty}X_n=a\}$ należy do $F_\infty$.
\end{problem}





\end{document}
