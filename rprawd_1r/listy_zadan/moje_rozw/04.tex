\documentclass{article}

\usepackage{../../../notatki}

\begin{document}

\subsection*{ZADANIE 2.}
\emph{\color{pink}Ola poszła do kasyna mając 100 złotych. Postanowiła grać tak długo aż albo zbankrutuje, albo osiągnie 500 złotych. W każdej pojedynczej grze może wygrać 10 zł z prawdopodobieństwem $\frac13$, przegrać 10 złotych z prawdopodobieństwem $\frac12$ lub utrzymać swój stan posiadania z prawdopodobieństwem $\frac16$. Pokaż, że z prawdopodobieństwem 1 Ola skończy grę w skończonym czasie}
\smallskip

Zadanie zrobimy jakbyśmy rozważali pijaka próbującego usilnie wrócić do domu na prostej drodze: każdy krok to początek nowej, wspaniałej przygody.

Jako, że przyrównamy fortunę Oli do pijaka, a jej odległość od $50$ do odległości pijaka od ukochanej szklanki soku po ogórkach kiszonych, to oznaczymy przez $A_i$ prawdopodobieństwo, że startując w punkcie $i$ pijak dopadnie źródła domowych elektrolitów. 
Tutaj dokonam jeszcze podmianki, żeby było mi wygodniej: ponieważ fortuna Oli będzie skakać po wartościach z $0$ na końcu (tzn. podzielnych przez $10$), to każdy krok pijaka będzie krokiem długości $10$. 
To znaczy Ola zdobywa banknoty $10$zł i liczy ich ilość żeby zdecydować czy gra dalej czy nie, a nie dokładną wartość swojego portfela.

Problem z zadania startuje w $i=10$ i wygrana będzie przybliżać nas do $50$ - bar był $10$ metrów od posterunku policji, a dom aż $50$ metrów. 

Liczy się, aby pijak dotarł gdzieś, gdzie ma wodę, więc $A_{50}=1$ i $A_0=1$, bo czy to w domu, czy w więzieniu, jakieś elektrolity się znajdą.
\medskip

Zastanówmy się jak opisać, że pijak startując w $i$-tym kroku dojdzie do domu? 
Możemy to zrobić korzystając z rekurencji. Jeżeli z prawdopodobieństwem $\frac13$ ruszy się w stronę domu, to zrzucamy całą robotę na $A_{i+1}$, jeśli się oddali od domu z prawdopodobieństwem $\frac12$, to będziemy liczyć $A_{i-1}$, a pozostanie w miejscu z prawdopodobieńśtwem $\frac16$, czyli dostajemy:
$$A_i=\frac13A_{i+1}+\frac12A_{i-1}+\frac16A_i$$
$$3A_i=A_{i+1}+\frac32A_{i-1}+\frac12A_i$$
$$\frac52A_i-\frac32A_{i-1}=A_{i+1}$$
i to jest już rekurencja, którą w teorii potrafię rozwiązać, a w praktyce zrobi to za mnie wolframalpha:
$$A_i=c_1\left(\frac32\right)^i+c_2$$
$$\begin{cases}A_0=1=c_1+c_2\\A_{50}=1=c_1{3^{50}\over 2^{50}}+c_2\end{cases}$$
To również rozwiązuje za mnie wolframalpha i mówi, że $c_1=0$ i $c_2=1$, czyli prawdopodobieństwo dojścia do elektrolitów (tudzież zakończenia gry) wynosi $A_{10}=1$.


\subsection*{ZADANIE 3.}
\emph{\color{yellow}Losujemy niezależnie nieskończenie wiele punktów z koła o promieniu $1$ i środku $(0, 0)$. Dla jakich wartości $\varepsilon$ z prawdopodobieństwem $1$ w kole o promieniu $\varepsilon$ i środku $(0, 0)$ znajdzie się nieskończenie wiele punktów?}
\smallskip

Zgaduję że dla $\varepsilon\geq\frac1{\sqrt2}$, bo wtedy te koła to będzie przynajmniej połowa całości.

No boże no, to widać że dla tych na pewno śmignie.

$A_n$ - w $n$-tym ruchu punkt wpada w moje koło. Prawdopodobieństwo wpadnięcia w kółko o promieniu $\varepsilon$ wynosi $\varepsilon^2$.
Coś pojebałam, albo to jest trywialne.


\subsection*{ZADANIE 4.}
\emph{\color{yellow}Zdarzenia $A_1,A_2,...$ są niezależne i $\prawdo{A_n}=p_n\in(0,1)$. Wykaż, że z prawdopodobieństwem $1$ zachodzi co najmniej jedno ze zdarzeń $A_n$ $\iff$ z prawdopodobieństwem $1$ zachodzi nieskończenie wiele zdarzeń $A_n$.}
\smallskip

$$\prawdo{\bigcup\limits_{n=1}A_n}=1\iff\prawdo{\bigcap\limits_{m=1}\bigcup\limits_{n=m}A_n}=1$$

$\impliedby$
dość trywialne, bo
$$\bigcap\bigcup A_n\subseteq\bigcup A_n\implies 1=\prawdo{\bigcap\bigcup A_n}\leq \prawdo{\bigcup A_n}\leq1$$

$\implies$
\begin{align*}
    1=\prawdo{\bigcup\limits_{n=1}^\infty A_n}&=\lim\limits_{N\to\infty}\prawdo{\bigcup\limits_{n=1}^N\prawdo{A_n}}=\lim\limits_{N\to \infty}\left(1-\prod\limits_{n=1}^N(1-\prawdo{A_n})\right)\geq\\
    &\geq\lim\limits_{N\to\infty}1-\prod\limits_{n=1}^Ne^{-\prawdo{A_n}}=\lim\limits_{N\to\infty}1-e^{-\sum\prawdo{A_n}}
\end{align*}
$$1=1-\lim e^{-\sum\prawdo{A_n}}\implies 0=\lim e^{-\sum\prawdo{A_n}}\implies \lim\sum\prawdo{A_n}=\infty$$
i tu już z twerdzenia B-C.


\subsection*{ZADANIE 5.}
\emph{\color{yellow}Rzucamy nieskończenie wiele razy monetą, w której orzeł wypada z prawdopodobieństwem $p\geq\frac12$. Niech $A_n$ oznacza zdarzenie, że pomiędzy rzutem $2^n$ a $2^{n+1}$ otrzymano ciąg $n$ kolejnych orłów. Pokaż, że zdarzenia $A_n$ z prawdopodobieństwem $1$ zachodzą nieskończenie wiele razy.}
\smallskip

Najpierw powinnam znaleźć sobie wzorek na prawdopodobieństwo wyrzucienie orła, pokazać, że to w nieskończoności nie zbiega do $0$, czyli suma jest nieskończona. Wypadałoby powiedzieć o niezależności i reszta to śmiga.

Jakie jest prawdopodobieństwo, że na $2^n$-tym miejscu wypadnie $n$ kolejnych orłów? $p^n$. Jaka jest suma czegoś takiego?
$$\sum\limits_{n=1}^\infty p^n\geq\sum\limits_{n=1}\left(\frac12\right)^n$$


Bardzo nieeleganckie szacowanie, ale co jeśli policzę prawdopodobieństwo, że orzeł wogóle $n$ razy nie wypadnie między $[2^n,2^{n+1})$?
\begin{align*}
    A_n^c&=\sum\limits_{i=0}^{n-1}p^i(1-p)^{2^n-i}=(1-p)^{2^n+1}\left({1-{p\over 1-p}^n\over1-2p}\right)
\end{align*}
W takim razie suma prawdopodobieństw, że wogóle będzie miało szansę wypaść $n$ razy pod rząd, to znaczy orzeł wypadnie co najmniej $n$ razy, wynosi
\begin{align*}
    \sum A_n&=\sum\left[1-(1-p)^{2^n+1}\left({1-{p^n\over(1-p)^n}\over1-2p}\right)\right]=\sum1-\sum(1-p)^{2^n-i}=1-c
\end{align*}
gdzie $c$ to jakaś stała wynikająca z tego, że ciąg w tej drugiej sumie zbiega do $0$, więc szereg jest zbieżny?



\end{document}
