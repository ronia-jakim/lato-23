\documentclass{article}

\usepackage{../../../notatki}

\begin{document}
\subsection*{ZADANIE 10.}
\emph{\color{pink}Niech $(\Omega, \set{F},\mathbb{P})$ będzie przestrzenią probabilistyczną taką, że $\Omega$ jest zbiorem dyskretnym (skończonym lub przeliczalnym). Pokaż, że nie istnieje rodzina niezależnych zdarzeń $\{A_i\}_{i\in \N}$ taka, że $\prawdo{A_n}=\frac12$ dla każdego $n$.}

Najpierw rozpracuję tę wersję nieskończoną przeliczalną.

Załóżmy nie wprost, że mam ciąg $\{A_i\}_{i\in \N}$ taki, że
$$\prawdo{A_i}=\frac12.$$
Wiem, że zdarzeń z $\Omega$ jest przeliczalnie wiele, czyli jest jakiś, którego prawdopodobieństwo jest różne od zera, bo singletony są rozłączne i sumują się do całości. No to weźmy sobie $\omega\in\Omega$ takie, że $\prawdo{\omega}=p\neq0$. Teraz rysuneczek:
\begin{illustration}
    \filldraw[pattern={Hatch[line width=0.5mm, distance=3mm]}, pattern color=black!85!gray!80!] (-1, 0) rectangle (8, 6);
    %\filldraw[pattern={Dots[radius=0.7mm, distance=3mm]}, pattern color=gray] (3.5, 3) circle (2);
    \filldraw[color=back, fill=back] (3.5, 1.27) arc (-120:120:2);
    \filldraw[color = back, fill=back] (3.5, 4.73) arc (60:300:2);
    \filldraw[color=blue, pattern={Lines[line width=1mm, distance=3mm]}, pattern color=black!85!blue!80!] (3.5, 1.27) arc (-120:120:2);
    \filldraw[color=orange, pattern={Dots[radius=1mm, distance=3mm]}, pattern color=black!85!orange!80!] (3.5, 4.73) arc (60:300:2);
    \filldraw[color=back, fill=back] (3.5, 1.27) arc (-60:60:2);
    \filldraw[color=back, fill=back] (3.5, 4.73) arc (120:240:2);
    \filldraw[color=back, pattern={Stars[radius=1mm, distance=3mm]}, pattern color=black!85!green!80!] (3.5, 1.27) arc (-60:60:2);
    \filldraw[color=back, pattern={Stars[radius=1mm, distance=3mm]}, pattern color=black!85!green!80!] (3.5, 4.73) arc (120:240:2);
    \draw[very thick, blue] (4.5, 3) circle (2);
    \draw[very thick, color=orange] (2.5, 3) circle (2);
    \draw[very thick, color=dark-green, dashed] (3.5, 1.27) arc (-60:60:2);
    \draw[very thick, color=dark-green, dashed] (3.5, 4.73) arc (120:240:2);
    \node at (2, 5.3) {$\color{orange}A_1$};
    \node at (6, 5.3) {$\color{blue}A_2$};
    \draw[very thick] (-1, 0) rectangle (8, 6);
    \node at (0, 0.5) {$\color{gray}(A_1\cup A_2)^c$};
    \node at (1.5, 3.5) {$\color{orange}A_1\cap A_2^c$};
    \node at (5.5, 3.5) {$\color{blue}A_2\cap A_1^c$};
    \node at (3.5, 3) {$\color{green}A_1\cap A_2$};
\end{illustration}
Czyli mamy cztery możliwości gdzie zawiera się $\omega$: $(A_1\cup A_2)^c,A_1\cap A_2,A_1\cap A_2^c,A_2\cap A_1^c$. Jak wygląda prawdopodobieństwo każdej z nich?
\begin{align*}
\prawdo{A_1\cup A_2}&=1-\prawdo{A_1\cap A_2}=1-\prawdo{A_1^c}\prawdo{A_2^c}=\\
&=1-(1-\prawdo{A_1})(1-\prawdo{A_2})=1-(1-\frac12)(1-\frac12)=1-\frac14=\frac34
\end{align*}
\begin{align*}
    \prawdo{A_1\cap A_2}&=\prawdo{A_1}\prawdo{A_2}=\frac14
\end{align*}
\begin{align*}
    \prawdo{A_1\cap A_2^c}&=\prawdo{A_1}-\prawdo{A_1\cap A_2}=\frac12-\frac14=\frac14
\end{align*}
\begin{align*}
    \prawdo{A_2\cap A_1^c}&=\prawdo{A_2}-\prawdo{A_1\cap A_2}=\frac12-\frac14=\frac14
\end{align*}
Co jeśli weźmiemy sobie trzy niezależne zbiory? Wtedy mamy $A_1, A_2,A_3$ i punkt należy albo do $\prawdo{A_1\cap A_2\cap A_3}=\frac18$, albo $\prawdo{A_1\cup A_2\cup A_3}=\frac78$, albo $\prawdo{A_1\cap A_2\cap A_3^c}=\prawdo{A_1\cap A_2\setminus A_3}=\prawdo{A_1\cap A_2}-\prawdo{A_1\cap A_2\cap A_3}=\frac38$ albo $\prawdo{[A_1\cup A_2\cup A_3]^c}=\prawdo{A_1^c\cap A_2\cap A_3^c}=\frac18$ tudzież inne kombinacje indeksów, ale to pokryłam na dole.

\begin{illustration}
    \draw[very thick] (-0.5, -0.5) rectangle (8, 6.5);
    \draw[orange, very thick] (2.5, 2.5) circle (2);
    \draw[blue, very thick] (4, 4) circle (2);
    \draw[green, very thick] (5, 2.5) circle (2);
    \node at (3.8, 3) {$\frac18$};
    \node at (3.8, 1.5) {$\frac38$};
    \node at (5, 3.2) {$\frac38$};
    \node at (2.8, 3.5) {$\frac38$};
    \node at (1.5, 2) {$\frac18$};
    \node at (6, 2) {$\frac18$};
    \node at (3.8, 5) {$\frac18$};
    \node at (0, 0) {$\frac78$};
\end{illustration}

\end{document}