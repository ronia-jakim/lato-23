\documentclass{article}

\usepackage{../../../notatki}

\begin{document}
\subsection*{ZADANIE 9.}
\emph{\color{pink}Niech $(\Omega,\set{F}, \mathbb{P})$ będzie przestrzenią probabilistyczną taką, że istnieje $n$ niezależnych zbiorów $B_1,...,B_n$ i takich, że $\prawdo{B_i}\in(0,1)$. Z ilu co najmniej elementów musi się składać $\Omega$?}



\subsection*{ZADANIE 10.}
\emph{\color{pink}Niech $(\Omega, \set{F},\mathbb{P})$ będzie przestrzenią probabilistyczną taką, że $\Omega$ jest zbiorem dyskretnym (skończonym lub przeliczalnym). Pokaż, że nie istnieje rodzina niezależnych zdarzeń $\{A_i\}_{i\in \N}$ taka, że $\prawdo{A_n}=\frac12$ dla każdego $n$.}

\acc{Lemat:} W przestrzeni probabilistycznej dyskretnej $(\exists\;\omega\in\Omega)\;\prawdo{\omega}>0$ a nawet więcej, $(\forall\;A\subseteq\Omega,\prawdo{A}>0)(\exists\;\omega\in A)\;\prawdo{A}$.

Dowód jest oczywisty: jeśli taka $\omega$ by nie istniała, to
$$1=\prawdo{\bigcup\limits_{\omega\in\Omega}\{\omega\}}=\sum\limits_{\omega\in\Omega}\prawdo{\omega}=\sum\limits_{\omega\in\Omega}0=0,$$
bo $\Omega$ jest co najwyżej przeliczalną sumą zbiorów $\{\omega\}$ i są one rozłączne dla różnych elementów $\Omega$. Tak samo powtarzamy dla $A$ - jest ono co najwyżej przeliczalną sumą $\{\omega\}$.
\medskip

\begin{center}
    \dyg{Pomysł jest taki, że dla dowolnego $\omega\in\Omega$ i dowolnego $n\in\N$ pokażę, że $\prawdo{\omega}<2^{-n}$, czyli jest równe zero, co jest sprzeczne z dyskretnością $\Omega$.}
\end{center}
\medskip

\acc{Lemacik:} dla każdego $n$ $A_1,...,A_n$ ciąg zbiorów
$$A_1^{c_1}\cap A_2^{c_2}\cap...\cap A_n^{c_n},$$
gdzie $c_i$ koduje czy bierzemy zbiór $A_i$ czy jego dopełnienie, pokrywa całe $\Omega$.

1. $n=2$, widać na obrazku:
\begin{illustration}
    \filldraw[pattern={Hatch[line width=0.5mm, distance=3mm]}, pattern color=black!85!gray!80!] (-1, 0) rectangle (8, 6);
    %\filldraw[pattern={Dots[radius=0.7mm, distance=3mm]}, pattern color=gray] (3.5, 3) circle (2);
    \filldraw[color=back, fill=back] (3.5, 1.27) arc (-120:120:2);
    \filldraw[color = back, fill=back] (3.5, 4.73) arc (60:300:2);
    \filldraw[color=blue, pattern={Lines[line width=1mm, distance=3mm]}, pattern color=black!80!blue!80!] (3.5, 1.27) arc (-120:120:2);
    \filldraw[color=orange, pattern={Dots[radius=1mm, distance=3mm]}, pattern color=black!80!orange!80!] (3.5, 4.73) arc (60:300:2);
    \filldraw[color=back, fill=back] (3.5, 1.27) arc (-60:60:2);
    \filldraw[color=back, fill=back] (3.5, 4.73) arc (120:240:2);
    \filldraw[color=back, pattern={Stars[radius=1mm, distance=3mm]}, pattern color=black!80!green!80!] (3.5, 1.27) arc (-60:60:2);
    \filldraw[color=back, pattern={Stars[radius=1mm, distance=3mm]}, pattern color=black!80!green!80!] (3.5, 4.73) arc (120:240:2);
    \draw[very thick, blue] (4.5, 3) circle (2);
    \draw[very thick, color=orange] (2.5, 3) circle (2);
    \draw[very thick, color=dark-green, dashed] (3.5, 1.27) arc (-60:60:2);
    \draw[very thick, color=dark-green, dashed] (3.5, 4.73) arc (120:240:2);
    \node at (2, 5.3) {$\color{orange}A_1$};
    \node at (6, 5.3) {$\color{blue}A_2$};
    \draw[very thick] (-1, 0) rectangle (8, 6);
    \node at (0, 0.5) {$\color{gray}(A_1\cup A_2)^c$};
    \node at (1.5, 3.5) {$\color{orange}A_1\cap A_2^c$};
    \node at (5.5, 3.5) {$\color{blue}A_2\cap A_1^c$};
    \node at (3.5, 3) {$\color{green}A_1\cap A_2$};
\end{illustration}

2. Załóżmy, że ciąg $A_1^{c_1}\cap...\cap A_n^{c_n}$ pokrywa całe $\Omega$. Dokładamy kolejny zbiór, $A$.

Weźmy dowolne $\omega$. Wiemy, że istnieje ciąg $c_1,...,c_n$, że $\omega\in A_1^{c_1}\cap....\cap A_n^{c_n}$. Co więcej, $\omega\in A$ lub $\omega\in A^c$. Możemy więc powiedzieć, że $\omega\in A^{c_{n+1}}$. skoro $\omega\in A_1^{c_1}\cap...\cap A_n^{c_n}$ i $\omega\in A^{c_{n+1}}$, to
$$\omega\in A_1\cap...\cap A_{n+1}^{c_{n+1}}$$
\newpage

\acc{Lemaciuś:} Dla dowolnego ciągu jak wyżej
$$\prawdo{A_1^{c_1}\cap...\cap A_n^{c_n}}={1\over2^n}$$

1. $n=2$

\begin{align*}
    \prawdo{A_1\cap A_2}&=\prawdo{A_1}\prawdo{A_2}=\frac14\\
    \prawdo{A_1\cap A_2^c}&=\prawdo{A_1\setminus A_2}=\prawdo{A_1}-\prawdo{A_1\cap A_2}=\frac12-\frac14=\frac14\\
    \prawdo{A_1^c\cap A_2}&=\prawdo{A_2\setminus A_1}=\prawdo{A_2}-\prawdo{A_1\cap A_2}=\frac14\\
    \prawdo{A_1^c\cap A_2^c}&=\prawdo{(A_1\cup A_2)^c}=1-\prawdo{A_1\cup A_2}=1-(\prawdo{A_1}+\prawdo{A_2}-\prawdo{A_1\cap A_2})=\\
    &=1-(\frac12+\frac12-\frac14)=\frac14
\end{align*}

2.  Załóżmy, że $\prawdo{A_1^{c_1}\cap...\cap A_n^{c_n}}=2^{-n}$. Niech teraz $A$ będzie kolejnym zbiorem (lub jego dopełnieniem). Poprzestawiajmy indeksy tak, żeby rozważany ciąg miał postać
$$A_1\cap...\cap A_m\cap A_{m+1}^c\cap...\cap A_n^c,$$
którą piszę się zdecydowanie przyjemniej, a jest równoważny. Dla $m=n$ mamy to z faktu, że $A, A_1,...,A_n$ są niezależne, dla $m=n-1$
\begin{align*}
    \prawdo{A\cap A_1^c\cap A_2\cap...\cap A_n}&=\prawdo{(A\cap A_2\cap...\cap A_n)\setminus A_1}=\prawdo{A\cap...\cap A_n}-\prawdo{A_1\cap A\cap A_2\cap...\cap A_n}=\\
    &={1\over 2^n}-{1\over 2^{n+1}}={2-1\over 2^{n+1}}=2^{-n-1}
\end{align*}

Czyli zostaje nam sprawdzić tezę dla $m<n-1$
\begin{align*}
    \prawdo{A\cap A_1\cap...\cap A_m\cap A_{m+1}^c\cap...\cap A_n^c}&=\prawdo{(A\cap A_1\cap...\cap A_m)\cap (A_{m+1}\cup...\cup A_n)^c}=\\
    &=\prawdo{(A\cap A_1\cap...\cap A_m)\setminus(A_{m+1}\cup...\cup A_n)}=\\
    &=\prawdo{A\cap A_1\cap ...\cap A_m}-\prawdo{(A\cap A_1\cap...\cap A_m)\cap(A_{m+1}\cup...\cup A_n)}=\\
    &=\prawdo{A}\prod\limits_{i=1}^m\prawdo{A_i}-\prawdo{\bigcup\limits_{i=m+1}^n(A\cap...\cap A_m)\cap A_{i}}=(\coffee)
\end{align*}

\begin{align*}
    \prawdo{\bigcup A\cap ...\cap A_m\cap A_i}&=\sum \prawdo{A\cap...\cap A_m\cap A_i}-\sum\sum \prawdo{A\cap...\cap A_m\cap A_i\cap A_j}+...+\\
    &+(-1)^{n-m}\prawdo{A\cap A_1\cap...\cap A_n}=\\
    &=\sum 2^{-m-2}-\sum\sum 2^{-m-3}+...+(-1)^{n-m}2^{-n-1}=\\
    &=2^{-m-2}{n-m\choose 1}-{n-m\choose 2}2^{-m-3}+...+(-1)^{n-m+1}2^{-n-1}
\end{align*}

\begin{align*}
    (\coffee)&=2^{-m-1}-2^{-m-2}{n-m\choose 1}+{n-m\choose 2}2^{-m-3}+...+(-1)^{n-m}2^{-n-1}=\\
    &=2^{-m-1}(1-2^{-1})^{n-m}=2^{-m-1}2^{-n+m}=2^{-n-1}
\end{align*}
\bigskip

Lemacik mówił, że każdy $\omega\in\Omega$ musi wpaść w 
$$A=A_1^{c_1}\cap...\cap A_n^{c_n},$$ natomiast Lemaciuś powiedział, że
$$(\forall\;n\in\N)\;\prawdo{A}<2^{-n}$$
czyli $(\forall\;\varepsilon>0)\;\prawdo{\omega}<\varepsilon\implies\prawdo{\omega}=0$.

Całkowicie pomijam na chwilę obecną przypadek skończony.

\proofend

\end{document}