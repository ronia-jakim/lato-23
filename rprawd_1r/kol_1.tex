\documentclass{article}

\usepackage[light-theme]{../lecture_notes}

\begin{document}
\begin{enumerate}
    \item Jest $10$ kostek do gry, z czego $2$ sfałszowane - zawsze wypada na nich szóstka. Wybieramy losowo dwie kostki i rzucamy nimi. Obliczyć:
    \begin{enumerate}
        \item Prawdopodobieństwo wyrzucenia w sumie $12$ oczek.
        \item Prawdopodobieństwo warunkowe, że minimum $1$ kostka jest sfałszowana, jeśli wyrzuciliśmy w sumie $12$.
    \end{enumerate}
    \item Zmienna losowa $X$ ma rozkład $f(x)=\frac{x}{2}\cdot\mathbb{1}_{[0,2]}$.
    \begin{enumerate}
        \item Znajdź rozkład $Y=X^2$. Wyznacz dystrybuantę i gęstość.
        \item Oblicz $\prob{X<Y}$
    \end{enumerate}
    \item Przypomnijmy, że zmienna losowa $X$ ma rozkład wykładniczy z parametrem $\lambda$, gdy jej gęstość jest zadana wzorem $f(x)=\lambda e^{-\lambda x}\cdot\mathds{1}_{(0,\infty)}(x)$, a dystrybuanta to $F(x)=1-e^{-\lambda x}$.

    Niech $X$ i $Y$ będą niezależne o rozkładzie z parametrem $1$. Niech $\lambda,\mu$ będą ściśle dodatnie, wyznacz rozkład
    $$Z=\min\{\frac{x}{\lambda},\frac{y}{\mu}\}$$
    \item Zmienna losowa ma rozkład jednostajny na kwadracie o wierzchołkach $(1, 0), (0, 1), (-1, 0), (0, -1)$.
    \begin{enumerate}
        \item Wyznacz rozkłady brzegowe zmiennych losowych $X$ i $Y$
        \item Czy $X$ i $Y$ są niezależne?
        \item Wyznacz rozkład zmiennej $Z=|X+Y|$
    \end{enumerate}
    \item Z odcinka $[0,1]$ losujemy niezależnie $a_1,a_2,...$. Udowodnij, że z prawdopodobieństwem $1$ ciąg $a_n$ zawiera ciąg rosnący.
\end{enumerate}
\end{document}
