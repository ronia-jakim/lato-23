\documentclass{article}

\usepackage{../../../../notatki}

\begin{document}

\begin{center}\begin{tabular}{| c | c | c | c | c | c | c | c | c | c | c |}
    \hline

    1 & 2 & 3 & 4 & 5 & 6 & 7 & 8 & $\Sigma$\\

    \hline

    - & - & - & - & - & - & - & - & ---\\

    \hline
\end{tabular}\end{center}

\subsection*{ZADANIE 1.}
\emph{Napisz rekurencyjne funkcje, które dla danego drzewa binarnego $T$ obliczają:}

Zakładam, że drzewa są pełne binarne (czyli albo ma dwoje dzieci, albo ani jednego)

\emph{(a) liczbę wierzchołków w $T$}

\begin{lstlisting}[language=Python]
# dupa

A[n][2] <- macierz sasiedztwa T, -1 jesli nie ma sasiadow

function cnt_vertices(v):
    if A[v][0] != -1:
        return cnt_vertices(A[v][0]) + cnt_vertices(A[v][1]) + 1
    else:
        return 1
\end{lstlisting}

\emph{(b) maksymalną odległość między wierzchołkami w $T$.}

\begin{lstlisting}[language=Python]
# dupa

A[n][2] <- macierz sasiedztwa T, -1 jesli nie ma sasiadow

function max_dist(v, m):
    if A[v][0] != -1:
        left <- max_dist(A[v][0])
        right <- max_dist(A[v][1])
        max_fork = max(left[0], right[0], left[1] + right[1] + 2)
        max_down = max(left[1], right[1]) + 1
        return [max_fork, max_down]
    return [0, 0]

ret <- max_dist(0)
OUTPUT max(ret[0], ret[1])
\end{lstlisting}

\subsection*{ZADANIE 3.}
\emph{\text{Porządkiem topologicznym} wierzchołków acyklicznego digrafu $G=(V, E)$ nazywamy taki liniowy porządek jego wierzchołków, w którym początek każdej krawędzi występuje przed jej końcem. Jeśli wierzchołki z $V$ utożsamimy z początkowymi liczbami naturalnymi, to każdy ich porządek liniowy można opisać permutacją liczb $1,...,|V|=n$; w szczególności pozwala to na porównanie leksykograficzne porządków.}

\emph{Ułóż algorytm, który dla danego acyklicznego digrafu znajduje pierwszy leksykograficznie porządek topologiczny.}
\smallskip

\subsection*{ZADANIE 4.}
\emph{Niech $u$ i $v$ będą dwoma wierzchołkami w grafi nieskierowanym $G=(V, E; c)$, gdzie $c:E\to \R_+$ jest funkcją wagową. Mówimy, że droga z $u=u_1,...,u_{k-1},u_k=v$ z $u$ do $v$ jest sensowna, jeśli dla każdego $i=2,...,k$ istnieje droga z $u_i$ do $v$ krótsza od każdej drogi z $u_{i-1}$ do $v$ (przez długość drogi rozumiemy sumę wag jej krawędzi).}

\emph{Ułóż algorytm, który dla danego $G$ oraz wierzchołków $u$ i $v$ wyznaczy liczbę sensownych dróg z $u$ do $v$.}

\subsection*{ZADANIE 5.}

\begin{lstlisting}{language=Python}
A <- lista sasiedztwa
IN <- stworzona przy wczytywaniu lista 


\end{lstlisting}

\end{document}