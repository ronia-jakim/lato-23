\documentclass{article}

\usepackage{../../../../notatki}

\begin{document}

\begin{center}\begin{tabular}{| c | c | c | c | c | c | c | c | c | c | c |}
    \hline

    1 & 2 & 3 & 4 & 5 & 6 & 7 & 8 & $\Sigma$\\

    \hline

    + & - & + & + & + & - & - & - & ---\\

    \hline
\end{tabular}\end{center}

\subsection*{ZADANIE 1.}
\emph{\color{pink}Napisz rekurencyjne funkcje, które dla danego drzewa binarnego $T$ obliczają:}

Zakładam, że drzewa są pełne binarne (czyli albo ma dwoje dzieci, albo ani jednego)

\emph{\color{pink}(a) liczbę wierzchołków w $T$}

\begin{lstlisting}[language=Python]
# dupa

A[n][2] <- macierz sasiedztwa T, -1 jesli nie ma sasiadow

function cnt_vertices(v):
    if A[v][0] != -1:
        return cnt_vertices(A[v][0]) + cnt_vertices(A[v][1]) + 1
    else:
        return 1
\end{lstlisting}

\emph{\color{pink}(b) maksymalną odległość między wierzchołkami w $T$.}

\begin{lstlisting}[language=Python]
# dupa

A[n][2] <- macierz sasiedztwa T, -1 jesli nie ma sasiadow

function max_dist(v, m):
    if A[v][0] != -1:
        left <- max_dist(A[v][0])
        right <- max_dist(A[v][1])
        max_fork = max(left[0], right[0], left[1] + right[1] + 2)
        max_down = max(left[1], right[1]) + 1
        return [max_fork, max_down]
    return [0, 0]

ret <- max_dist(0)
OUTPUT max(ret[0], ret[1])
\end{lstlisting}

\subsection*{ZADANIE 3.}
\text{\color{pink}Porządkiem topologicznym} \emph{\color{pink}wierzchołków acyklicznego digrafu $G=(V, E)$ nazywamy taki liniowy porządek jego wierzchołków, w którym początek każdej krawędzi występuje przed jej końcem. Jeśli wierzchołki z $V$ utożsamimy z początkowymi liczbami naturalnymi, to każdy ich porządek liniowy można opisać permutacją liczb $1,...,|V|=n$; w szczególności pozwala to na porównanie leksykograficzne porządków.}

\emph{\color{pink}Ułóż algorytm, który dla danego acyklicznego digrafu znajduje pierwszy leksykograficznie porządek topologiczny.}
\smallskip

\begin{lstlisting}[language=Python]
A <- macierz sasiedztwa
IN <- tablica ilosci krawedzi wchodzacych do wierzcholka

ret <- []
golaski <- [] kolejka piorytetowa

i <- 0

# sprawdzam potencjalne poczatki mojej permutacji i pamietam je w kolejnosci rosnacej
while i < n:
    if IN[i] == 0:
        golaski.insert-node(i)

# dopoki mam potencjalne wierzcholki do wstawienia
while len(golaski) > 0:
    m <- find-min(golaski)
    ret.append(m)
    golaski.deletemin()

    # obcinam golaska od jego sasiadow
    for i in A[m]:
        IN[i]--
        if IN[i] == 0:
            golaski.push_back(i)
    
OUTPUT ret
\end{lstlisting}

\subsection*{ZADANIE 4.}
\emph{\color{pink}Niech $u$ i $v$ będą dwoma wierzchołkami w grafi nieskierowanym $G=(V, E; c)$, gdzie $c:E\to \R_+$ jest funkcją wagową. Mówimy, że droga z $u=u_1,...,u_{k-1},u_k=v$ z $u$ do $v$ jest sensowna, jeśli dla każdego $i=2,...,k$ istnieje droga z $u_i$ do $v$ krótsza od każdej drogi z $u_{i-1}$ do $v$ (przez długość drogi rozumiemy sumę wag jej krawędzi).}

\emph{\color{pink}Ułóż algorytm, który dla danego $G$ oraz wierzchołków $u$ i $v$ wyznaczy liczbę sensownych dróg z $u$ do $v$.}

\begin{lstlisting}[language=Python]
A <- tablica somsiadow

D <- robimy Dijkstre, dostajemy tablice

i <- 0

S <- [[] * n]

while i < n:
    for j in A[i]:
        if D[i] > D[j]:
            S[i].push_back(j)
        else if D[j] > D[i]:
            S[j].push_back(i)

P <- topo sort <3

Dupa <- puste n * []
Dupa[u] <- 1

for i in P:
    for j in S[i]:
        Dupa[j] += Dupa[i]

OUTPUT Dupa[v]
\end{lstlisting}

\subsection*{ZADANIE 5.}
\emph{\color{pink}Ułóż algorytm, który dla zadanego acyklicznego grafy skierowanego $G$ znajduje długość najdłuższej drogi w $G$. Następnie zmodyfikuj swój algorytm tak, by wypisywał drogę o największej długości (jeśli jest kilka takich dróg, to Twój algorytm powinien wypisać dowolną z nich).}


\begin{lstlisting}[language=Python]
A <- lista sasiedztwa
IN <- stworzona przy wczytywaniu lista 

golaski <- []

MAX_DROGA <- tablica n razy 0

i <- 0
while i < n:
    if IN[i] == 0:
        golaski.append(i)

while len(golaski) > 0:
    m <- golaski[len(golaski)]
    golaski.pop_back()
    
    for i in A[m]:
        IN[i]--
        if IN[i] == 0:  
            golaski.push_back(i)

        # tutaj maksymalna droga do i to jest albo to co bylo, albo droga do m powiekszona o 1?
        MAX_DROGA[i] = max(MAX_DROGA[m] + 1, MAX_DROGA[i])

OUTPUT max(MAX_DROGA)

\end{lstlisting}

\end{document}