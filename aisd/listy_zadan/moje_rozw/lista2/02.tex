\documentclass{article}

\usepackage{../../../../lecture_notes}

\begin{document}
\subsection*{ZADANIE 2.}
\excercise[d]{Udowodnij, że algorytm Kruskala znajduje minimalne drzewa spinające poprzez przyrównanie tych drzew do drzew optymalnych}

Po pierwsze, rozpiszmy co robi ten algorytm:


\begin{lstlisting}
E' <- [] pusty zbior
C <- E

while |E'|<n+1
    e <- min(C)
    if (E' + e nie ma cyklu)
        E' <- E' + e
    C <- C - e
\end{lstlisting}

Niech $T$ będzie drzewem minimalnym, a $E'$ będzie wynikiem algorytmu. Chcę pokazać, że $E'$ jest minimalnym spinning tree. To że jest wogóle tree, to widać, elo.

Co, jeśli istnieje $e\in E'\setminus T$? Wtedy $e\cup T$ będzie miało cykle, bo $T$ zachaczało o wszystkie krawędzie, jara jara jara. Czyli w ten sposób tworzymy sobie pewien cykl. Super. To teraz w $T$ musiałam mieć jakieś inne przejście w tym cyklu, niech to będzie $f$. No ale miara $e$ była mniejsza, bo inaczej to był $f$ próbowała dołączyć przed $e$ do $E'$ i by nie stwierdziło, że się zacykli. Czyli graf $E'\setminus e\cup f$ będzie miał troszkę większą sumę niż $E'$. 

Robiąc tak indukcyjnie aż nam się wszystkie wierzchołki pokryją, dojdziemy do grafu pokrywającego się z $T$, ale o większej mocy niż $E'$. Czyli to nie mogło być tak, że takie usuwanie krawędzi faktycznie zwiększało sumę, tylko to wszystko musiało zostawać tak samo, więc suma z $E'$ to to samo co te sumy w międzyczasie, a one z kolei równały się sumie $T$.

\subsection*{ZADANIE 3.}
\excercise[d]{Danych jest $n$ odcinków $I_j=[p_j,k_j]$ leżących na osi $OX$, $j=1,...,n$. Ułóż algorytm znajdujący zbiór $S\subseteq \{I_1,...,I_n\}$ nieprzecinających się przecinków, o największej mocy.}

Mam listę $P$ i $K$, odpowiednio początków i końców tych pyśków. Może jakoś od razu sobie założę, że mam dwójkę? Indeks i wartość początku/końca? wtedy mogę łatwo wiedzieć gdzie był czyj koniec?

Dobra, to teraz biorę pierwszego pyśka z końców i wkładam go sobie do $S$. W ten sposób gwarantuję sobie, że śmignie, bo nawet jeśli jakiś z dalszym końcem da mi ten sam wynik, to nie wiem startowo że tak będzie, więc po prostu wybieram to, co niszczy mi najmniej innych wyborów.

W następnej kolejności idę dalej przez $K$ aż znajdę pierwszy koniec, który ma początek ostro większy niż to co jako pierwsze wybrałam. Tutaj jest ta sama historia. No i tak dalej aż do końca listy $K$.


\subsection*{ZADANIE 4.}
\excercise[d]{Rozważ następująca wersję problemu wydawania reszty: dla danych liczb naturalnych $a,b, (a\leq b)$ chcemy przedstawić ułamek $\frac{a}{b}$ jako sumę różnych ułamków o licznikach równych $1$. Udowodnij, że algorytm zachłanny zawsze daje rozwiązanie. Czy zawsze jest to rozwiązanie optymalne (tj. o najmniejszej liczbie składników?)}

Rozumiem, że algorytm zachłanny po prostu leci przez kolejne liczby naturalne i sprawdza, czy dodając je dostaję coś większego, czy się mieszczę? I jeszcze sprawdzam, czy jest sens iść dalej, to znaczy czy mam już dokładnie ten ułamek który chciałam.

%Chyba daje to najlepsze, bo na pewno mogę każdy ten ułamek $\frac1p$ rozbić dalej, ale to wtedy zwiększam $p$ niepotrzebnie i najpierw te mniejsze zaliczę w pierwszej kolejności. Jeżeli istniałaby reprezentacja bardziej optymalna, to miałaby jeden ułamek z mniejszym mianownikiem, ale my szliśmy od największych mianowników, czyli sprzeczność?

Otóż nie daje najlepszego, bo mi podusia powiedziała, że istnieje
$$\frac{9}{20}=\frac13+\frac19+\frac1{180}$$
według zachłana, a można też zrobić
$$\frac9{20}=\frac14+\frac15$$

Niech $k$ będzie najmniejszą taką liczbą naturalną, że
$$\frac{a}{b}-\frac1k=\frac{ak-b}{bk}>0.$$
Chcę pokazać, że $a>ak-b$. Wtedy zakończenie algorytmu wynika z nieistnienia nieskończonego, malejącego ciągu liczb naturalnych.
$$a>ak-b$$
$$b>ak-a=a(k-1)$$
$$\frac{b}{a}>k-1$$
$$\frac{1}{k-1}>\frac{a}{b}$$
co jest prawdą, bo $k$ było najmniejsze takie, że $\frac{a}{b}\geq\frac{1}{k}$



\subsection*{ZADANIE 5.}
\excercise[u]{Ułóż algorytm, który dla danego $n$-wierzchołkowego drzewa i liczby $k$, pokoloruje jak najwięcej wierzchołków tak, ba na każdej ścieżce prostej było nie więcej niż $k$ pokolorowanych wierzchołków.}

Czy ja chcę pokolorować wszystkie wierzchołki, uciąć je i powtórzyć to samo dla $k-2$ na tym nowym drzewie? A jeśli $k=1$, to wtedy chyba losowy jeden mogę pomalować.

\subsection*{ZADANIE 6.}
\excercise{Ułóż algorytm, który dla danego spójnego grafu $G$ oraz krawędzi $e$ sprawdza w czasie $O(n+m)$, czy krawędź $e$ należy do jakiegoś minimalnego drzewa spinającego grafu $G$. Możesz założyć, że wszystkie wagi krawędzi są różne.}

\begin{problem}[9]{}
Operacja \emph{swap(i,j)} na permutacji powoduje przestawienie elementów znajdujących się na pozycjach $i$ oraz $j$. Koszt takiej operacji określamy na $|i-j|$. Kosztem ciągu operacji \emph{swap} jest suma kosztów poszczególnych operacji.
\end{problem}

U mnie indeks tablicy to aktualna liczba którą przestawiamy, a zawartość tablicy mówi gdzie ona stoi

\begin{lstlisting}
P <- permutacja pi
S <- permutacja sigma
left <- [] * n
right <- [] * n

op = []

for i in {0, 1, ..., n-1}:
    if P[i] > S[i]: # jesli element stoi bardziej w prawo w pi
        left += [i]
    if P[i] < S[i]: # element w pi jest bardziej na lewo
        right += [i]

for i in right:
    for j in left:
        # zamieniam to co chce w lewo z tym co chce w prawo
        pom = P[i]
        P[i] = P[j]
        P[j] = pom

        if P[j] == S[j]:
            left.remove(j)

        if P[i] == S[i]:
            right.remove(i)
            break
\end{lstlisting}

\begin{problem}[2]{}
Z LISTY 3

To z prostymi
\end{problem}

\begin{lstlisting}[language=Python]
proste <- lista par (a, b)

merge_sort (L):
    if |L| == 1:
        return L
    if |L| == 2:
        if L[0][0] == L[1][0]:
            return [[L[0][0], max(L[0][1], L[1][1])]]
        else:
            if L[0][0] < L[1][0]:
                return [L[0], L[1]]
            else:
                return [L[1], L[0]]
    A = L[0::|L|/2]
    B = L[|L|/2 + 1, |L|-1]
    if (A[0][0] < B[0][0]):
        return A + B
    else:
        return B + A

P = merge_sort(proste)
n = |P|

RET = [P[0], P[1]]
for i in {2, ..., n-1}:
    j = i-1
    przec = (RET[j-1][1] - P[i][1]) / (P[i][0]-RET[j-1][0])
    while i >= 0 && (P[i][0] * przec + P[i][1]) >= (RET[j][0] * przec + RET[j][1]):
        RET.pop_back()
        przec = (RET[j-1][1] - P[i][1]) / (P[i][0]-RET[j-1][0])
        j = j-1
    RET += P[i]

    

\end{lstlisting}











\end{document}
