\documentclass{article}

\usepackage{../../../notatki}

\begin{document}

\begin{center}\begin{tabular}{| c | c | c | c | c | c | c | c | c | c | c |}
    \hline

    1 & 2 & 3 & 4 & 5 & 6 & 7 & 8 & 9 & 10 & $\Sigma$\\

    \hline

    - & - & - & + & - & + & + & + & - & + & ???\\

    \hline
\end{tabular}\end{center}

\subsection*{ZADANIE 4.}
\emph{Udowodnij, że algorytm mnożenia liczb "po rosyjsku" jest poprawny. Jaka jest jego złożoność czasowa i pamięciowa przy:}

\emph{1. jednorodnym kryterium kosztów,}

\emph{2. logarytmicznym kryterium kostów?}
\medskip

\textbf{Dowodzik:}

Ustalmy dowolne $b\in\N$. Pokażemy przez indukcję, że dla dowolnego $a\in\N$ wynik algorytmu jest równy $ab$. Przypadek bazowy, czyli $a=1$ jest trywialny. 

Załóżmy teraz, że dla dowolnego $a'\leq n$ algorytm działa i niech $a=n+1$. Rozważmy dwa przypadki:

1. $(n+1)$ jest nieparzyste. Wtedy $a_1=(n+1)$ jest nieparzyste, wpp. do $a_1'=n$. Dalej, $a_2=\begin{floor}{n+1\over 2}\end{floor}=\begin{floor}{n\over 2}\end{floor}=a_2'$, czyli od drugiego miejsca ciąg $a_i$ dla $n+1$ jest taki sam jak dla $n$, więc:
\begin{align*}
    \sum\limits_{i=1,a_i \;np}^kb_i=b_1+\sum\limits_{i=2,a_i'\; np} b_i=b_1+n\cdot b=b+bn=b(n+1)
\end{align*}

2. $(n+1)$ jest parzyste. Wtedy $a_1=(n+1)$ jest parzyste, więc $b_1$ nie zostanie użyte w sumie. Natomiast $\begin{floor}{n+1\over 2}\end{floor}=\begin{floor}{n\over 2}\end{floor}+1=a_2'+1$, czyli od trzeciego indeksu $a_i$ jest taki sam jak $a_i'$
\begin{align*}
    \sum\limits_{i=1, a_i \;np}^kb_i=a_2b_2+\sum\limits_{i=3,a_i\;np}^kb_i=(a_2'+1)2b+\sum\limits_{i=3}^k b_i=2b+\sum\limits_{i=2,a_i'\;np}^kb_i=2b+\sum\limits_{i=1}^kb_i-b_1=b+\sum\limits_{i=1}^kb_i=b+nb
\end{align*}

\textbf{Złożoność:}

1. Będziemy obliczać wyrazy ciągu $a_i$ $\log_2(a)$ razy, za każdym razem będziemy dzielić, sprawdzać podzielność, mnożyć $b$ i ewentualnie dodawać, co jest mniej więcej stałą liczbą operacji.

2. {\color{orange}?????}

\subsection*{ZADANIE 5.}
\emph{Oszacuj z dokładnością do $\Theta$ złożoność poniższego fragmentu programu:}
\begin{lstlisting}
res <- 0
for i <- 1 to n do
    j <- i
    while (j jest parzyste) j <- j/2
    res <- res + j
\end{lstlisting}

\subsection*{ZADANIE 6.}
\emph{Pokaż, w jaki sposób algorytm "macierzowy" obliczania $n$-tej liczby Fibonacciego można uogólnić na inne ciągi, w których kolejne elementy definiowane są liniową kombinacją skończonej liczby elementów wcześniejszych. Następnie uogólnij swoje rozwiązanie na przypadek, w którym $n$-ty element ciągu definiowany jest jako suma kombinacji liniowej skończonej liczby elementów wcześniejszych oraz wielomianu zmiennej $n$.}
\medskip

Pierwsza część jest dość prosta. Rozważamy ciąg zdefiniowany rekurencyjnie
$$a_n=\alpha_1a_{n-1}+\alpha_2a_{n-2}+...+\alpha_ka_{n-k}.$$
Popatrzmy na macierz
$$A=\begin{bmatrix}
    \alpha_1 & \alpha_2 & \alpha_3 & ... &\alpha_{k-1} & \alpha_k\\
    1 & 0 & 0 & ... & 0 & 0\\
    0 & 1 & 0 & ... & 0 & 0\\
    ... & ... & ... & ... & ... & ...\\
    0 & 0 & 0 & ... & 1 & 0
\end{bmatrix}$$
Mnożąc $A^{n-1}$ przez wektor
$$\begin{bmatrix}
    a_{k-1}\\
    a_{k-2}\\
    ...\\
    a_0
\end{bmatrix}$$
dostajemy wektor zawierający $a_n$ na pierwszym miejscu oraz wszystkie poprzednie miejsca na pozostałych miejscach.
\smallskip

Teraz co się dzieje dla ciągu zawierającego wielomian zmiennej $n$?

Przyjrzyjmy się ciągowi zdefiniowanemu rekurencyjnie ze szczyptą wielomianu:
$$a_n=\sum\limits_{i=1}^k\alpha_ia_{n-i}+\sum\limits_{i=0}^m\beta_in^i$$
Mogę zacząć od tego, że
$$\begin{bmatrix}
    a_n\\a_{n-1}\\a_{n-2}\\...\\a_{n-k+1}
\end{bmatrix}=\begin{bmatrix}
    \alpha_1 & \alpha_2 & \alpha_3 & ... &\alpha_{k-1} & \alpha_k\\
    1 & 0 & 0 & ... & 0 & 0\\
    0 & 1 & 0 & ... & 0 & 0\\
    ... & ... & ... & ... & ... & ...\\
    0 & 0 & 0 & ... & 1 & 0
\end{bmatrix}\begin{bmatrix}
    a_{n-1}\\a_{n-2}\\a_{n-3}\\...\\a_{n-k}
\end{bmatrix}+\begin{bmatrix}
    \sum\beta_in^i\\0\\0\\...\\0
\end{bmatrix}$$

{\color{orange}Czy jest coś więcej, co mogę o tym cudeńku powiedzieć?}

\subsection*{ZADANIE 7.}
\emph{Rozważ poniższy algorytm, który dla danego (wielo)zbioru $A$ liczb całkowitych wylicza pewną wartość. Twoim zadaniem jest napisanie programu (w pseudokodzie), możliwie najoszczędniejszego pamięciowo, który wylicza tę samą wartość.}

\begin{lstlisting}
while |A| > 1 do
    a <- losowy element z A
    A <- A \ {a}
    b <- losowy element z A
    A <- A \ {b}
    A <- A u {a-b}

output (x mod 2), gdzie x jest elementem ze zbioru A
\end{lstlisting}
\smallskip

Zauważmy, że wynik zależy od ilości elementów nieparzystych. To znaczy, jeżeli elementów nieparzystych jest parzyście wiele, to dostaniemy $0$, wpp. dostaniemy $1$, gdyż te zbędne jedynki nieparzystości nie zniosą się do końca.

Nie musimy więc trzymać całego $A$ w pamięci przez cały czas, a wystarczy wczytywać je element po elemencie, sprawdzać jego podzielność i odpowiednio modyfikować aktualną wartość wyniku:
\begin{lstlisting}
ret = 0
while A ma niewczytane elementy:
    a <- kolejny element A
    if a % 2
        ret = (ret + 1) % 2
    
output ret
\end{lstlisting}

\subsection*{ZADANIE 8.}
\emph{Ułóż algorytm, który dla drzewa $T=(V, E)$ oraz listy par wierzchołków $\{v_i,u_i\}$ ($i=1,...,m$) sprawdza, czy $v_i$ leży na ścieżce z $u_i$ do korzenia. Przyjmij, że drzewo zadane jest jako lista $(n-1)$ krawędzi $(p_i,a_i)$ takich, że $p_i$ jest ojcem $a_i$ w drzewie.}

Koncepcja jest taka, że idziemy od $u_i$ w górę, jeśli trafimy na $v_i$, piszemy TAK, wpp gdy jesteśmy na korzeniu, piszemy NIE

\subsection*{ZADANIE 10.}
\emph{Ułóż algorytm dla następującego problemu:}

\emph{\textbf{PROBLEM}}

\emph{dane: $n,m\in\N$}

\emph{wynik: wartość współczynnika przy $x^2$ (wzięta modulo $m$) wielomiany $(...((x-2)^2-2)^2...-2)^2$, gdzie nawiasów ogółem jest $n$.}

\emph{Czy widzisz zastosowanie metody użytej w szybkim algorytmie obliczania $n$-tej liczby Fibonacciego do rozwiązania tego problemu?}
\medskip

Tak naprawdę wystarczy, że będę trzymać to, co się dzieje przy wyrazie stałym, wyrazie z $x$ i wyrazie z $x^2$. 

Wyraz stały w tym cudeńku to zawsze będzie $4$, bo z tego nawiasu w środku zawsze mamy $4$ na końcu, odejmujemy $2$ i podnosimy do kwadratu wielomian z $2$ jako wyrazem stałym. To widać. 

Wyraz przy $x$ to będą kolejne potęgi $4$ na minusie? Można to pokazać przy pomocy ciągu rekurencyjnego $a_n$:
$$a_1=-4$$
$$a_n=4a_{n-1}=-4^n$$
bo
\begin{align*}
    p_n(x)&=(...((x-2)^2-2)^2...-2)^2=4+a_nx+....\\
    (p_n(x)-2)^2&=p_n(x)^2-4p_n(x)+4=(4+a_nx+...)^2-4(4+a_nx+...)+4=\\
    &=8a_nx-4a_nx+...=4a_nx
\end{align*}

To teraz pozostaje mi napisać rekurencję na $b_n$, czyli wyraz przy $x^2$.
\begin{align*}
    p_n(x)&=(...((x-2)^2-2)^2...)^2=4-4^nx+b_nx^2+...\\
    (p_n(x)-2)^2&=p_n(x)^2-4p_n(x)+4=(4-4^nx+b_nx^2+...)^2-4(4-4^nx+b_nx^2+...)+4=\\
    &=8b_nx^2+4^{2n}x^2-4b_nx^2+...
\end{align*}
Czyli
$$b_1=1$$
$$b_n=4b_n+16^{n-1}$$

\end{document}