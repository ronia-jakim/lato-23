\documentclass{article}

\usepackage{amsmath}
\usepackage{amssymb}

\usepackage{amsthm}


\title{Gaußian elimination}
\author{Weronika Jakimowicz}

\usepackage{coffeestains}
\newcommand{\rank}[1]{\text{rank}(#1)}
\newcommand{\colctn}[1]{\text{colCount}(#1)}
\newcommand{\Ab}{(A|b)}

\usepackage{hyperref}
\usepackage{cleveref}

\newtheorem{lemma}{Lemma}[subsection]
\newtheorem{remark}[lemma]{Remark}
\newtheorem{thm}[lemma]{Theorem}
\newtheorem{proposition}[lemma]{Propozycja}

\theoremstyle{definition}
\newtheorem{deff}[lemma]{Definition}
\begin{document}
\maketitle

\begin{abstract}
We will present some variant of Gaußsian elimination. In particular, the method we present is useful in fining the number of solutions to a given system of linear equations. We note that our method does not cover finding the solution set.
\end{abstract}

\setcounter{section}{-1}
\coffeestainA{0.3}{0.85}{-25}{5cm}{1.3cm}

\section{Preliminaries}
All matrices will be over a fixed field $\mathbb{F}$

\begin{deff}
A matrix if of \emph{reduced echelon form $(REF)$} if:
\begin{itemize}
\item All rows consisting of only zeroes are at the bottom.
\item The leading entry (that is the left-most nonzero entry) of every nonzero row is to the right of the leading entry of every row above.
\end{itemize}
\end{deff}

Note that this variant doesn't require leading entries to be $1$.

\begin{deff}
Two matrices $A,B$ are \emph{row equivalent} denoted $A\sim B$ if $A$ is a product of elementary row operations on $B$.
\end{deff}

\begin{lemma}
$\sim$ is an equivalence relation.
\end{lemma}

\begin{deff}
Let $A$ be a matrix. Then $\colctn(A)$ is the number of columns in $A$.
\end{deff}

\begin{deff}
Let $A$ be an REF matrix. Then $\rank{A}$ is the number of non-zero rows in $A$, equivalantly, the number of leading entries.
\end{deff}

\begin{remark}\label{rem:0:0:6}
Let $A$ be an REF matrix. Then $\rank{A}\leq\colctn{A}$.
\end{remark}

\section{Gaußian elimination}
\subsection{The results}
\begin{thm}
For every matrix $A$, there is some matrix $B$ such that $B$ is REF and $A\sim B$.
\end{thm}

\begin{lemma}\label{lemma:1:1:2}
Let $(A|b)$ be an REF matrix. Then $\rank{A|b}\geq\rank{A}$.
\end{lemma}

\begin{proposition}\label{proposition:1:1:3}
Let $(A|b)$ be an REF matrix representing a system of linear equations, such that $\rank{A}=\rank{\Ab}$. Then the system of linear equations has at least one solution.
\end{proposition}

\begin{proposition}\label{proposition:1:1:4}
Let $\Ab$ be an REF matrix representing a system of linear equations, such that $\rank{A}<\rank{\Ab}$. Then the system of linear equations has no solutions.
\end{proposition}

\begin{proposition}\label{proposition:1:1:5}
Let$\Ab$ be an REF matrix representing a system of linear equations, such that $\rank{A}<\colctn{A}$. If the system of equations has at least one solution, then it has at least $|\mathbb{F}|$ many solutions.
\end{proposition}

\begin{proposition}\label{proposition:1:1:6}
Let $\Ab$ be an REF matrix representing a system of linear equations, such that $\rank{A}=\colctn{A}$. Then the system of equations has at most one solution.
\end{proposition}

\subsection{summary}\label{summ}
We summarize the results above in \Cref{tab:1} below.

\begin{table}[h]
\begin{tabular}{|c|c||c|c|}
\hline

\multicolumn{2}{|c||}{$\rank{A}=\rank{\Ab}$} & \multicolumn{2}{|c|}{$\rank{A}<\rank{\Ab}$}\\
\hline

$\rank{A}=\colctn{A}$ & $\rank{A}>\colctn{A}$ & $\rank{A}=\colctn{A}$ & $\rank{A}>\colctn{A}$\\
\hline

$\left(\begin{matrix}1&2\\0&4\\0&0\end{matrix}\right|\left.\begin{matrix}3\\5\\0\end{matrix}\right)$
& $\left(\begin{matrix}1&2&3\\0&0&4\\0&0&0\end{matrix}\right|\left.\begin{matrix}5\\6\\0\end{matrix}\right)$

&
$\left(\begin{matrix}1&2\\0&3\\0&0\end{matrix}\right|\left.\begin{matrix}4\\4\\4\end{matrix}\right)$
& $\left(\begin{matrix}1&2&3\\0&2&3\\0&0&0\end{matrix}\right|\left.\begin{matrix}4\\4\\4\end{matrix}\right)$
\\
\hline

\coffeestainC{0.4}{0.6}{-40}{-6cm}{1cm}
One solution & $\geq|\mathbb{F}|$ solutions&\multicolumn{2}{|c|}{No solutions}\\
\hline
\end{tabular}
\caption{summary}\label{tab:1}
\end{table}

By \Cref{rem:0:0:6} and \cref{lemma:1:1:2} the table covers all possible cases. By \Cref{proposition:1:1:3,proposition:1:1:4,proposition:1:1:5,proposition:1:1:6}, we obtain the number of solutions depicted in the \nameref{summ}.
\end{document} 
