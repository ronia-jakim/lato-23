\documentclass{article}

\usepackage{amsmath}
\usepackage{amssymb}
\usepackage{amsthm}
\usepackage{hyperref}
\usepackage{cleveref}


\title{Linear algebra stuff}
\author{Weronika Jakimowicz}

\newtheorem{proposition}{Proposition}[section]
\newtheorem{df}[proposition]{Definition}
\newtheorem{fct}[proposition]{Fact}
\newtheorem{tm}[proposition]{Theorem}

\newcommand{\tbt}[1]{\textbf{\emph{#1}}}

\begin{document}
\maketitle
\begin{abstract}
We do cool stuff in linear algebra
\end{abstract}

\section{Matrices and linear equations}
\subsection{Elementary definitions and results}
Let $A=[a_{i,j}]$ where $a_{i,j}$ are the entries of $A$. The matrix $A$ can be explicitly written as:
$$
A=\begin{bmatrix}a_{1,1}&a_{1,2}&\dots&a_{1,n}\\
a_{2,1}&a_{2,2}&\dots&a_{2,n}\\
\vdots&\vdots&\ddots&\vdots\\
a_{n,1}&a_{n,2}&\dots&a_{n,n}
\end{bmatrix}
$$
and consider the following system of linear equations:
%\begin{equation}
%\label{eq:1}
%\begin{cases}
%    a_{1,1}x_1\;+\;a_{1,2}x_2\;+\;\dots\;+\;a_{1,n}x_n&=\;0\\
%    a_{2,1}x_1\;+\;a_{2,2}x_2\;+\;\dots\;+\;a_{2,n}x_n&=\;0\\
%    \quad\vdots\\
%    a_{n,1}x_1\;+\;a_{n,2}x_2\;+\;\dots\;+\;a_{n,n}x_n&=\;0
%\end{cases}
%\end{equation}

\begin{equation}\label{eq:1}
\left\{\begin{matrix}
a_{1,1}x_1&+&a_{1,2}x_2&+&\dots&+&a_{1,n}x_n&=&0\\
a_{2,1}x_1&+&a_{2,2}x_2&+&\dots&+&a_{2,n}x_n&=&0\\
\vdots &\\
a_{n,1}x_1&+&a_{n,2}x_2&+&\dots&+&a_{n,n}x_n&=&0
\end{matrix}\right.
\end{equation}

\begin{proposition}\label{prop:1}
The equations in \Cref{eq:1} have a unique solution $\iff$ $A$ is row equivalent to the identity matrix $I_n$.
\end{proposition}

\section{Invertible matrices}
\begin{df} 
An $n\times n$ matrix $B$ is invertible if there is some matrix $C$ such that $B\cdot C=I$, where $I$ is the identity matrix.
\end{df}

\begin{fct}\label{fact:2}
Let $A$ be an $n\times n$ matrix. Then the following statements are equivalent:
\begin{enumerate}
    \item $A$ is invertible
    \item $A$ is row equivalent to the identity matrix $I_n$
\end{enumerate}
\end{fct}

\section{Conclusion}
\begin{tm}
$A$ is row equivalent to the identity matrix $I_n$  $\iff$ The equations in \ref{eq:1} have a unique solution.
\end{tm}
\begin{proof}
    The theorem follows from \Cref{prop:1} and \cref{fact:2}.
\end{proof}

\tableofcontents

\Huge
\begin{flushleft}\tbt{That's}\end{flushleft}
\begin{center}\tbt{all,}\end{center}
\begin{flushright}\tbt{folks!}\end{flushright}





\end{document}
