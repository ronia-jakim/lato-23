\documentclass{article}

% ustawiamy geometrię strony, tj. jakiej wielkości są marginesy etc.
%\usepackage[showframe]{geometry} % rysuje ramkę na marginesy
\usepackage{geometry}
\geometry{a4paper, total={190mm, 277mm}}

% dla polskich znaków
\usepackage[utf8]{inputenc}
\usepackage[T1]{fontenc}

% żeby przerzucało troszkę ładniej do nowej linijki
\usepackage{microtype}

% takie basic paczuszki
\usepackage{amsthm} % do twierdzeń
\usepackage{thmtools} % też do twierdzeń
\usepackage{amsmath}
\usepackage{amssymb}

% to takie moje magiczne rzeczy, pozwala dodawać troszkę logiki do poleceń i środowisk
\usepackage{xparse}

% do tworzenia kolumn
\usepackage{multicol}

% żeby zmieniać styl enumerate
\usepackage{enumitem}

\newtheoremstyle{zadankoStyle}
{2pt}
{2pt}
{}
{}
{\bfseries}
{.}
{ }
{\thmname{#1}\thmnumber{ #2}\thmnote{}}

\declaretheorem[title={}, style=zadankoStyle]{cw}

\NewDocumentEnvironment{podpunkty}{ o }
{%
  \IfNoValueTF{#1}
  {} % nie chcemy wielu kolumn
  {  % chcemy podpunkty w kolumnach
    \begin{multicols}{#1}
  }
  \begin{enumerate}[label=\alph*), leftmargin=12mm]
}
{%
  \end{enumerate}
  \IfNoValueTF{#1}
  {}
  {%
    \end{multicols}
  }
}

\NewDocumentEnvironment{zadanko}{ o }{%
    \IfNoValueTF{#1}
    {\begin{cw}} % kontynuujemy numerację
    {%\renewcommand{\thecw}{#1}\begin{cw} % zmieniamy numerację
      \setcounter{cw}{#1}\begin{cw}
    }
}
    {\end{cw}}

\title{Zagadnienia 1, 2, 3}
\author{}
\date{}

\begin{document}
\maketitle

\begin{zadanko}
  Wskaż liczbę, która nie jest wielokrotnością $3$:
  
  \begin{podpunkty}[4]
    \item 30
    \item 452
    \item 99012
    \item 27
  \end{podpunkty}
\end{zadanko}

\begin{zadanko}
  Wskaż liczbę, która jest wielokrotnością $9$:
  
  \begin{podpunkty}[2]
    \item 2727
    \item 330
    \item 17
    \item 77
  \end{podpunkty}
\end{zadanko}

\begin{zadanko}[50]
  Wybierz wszystkie liczby podzielne przez $2$ lub $5$:

  \begin{podpunkty}[3]
    \item 23456
    \item 230
    \item 45
    \item 123
    \item 985
  \end{podpunkty}
\end{zadanko}

\begin{zadanko}
  Podaj przykład liczby całkowitej, ale nie naturalnej.
\end{zadanko}

\begin{zadanko}
  Wskaż wszystkie liczby pierwsze spośród podanych:
  
  \begin{podpunkty}[3]
  \item 13
  \item 50
  \item 23
  \item 49
  \item 34
  \item 17
  \end{podpunkty}
\end{zadanko}

\begin{zadanko}
  Wyznacz wszystkie dzielniki liczb:
  
  \begin{podpunkty}
  \item 33
  \item 24
  \end{podpunkty}
\end{zadanko}

\begin{zadanko}
  Rozłóż na czynniki pierwsze:
  
  \begin{podpunkty}
  \item 12
  \item 99
  \end{podpunkty}
\end{zadanko}

\end{document}
