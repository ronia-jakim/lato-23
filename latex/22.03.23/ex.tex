\documentclass{article}

\usepackage{amsmath}
\usepackage{amssymb}
\usepackage{graphicx}

\usepackage{amsthm}
\usepackage{thmtools}

\newtheoremstyle{qqStyle}
    {\topskip}
    {\topskip}
    {}
    {}
    {\bfseries}
    {.}
    {3pt}
    {}


\newtheoremstyle{itStyle}
    {\topskip}
    {\topskip}
    {\slshape}
    {}
    {\bfseries}
    {.}
    {3pt}
    {}

\theoremstyle{qqStyle}

\graphicspath{{./}}

\title{Functions}
\author{Weronika Jakimowicz}

\newcommand{\R}{\mathbb{R}}
\newcommand{\Q}{\mathbb{Q}}

\declaretheorem[title={Question}]{qq}
%\newtheorem{qq}{Question}
%\newtheorem{th}{Theorem}
\declaretheorem[title={Theorem}, numberlike=qq, style=itStyle]{theorem}
\declaretheorem[title={Question}, numberlike=qq, style=itStyle]{qq2}

\usepackage{hyperref}
\usepackage{cleveref}

\newcommand{\Y}{YES}

\begin{document}
\maketitle

The modern concept of functions $f:\R\to\R$ has a surprisingly complicated history and grew mostly out of an attempt to solve various problems arising from earlier concepts.

One such problem is the following related to differentiability:

\begin{qq}\label{q1}
Are all functions $f:\R\to\R$ differentiable?
\end{qq}

The answer here is obviously no, as seen by the following example:
$$
f(x):=|x|:=\Bigg\{\begin{array}{ll}&x+1\text{, if }x\leq 0\\&x\text{, if }x<0\end{array}
$$

Obviously, the only problem here is the point $x=0$. So one might ask the following question:
\begin{qq}\label{q2}
Are all functions $f$ differentiable at all but finitely many points?
\end{qq}

Again, the answer is no, as seen by the following example:
$$
f(x):=\chi_\Q(x):=\left\{\begin{array}{ll}&1,\text{ if }x\in\Q\\&0,\text{ if }x\notin\Q\end{array}\right.
$$

Clearly, this function is discountinuous at all points though and hence one can ask the following even more precise question:

\begin{qq2}\label{q3}
Are all continuous functions differentiable everywhere but a countable number of points?
\end{qq2}

This however is also wrong:

\begin{theorem}\label{thm4}
There are functions $f:\R\to\R$ which are everywhere continuous but nowhere differentiable.
\end{theorem}

Examples like the one in \Cref{thm4} were found by Karl Weierstraß pictured in \Cref{fig1} contemplating how to destroy $19$th century real Analysis.

\begin{figure}[!h]
\begin{center}
\includegraphics[scale=4]{./kotecek.png}
\caption{An utter madman!}\label{fig1}
\end{center}
\end{figure}

\begin{equation}
\begin{array}{|| c | c | c ||}
\hline \hline
f(x):= & f'(x) & \int f(t)dt\\
\hline

x & 1 & \frac{x^2}{2}\\
\hline
e^x & e^x & e^x\\
\hline
\sin(x) & \cos(x) & -\cos(x)\\
\hline
x^{-1} & -x^{-2} & \log(x)\\
\hline \hline
\end{array}\label{tab1}
\end{equation}
\medskip

We want to note differentials and integral of some well-known functions in \labelcref{tab1}. Summarising some of the examples above we obtain in \Cref{tab2} below. One can reasonably ask what the issue with the integrability of $\chi_\Q$ is. We shall return to this later.

\begin{table}[h]
\centering
\begin{tabular}{|c||c|c|c|}
\hline
$f(x):=$ & continuous & differentiable & integrable\\
\hline \hline
$x$ & \Y&\Y&\Y\\
\hline
$|x|$&\Y&Everywhere but $0$&\Y\\
\hline
$\chi_\Q(x)$&\multicolumn{2}{|c|}{Nowhere continuous or differentiable}&Ehhh...\\
\hline

Weierstraß functions&\Y&NO&\Y\\
\hline
\end{tabular}

\caption{This whole mess!}\label{tab2}
\end{table}

(Remark: Question 3 must not be italicised by using a font-environment like \emph{emph}.)

\end{document}
