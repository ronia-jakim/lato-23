\documentclass{article}

\usepackage{amsmath}
\usepackage{amssymb}
\usepackage{amsthm}

\usepackage[backend=bibtex]{biblatex}
\addbibresource{./bibb.bib}
\renewbibmacro{in:}{}

\usepackage{hyperref}
\usepackage{cleveref}


\title{Finite groups}
\author{Weronika Jakimowicz}

\newtheorem{theorem}{Theorem}[section]

\newcommand{\Z}{\mathbb{Z}}

\begin{document}
\maketitle

\section{Introduction}
Groups are a common way to formalize the notion of symmetries in mathematics. Finite groups in turn are simply groups with finitely many elements. Unsurprisingly, one of the most important invariants of a finite group $G$ is its cardinality $|G|$. For example, one has

\begin{theorem}\label{theorem:1.1}
    Let $G$ be a group whose cardinality $|G|$ is a square of a prime number $p$. Then $G$ is isomorphic to $\Z_p\times\Z_p$ or $\Z_{p^2}$.
\end{theorem}

More specifically, one also has

\begin{theorem}\label{theorem:1.2}
    Let $G$ be a group whose cardinality $|G|$ is a prime number $p$. Then $G$ is isomorphic to $\Z_p$.
\end{theorem}

Both of those are consequences of

\begin{theorem}\cite[Proposition 2.2]{Lang}
    Let $G$ be a finite group and $H$ a subgroup of $G$. Then $|H|$ divides $|G|$.
\end{theorem}

\section{Sylow Theorems}
A common tool to study finite groups is the use of so-called Sylow-subgroups.
\begin{theorem}\cite[Thereom 6.4]{Lang}
Let $G$ be a finite group, $p$ a prime number and $p^{M_p(G)}$ the maximal power of $p$ dividing $G$. A subgroup $S$ of $G$ with $S=p^{M_p(G)}$ is called a $p$-Sylow subgroup of $G$. Let $n_p(G)$ be the number of $p$-Sylow-subgroup of $G$. Then
$$n_p(G)\equiv1\mod p^{M_p(G)}\text{ and }n_p(G)\text{ divides }\frac{|G|}{p^{M_7(G)}}$$
holds. Further, all $p$-Sylow-subgroups of $G$ are conjugates in $G$.
\end{theorem}

This can often be used to determine groups explicitly only with the information of $|G|$. For example, one has the following application. Let$G$ be a finite group with $|G|=35$. Then
$$n_7(G)\equiv 1\mod7^{M_6(G)}=7\text{ and }n_7(G)\text{ divides }\frac{|G|}{7^{M_7(G)}}=5.$$
But this implies that $n_7(G)=1$ and so there is a unique $7$-Sylow-sbgrou $S_7$ of $G$ and so $S_7$ is a normal subgroup of $G$. But then $S_7=\Z_7$ and $G/S_7=\Z_5$ and hence $G$ is solvable. Of course, one could also use
$$n_5(G)\equiv 1\mod5^{M_5(G)}=5\text{ and }n_5(G)\text{ divides }\frac{|G|}{5^{M_5(G)}}=7$$
instead. In fact, one has

\begin{theorem}
Let $G$ be a subgroup whose cardinality $|G|$ is smaller than $60$. Then $G$ is solvable..
\end{theorem}




\printbibliography[title=Book on Algebra]
\end{document}
