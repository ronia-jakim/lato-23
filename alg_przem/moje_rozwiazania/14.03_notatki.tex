\documentclass{article}

\usepackage{../../notatki}

\title{Algebra przemienna\medskip\\{\normalsize notatki}}
\author{}

\begin{document}
\maketitle

\dyg{Dygresja, która ma się przydać na zajęciach 15.03}

Jak mamy pierścień $R$ i na nim mamy $Spec(R)$, to bierzemy element $f\in R$
$$\hat{f}(\pideal):=[f]_\pideal\in R/\pideal$$
i to ma przypominać wiązkę wektorową. Ewaluacja funkcji w punkcie.

Można też powiedzieć, że dla dowolnego ideału $\mathfrak{a}$ mamy  $f(\mathfrak{a})=[f]_\mathfrak{a}\in R/\mathfrak{a}$, ale ten ideał $\mathfrak{a}$ daje $V(\mathfrak{a})$ i to wygląda tak, jakbyśmy jaką funkcję na spektrum chcieli obciąć do $V(\mathfrak{a})$.
\medskip

Jeśli $M$ jest $R$ modułem, to konstrukcję wyżej możemy powtórzyć. To znaczy jeśli $\pideal$ jest ideałem pierwszym i jeśli weźmiemy 
$$M\otimes_R R/\pideal$$
to to jest naturalny $R/\pideal$ moduł.

\begin{illustration}
\draw(0,0)--(7,0);
\filldraw (5, 0) circle (2pt) node [below] {$\pideal$};
\draw (5.3, 0.3)--(5.3, 4) node [right] {$R/\pideal$};
\draw (4.7, 2)--(4.7, 4) node [left] {$M/\pideal$};
\filldraw (3, 0) circle (2pt);
\filldraw (1, 0) circle (2pt);
\filldraw (6, 0) circle (2pt);
\node at (7, -0.5) {$R$};
\end{illustration}
\medskip

\podz{dark-green}
\medskip

Ponoć nie będą nam Boole potrzebne i Januszkiewicz uważa, że to jest szczególnie nudne.

Niech $X$ będzie przestrzenią zwartą. Weźmy zbiór wszystkich ciągłych funkcji z $X$ w pierścień $\Z_2$ [$F(X, \Z_2$]. Czy są fajne własności? $f\circ f=f$ i $f+f=0$, czyli to jest pierścień Boole'a.

Jakie ten pierścień ma ideały? Wszystkie ideały pierwsze są maksymalne (to jakieś zadanie było).
Jeśli weźmiemy spójny podzbiór $X$, to wtedy $F(X, \Z_2)$ musi być stały w pewnym miejscu. Czyli $Spec(F)$ to zbiór składowych spójnych $X$.

Weźmy wszystkie funkcje ciągłe w $\Z_2$ i zamieńmy te funkcje na jedną, solidną funkcję w $\Z_2^R$. Taka funkcja to nie jest włożenie, ale idzie na $Im(X)\subseteq \Z_2^R$
\begin{illustration}
\node (1) at (0, 0) {X};
\node (2) at (2, 0) {$\Z_2$};
\node (3) at (4, 0) {$X$};
\node (4) at (6, 0) {$\Z_2^R$};
\node (5)  at (6, -2) {$Im(X)$};
\draw[->] (1)--(2);
\draw[->](0.25, 0.3)--(1.7, 0.3) node [midway, above] {$\psi_1$};
\draw[->](0.25, -0.3)--(1.7, -0.3) node [midway, below] {$\psi_2$};
%\draw[dashed, ->] (2)--(3);
\draw[->] (3)--(4) node [midway, above] {$\phi$};
\draw[->] (3)--(5);
\node[rotate=90] at (6, -1) {$\subseteq$};

\node at (10, -2) {$F(X,\Z_2)=F(Im\phi,\Z_2)$};
\node at (5, -4) {$\phi(x)=(\psi_1(x),\psi_2(x)$};
\end{illustration}

Czyli $Spec(F(X, \Z_2))=Im\phi$.

\subsection*{ZADANIE 23}
(i)

Otwarte są z definicji, a dla domkniętości mówimy, że $V(f)\sqcup V(1-f)=X$.

(ii)

\begin{align*}
X_{f_1}\cup...\cup X_{f_n}&=\left(\bigcap V(f_i)\right)^c=V(f_1,...,f_n)^c=V(f)^c=X_f
\end{align*}

(iii)

Wskazówka to rozwiązuje.

(iv)

Kwasi-zwartość mamy, bo każde spektrum takie jest, więc pozostaje Hausdorff.

Bierzemy dwa punkty ze spektrum $p_1,p_2\in Spec(A)$. Ideały pierwsze są maksymalne, czyli mamy $f\in p_1\setminus p_2$ i na odwrót, czyli $g\in p_2\setminus p_1$. Weźmy 
$$p_1\in V(f)=\{\pideal\;:\;f\in\pideal\}\not\ni p_2.$$ 
I to jest zbiór otwarto domknięty, czyli jego dopełnienie jest otwartym otoczeniem $p_2$.

\subsection*{ZADANIE 24}
80$\%$ tego zadania to definicje, a potem jest część tego zadania gdzie przestajemy definiować i zaczynamy to robić.

\subsection*{ZADANIE 25.}
Że bijekcja $f\mapsto V(f)$ zachowuje operacje mnogościowe $\cup,\cap$.

Generalnie zadanie jest nudne, ale fajnym wyzwaniem jest rozróżnic algebraicznie $(\frac1n,0)$ i $(0, \frac1n+\frac1m)$.

\subsection*{ZADANIE 26}
Jest zrobione w tych zadaniach i cała zabawa została nam zabrana.

\subsection*{ZADANIE 27}
Prawdopodobnie chodzi o to, że jeden zbiór może mieć bardzo dużo definiujących równań.

Jak mam wielomiany $(f_1,...,f_k)$ $f\in K[x_1,...,x_m]$. Zbiór $Zer(f_1,...,f_k)$ to jest zbiór tych punktów $K^m$ takich, że $f$ ewaluwane w nich daje $0$. To daje ideał $I(Zer)$, czyli wszystkie funkcje $g$ takie, że $g(z)=0$ dla $z\in Zer$ (czyli to co na algebrze 2R).

I to nawet chyba jest tak, że $I(Zer)\supseteq \sqrt{\begin{pair}f_1,...,f_k\end{pair}}$. Tutaj jest równość i to jest kwestia geometrii algebraicznej.

$P(X)$ to jest jedna z dwoch rzeczy:
\begin{enumerate}
    \item $k[x_1,...,x_m]/\begin{pair}f_1,...,f_k\end{pair}$
    \item $k[x_1,...,x_m]/I(Zer)$
\end{enumerate}

$P(X)$ to funkcje wielomianowe na $Zer$. $(1)=(2)$.

Jest jeszcze mała uwaga, że klasy abstrakcji $[x_1],[x_2],...,[x_m]$ generują $P(X)$ jak $k$-algebrę.

\dyg{Co się stanie, jeśli weźmiemy $\mathfrak{m}\subseteq P(X)$?} Jeśli mamy $x\in Zer$, to mamy $\mathfrak{m}_x<P(X)$. Jeśli weźmiemy punkt, w którym wszystkie $f$ znikają, to dostajemy homomorfizm ewaluacji, którego jądro to $ker(ev_x)$.

\acc{Oni mówią, że odwzorowanie $Zer\ni x\mapsto \mathfrak{m}_x$ jest iniekcją.} Jeśli $x=[a_1,...,a_m]$ i $y=[b_1,...,b_m]$ i one są różne na którejś współrzędnej, to chcemy znaleźć funkcję, która w $x$ znika, a w $y$ nie znika i odwrotnie.

\acc{Suriektywnośc tego odworowania} natomiast jest tym samym co równość $I(Zer)=\sqrt{\begin{pair}f_1,...,f_k\end{pair}}$. To jest etap, na którym potrzebujemy algebraicznej domkniętości $k$.

\subsection*{ZADANIE 28}
Czytanie ze zrozumieniem.

\deff{Praca domowa: przegooglować Jacobian Conjecture}.

Wiemy co to są odwzorowania $k^m\to k^n$. Jeżeli $n$ to jest $1$, to są wielomiany.
Teraz chcemy o $k^m\supseteq V\to W\subseteq k^n$ i taki odwzorowanie jest affiniczne, jeśli znajdę fajne $k^m\to k^n$. Jeśli funkcja znika na $W$, to chcę, żeby ona po przeprowadzeniu z powrotem znikała na $V$. Te strzałki $k^m\to k^n$ to $n$ wielomianów $m$ zmiennych.

I w sumie to tutaj zgubiłam uwagę.


\section*{COŚ O FUNKTORACH POCHODNYCH}

Mamy moduł $M$ i możemy założyć, że jest on skończenie generowany i jest on Noetherowski. Jeśli tak jest, to on ma rezorwenta, a ten rezorwenta ma jadro:
\begin{tikzcd}
0 & M\arrow[l] & R^{k_1}\arrow[l] & R^{k_2}\arrow[l] & R^{k_3}\arrow[l] &...\arrow[l] & \text{<- dokładna rezolwenta wolna}
\end{tikzcd}

Powiedzmy, że mamy jakiś funktor, to wtedy

\begin{tikzcd}
0 & F(M)\arrow[l] & F(R^{k_1})\arrow[l] & F(R^{k_2})\arrow[l] &...\arrow[l]& \text{<- kompleks zależny od rezolwenty}
\end{tikzcd}

Coś, co nie zależy od rezolwenty to jest funktor pochodny i to, czy to są homologie czy nie to nie jest ważne, ważne żeby było rezolwenty w to nie involve.

Popatrzmy na konretne funktory. Weźmy sobie $k$-moduł $M$, na który działa grupa $G$. Innymi słowy, mamy $G\to Aut_k(M)$, czyli $k[G]$.
Jak już coś takiego mamy, to możemy przypisać
$M\xrightarrow[]{\phi} M_G$, gdzie $M_G$ to jest największy taki obiekt, że $G$ działa na niego trywialnie. i $\phi(gm)=g\phi(m)=\phi(m)$

Szukamy jądra w $M$, które przez $\phi$ przejdzie na zero: $m-gm=0$ dla wszystkich $g$. Czyli bierzemy podmoduł $(m-gm)$ i koniec:
$$M_G=M/\begin{pair}m-mg\end{pair}$$
Ta konstrukcja nazywa się \deff{koniezmiennikami}.

Jest też inna konstrukcja, która nazywa się niezmiennikami. Bierzemy $M$ i szukamy modułu 
$$M^G\xrightarrow[]{\phi}M$$ 
takiego, że $G$ działa trywialnie. 
$$M^G=(Im\phi)=\{m\;:\;gm=m\}$$
Czyli tutaj mamy niezmienniki i to jest istotnie dualne do koniezmienników.
\medskip

Teraz chcemy zrobić funktory pochodne, pytanie jak?

Chcemy wybudować rezolwentę w kategorii $k$-$G$-modułów (jak to się pisze xd)
$$H^0*G, M)=M^G$$
$$H_0(G, M)=M_G$$
i pierwszy funktor pochodny
$$H'(G, M)$$
homologie $G$ o współczynnikach w $M$.

$H^0(G, M)$ jest dwufunktorialny, czyli z jednej strony mówi coś o $G$, a z drugiej mówi coś o $M$.


$N$, $M$ sa skończenie generowane

\begin{tikzcd}
N\arrow[d, "f" right] & R^N\arrow[l]\arrow[d] & ker\arrow[l]\arrow[d,"no\; tak"]\\
M & R^n\oplus R^m\arrow[l]&ker\arrow[l]
\end{tikzcd}
\begin{center}$\star\star\star\star$ I ŻYLI DŁUGO I SZCZEŚLIWIE I WSZYSTKO IM KOMUTOWAŁO $\star\star\star\star$\end{center}








\end{document}
