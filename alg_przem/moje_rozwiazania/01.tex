\documentclass{article}

\usepackage{../../notatki}

\begin{document}
\subsection*{ZADANIE 18}
\emph{\color{pink}Z psychologicznych powodów, czasem wygodniej jest oznaczać ideał pierwszy pierścienia $A$ literami $x,y$ jeśli myślimy o nich jako o punkcie $X=Spec(A)$. Kiedy myślimy o $x$ jako o ideale pierwszym $A$, oznaczamy go przez $\mathfrak{p}_x$ (oczywiście jest to ta sama rzecz). Pokaż, że}

\emph{\color{pink}(i) zbiór $\{x\}$ jest domknięty w $Spec(A)\iff\mathfrak{p}_x$ jest maksymalny}
\medskip

$\impliedby$

Jeśli $\mathfrak{p}_x$ jest ideałem maksymalnym, to $\{x\}=V(\mathfrak{p}_x)$, gdyż żaden inny ideał pierwszy nie zawiera $\mathfrak{p}_x$. My definiowaliśmy $V(E)$ jako zbiory domknięte, więc $\{x\}$ też taki jest.

$\implies$

Wiem, że $\{x\}$ jest zbiorem domkniętym. Czyli jest przekrojem pewnej rodziny domkniętych zbiorów bazowych
$$\{x\}=\bigcap\limits_{i\in I}V(E_i)=V\Big(\bigcup\limits_{i\in I}(E_i)\Big)$$
ale jeśli taka suma zawiera się w jednym, jedynym ideale pierwszym, to jest on maksymalny.
\medskip

\emph{\color{pink}(ii) $\overline{\{x\}}=V(\mathfrak{p}_x)$}

$\subseteq$

Jest raczej prostym zawieraniem: $\overline{\{x\}}$ jest najmniejszym zbiorem domkniętym zawierającym $\{x\}$, a $V(\mathfrak{p}_x)$ z pewnością to spełnia.

$\supseteq$

Po pierwsze zauważmy, że
$$V(\pideal_x)=\bigcap\limits_{E\subseteq \pideal_x}V(E) = V\Big(\bigcup\limits_{E\subseteq\pideal_x}V(E)\Big),$$
bo to są wszystkie te ideały pierwsze, które zawierają jakiś podzbiór $\pideal_x$, czyli obcinamy te mniejsze podzbiory $\pideal_x$ w trakcie brania przekroju.

Wiemy, że $\bigcap\limits_{E\subseteq \pideal_x}V(E)$ jest zbiorem domkniętym. Wiemy, że $x\in\bigcap V(E)$, czyli dostajemy, że $V(\pideal_x)$ jest przekrojem wszystkich zbiorów domkniętych zawierających $x$, czyli jest najmniejszym zbiorem domkniętym zawierającym $x$, czyli domknięciem $x$.
\medskip

\emph{\color{pink}(iii) $y\in\{\overline x\}\iff\mathfrak{p}_x\subseteq\mathfrak{p}_y$}

$\impliedby$

Niech $x,y\in X$ takie, że $\mathfrak{p}_x\subseteq\mathfrak{p}_y$. Wówczas, $x\in V(E)\implies y\in V(E)$. Ponieważ $\overline{\{x\}}$ jest przekrojem zbiorów $V(E_i)$, który zawiera $x$, to w szczególności każdy z tych zbiorów zawiera również $y$, stąd $y\in\overline{\{x\}}$.

$\implies$

Trywialne z \emph{(ii)}.
\medskip

\emph{\color{pink}(iv) $X$ jest $T_0$-przestrzenią (jeśli $x,y$ są rozróżnialnymi punktami $X$, to albo istnieje otoczenie $x$ które nie zawiera $y$, albo istnieje otocznie $y$, które nie zawiera $x$).}

Weźmy dowolne punkty $x,y\in X$. Rozważmy dwa przypadki:

1. $\pideal_x\subseteq\pideal_y$ (lub $\pideal_y\subseteq\pideal_x$, ale WLOG pierwsza wersja)

Wtedy $x\in X\setminus V(\pideal_y)$, które jest zbiorem otwartym takim, że $y\notin X\setminus V(\pideal_y)$.

2. $\pideal_x\not\subseteq\pideal_y$ i $\pideal_y\not\subseteq\pideal_x$

Wtedy $y\notin\overline{\{x\} }$ i $x\notin\overline{\{y\} }$. Czyli $y\in X\setminus\{x\}$ jest otwartym zbiorem zawierającym $y$ ale niezawierającym $x$.

\subsection*{ZADANIE 19.}
\emph{\color{blue}Przestrzeń topologiczna $X$ jest nieredukowalna, jeśli $X\neq\emptyset$ i jeśli każda para niepustych otwartych podzbiorów $X$ się przecina (równoważnie, każdy niepusty podzbiór otwarty jest gęsty w $X$). Pokaż, że $Spec(A)$ jest nieredukowalny $\iff$ nilradykał $A$ jest ideałem pierwszym.}

$\implies$

Chcę mieć punkt, którego dopełnienie jest wszystkim. 

$\impliedby$

Niech $r$ będzie nilradykałem pierścienia $A$ i niech $r$ będzie ideałem pierwszym. Weźmy dowolne $\mathfrak{a}_1,\mathfrak{a}_2\normalsubgroup A$ i rozpatrzmy $U_1=V(\mathfrak{a}_1)^c,U_2=V(\mathfrak{a}_2)^c$.

\subsection*{ZADANIE 20}
\emph{\color{pink}Niech $X$ będzie przestrzenią topologiczną}

\emph{\color{pink}(i) Jeśli $Y$ jest nieredukowalną podprzestrzenią $X$, wtedy domknięcie $\overline Y$ w $X$ jest nieredukowalne.}

Załóżmy nie wprost, że $\overline Y$ nie jest nieredukowalna. Wtedy istnieją $U, V\subseteq \overline Y$ takie, że $U\cap V=\emptyset$. Ale zbiór $U\cap Y$ jest albo pusty albo jest zbiorem otwartym w $Y$. Tak samo dla $V\cap Y$. To znaczy, że $(U\cap Y)\cap(V\cap Y)=\emptyset$ co jest sprzeczne z nieredukowalnością $Y$.

\emph{\color{pink}(ii) Każda nieredukowalna podprzestrzeń $X$ jest zawarta w pewnej nieredukowalnej podprzestrzeni $X$.}

Niech $S$ będzie zbiorem nieredukowalnych podprzestrzeni $X$. Rozważmy łańcuch
$$Y_1\subseteq Y_2\subseteq...$$
podprzestrzeni z $S$. Niech $Y=\bigcup Y_i$. Musimy pokazać, że $Y\in S$, czyli $Y$ jest nieredukowalny.

Niech $U,V\subseteq Y$. Wtedy istnieje $i$ takie, że $U,V\subseteq Y_i$. Ponieważ $Y_i$ jest nieredukowalna, to $U\cap V\neq\emptyset$. W takim razie każde dwa zbiory otwarte z $Y$ tną się niepusto, a więc $Y$ jest nieredukowalne.

Wystarczy użyć lematu Zorna dla zbioru $S_A=\{Y\subseteq X\;:\;Y\text{ nieredukowalna i }A\subseteq Y\}$.

\emph{\color{pink}(iii) Maksymalne nieredukowalne podprzestrzenie $X$ są domknięte i pokrywają $X$. Nazywamy je składowe nieredukowalne $X$. Jakie są składowe nieredukowalne przestrzeni Hausdorffa?}

Niech $M\subseteq X$ będzie maksymalną podprzestrzenią nieredukowalną $X$. Domkniętość $M$ wynika wprost z (ii). Gdyby $M$ nie było domknięte, to $M\subsetneq\overline M$, a $\overline M$ też jest nieredukowalne i mamy sprzeczność z maksymalnością $M$.

Dlaczego pokrywają? Bo dla każdego $\{x\}$, $x\in X$ możemy rozpatrzeć zbiór wszystkich nieredukowalnych zbiorów takich, że $\{x\}\subseteq A$ i w ten sposób znajdziemy maksymalne zbiory nieredukowalne zawierające każdy element $X$, czyli pokrywające $X$.

W Hausdorffie możemy każde dwa punkty oddzielić dwoma rozłącznymi otwartymi otoczeniami, więc maksymalne nieredukowalne podprzestrzenie to singletony. I to właśnie przypadek, który mnie natchnął do wytłumaczenia jak pokryć $X$.

\emph{\color{yellow}(iv) Jeśli $A$ jest pierścieniem i $X=Spec(A)$, wtedy składowe nieredukowalne $X$ to zbiory domknięte $V(\pideal)$, gdzie $\pideal$ to najmniejsza zbiory pierwsze $A$.}

\subsection*{ZADANIE 21}
\emph{\color{pink}Niech $\phi:A\to B$ będzie homomorfizmem pierścieni. Niech $X=Spec(A)$ i $Y=Spec(B)$. Jeśli $\mathfrak{q}\in Y$, to $\phi^{-1}(\mathfrak{q})$ jest ideałem pierwszym w $A$, w szczególności punktem $X$. Z tego powodu $\phi$ indukuje przekształcenie $\phi^*:Y\to X$. Pokaż, że}

\emph{\color{pink}(i) Jeśli $f\in A$, wtedy $\phi^{*-1}(X_f)=Y_{\phi(f)}$ i dlatego $\phi^*$ jest ciągłe.}
\begin{align*}
    \phi^{*-1}(X_f)&=\{y\in Y\;:\;(f)\subsetneq\phi^*(y)\}=\{y\in Y\;:\;(f)\subsetneq\phi^{-1}(y)\}=\{y\in Y\;:\;\phi(f)\subseteq y\}=Y_{\phi(f)}
\end{align*}

\emph{\color{pink}(ii) Jeśli $\mathfrak{a}$ jest ideałem $A$, wtedy $\phi^{*-1}(V(\mathfrak{a}))=V(\mathfrak{a^e})$ }

$\mathfrak{a}^e$ to rozszerzenie obrazy $\phi(\mathfrak{a})$ do ideału w $B$.
\begin{align*}
    \phi^{*-1}(V(\mathfrak{a}))=\{y\in Y\;:\;\mathfrak{a}\subseteq \phi^*(y)\}=\{y\in Y\;:\;\mathfrak{a}\subseteq\phi^{-1}(y)\}=\{y\in Y\;:\;\phi(\mathfrak{a})\subseteq y\}=V(\phi(\mathfrak{a}))=V(\mathfrak{a}^e)
\end{align*}
Ta ostatnia równość z jakiegoś poprzedniego zadanka, bo wtedy mam, że $V(E)=V((E))$ i to jest to samo.

\emph{\color{pink}(iii) Jeśli $\mathfrak{b}$ jest ideałem $B$, wtedy $\overline{\phi^*(V(\mathfrak{b}))}=V(\mathfrak{b}^c)$}
\begin{align*}
    \phi^*(V(\mathfrak{b}))&=\{\phi^{-1}(y)\;:\;\mathfrak{b}\subseteq y\in Y\}=\{x\in X\;:\;\phi(x)=y\supseteq \mathfrak{b}\}=\\
    &=\{x\in X\;:\;\mathfrak{b}\subseteq\phi(x)\}\overset{\star}{=}\{x\in X\;:\;\phi^{-1}(\mathfrak{b})\subseteq x\}=V(\phi^{-1}(\mathfrak{b}))=V(\mathfrak{b}^c)
\end{align*}

\emph{\color{yellow}(iv) Jeżeli $\phi$ jest suriekcją, wtedy $\phi^*$ jest homeomorfizmem z $Y$ w zamknięty zbiór $V(ker(\phi))$ $X$. (W szczególności $Spec(A)$ i $Spec(A/\mathfrak{R})$, gdzie $\mathfrak{R}$ to nilradykał $A$, są naturalnie homeomorficzne.)}


\emph{\color{yellow}(v) Jeśli $\phi$ jest 1-1, to $\phi^*(Y)$ jest gęste w $X$. Dokładniej, $\phi^*(Y)$ jest gęste w $X$ $\iff ker(\phi)\subseteq\mathfrak{R}$}

$\implies$

$$Y=V(0)=V(\phi(ker(\phi)))$$
Z gęstości $Y$ mam:
$$V(0)=V(\mathfrak{R})=X=\overline{\phi^*(Y)}=\overline{\phi^*(V(\phi(ker(\phi))))}=V(\phi^{-1}(\phi(ker(\phi))))=V(ker(\phi))$$
A ponieważ $\mathfrak{R}$ jest ideałem pierwszym, to $\ker(\phi)\subseteq\mathfrak{R}$

$\impliedby$

Wiem, że $ker(\phi)\subseteq\mathfrak{R}$. No to wtedy
$$V((ker\phi))=X$$
No to wtedy
$$V(\phi(ker(\phi)))=V((0))=Y$$
\begin{align*}
    \phi^*(Y)&=\phi^*(V(\phi(ker(\phi))))=\{\phi^{-1}(y)\;:\;\phi(ker(\phi))\subseteq y\in Y\}=\\
    &=\{x\in X\;:\;\phi(ker(\phi))\subseteq\phi(x)\}=V(ker(\phi))=X
\end{align*}

\subsection*{ZADANIE 22.}
\emph{\color{yellow}Niech $A=\prod\limits_{i=1}^nA_i$ będzie produktem prostym pierścieni $A_i$. Pokaż, że $Spec(A)$ jest rozłączną sumą otwartych (i zamkniętych) podzbiorów $X_i$, gdzie $X_i$ jest kanonicznie homeomorficzna z $Spec(A_i)$}

To widać. Kanoniczny homeomorfizm to będzie identyczność na jednej współrzędnej i resztę wyrzucamy. 

\emph{\color{pink}Teraz niech $A$ będzie dowolnym pierścieniem. Pokaż, że następujące są równoważne:}

\emph{(i) $X=Spec(A)$ jest niespójna}

\emph{(ii) $A\cong A_1\times A_2$ gdzie żaden z $A_1,A_2$ nie jest pierścieniem zerowym.}

\emph{(iii) $A$ zawiera idempotent $\neq0,1$}
\medskip

$(i)\implies (ii)$

Skoro $X$ jest niespójna, to jest sumą dwóch rozłącznych zbiorów $V(\mathfrak{a}), V(\mathfrak{b})$.  Mam też, że te zbiory są rozłączne, czyli
$$\emptyset=V(\mathfrak{a})\cap V(\mathfrak{b})=V(\mathfrak{a}\cup\mathfrak{b})$$
ale $\mathfrak{a}\cup\mathfrak{b}\subseteq\mathfrak{a}+\mathfrak{b}$, które w tym wypadku wyszłoby $(1)$, bo
$$V(1)=\emptyset=V(\mathfrak{a}\cup\mathfrak{b})\supseteq V(\mathfrak{a}+\mathfrak{b})$$
% Dodatkowo,
% $$V(\mathfrak{a})\cup V(\mathfrak{b})=X$$
% a z zadania 15 mamy, że wtedy
% $$V(\mathfrak{a})\cup V(\mathfrak{b})=V(\mathfrak{a}\cap \mathfrak{b})=V(\mathfrak{a}\mathfrak{b})$$
% czyli $\mathfrak{R}=\mathfrak{a}\mathfrak{b}$.

Czyli mam tutaj $\mathfrak{a},\mathfrak{b}$ wzajemnie pierwsze, czyli $\phi:A\to A/\mathfrak{a}\times A/\mathfrak{b}$ jest izomorfizmem pierścieni.

$(ii)\implies (iii)$

$$x=(1, 0)$$
$$x^2=(1^2, 0^2)$$

$\color{yellow}(iii)\implies(i)$

Jeśli przestrzeń jest niespójna, to $A\cup B=X$, ale jeśli weźmiemy dopełnienie, to mamy 
$$\emptyset=X^c=(A\cup B)^c=A^c\cap B^c$$
$$X=\emptyset^c=(A\cap B)^c=A^c\cap B^c,$$ czyli mogę równoważnie szukać dwóch rozłącznych zbiorów domkniętych sumujących się do całości.

Mamy idempotent, chcemy pokazać, że wtedy da się rozdzielić $A$ na sumę dwóch rozłącznych. Niech $e$ będzie idempotentem i niech $e'=1-e$. Wtedy
$$e'+e=1$$
$$e'e=(1-e)e=e-e^2=0$$
$$X=V(0)=V(e'e)=V(e')\cup V(e)$$
$$\emptyset=V(1)=V(e'+e)$$

\subsection*{ZADANIE 26.}
\emph{\color{pink}Niech $A$ będzie pierścieniem. Podprzestrzeń $Spec(A)$ zawierająca ideały maksymalne z $A$, razem z indukowaną topologią, jest nazywana}\text{\color{pink} maksymalnym spektrum $A$}\emph{\color{pink} i oznaczamy ją $Max(A)$. Dla dowolnego pierścienia przemiennego nie ma to ładnych funktorowych własności (zad. 21) $Spec(A)$, bo odwrotny obraz ideału maksymalnego przez homomorfizm pierścieni nie musi być maksymalny.}

\emph{\color{pink}Niech $X$ będzie zwartą przestrzenią Hausdorffa i niech $C(X)$ oznacza pierścień wszystkich ciągłych liniowych funkcji rzeczywistych z $X$. Dla każdego $x\in X$ niech $\mathfrak{m}_x$ będzie zbiorem wszystkich $f\in C(X)$ takich, że $f(x)=0$. Ideał $\mathfrak{m}_x$ jest maksymalny, ponieważ jest jądrem pewnego suriektywnego homomorfizmu $C(X)\to\R$, który mapuje $f\to f(x)$. Jeśli $\overline X$ będzie oznaczało $Max(C(X))$, mamy wtedy mapę $\mu :X\to\overline X$ $x\mapsto \mathfrak{m}_x$.}

\emph{\color{pink}Pokażemy, że $\mu$ jest homeomorfizmem między $X$ w $\overline X$.}

{\color{yellow}\emph{(i) Niech $\mathfrak{m}$ będzie ideałem maksymalnym $C(X)$ i niech $V =V(\mathfrak{m})$ będzie zbiorem zawierającym wszystkie wspólne zera funkcji z $\mathfrak{m}$}
$$V=\{x\in X\;:\;(\forall\;f\in\mathfrak{m})\;f(x)=0\}$$
\emph{Załóż, że $V$ jest pusty. Wtedy dla każdego $x\in X$ istnieje $f_x\in\mathfrak{m}$ takie, że $f_x(x)\neq 0$. Ponieważ $f_x$ jest ciągłe, istnieje otwarte otoczenie $x\in U_x\subseteq X$ na którym $f_x$ nie znika. Ze zwartości mamy skończoną liczbę takich sąsiedztw, nazwijmy je $U_{x_1},...,U_{x_n}$ pokrywających $X$. Niech}
$$f=f_{x_1}^2+...+f_{x_n}^2$$
\emph{ponieważ $f$ nie znika w żadnym punkcie $X$, $f$ jest odwracalne. Ale to jest sprzeczne z $f\in\mathfrak{m}$, stąd $V$ nie jest pusty.}

\emph{Niech $x$ będzie punktem $V$. Wtedy $\mathfrak{m}\subseteq\mathfrak{m}_x\implies \mathfrak{m}=\mathfrak{m}_x$, bo $\mathfrak{m}$ jest maksymalny. Stąd $\mu$ jest surjiekcją.}}

Czy ja mam potwierdzić, że $\mu$ jest suriekcją? To znaczy no to widać?

Znaczy tak, jeśli coś jest nieodwracalne, to jest zawarte w pewnym ideale maksymalnym a więc przejdzie na coś z $\overline X$. Jeśli coś jest odwracalne, to nie jest zawarte w żadnym ideale maksymalnym

No to ze jest na to widac no

ideały max przejda na siebie i pokryja $\overline X$

\emph{\color{yellow}(ii) Z lematu Urysowa (to jest jedyny nietrywialny fakt potrzebny w tym argumencie) ciągła funkcja rozdziela punkty $X$. Stąd $x\neq y\implies\mathfrak{m}_x\neq\mathfrak{m}_y$ i $\mu$ jest injiekcją.}

Niech $x, y\in X$ i niech $f\in C(X)$ będzie funkcją je rozdzielającą.

\emph{\color{yellow}(iii) Niech $f\in C(X)$, niech}
$$U_f=\{x\in X\;:\;f(x)\neq 0\}$$
\emph{\color{yellow}i niech}
$$\overline{U_f}=\{\mathfrak{m}\in\overline X\;:\;f\notin \mathfrak{m}\}$$
\emph{\color{yellow}Pokaż, że $\mu(U_f)=\overline{U_f}$. Zbiory otwarte $U+f$ (odpowiednio $\overline{U_f}$) tworzą bazę topologii $X$ (odpowiednio $\overline X$) i z tego powodu $\mu$ jest homeomorfizmem. Więc $X$ można odtworzyć z pierścienia funkcji $C(X)$}

\end{document}