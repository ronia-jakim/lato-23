\documentclass{article}

\usepackage{../../notatki}

\begin{document}
\subsection*{ZADANIE 18}
\emph{\color{pink}Z psychologicznych powodów, czasem wygodniej jest oznaczać ideał pierwszy pierścienia $A$ literami $x,y$ jeśli myślimy o nich jako o punkcie $X=Spec(A)$. Kiedy myślimy o $x$ jako o ideale pierwszym $A$, oznaczamy go przez $\mathfrak{p}_x$ (oczywiście jest to ta sama rzecz). Pokaż, że}

\emph{\color{pink}(i) zbiór $\{x\}$ jest domknięty w $Spec(A)\iff\mathfrak{p}_x$ jest maksymalny}
\medskip

$\impliedby$

Jeśli $\mathfrak{p}_x$ jest ideałem maksymalnym, to $\{x\}=V(\mathfrak{p}_x)$, gdyż żaden inny ideał pierwszy nie zawiera $\mathfrak{p}_x$. My definiowaliśmy $V(E)$ jako zbiory domknięte, więc $\{x\}$ też taki jest.

$\implies$

Wiem, że $\{x\}$ jest zbiorem domkniętym. Czyli jest przekrojem pewnej rodziny domkniętych zbiorów bazowych
$$\{x\}=\bigcap\limits_{i\in I}V(E_i)=V\Big(\bigcup\limits_{i\in I}(E_i)\Big)$$
ale jeśli taka suma zawiera się w jednym, jedynym ideale pierwszym, to jest on maksymalny.
\medskip

\emph{\color{pink}(ii) $\overline{\{x\}}=V(\mathfrak{p}_x)$}

$\subseteq$

Jest raczej prostym zawieraniem: $\overline{\{x\}}$ jest najmniejszym zbiorem domkniętym zawierającym $\{x\}$, a $V(\mathfrak{p}_x)$ z pewnością to spełnia.

$\supseteq$

Po pierwsze zauważmy, że
$$V(\pideal_x)=\bigcap\limits_{E\subseteq \pideal_x}V(E) = V\Big(\bigcup\limits_{E\subseteq\pideal_x}V(E)\Big),$$
bo to są wszystkie te ideały pierwsze, które zawierają jakiś podzbiór $\pideal_x$, czyli obcinamy te mniejsze podzbiory $\pideal_x$ w trakcie brania przekroju.

Wiemy, że $\bigcap\limits_{E\subseteq \pideal_x}V(E)$ jest zbiorem domkniętym. Wiemy, że $x\in\bigcap V(E)$, czyli dostajemy, że $V(\pideal_x)$ jest przekrojem wszystkich zbiorów domkniętych zawierających $x$, czyli jest najmniejszym zbiorem domkniętym zawierającym $x$, czyli domknięciem $x$.
\medskip

\emph{\color{pink}(iii) $y\in\{\overline x\}\iff\mathfrak{p}_x\subseteq\mathfrak{p}_y$}

$\impliedby$

Niech $x,y\in X$ takie, że $\mathfrak{p}_x\subseteq\mathfrak{p}_y$. Wówczas, $x\in V(E)\implies y\in V(E)$. Ponieważ $\overline{\{x\}}$ jest przekrojem zbiorów $V(E_i)$, który zawiera $x$, to w szczególności każdy z tych zbiorów zawiera również $y$, stąd $y\in\overline{\{x\}}$.

$\implies$

Trywialne z \emph{(ii)}.
\medskip

\emph{\color{pink}(iv) $X$ jest $T_0$-przestrzenią (jeśli $x,y$ są rozróżnialnymi punktami $X$, to albo istnieje otoczenie $x$ które nie zawiera $y$, albo istnieje otocznie $y$, które nie zawiera $x$).}

Weźmy dowolne punkty $x,y\in X$. Rozważmy dwa przypadki:

1. $\pideal_x\subseteq\pideal_y$ (lub $\pideal_y\subseteq\pideal_x$, ale WLOG pierwsza wersja)

Wtedy $x\in X\setminus V(\pideal_y)$, które jest zbiorem otwartym takim, że $y\notin X\setminus V(\pideal_y)$.

2. $\pideal_x\not\subseteq\pideal_y$ i $\pideal_y\not\subseteq\pideal_x$

Wtedy $y\notin\overline{\{x\} }$ i $x\notin\overline{\{y\} }$. Czyli $y\in X\setminus\{x\}$ jest otwartym zbiorem zawierającym $y$ ale niezawierającym $x$.

\subsection*{ZADANIE 19.}
\emph{\color{blue}Przestrzeń topologiczna $X$ jest nieredukowalna, jeśli $X\neq\emptyset$ i jeśli każda para niepustych otwartych podzbiorów $X$ się przecina (równoważnie, każdy niepusty podzbiór otwarty jest gęsty w $X$). Pokaż, że $Spec(A)$ jest nieredukowalny $\iff$ nilradykał $A$ jest ideałem pierwszym.}

$\implies$

Chcę mieć punkt, którego dopełnienie jest wszystkim. 

$\impliedby$

Niech $r$ będzie nilradykałem pierścienia $A$ i niech $r$ będzie ideałem pierwszym. Weźmy dowolne $\mathfrak{a}_1,\mathfrak{a}_2\normalsubgroup A$ i rozpatrzmy $U_1=V(\mathfrak{a}_1)^c,U_2=V(\mathfrak{a}_2)^c$.

\subsection*{ZADANIE 20}
\emph{\color{pink}Niech $X$ będzie przestrzenią topologiczną}

\emph{\color{pink}(i) Jeśli $Y$ jest nieredukowalną podprzestrzenią $X$, wtedy domknięcie $\overline Y$ w $X$ jest nieredukowalne.}

Załóżmy nie wprost, że $\overline Y$ nie jest nieredukowalna. Wtedy istnieją $U, V\subseteq \overline Y$ takie, że $U\cap V=\emptyset$. Ale zbiór $U\cap Y$ jest albo pusty albo jest zbiorem otwartym w $Y$. Tak samo dla $V\cap Y$. To znaczy, że $(U\cap Y)\cap(V\cap Y)=\emptyset$ co jest sprzeczne z nieredukowalnością $Y$.

\emph{\color{pink}(ii) Każda nieredukowalna podprzestrzeń $X$ jest zawarta w pewnej nieredukowalnej podprzestrzeni $X$.}

Niech $S$ będzie zbiorem nieredukowalnych podprzestrzeni $X$. Rozważmy łańcuch
$$Y_1\subseteq Y_2\subseteq...$$
podprzestrzeni z $S$. Niech $Y=\bigcup Y_i$. Musimy pokazać, że $Y\in S$, czyli $Y$ jest nieredukowalny.

Niech $U,V\subseteq Y$. Wtedy istnieje $i$ takie, że $U,V\subseteq Y_i$. Ponieważ $Y_i$ jest nieredukowalna, to $U\cap V\neq\emptyset$. W takim razie każde dwa zbiory otwarte z $Y$ tną się niepusto, a więc $Y$ jest nieredukowalne.

Wystarczy użyć lematu Zorna dla zbioru $S_A=\{Y\subseteq X\;:\;Y\text{ nieredukowalna i }A\subseteq Y\}$.

\emph{\color{pink}(iii) Maksymalne nieredukowalne podprzestrzenie $X$ są domknięte i pokrywają $X$. Nazywamy je składowe nieredukowalne $X$. Jakie są składowe nieredukowalne przestrzeni Hausdorffa?}

Niech $M\subseteq X$ będzie maksymalną podprzestrzenią nieredukowalną $X$. Domkniętość $M$ wynika wprost z (ii). Gdyby $M$ nie było domknięte, to $M\subsetneq\overline M$, a $\overline M$ też jest nieredukowalne i mamy sprzeczność z maksymalnością $M$.

Dlaczego pokrywają? Bo dla każdego $\{x\}$, $x\in X$ możemy rozpatrzeć zbiór wszystkich nieredukowalnych zbiorów takich, że $\{x\}\subseteq A$ i w ten sposób znajdziemy maksymalne zbiory nieredukowalne zawierające każdy element $X$, czyli pokrywające $X$.

W Hausdorffie możemy każde dwa punkty oddzielić dwoma rozłącznymi otwartymi otoczeniami, więc maksymalne nieredukowalne podprzestrzenie to singletony. I to właśnie przypadek, który mnie natchnął do wytłumaczenia jak pokryć $X$.

\emph{\color{yellow}(iv) Jeśli $A$ jest pierścieniem i $X=Spec(A)$, wtedy składowe nieredukowalne $X$ to zbiory domknięte $V(\pideal)$, gdzie $\pideal$ to najmniejsza zbiory pierwsze $A$.}

\subsection*{ZADANIE 21}
\emph{\color{pink}Niech $\phi:A\to B$ będzie homomorfizmem pierścieni. Niech $X=Spec(A)$ i $Y=Spec(B)$. Jeśli $\mathfrak{q}\in Y$, to $\phi^{-1}(\mathfrak{q})$ jest ideałem pierwszym w $A$, w szczególności punktem $X$. Z tego powodu $\phi$ indukuje przekształcenie $\phi^*:Y\to X$. Pokaż, że}

\emph{\color{pink}(i) Jeśli $f\in A$, wtedy $\phi^{*-1}(X_f)=Y_{\phi(f)}$ i dlatego $\phi^*$ jest ciągłe.}
\begin{align*}
    \phi^{*-1}(X_f)&=\{y\in Y\;:\;(f)\subsetneq\phi^*(y)\}=\{y\in Y\;:\;(f)\subsetneq\phi^{-1}(y)\}=\{y\in Y\;:\;\phi(f)\subseteq y\}=Y_{\phi(f)}
\end{align*}

\emph{\color{pink}(ii) Jeśli $\mathfrak{a}$ jest ideałem $A$, wtedy $\phi^{*-1}(V(\mathfrak{a}))=V(\mathfrak{a^e})$ }

$\mathfrak{a}^e$ to rozszerzenie obrazy $\phi(\mathfrak{a})$ do ideału w $B$.
\begin{align*}
    \phi^{*-1}(V(\mathfrak{a}))=\{y\in Y\;:\;\mathfrak{a}\subseteq \phi^*(y)\}=\{y\in Y\;:\;\mathfrak{a}\subseteq\phi^{-1}(y)\}=\{y\in Y\;:\;\phi(\mathfrak{a})\subseteq y\}=V(\phi(\mathfrak{a}))=V(\mathfrak{a}^e)
\end{align*}
Ta ostatnia równość z jakiegoś poprzedniego zadanka, bo wtedy mam, że $V(E)=V((E))$ i to jest to samo.

\end{document}