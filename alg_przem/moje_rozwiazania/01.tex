\documentclass{article}

\usepackage{../../notatki}

\begin{document}
\subsection*{ZADANIE 18}
\emph{Z psychologicznych powodów, czasem wygodniej jest oznaczać ideał pierwszy pierścienia $A$ literami $x,y$ jeśli myślimy o nich jako o punkcie $X=Spec(A)$. Kiedy myślimy o $x$ jako o ideale pierwszym $A$, oznaczamy go przez $\mathfrak{p}_x$ (oczywiście jest to ta sama rzecz). Pokaż, że}

\emph{(i) zbiór $\{x\}$ jest domknięty w $Spec(A)\iff\mathfrak{p}_x$ jest maksymalny}
\medskip

$\impliedby$

Jeśli $\mathfrak{p}_x$ jest ideałem maksymalnym, to $\{x\}=V(\mathfrak{p}_x)$, gdyż żaden inny ideał pierwszy nie zawiera $\mathfrak{p}_x$. My definiowaliśmy $V(E)$ jako zbiory domknięte, więc $\{x\}$ też taki jest.

$\implies$

Wiem, że $\{x\}$ jest zbiorem domkniętym. Czyli jest przekrojem pewnej rodziny domkniętych zbiorów bazowych
$$\{x\}=\bigcap\limits_{i\in I}V(E_i)=V\Big(\bigcup\limits_{i\in I}(E_i)\Big)$$
Jeżeli którykolwiek z $E_i$ zawieraj element odwracalny, to $\mathfrak{p}_x=(1)$. W przeciwnym przypadku

\end{document}