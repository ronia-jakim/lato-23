\documentclass{article}

\usepackage{../../notatki}

\begin{document}

\subsection*{ZADANIE 1.}
\emph{\color{pink}Pokaż, że $(\Z/m\Z)\otimes_{\Z}(\Z/n\Z)=0$ jeśli $m,n$ są względnie
pierwsze.}
\smallskip

Załóżmy, że $m,n$ są względnie pierwsze, czyli z równości Bezout'a:
$$am+bn=1$$
teraz popatrzmy na dowolny element produkciku:
$$x\otimes y=(xy)\otimes (am+bn)=(xy)\otimes(am)+(xy)\otimes(bn)=(amx)\otimes y+(xy)\otimes
0=0\otimes y+(xy)\otimes 0=0+0=0$$
Czyli każdy element jest $0$, więc całość też jest $0$.

\subsection*{ZADANIE 2.}
\emph{\color{pink}Niech $A$ będzie pierścieniem, $\mathfrak{a}$ ideałem, a $M$
$A$-modułem. Pokaż, że $(A/\mathfrak{a})\otimes_A M$ jest izomorficzne do
$M/\mathfrak{a}M$.}
\emph{[Stensoruj ciąg dokładny $0\to\mathfrak{a}\to A\to A/\mathfrak{a}\to0$ z $M$}
\smallskip

To jest tak, że jak miałam sobie
$$\mathfrak{a}\to A\to A/\mathfrak{a}\to 0$$
i jakiś losowy $A$-moduł $M$, to
$$\mathfrak{a}\otimes M\to A\otimes M\to A/\mathfrak{a}\otimes M\to 0$$
też jest ciągiem dokładnym!

Zajebiście, to teraz jak te pyśki szły? Pierwszy jest iniekcją, drugi jest suriekcją i ten drugi indukuje izomorfizm $Coker(f)=M/f(M')$ na $M''$ ($f$ to pierwsza funkcja, a myśki lecą $M'\to M\to M''$.) 

Czyli co? Jak wygląda ta iniekcja $\mathfrak{a}\to A$? To jest identyczność na $\mathfrak{a}$ lol.

Jak na razie mam, że
$$A/\mathfrak{a}\otimes M\cong (A\otimes M)/(\mathfrak{a}\otimes M)\cong AM/\mathfrak{a}M=M/\mathfrak{a}M$$

\subsection*{ZADANIE 3.}
\emph{\color{yellow}Niech $A$ będzie pierścieniem lokalnym, $M,N$ skończenie generowanymi $A$-modułami. Udowodnij, że $M\otimes N=0$ wtedy $M=0$ lub $N=0$.}

\emph{\color{yellow}[Niech $\mathfrak{m}$ będzie ideałem maksymalnym, $k=A/\mathfrak{m}$ będzie residue filed (to jest ciało zrobione przez wytentegowanie z tym tym). Niech $M_k=k\otimes_AM\cong M/\mathfrak{m}M$ na mocy zadania 2. Z lematu Nakayamy mamy, że $M_k=0\implies M=0$. Ale $M\otimes_AN=0\implies (M\otimes_AN)_k=0\implies M_k\otimes_kN_k=0\implies M_k=0$ or $N_k=0$, since $M_k,N_k$ są przestrzeniami wektorowymi nad ciałem.]}
\smallskip

Czyli co, ja mam uzasadnić po prostu przejścia w tym łańcuszku?
$$M\otimes_AN=0\implies(M\otimes_AN)_k=0\overset{(\star)}{\implies} M_k\otimes_kN_k=0\overset{(\heartsuit)}{\implies} M_k=0\;\lor\;N_k=0$$
Bo cała reszta wydaje się mieć sens?

($\star$) $k\otimes_A (M\otimes_AN)=0\implies(k\otimes_AM)\otimes_k(k\otimes_AN)=0$

Jeśli $k\otimes_A(M\otimes_AN)=0$, to $(k\otimes_AM)\otimes_AN)=0$, czyli $k\otimes_AM$

A to to jest raczej proste, bo jeśli $k\otimes_A(M\otimes_AN)=0$, to tym bardziej $k\otimes_k(k\otimes_A(M\otimes_AN))=0$ a jak się poprzestawia, bo to raczej jest izomorficzne, chyba że nagle świat staną na głowie, to dostaję $k\otimes_AM\otimes_kk\otimes_AN$.

($\heartsuit$) $M_k\otimes_kN_k=0\implies M_k=0\;\lor\;N_k=0$? Nie no, to jest raczej oczywiste z tego ten ten na N.

{\color{orange}POKOPAŁAM TE RÓWNOŚCI I CO JEST CZYM AAAAAAAAAAA -- zapytać jak się zmienia to nad czym tensorujemy}

Chwila, bo $0=k\otimes_A(M\otimes_AN)=(k\otimes_AM)\otimes_AN$

\subsection*{ZADANIE 4.}
\emph{\color{yellow}Niech $M_i\;(i\in I)$ będzie dowolną rodziną $A$-modułów i niech $M$ będzie ich sumą prostą. Pokaż, że $M$ jest płaski $\iff$ każdy $M_i$ jest płaski}
\smallskip

Mamy funktor $T_N:M\mapsto M\otimes_A N$ i on jest na kategorii $A$-modułów i homomorfizmów. Jeśli $T_N$ jest dokładny, czyli tensorowanie z $N$ przekształca wszystkie ciągi dokładne na ciągi dokładne, wtedy $N$ jest {\color{blue}flat} $A$-modułem.

$\impliedby$ pójdzie chyba z faktu, że $(M\oplus N)\otimes P\cong (M\otimes P)\oplus(N\otimes P)$

$\implies$ 

Wiem, że jeśli mam ciąg dokładny
$$0\to N_1\to N_2\to N_3\to 0$$
dla dowolnych $N_i$, to wtedy tensorowanie przez $M$ zachowuje dokładność, tzn ciąg
$$0\to N_1\otimes M\to N_2\otimes M\to N_3\otimes M\to 0$$
jest nadal dokładny.

Co by się stało, jeśli któraś współrzędna $M$ nie jest flat? Wtedy mogłam $N$ wybrać tak, żeby
$$0\to N_1\otimes M_i\to N_2\otimes M_i\to N_3\otimes M_i\to 0$$
nie było dokładne, czyli tutaj psuje się iniekcja
$$f_1:N_1\otimes M_i\to N_2\otimes M_i$$
No dobra, ale ja mogę zapisać sobie
$$M=M_i\bigoplus\limits_{j\neq i}M_j$$
i zrobić
$$F_1:N_1\otimes(M_i\bigoplus M_j)\to N_2\otimes(M_i\bigoplus M_j)$$
czyli coś typu $n_1\otimes (m_i, m)\mapsto n_2\otimes (m_i, m)$, ale mam też izomorfizmy
$$n_1\otimes (m_i, m)\mapsto (n_1, m_i)\otimes (n_1,m)$$
$$n_2\otimes (m_i, m)\mapsto(n_2,m_i)\otimes (n_2, m)$$
no i tak jakby iniekcyjność $F_1$ jest psuta przez brak inikcyjności w $f_1$, czyli sprzeczność?
Bo przecież $F_1=f_1\otimes F'$ dla jakiejś ładnej iniekcji $F'$.

\subsection*{ZADANIE 5.}
\emph{\color{pink}Niech $A[X]$ będzie pierścieniem wielomianów jednej zmiennej nad pierścieniem $A$. Pokaż, że $A[X]$ jest płaską $A$-algebrą.}
\smallskip

No jak dla mnie to $A[X]$ to jest suma prosta $\bigoplus_{n\in\N}Ax^n\cong \bigoplus_{n\in\N}A$ i $A[X]$ to moduł wolny.
Ah, no i teraz korzystam z tego, że $A\otimes M=M$ i śmiga.


\subsection*{ZADANIE 6.}
{\color{yellow}\emph{Dla dowolnego $A$-moduły, niech $M[X]$ będzie oznaczało zbiór wszystkich wielomianów w $x$ o współczynnikach z $M$, to znaczy wyrażenia formy}
$$m_0+m_1x+...+x_rx^r$$
Zdefiniuj iloczyn elementu $A[X]$ z elementem $M[X]$ w oczywisty sposób, pokaż że $M[X]$ jest $A[X]$-modułem. Pokaż, że $M[X]\cong A[X]\otimes_AM$.}
\smallskip

Jak to leciało dla $A$-modułu? $a,b\in A, x,y\in M$
\begin{multline*}\\
    a(x+y)=ax+ay\\
    (a+b)x=ax+by\\
    (ab)x=a(bx)\\
    1x=x\\
\end{multline*}

Czy ja chce brać sobie $w,v\in M[X]$ oraz $p,r\in A[X]$ i robić zwykłe mnożenie wielomianów? Chyba tak XD
\begin{align*}
    p(w+v)&=\left(\sum p_ix^i\right)\left(\sum w_ix^i+\sum v_ix^i\right)=\left(\sum p_ix^i\right)\left(\sum(w_i+v_i)x^i\right)=\\
    &=\sum\limits_{k=0}^n\left(\sum\limits_{i+j=k} p_i(w_j+v_j)x^k\right)=\sum\left(\sum p_iw_jx^k+\sum p_iv_jx^k\right)=\\
    &=\sum\sum p_iw_jx^k+\sum\sum p_iv_jx^k=pw+pv
\end{align*}
I reszty sprawdzania to mi się nie chce.

Homomorfizm na
$$f:M[X]\to A[X]\otimes_AM$$
$$f(\sum m_jx^j)=\sum(x^j\otimes m_j)$$
jest $1-1$, bo każdy wielomian jest unikalny ze względu na współczynniki przy kolejnych potęgach, bla bla bla. Widać. Nawet mi się nie chce tego pisać ładnie

To teraz w drugą stronę jest też dość prosty
$$g:A[X]\otimes_AM\to M[X]$$
$$g(\left(\sum a_ix^i\right)\otimes m)=\sum a_imx^i$$











\end{document}
