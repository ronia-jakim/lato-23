\documentclass{article}

\usepackage{../../lecture_notes}

\begin{document}

\subsection*{ZADANIE 1.}
\emph{\color{pink}Pokaż, że $(\Z/m\Z)\otimes_{\Z}(\Z/n\Z)=0$ jeśli $m,n$ są względnie
pierwsze.}
\smallskip

Załóżmy, że $m,n$ są względnie pierwsze, czyli z równości Bezout'a:
$$am+bn=1$$
teraz popatrzmy na dowolny element produkciku:
$$x\otimes y=(xy)\otimes (am+bn)=(xy)\otimes(am)+(xy)\otimes(bn)=(amx)\otimes y+(xy)\otimes
0=0\otimes y+(xy)\otimes 0=0+0=0$$
Czyli każdy element jest $0$, więc całość też jest $0$.

\subsection*{ZADANIE 2.}
\emph{\color{pink}Niech $A$ będzie pierścieniem, $\mathfrak{a}$ ideałem, a $M$
$A$-modułem. Pokaż, że $(A/\mathfrak{a})\otimes_A M$ jest izomorficzne do
$M/\mathfrak{a}M$.}
\emph{[Stensoruj ciąg dokładny $0\to\mathfrak{a}\to A\to A/\mathfrak{a}\to0$ z $M$}
\smallskip

To jest tak, że jak miałam sobie
$$\mathfrak{a}\to A\to A/\mathfrak{a}\to 0$$
i jakiś losowy $A$-moduł $M$, to
$$\mathfrak{a}\otimes M\to A\otimes M\to A/\mathfrak{a}\otimes M\to 0$$
też jest ciągiem dokładnym!

Zajebiście, to teraz jak te pyśki szły? Pierwszy jest iniekcją, drugi jest suriekcją i ten drugi indukuje izomorfizm $Coker(f)=M/f(M')$ na $M''$ ($f$ to pierwsza funkcja, a myśki lecą $M'\to M\to M''$.) 

Czyli co? Jak wygląda ta iniekcja $\mathfrak{a}\to A$? To jest identyczność na $\mathfrak{a}$ lol.

Jak na razie mam, że
$$A/\mathfrak{a}\otimes M\cong (A\otimes M)/(\mathfrak{a}\otimes M)\cong AM/\mathfrak{a}M=M/\mathfrak{a}M$$

\subsection*{ZADANIE 3.}
\emph{\color{yellow}Niech $A$ będzie pierścieniem lokalnym, $M,N$ skończenie generowanymi $A$-modułami. Udowodnij, że $M\otimes N=0$ wtedy $M=0$ lub $N=0$.}

\emph{\color{yellow}[Niech $\mathfrak{m}$ będzie ideałem maksymalnym, $k=A/\mathfrak{m}$ będzie residue filed (to jest ciało zrobione przez wytentegowanie z tym tym). Niech $M_k=k\otimes_AM\cong M/\mathfrak{m}M$ na mocy zadania 2. Z lematu Nakayamy mamy, że $M_k=0\implies M=0$. Ale $M\otimes_AN=0\implies (M\otimes_AN)_k=0\implies M_k\otimes_kN_k=0\implies M_k=0$ or $N_k=0$, since $M_k,N_k$ są przestrzeniami wektorowymi nad ciałem.]}
\smallskip

Czyli co, ja mam uzasadnić po prostu przejścia w tym łańcuszku?
$$M\otimes_AN=0\implies(M\otimes_AN)_k=0\overset{(\star)}{\implies} M_k\otimes_kN_k=0\overset{(\heartsuit)}{\implies} M_k=0\;\lor\;N_k=0$$
Bo cała reszta wydaje się mieć sens?

($\star$) $k\otimes_A (M\otimes_AN)=0\implies(k\otimes_AM)\otimes_k(k\otimes_AN)=0$

Jeśli $k\otimes_A(M\otimes_AN)=0$, to $(k\otimes_AM)\otimes_AN)=0$, czyli $k\otimes_AM$

A to to jest raczej proste, bo jeśli $k\otimes_A(M\otimes_AN)=0$, to tym bardziej $k\otimes_k(k\otimes_A(M\otimes_AN))=0$ a jak się poprzestawia, bo to raczej jest izomorficzne, chyba że nagle świat staną na głowie, to dostaję $k\otimes_AM\otimes_kk\otimes_AN$.

($\heartsuit$) $M_k\otimes_kN_k=0\implies M_k=0\;\lor\;N_k=0$? Nie no, to jest raczej oczywiste z tego ten ten na N.

{\color{orange}POKOPAŁAM TE RÓWNOŚCI I CO JEST CZYM AAAAAAAAAAA -- zapytać jak się zmienia to nad czym tensorujemy}

Chwila, bo $0=k\otimes_A(M\otimes_AN)=(k\otimes_AM)\otimes_AN$

\subsection*{ZADANIE 4.}
\emph{\color{yellow}Niech $M_i\;(i\in I)$ będzie dowolną rodziną $A$-modułów i niech $M$ będzie ich sumą prostą. Pokaż, że $M$ jest płaski $\iff$ każdy $M_i$ jest płaski}
\smallskip

Mamy funktor $T_N:M\mapsto M\otimes_A N$ i on jest na kategorii $A$-modułów i homomorfizmów. Jeśli $T_N$ jest dokładny, czyli tensorowanie z $N$ przekształca wszystkie ciągi dokładne na ciągi dokładne, wtedy $N$ jest {\color{blue}flat} $A$-modułem.

$\impliedby$ pójdzie chyba z faktu, że $(M\oplus N)\otimes P\cong (M\otimes P)\oplus(N\otimes P)$

$\implies$ 

Wiem, że jeśli mam ciąg dokładny
$$0\to N_1\to N_2\to N_3\to 0$$
dla dowolnych $N_i$, to wtedy tensorowanie przez $M$ zachowuje dokładność, tzn ciąg
$$0\to N_1\otimes M\to N_2\otimes M\to N_3\otimes M\to 0$$
jest nadal dokładny.

Co by się stało, jeśli któraś współrzędna $M$ nie jest flat? Wtedy mogłam $N$ wybrać tak, żeby
$$0\to N_1\otimes M_i\to N_2\otimes M_i\to N_3\otimes M_i\to 0$$
nie było dokładne, czyli tutaj psuje się iniekcja
$$f_1:N_1\otimes M_i\to N_2\otimes M_i$$
No dobra, ale ja mogę zapisać sobie
$$M=M_i\bigoplus\limits_{j\neq i}M_j$$
i zrobić
$$F_1:N_1\otimes(M_i\bigoplus M_j)\to N_2\otimes(M_i\bigoplus M_j)$$
czyli coś typu $n_1\otimes (m_i, m)\mapsto n_2\otimes (m_i, m)$, ale mam też izomorfizmy
$$n_1\otimes (m_i, m)\mapsto (n_1, m_i)\otimes (n_1,m)$$
$$n_2\otimes (m_i, m)\mapsto(n_2,m_i)\otimes (n_2, m)$$
no i tak jakby iniekcyjność $F_1$ jest psuta przez brak inikcyjności w $f_1$, czyli sprzeczność?
Bo przecież $F_1=f_1\otimes F'$ dla jakiejś ładnej iniekcji $F'$.

\subsection*{ZADANIE 5.}
\emph{\color{pink}Niech $A[X]$ będzie pierścieniem wielomianów jednej zmiennej nad pierścieniem $A$. Pokaż, że $A[X]$ jest płaską $A$-algebrą.}
\smallskip

No jak dla mnie to $A[X]$ to jest suma prosta $\bigoplus_{n\in\N}Ax^n\cong \bigoplus_{n\in\N}A$ i $A[X]$ to moduł wolny.
Ah, no i teraz korzystam z tego, że $A\otimes M=M$ i śmiga.


\subsection*{ZADANIE 6.}
{\color{pink}\emph{Dla dowolnego $A$-moduły, niech $M[X]$ będzie oznaczało zbiór wszystkich wielomianów w $x$ o współczynnikach z $M$, to znaczy wyrażenia formy}
$$m_0+m_1x+...+x_rx^r$$
Zdefiniuj iloczyn elementu $A[X]$ z elementem $M[X]$ w oczywisty sposób, pokaż że $M[X]$ jest $A[X]$-modułem. Pokaż, że $M[X]\cong A[X]\otimes_AM$.}
\smallskip

Jak to leciało dla $A$-modułu? $a,b\in A, x,y\in M$
\begin{multline*}\\
    a(x+y)=ax+ay\\
    (a+b)x=ax+by\\
    (ab)x=a(bx)\\
    1x=x\\
\end{multline*}

Czy ja chce brać sobie $w,v\in M[X]$ oraz $p,r\in A[X]$ i robić zwykłe mnożenie wielomianów? Chyba tak XD
\begin{align*}
    p(w+v)&=\left(\sum p_ix^i\right)\left(\sum w_ix^i+\sum v_ix^i\right)=\left(\sum p_ix^i\right)\left(\sum(w_i+v_i)x^i\right)=\\
    &=\sum\limits_{k=0}^n\left(\sum\limits_{i+j=k} p_i(w_j+v_j)x^k\right)=\sum\left(\sum p_iw_jx^k+\sum p_iv_jx^k\right)=\\
    &=\sum\sum p_iw_jx^k+\sum\sum p_iv_jx^k=pw+pv
\end{align*}
I reszty sprawdzania to mi się nie chce.

Homomorfizm na
$$f:M[X]\to A[X]\otimes_AM$$
$$f(\sum m_jx^j)=\sum(x^j\otimes m_j)$$
jest $1-1$, bo każdy wielomian jest unikalny ze względu na współczynniki przy kolejnych potęgach, bla bla bla. Widać. Nawet mi się nie chce tego pisać ładnie

To teraz w drugą stronę jest też dość prosty
$$g:A[X]\otimes_AM\to M[X]$$
$$g(\left(\sum a_ix^i\right)\otimes m)=\sum a_imx^i$$

\subsection*{ZADANIE 7.}
{\color{yellow}\emph{Niech $\mathfrak{p}$ będzie ideałem pierwszym w $A$. Pokaż, że $\mathfrak{p}[X]$ jest ideałem pierwszym w $A[X]$. Czy jeśli $\mathfrak{m}$ jest ideałem pierwszym w $A$, to $\mathfrak{m}[X]$ jest ideałem maksymalnym w $A[X]$?}}
\smallskip

Z poprzedniego zadania wiem, że $\mathfrak{p}[X]\cong A[X]\otimes_A\mathfrak{p}$, bo każdy ideał jest $A$-modułem.

Czy mogę określić sobie homomorfizm (ewualuację w $x=1$)
$$f:A[X]\to A$$
$$f(\sum a_ix^i)=\sum a_i$$
i wtedy $f^{-1}[\mathfrak{p}]$ jest całością $\mathfrak{p}[X]$ jest ideałem pierwszym jako przeciwobraz ideału pierwszego przez homomorfizm. 

Alternatywnie
$$(A[X])/(\mathfrak{p}[X])\cong (A/\mathfrak{p})[X]$$
w pierwszym zadaniu z poprzedniego rozdziału pokazywaliśmy, że $f\in A[X]$ jest dzielnikiem zera $\iff$ $af=0$ dla pewnego $a\in A\setminus\{0\}$,
czyli $\iff$ w $A$ są dzielniki zera.
Ale w $(A/\mathfrak{p})$ dzielników zera nie ma, bo wszystkie są w $\mathfrak{p}$ który to wyrzuciliśmy, więc śmiga.

\subsection*{ZADANIE 9.}
\emph{\color{yellow}TO WYPADAŁOBY ZROBIĆ, ALE NIEEE CHCEEE MIII SIEEEE}



\subsection*{ZADANIE 10.}
\emph{\color{blue}Niech $A$ będzie pierścieniem i $\mathfrak{a}$ ideałem zawartym w radykalne Jacobsona. Niech $M$ będzie $A$-modułem, a $N$ niech będzie skończenie generowanym $A$-modułem. Niech $u:M\to N$ będzie homomorfizmem. Pokaż, że jeżeli indukowany homomorfizm $\overline:M/\mathfrak{a}M\to N/\mathfrak{a}N$ jest surjektywny, to również $u$ taki jest.}

Najpierw rysuneczek:

\begin{center}
\begin{tikzcd}
    M/\mathfrak{a}M\arrow[r,"\overline{u}"]\arrow[d, "\cong"] & N/\mathfrak{a}N \arrow[r]\arrow[d, "\cong"] & Coker(\overline{u})=0\arrow[d, "\cong"]\\
    M\otimes A/\mathfrak{a}\arrow[r, "u\otimes 1"] & N\otimes A/\mathfrak{a}\arrow[r] & 0=???=Coker(u)\otimes A/\mathfrak{a}\\
    M\arrow[r, "u"] & N\arrow[r] & Coker(u)
\end{tikzcd}
\end{center}

No i to jest tak, że to co jest w ??? jest izomorficzne z $Coker(\overline{u})$, bo no izomorfizm w dół mi nie popsuje $Coker(\overline{u})$, które było równe $0$. Czyli $???=0$. Z drugiej strony, to co jest w $???$ jest równe $Coker(u)\otimes A/\mathfrak{a}$. Skoro $N$ było skończenie generowane, to takie jest też $Coker(u)$, bo przecież wychodzi z $N$. Czyli mam, że
$$0=Coker(u)\otimes A/\mathfrak{a}\;\cong\; Coker(u)/\mathfrak{a}Coker(u)$$
i z tego wynika, że $Coker(u)=\mathfrak{a}Coker(u)$ i z lematu Nakayamy wiem, że $Coker(u)=0$.


\emph{\color{orange}Nieskończenie generowany moduł, który nie spełnia lematu Nakayamy. Wyzwanie: znaleźć pierścień $R$, moduł $M$ i ideał $\mathfrak{a}$ taki, że $M=\mathfrak{a}M$ i $M\neq 0$}

Mam pierścień $k[x_1,...,x_n,...]$

\subsection*{ZADANIE 11.}
{\slshape\color{blue}Niech $A$ będzie pierścieniem $\neq0$. Pokaż, że $A^n\cong A^m\implies m=n$.

[Niech $\mathfrak{m}$ będzie ideałem maksymalnym w $A$ i niech $\phi:A^n\to A^m$ będzie izomorfizmem. Wtedy $1\otimes\phi:(A/\mathfrak{m})\otimes A^n\to (A/\mathfrak{m})\otimes A^m$ jest izomorfizmem pomiędzy przestrzeniami liniowymi wymiaru $m$ i $n$ nad ciałem $k=A/\mathfrak{m}$. Czyli $m-n$.]

\begin{itemize}
    \item Jeżeli $\phi:A^m\to A^n$ jest surjekcją, to $m\geq n$
    \item Czy jeżeli $\phi:A^m\to A^n$ jest iniekcją, to $m\leq n$?
\end{itemize}}

Mamy $A^m\cong A^n$ i $\mathfrak{m}\triangleleft A$. 

\begin{center}
\begin{tikzcd}
    A^n\arrow[r, "\cong"]\arrow[d]  &A^m\arrow[d]\\ 
    (A/\mathfrak{m})^n\arrow[r, "\cong"]&(A/\mathfrak{m})^m
\end{tikzcd}
\end{center}

i to niżej to jest przestrzeń liniowa, korzystamy z fakty dobrej określonowości wymiaru takich przestrzeni.


Na surjekcję to działa, ale przy iniekcji niekoniecznie to się przenosi.

Zakładamy nie wprost, że $m>n$ i mamy strzałkę $\phi:A^m\to A^n$. Będziemy uzasadniać, że ona ma nietrywialne jądro.

\begin{center}
\begin{tikzcd}
    A^m\arrow[r]\arrow[rr, bend right=30, "\psi" below] &A^n\arrow[r, hookrightarrow] & A^m
\end{tikzcd}
\end{center}
Niech $M$ będzie modułem z $A^k$.
i $\psi\in End(A^m)$. Mamy, że dla $a_i\in A$
$$\psi^k+...+a_1\psi^{k-1}+...+a_kid_{a^m}=0$$

\begin{center}\begin{tikzcd}
    A^k\arrow[r]\arrow[d]\arrow[dr, to path=|- (\tikztotarget)] & A^k\arrow[d]\\
    M\arrow[r, "\psi" below] & M
\end{tikzcd}\end{center}

\subsection*{ZADANIE 12.}
{\slshape\color{blue}Niech $M$ będzie skończenie generowanym $A$-modułem i $\phi:M]to A^n$ będzie surjektywnym homomorfizmem. Pokaż, że $ker(\phi)$ jest skończenie generowany.

[Niech $e_1,...,e_n$ będzie bazą $A^n$ i wybierzmy $u_i\in M$ takie, że $\phi(u_i)=e_i$. Pokaż, że $M$ jest sumą prostą $ker(\phi)$ i podmodułów generowanych przez $u_1,...,u_n$.}

%\begin{center}
%\begin{tikzcd}
%    M\arrow[r, "\phi"]\arrow[d, leftrightarrow] & A^n\\
%    \bigoplus\limits_{i\leq n}M_i\oplus ker(\phi)\arrow[ru, "\psi" below]
%\end{tikzcd}
%\end{center}
%
%$$\psi\begin{pmatrix}x\\a_1u_1\\...\\a_nu_n\end{pmatrix}=\begin{pmatrix}a_1e_1\\...\\a_ne_n\end{pmatrix}$$
%
%\podz{fore}

Korzystamy ze wskazówki, czyli te $u_i$ istnieją tak jak chcemy. Niech $m\in M$, wtedy
$$\phi(m)=\sum a_ie_i\implies m-\sum a_iu_i\in ker(\phi)$$
Czyli $m\in M$ to jest suma czegoś z $\langle u_i\rangle$ i czegoś z $ker(\phi)$.

Z tego wnioskujemy, że $ker(\phi)\cong M/\langle u_i\rangle$ i my mówimy, że to jest skończenie generowane, bo jest suriekcja z $M$ w to cóś.

\subsection*{ZADANIE 13}
{\slshape\color{blue}Niech $f:A\to B$ będzie homomorfizmem pierścieni i niech $N$ będzie $B$-modułem. Patrzenie na $N$ jako na $A$-moduł poprzez restrykcję skalarów daje nam $B$-moduł $N_B=B\otimes_AN$. Pokaż, że homomorfizm $g:N\to N_B$ taki, że $y\mapsto1\otimes y$ jest iniekcją i że $g(N)$ jest składnikiem sumy $N_B$ (czyli $N_B=g(N)\oplus C$ dla pewnego $C$).

[Zdefiniuje $p:N_B\to N$ przez $p(b\otimes y)=by$ i pokaż, że $N_B=Im(g)\oplus ker(p)$.]}

To, że $g$ jest iniekcją to raczej widać. Bo wpp. 
$$g(y)=1\otimes y=1\otimes y'=g(y')$$
ale $y\neq y'$, czyli $1\otimes y\neq 1\otimes y'$.

Chyba mam pokazać, że $ker(p)$ i $Im(g)$ są rozłączne i coś z $N$ jest albo w $ker$ albo w $Im$. Ale to chyba widać.

Niech $b\otimes n\in N_B$. Wtedy mamy dwie opcje:
\begin{itemize}
    \item $bn=0$, wtedy $b\otimes n\in ker(p)$
    \item $bn\neq 0$, wtedy $b\otimes n=1\otimes bn\in Im(g)$.
\end{itemize}

Wypadałoby też pokazać, że $p$ jest liniowe, ale sobie pominiemy.

Wiemy, że $g\circ p=id$ i chcemy, by $ker(g)=0$

\subsection*{ZADANIE 14}
{\slshape\color{yellow}Częściowo uporządkowany zbiór $I$ jest skierowany, jeżeli dla każdej pary $i,j\in I$ istnieje $k\in I$ takie, że $i\leq k\;\land\;j\leq k$.

Niech $A$ będzie pierścieniem, $I$ będzie skierowanym zbiorem i niech $\{M_i\}_{i\in I}$ będzie rodziną $A$ modułów indeksowanych przez $I$. Dla każdej pary $i,j\in I$ takiej, że $i\leq j$ niech $\mu_{ij}:M_i\to M_j$ będzie $A$-homomorfizmem i załóżmy, że poniższe aksjomaty są spełnione:
\begin{itemize}
    \item $\mu_{ii}$ jest identycznością na $M_i$
    \item $\mu_{ik}=\mu_{jk}\circ\mu_{ij}$ zawsze gdy $i\leq j\leq k$.
\end{itemize}
Wtedy moduł $M_i$ i homomorfizmy $\mu_{ij}$ są skierowanym systemem $\mathbb{M}=(M_i,\mu_{ij})$ nad zbiorem skierowanym $I$.

Skonstruujemy $A$-moduł $M$ nazywany skierowaną granicą skierowanego systemu $\mathbb{M}$. Niech $C$ będzie sumą prostą $M_i$ i zidentyfikuj każdy moduł $M_i$ z jego kanonicznym obrazem w $C$. Niech $D$ będzie podmodułem $C$ generowanym przez wszystkie elementy postaci $x_i-\mu_{ij}(x_i)$ dla $i\leq j$ i $x_i\in M_i$. Niech $M=C/D$ i $\mu:C\to M$ będzie projekcją, $\mu_i$ oznacza restrykcję $\mu$ do $M_i$.

Moduł $M$, albo bardziej dokładnie, para składająca się z $M$ i rodziny homomorfizmów $\mu_i:M_i\to M$ jest nazywana skierowaną granicą skierowanego systemu $\mathbb{M}$ i piszemy $\lim\limits_{\tikz[baseline] \draw[-stealth,line width=.4pt] (0ex,0.4ex) -- (3ex,0.4ex);} M_i$. Z tej konstrukcji jasno wynika, że $\mu_i=\mu_j\circ\mu_{ij}$ gdy tylko $i\leq j$.
}

No konstrukcja, coś tutaj mam zrobić sama?

\subsection*{ZADANIE 15.}
{\slshape\color{blue}
W sytuacji jak w zadaniu 14, pokaż, że każdy element $M$ może być zapisany w formie $\mu_i(x_i)$ dla pewnego $i\in I$ i pewnego $x_i\in M_i$.

Pokaż, że jeśli $\mu_i(x_i)=0$, wtedy istnieje $j\geq i$ takie, że $\mu_{ij}(x_i)=0$ w $M_j$.
}

Widzę to. Teraz pokazać.

%\begin{center}
%\begin{tikzcd}
%    C=\bigoplus M_i\arrow[r, "\mu"] & M\\
%    M_i\arrow[ur, "\mu_i"]\arrow[u]&\\
%\end{tikzcd}
%\end{center}
Weźmy element $x\in M$. Wtedy $x=\sum x_i+D$ dla $x_i\in M_i$. Ale mamy zbiór uporządkowany, czyli istnieje $j$ takie, że $\sum x_i+D=\sum\limits_{i\leq j}x_i+D$. To mogę sobie wszystko rzucić na $M_j$, to znaczy zrobić
$$y_j=\sum\mu_{ij}(x_i)\in M_j.$$
 Wtedy
 $$\sum\limits_{i\leq j}x_i-\sum\limits_{i\leq j}\mu_{ij}(x_i)=\sum\limits_{i\leq j}(x_i-\mu_{ij}(x_i))\in D$$
Czyli $\sum\limits_{i\leq j}x_i=\sum\limits_{i\leq j}\mu_{ij}(x_i)$ w środku $C/D=M$, czyli
$$x=y_j+D=\mu_j(y_j)$$
\medskip

Jeżeli $\mu_i(x_i)=0$, to $x_i\in D$? A to znaczy, że $x_i-\mu_{ij}(x_i)=x_i$ dla pewnego $j$, czyli $\mu_{ij}(x_i)=0$? 

\subsection*{ZADANIE 16.}
{\slshape\color{yellow}
Pokaż, że skierowana granica jest wyznaczona (z dokładnością do izomorfizmu) przez następującą własność. Niech $N$ będzie $A$-modułem i dla każdego $i\in I$ niech $\alpha_i:M_i\to N$ będzie homomorfizmem  $A$-modułów takim, że $\alpha_i=\alpha_j\circ\mu_{ij}$ kiedy $i\leq j$. Wtedy istnieje jedyny homomorfizm $\alpha:M\to N$ taki, że $\alpha_i=\alpha\circ\mu_i$ dla wszystkich $i\in I$.
}

\begin{center}
\begin{tikzcd}
    & M \arrow[dr, "\alpha", dashed]&\\
    M_i\arrow[rr, "\alpha_i"]\arrow[dr, "\mu_{ij}"]\arrow[ur, "\mu_i"] &  & N\\
    & M_j\arrow[ur, "\alpha_{j}"]
\end{tikzcd}
\end{center}

To, że taka istnieje to widać z diagramu. Teraz co, gdyby były dwie takie strzałki? $\alpha'$ i $\alpha$? Wtedy
$$\alpha_i=\alpha\circ\mu_i$$
$$\alpha_i=\alpha'\circ\mu_i$$
a z drugiej strony
$$\alpha_i=\alpha_j\circ\mu_{ij}=(\alpha\circ\mu_j)\circ\mu_{ij}$$
$$\alpha_i=\alpha_j\circ\mu_{ij}=(\alpha'\circ\mu_j)\circ\mu_{ij}$$

\section*{ZADANIE 17.}
\excercise{Niech $\mathbb{M}=(M_i,\mu_{ij}),\mathbb{N}=(N_i,\nu_{ij})$ będą skierowanymi układami $A$ modułów nad tym samym skierowanym zbiorem. Niech $M,N$ będą skierowaną granicą i $\mu_i$}



















\end{document}
