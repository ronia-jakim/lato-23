\section{Pierścienie i ideały}
\emph{Szybkie powtórzenie notacji i podstawowych definicji, z małym dodatkiem ponad algebrę 1r.}

\subsection{Pierścienie i homomorfizmy pierścieni}

\deff{Pierścien} $A$ to zbiór z dwoma binarnymi operacjami (dodawanie i mnożenie) takimi, że
\smallskip

\indent 1. $A$ jest abelową grupą względem dodawania,

\indent 2. mnożenie jest łączne i rozłączne względem dodawania,

\indent 3. dla nas dodatkowo mnożenie jest przemienne,

\indent 4. i ma element neutralny.
\smallskip

Czyli rozważamy tylko \emph{pierścienie przemienne z jednością}. Warto zaznaczyć, że nie wykluczamy że $1=0$, ale wtedy $A$ ma tylko jeden element i jest pierścieniem zerowym, oznaczanym przez $0$.
\medskip

\deff{Homomorfizm pierścieni} to funkcja $f$ z pierścienia $A$ w pierścień $B$ taka, że
\smallskip

\indent 1. $f(x+y)=f(x)+f(y)$,

\indent 2. $f(xy)=f(x)f(y)$,

\indent 3. $f(1)=1$.

\subsection{Ideały, pierścienie ilorazowe}

\deff{Ideał} $I$ pierścienia $A$ to podzbiór $A$ taki, że jest podgrupą względem dodawania i taki, że $AI\subseteq I$. Grupa ilorazowa $A/I$ zachowuje mnożenie zdefiniowane w $I$, co sprawia, że jest pierścieniem, nazywanym \deff{pierścieniem ilorazowym} [lub \emph{residue-class ring}]. Elementami $A/I$ są warstwy $I$ w $A$, a funkcja $\phi:A\to A/I$ taka, że $\phi(x)=x+I$ jest surjiektywnym homomorfizmem.
\smallskip

\deff{\large Twierdzenie:} Istnieje funkcja $1-1$ zachowująca porządek zależności pomiędzy ideałami $I\subseteq J\normalsubgroup A$ oraz ideałami $J'\normalsubgroup A/I$ zadana przez $J=\phi^{-1}(J')$.

\textbf{Dowód:} Jeśli $f:A\to B$ jest homomorfizmem pierścieni, to jądro $f$ jest ideałem $I$ w $A$ oraz obraz $f$ jest podpierścieniem $C\subseteq B$. $f$ indukuje izomorfizm pierścieni $A/I\cong C$.

\proofend

W dalszej części możemy stosować oznaczenie $x\equiv y\mod I$ żeby powiedzieć, że $x-y\in I$.

\subsection{Dzielniki zera, elementy nilpotentne i odwracalne}

\deff{Dzielnik zera} pierścienia $A$ to element $x$ taki, że istnieje dla niego $y\neq 0$ takie, że $xy=0$. Pierścień, który nie posiada dzielników zera różnych od $0$ jest nazywany \deff{dziedziną całkowitą} [\dyg{integral domain}]. 

Element $x\in A$ jest \deff{nilpotentny}, jeżeli istnieje $n>0$ takie, że $x^n=0$. Element nilpotenty jest zawsze dzielnikiem zera, ale odwrotna zależność nie zawsze zachodzi. 

\deff{Element odwracalny} $x\in A$ to element "dzielący zero", czyli istnieje unikalne $y\in A$ takie, że $xy=1$. Zwykle oznaczamy $y=x^{-1}$. Wszystkie elementy odwracalne pierścienia $A$ tworzą \deff{grupę multiplikatywną} [\emph{multiplicative group}], która jest abelową.
\smallskip

Wielokrotności $ax$ elementu $x\in A$ tworzą \deff{ideał główny} [\emph{principal ideal}] pierścienia $A$, co oznaczamy przez $(x)$. Jeżeli $x$ jest odwracalny, to $(x)=A=(1)$. Ideał generowany przez $0$ jest zwykle oznaczany $(0)=0$.

\deff{Ciało} to pierścień $A$ w którym $1\neq0$ i każdy niezerowy dzielnik zera jest odwracalny. Każde ciało jest domeną całkowitą.
\smallskip

\deff{\large Twierdzenie:} Niech $A$ będzie pierścieniem, wtedy poniższe są równoważne:

\indent I $A$ jest ciałem,

\indent II jedyne ideały w $A$ są $0$ lub $(1)$,

\indent III każdy homomorfizm z $A$ w niezerowy pierścień $B$ jest iniekcyjną.

\textbf{Dowód:} 

I $\implies$ II: Niech $I\neq 0$ będzie ideałem w $A$. Wtedy $I$ zawiera niezerowy element $x$, który jest odwracalny. W takim razie $(x)\subseteq I$, a ponieważ $(x)=(1)$, to $I=(1)$. 

II $\implies$ III: Niech $\phi:A\to B$ będzie homomorfizmem pierścieni. Wtedy $ker(\phi)$ jest ideałem różnym od $(1)$, czyli $ker(\phi)$ musi byc zerem, a więc jest funkcją $1-1$.

III $\implies$ I: Niech $x$ będzie elementem $A$, który nie jest odwracalny. WTedy $(x)\neq(1)$, czyli $B=A/(x)$ nie jest pierścieniem zerowym. Niech $\phi:A\to B$ będzie naturalnym homomorfizmem $A$ w $B$ z jądrem $(x)$. Przez hipotezę $\phi$ jest $1-1$, czyli $(x)=0$, więc $x=0$.

\proofend

\subsection{Ideały główne i maksymalne}

Ideał $I\normalsubgroup A$ jest \deff{ideałem pierwszym}, jeżeli $I\neq(1)$ oraz $xy\in I\implies x\in I\;albo\;y\in I$. Ideał $I\normalsubgroup A$ jest z kolei \deff{ideałem maksymalnym}, jeżeli $I\neq (1)$ i nie istnieje ideał $J$ taki, że $I\subsetneq J\subsetneq(1)$. Równoważnie:

\indent \point $I$ jest ideałem pierwszym $\iff$ $A/I$ jest domeną całkowitą,

\indent \point $J$ jest ideałem maksymalnym $\iff$ $A/J$ jest ciałem.

Stąd też, ideał maksymalny jest zawsze pierwszy, ale nie każdy ideał pierwszy jest ideałem maksymalnym.
\medskip

Jeżeli $f:A\to B$ jest homomorfizmem pierścieni i $I$ jest ideałem pierwszym w $B$, wtedy $f^{-1}(I)$ jest ideałem pierwszym w $A$, ale jeżeli $J$ jest ideałem maksymalnym to $f^{-1}(J)$ niekoniecznie musi byc ideałem maksymalnym.
\medskip

\deff{\large Twierdzenie:} Każdy pierścień $A\neq 0$ ma co najmniej jeden ideał maksymalny. 

\textbf{Dowód:} Standardowe zastosowanie lematu Zorna\footnote{Niech $S$ będzie niepustym, częściowo uporządkowanym zbiorem, wtedy jeśli każdy jego łańcuch $T$ ma górną granicę w $S$, to $S$ ma co najmniej jeden element maksymalny.}. Niech $\Sigma$ będzie zbiorem wszystkich ideałów różnych od $(1)$. Uporządkujmy $\Sigma$ przez inkluzję. $\Sigma$ jest zbiorem niepustym, bo $0\in\Sigma$. Musimy pokazać, że każdy łańcuch w $\Sigma$ jest ograniczony od góry. Niech $\{I_n\}$ będzie ciągiem ideałów z $\Sigma$, wtedy $I=\bigcup I_n$ też jest ideałem i nie zawiera $1$, bo nic w ciągu $1$ nie zawierało. Wskazaliśmy więc górne ograniczenie dowolnego łańcucha z $\Sigma$, więc z lematu Zorna $\Sigma$ ma element maksymalny.

\proofend

Jeśli $I\neq(1)$ jest ideałem w $A$, to istnieje ideał maksymalny w $A$ zawierający $I$. Trywialne.

Każdy nieodwracalny element $A$ jest zawarty w pewnym maksymalnym ideale. Też trywialne.
\smallskip

Zauważmy, że jeśli pierścień jest noetherowski, to nie musimy używać Zorna w dowodzie wyżej. Dalej, istnieją pierścienie mające dokładnie jeden pierścień maksymalny, na przykład ciała. Pierścień zawierający dokładnie jeden pierścień maksymalny $I$ jest nazywany \deff{pierścieniem lokalnym} [\emph{local ring}], a ciało $k=A/I$ jest nazywane \acc{residue field} pierścienia $A$.

\deff{\large Twierdzonko:}

\indent I. Jeżeli $A$ jest pierścieniem, a $I\neq(1)$ jego ideałem takim, że dla każdego $x\in A\setminus I$ $x$ jest elementem odwracalnym, to $A$ jest pierścieniem lokalnym.

\indent II. JEżeli $A$ jest pierścieniem i $I$ jego ideałem maksymalnym takim, że każdy element $1+I$ (czyli $1+x,x\in I$) jest odwracalny w $A$, to $A$ jest pierścieniem lokalnym.

\textbf{Dowód:}

I. Każdy ideał składa się z elementów nieodwracalnych,  więc jest zawarty w $I$. Czyli $I$ jest jedynym pierścieniem maksymalnym $A$.

II. Niech $x\in A\setminus I$. Skoro $I$ jest maksymalny, to ideał generowany przez $x$ i $I$ jest równy $(1)$, więc istnieje $y\in A$ i $t\in I$ takie, że $xy+t=1$. Stąd, $xy=1-t$ należy do $1+I$ i jest odwracalny. Teraz używamy punktu I i śmiga.

\proofend

Pierścień \deff{półlokalny} to pierścień zawierający skończoną liczbę ideałów maksymalnych.

\deff{Dziedzina ideałów głównych} [\emph{Principal ideal domain}, \dyg{PID}] to dziedzina całkowita w której każdy ideał jest ideałem głównym.

\subsection{Nilradykał i radykał Jacobsona}

Zbiór $\mathfrak{R}$ zawierający wszystkie nilpotentne elementy pierścienia $A$ jest nazywany jego \deff{nilradykałem} i jest ideałem. Później zostanie podana równoważna definicja nilradykału, ale najpierw twierdzonko.

\deff{\large Twierdzenie:} Nilradykał $\mathfrak{R}$ jest ideałem i $A/\mathfrak{R}$ nie posiada elementów nilpotentych różnych od $0$.

\textbf{Dowód:} Jeśli $x\in\mathfrak{R}$, to $ax\in\mathfrak{R}$ dla dowolnego $a\in A$. Niech $x, y\in\mathfrak{R}$ takie, że $x^m=0=y^n$. Wtedy również $(x+y)^{n+m-1}$ jest sumą wielokrotności $x^ry^s$ takich, że $r+s=m+n-1$. Wiemy, że $r<m$ i $s<n$, stąd też każdy produkt z nich znika i mamy,że $(x+y)^{n+m-1}=0$. Czyli $x+y\in\mathfrak{R}$, więc $\mathfrak{R}$ w istocie jest ideałem.

Niech $\overline x\in A/\mathfrak{R}$ będzie reprezentowane przez $x\in A$. Wtedy $\overline x^n$ jest reprezentowane przez $x^n$, więc jeśli $\overline x^n=0$, to również $x^n=0$ i $x\in\mathfrak{R}$, a więc $\overline x=0$.

\proofend

Druga definicja \acc{nilradykału} to przekrój wszystkich pierwszych ideałów pierścienia $A$.

\textbf{Dowód:} Niech $\mathfrak{R}'$ oznacza przekrój wszystkich pierwszych ideałów pierścienia $A$. Wtedy jeśli $f\in A$ jest nilpotentne i $I$ jest ideałem pierwszym, to $f^n=0\in I$, stąd też $f\in I$, bo $I$ jest ideałem pierwszym. Stąd też $f\in\mathfrak{R}'$.

Z drugiej strony, co jeśli $f$ nie jest nilpotentny? Niech $\Sigma$ będzie zbiorem wszystkich ideałów z własnością $n>0\implies f^n\notin I$. Wtedy $\Sigma$ nie jest pusty, ponieważ $0\in\Sigma$. Znowu możemy śmignąć Zornem przy porządkowaniu przez inkluzję i $\Sigma$ ma pewien element maksymalny, nazwijmy go $J$. Pokażemy, że $J$ jest ideałem pierwszym. Niech $x,y\notin J$. Wtedy ideały $J+(x)$ i $J+(y)$ zawierają właściwie $J$ i stąd też nie należą do $\Sigma$. Stąd też $f^m\in J+(x)$ oraz $f^n\in J+(y)$ dla pewnych $m,n$. W takim razie $f^{m+n}\in J+(xy)$ i ideał $J+(xy)$ nie jest w $\Sigma$, czyli $xy\notin J$. W takim razie mamy ideał pierwszy $J$ taki, że $f\notin J$ i $f\notin\mathfrak{R}'$.

\proofend
\medskip

\deff{Radykał Jacobsona} $\mathfrak{R}$ to przekrój wszystkich maksymalnych ideałów pierścienia $A$. Spełnia on:
$$x\in\mathfrak{R}\iff 1-xy\text{ jest odwracalne dla wszystkich} y$$
\textbf{Dowód:} 

$\implies$ Załóżmy, że $1-xy$ nie jest odwracalne. Wtedy jest zawarte w pewnym ideale maksymalnym $I$. Ale skoro $x\in\mathfrak{R}\subseteq I$, to $xy\in I$ i $1\in I$, co jest sprzecznością.

$\impliedby$ Załóżmy, że $x\notin I$ dla pewnego ideału maksymalnego $I$. Wtedy $I$ i $x$ generują ideał $(1)$, więc dla pewnego $u\in I$ oraz $y\in A$ mamy $u+xy=1$. Stąd też $1-xy\in I$, a więc nie jest elementem odwracalnym i mamy sprzeczność.

\proofend

\subsection{Operacje na ideałach}

Sumę dwóch ideałów $I,J\normalsubgroup A$ definiujemy jako zbiór wszystkich sum $x+y$, gdzie $x\in I$ oraz $y\in J$. Jest to najmniejszy ideał zawierający $I$ oraz $J$. W ogólności, jeśli mamy jakąś rodzinę ideałów $I_\alpha$, to $\sum I_\alpha$ jest definiowane jako zbiór elementów $\sum x_\alpha$, gdzie $x_\alpha\in I_\alpha$. Znowu, jest to najmniejszy ideał zawierający wszystkie ideały $I_\alpha$.

Przekrój ideałów jest nadal ideałem, to wiemy, ale nie wiemy, że tworzą one pełną sałatę względem zawierania.

Produkt dwóch ideałów $I, J$ to ideał $IJ$ generowany przez wszystkie $xy$ dla $x\in I$ oraz $y\in J$. Możemy to uogólnić na zbiór wszystkich $\sum x_\alpha y_\alpha$ dla $x_\alpha\in I$ i $y_\alpha\in J$. Analogicznie możemy zapisać produkt dowolnej, skończonej rodziny ideałów. W szczególności, potęgi $I^n$ ideału $I$ to dobrze zdefiniowane ideały. 

Wszystkie powyżej zdefiniowane operacje sa przemienne i łączne. Co więcej, działa rozłączność mnożenia względem dodawania (czy tam na odwrót). Dodatkowo mamy prawo modułu(?) [\dyg{modular law}], czyli jeśli $J\subseteq I$ albo $L\subseteq I$, to
$$I\cap(J+L)=I\cap J+I\cap L.$$
Z ciekawych rzeczy, w $\Z$ $\cap$ i $+$ są rozdzielne względem siebie oraz $(I+J)(I\cap J)=IJ$, ale nie jest to regułą ogólną, zwykle tylko $(I+J)(I\cap J)\subseteq IJ$.

Dwa ideały $I$ oraz $J$ są \deff{względnie pierwsze} lub względnie maksymalne [\emph{coprime or comaximal}], jeżeli $I+J=(1)$. W takim przypadku mamy $I\cap J=IJ$. Jasno widać, że $I$ i $J$ są względnie pierwsze $\iff$ istnieją $x\in I$ oraz $y\in J$ takie, że $x+y=1$.
\smallskip

Niech $A_1,...,A_n$ będą pierścieniami. Wtedy ich iloczyn prosty \dyg{direct product}
$$A=\prod A_i$$
jest zbiorem wszystkich ciągów $x=(x_1,...,x_n)$ dla $x_i\in A_i$ i dodawaniem oraz mnożeniem po współrzędnych.

Niech $A$ będzie pierścieniem, a $I_1,...,I_n$ jego ideałami. Możemy zdefiniować homomorfizm
$$\phi:A\to \prod(A/I_i)$$
$$\phi(x)=(x+I_1,...,x+I_n).$$

\deff{\large Twierdzenie:}

\indent I. Jeżeli $I_i,I_j$ są względnie pierwsze, wtedy $\prod I_i=\bigcap I_i$

\indent II. $\phi$ jak wyżej jest "na" $\iff$ $I_i, I_j$ są względnie pierwsze

\indent III. $\phi$ jest 1-1 $\iff$ $\bigcap I_i=(0)$

\textbf{Dowód:}

I. Indukcją po $n$. Przypadek dla $n=2$ jest już rozpykany. Załóżmy, że $n>2$. Niech $J=\prod_{i=1}^{n-1}I_i=\bigcap I_i$. 

STRONA 7 ŚRODEK DOWODU