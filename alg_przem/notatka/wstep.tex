Pomoce dydaktyczne:

\href{https://www.youtube.com/playlist?list=PL8yHsr3EFj53rSexSz7vsYt-3rpHPR3HB}{playlista z losowymi wykladami}

\section{Wstęp}

\subsection{Gradacje, filtracje}

Są pierścienie i ideały, $St_K$ to struktura pierścienia, dalej na środku mamy przykład pierścienia.

Są pierścienie, które są zgradowane i są pierścienie, które są zfiltrowane. Czemu nas to interesuje? bo mamy ciąg liczb. Stanley jest zgradowany.

Jak jest zgradowany, to ma ciąg wymiarów. Jakie są wymiary stopni gradacji?

Liczymy $\sum\limits_{i=0}^\infty dimR_i\cdot t^i$ dla punktu i dwóch punktów. Jeżeli $K$ to ${1\over1-t}$, dla połączonych dwóch punktów to $\left({1\over1-t}\right)^2$, a dla dwóch niepołączonych punktów ${2\over1-t}-1$.
\medskip

Topologia <3

Pierwszy rodzaj pierścieni pojawiających się w topologii to twory oznaczane
$$H^\cdot(X, R),$$
gdzie $R$ to pierścień, a $X$ to przestrzeń topologiczna. To jest chwilowo blackbox i my potem to wytłumaczymy. To coś jest zgradowane.

Taki pierścień to na przykład $R[X]/x^2=0$.To jest pierścień wielomianów jednej zmiennej. Teraz dla dwóch zmiennych $R[X, Y]:/X^2=0=y^2, xy=-yx$. Pierwsze odpowiada okręgowi $[S^1]$, a drugie odpowiada torusowi $[T^2]$. Czyli torusowi przypisujemy taki pierścień, o to mniej więcej tutaj chodzi.

Te obiekty, o których algebra przemienna chce mówić to są zgradowane przemienne obiekty. Czyli $R=\otimes R_i$, a potem przemienność ma być taka, że $r_ir_j=(-1)^{\alpha ij}r_jr_i$. Możemy na przykład mieć $\alpha=1$.

Pierścienie grupowe: $k[G]$, gdzie $k$ jest być może ciałem, a $G$ jest grupą. I teraz jeżeli $G$ jest nieprzemienne, to to jest bardzo nieprzemienne. Teoria reprezentacji zajmuje się badaniem takich pysi. W topologii jak mamy przestrzeń $X$, to nad nią wisi $\overline X$ razem z działaniem grupy $G$ takie, że $\overline X/G=X$ i to się nazywa pokryciem uniwersalnym. Iloraz jest $X$ i to działa nakrywająco, to znaczy każda orbita $G$ to jest zawsze otoczenie punktu który wybraliśmy. Zawsze możemy rozłożyć to jakoś trudne słowo, triangulacja. To co działa początkowo na $\overline X$, to działą teraz na traingulacji XDDD. $C_k(\overline X)$ to formalne kombinacje liniowe o współczynnikach w $k[G]$ $k$-sympleksów. Operatory brzegów. Mam wrażenie, że to akurat jest jakaś losowa baja o trójkącikach.