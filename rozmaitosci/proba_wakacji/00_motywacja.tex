
\begin{illustration}
\draw[rounded corners=35pt](6,-1)--(4.2,-1)--(2,-2)--(0,0)--(2,2)--(4.2,1)--(7,1)--(9.2,2)--(11,0)
--(9,-2)--(6,-1);
\draw (1.5,0.2) arc (175:315:1cm and 0.5cm);
\draw (3,-0.28) arc (-30:180:0.7cm and 0.3cm);
\draw (5.8,0) arc (0:360:0.5cm and 1cm);
\draw (7.5,0.2) arc (175:315:1cm and 0.5cm);
\draw (9,-0.28) arc (-30:180:0.7cm and 0.3cm);
%\node (a) at (-13:5.8) {$\partial M$};
%\node (a) at (26:2.5) {$\tilde{M}$};
\end{illustration}

\textbf{\large\color{orange}Motywacja}

Rozmaitości dostarczają narzędzi do badania abstrakcyjnych powierzchni o dowolnym wymiarze. Pozwalają działać na przestrzeniach opisywalnych (lokalnie) za pomocą ustalonej skończonej liczby parametrów, takich jak na przykład przestrzenie konfuguracyjne układów fizycznych. W bardziej przyziemnym ujęciu są to podzbiory $\R^n$ lub $\C^n$ opisywane równaniami algebraicznymi, jak np. $x^2+y^2+z^2$ w $\R^3$ (znana jako $S^2$).

Dzięki sprowadzaniu działań między rozmaitościami do działań między przestrzeniami $\R^n$ możemy rozszerzać Analizę Matematyczną I, II i III na przyglądanie się funkcjom między abstrakcyjnie wyrażonymi przestrzeniami.

\begin{flushright} 
Ahoj przygodo!
\end{flushright}
