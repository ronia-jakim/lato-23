\section{Podrozmaitości}

\begin{definition}
  Podzbiór $N\subseteq M^n$ dla gładkiej rozmaitości $M$ jest \important{podrozmaitością wymiaru $n$}, jeśli każdy punkt $p\in N$ posiada mapowe otoczenie otwarte $U_p\subseteq M$ oraz mapę $\phi:U_p\to V=\phi(U_p)\subseteq\R^m$ takie, że
  $$\phi(U_p\cap N)=\{(x_1,...,x_m)\in V\;:\;x_{n+1}=...=x_m=0\}$$
\end{definition}

\begin{illustration}
  %\draw[orange, step=0.2] (-2, -2) grid (6, 6);

  \path (-1, -2) edge [bend left=20] (1, -1.5);
  \path (1, -1.5) edge [bend right=20] (0.5, 0.5);
  \path (0.5, 0.5) edge [bend right=20] (-1.5, 0);
  \path (-1.5, 0) edge [bend left=20] (-1, -2);

  %\filldraw[pattern=crosshatch] (0, -0.3) circle (0.5);
  \filldraw[color=orange, pattern=crosshatch, pattern color=orange, smooth cycle, rounded corners=0.5mm] plot coordinates{(-0.5, -0.3) (-0.1, -0.8) (0.7, -0.2) (0, 0.3)};

  \draw[->] (3, -0.5)--(3, 3.5);
  \draw[->] (1, 1.5)--(5, 1.5);

  \draw[orange, pattern={Stars[points=7, angle=20, radius=3pt, distance=7pt]}, pattern color = orange!40] (3, 1.5) circle (1.3);

  \path (-1, 1) edge [bend left=10] (-0.2, 0.3);
  \node at (-1.2, 1.3) {$\color{orange}U_p$};

  \node at (-1.3, -2) {$M$};
  \path[very thick, green] (-1, -1.2) edge [bend right=30] (-0.2, -0.5);
  \path[very thick, green] (-0.2, -0.5) edge [bend left=30] (0.6, 0.4);
  \node at (-0.8, -0.8) {$\color{green}N$};

  \draw[ultra thick, green] (1.7, 1.5)--(4.3, 1.5);
  \draw[ultra thick, green] (1.7, 1.55)--(4.3, 1.55);
  \draw[ultra thick, green] (1.7, 1.45)--(4.3, 1.45);

  \node (A) at (4, -0.5) {$\color{green}\phi(U_p\cap N)$};
  \path [<-, green] (3.5, 1.4) edge [bend right=10] (A);

  \path[->] (0.2, -0.3) edge [bend right=20] node [midway, above] {$\phi$} (2, 1);

  \node at (3.4, 3.3) {$\R^{m-n}$};
  \node at (4.7, 1.2) {$\R^n$};
  \node at (2, 0) {$\R^m$};

  \node at (4.3, 2.3) {$\color{orange}V$};
\end{illustration}

Możemy to również rozumieć, że wokół każdego $p\in N$ istnieje lokalny układ współrzędnych $(x_1,...,x_m)$ na otwartym otoczeniu $U_p\subseteq M$ taki, że $U_p\cap N$ wyraża się w tym układzie jako $\{x_{n+1}=...=x_m=0\}$.

\begin{remark}
  Każda $n$-wymiarowa podrozmaitość $N\subseteq M^m$ jest $n$-wymiarową gładką rozmaitością.
\end{remark}

\begin{proof}
  Wybierzemy na $N$ atlas, a następnie udowodnimy jego zgodność.

  Jako mapy wybierzemy pary postaci $(U_p\cap N, \Pi_n^m\circ \phi)$, gdzie $(U_p,\phi)$ są mapami na $M$ jak w definicji wyżej, natomiast 
  $$\Pi_n^m:\R^m\to \R^n$$
  jest rzutowaniem, tzn:
  $$\Pi_n^m(x_1,...,x_n,x_{n+1},...,x_m)=(x_1,...,x_n).$$
  W takim razie $\Pi_n^m\circ\phi$ to pierwsze $n$ współrzędnych gładkiej mapy $\phi$, czyli jest gładką funkcją z $N$ w $\R^n$.

  \begin{illustration}
    %\draw[orange, step=0.2] (-2, -2) grid (6, 6);

    \path (-1, -2) edge [bend left=20] (2, -1.5);
    \path (2, -1.5) edge [bend right=20] (1.5, 0.5);
    \path (1.5, 0.5) edge [bend right=20] (-1.5, 0);
    \path (-1.5, 0) edge [bend left=20] (-1, -2);

    \path[very thick, green] (-1, -1.2) edge [bend right=30] (0.3, -0.5);
    \path[very thick, green] (0.3, -0.5) edge [bend left=30] (1.6, 0.4);
    
    \draw (-0.1, -0.5) ellipse (0.6 and 0.5);
    \node at (-0.7, -1) {$U_p$};
    \filldraw (0.1, -0.75) circle (1.5pt) node [below] {$p$};

    \path[->] (0.1, -0.6) edge [bend left=60] node [midway, above] {$\psi=\Pi_n^m\phi$} (2.2, 0.9);

    \draw[rotate around={30:(0.7, -0.2)}] (0.7, -0.2) ellipse (0.6 and 0.4);
    \node at (1.3, -0.7) {$U_p'$};
    \filldraw (0.7, 0) circle (1.5pt) node [below] {$p'$};

    \path[->] (0.7, -0.1) edge [bend right=60] (2.2, -2.1);
    \node at (0.5, -1.7) {$\psi'=\Pi_n^m\phi'$};

    \draw[->] (3, 0)--(3, 2);
    \draw[->] (2, 1)--(4, 1);

    \draw (3, 1) ellipse (0.6 and 0.4);
    \draw[ultra thick, green] (2.4, 1)--(3.6, 1);
    
    \node at (3.6, 1.4) {$V$};

    \draw[->] (3, -3)--(3, -1);
    \draw[->] (2, -2)--(4, -2);

    \draw (3, -2) ellipse (0.6 and 0.5);
    \draw[ultra thick, green] (2.4, -2)--(3.6, -2);
    
    \node at (3.6, -1.5) {$V'$};
  \end{illustration}

  Wybierzmy mapy $(U_p,\phi)$ i $(U_p',\phi')$ jak wyżej i posługujmy się notacją jak na ilustracji. Chcemy sprawdzić, czy $\psi'\psi^{-1}$ jest mapą gładką.
  $$\psi'\psi^{-1}=(\Pi_n^m\phi')(\Pi_n^m\phi)^{-1}=(\Pi_n^m\phi')(\phi^{-1}i),$$
  gdzie $i:\psi(N\cap U_p)\to V$ jest włożeniem zadanym wzorem
  $$i(x_1,...,x_n)=(x_1,...,x_n, \underbrace{0...,0}_{m-n\text{ zer}}).$$
  Wiemy już, że $\Pi_n^m$, $\phi$, $\phi'$ oraz $i$ są gładkie, czyli również $\psi'\psi^{-1}$ jako ich złożeniem jest funkcją gładką.
\end{proof}

\begin{example}
  \item Dla $m$-rozmaitości $M$ oraz $n$-rozmaitości $N$ i otwartego $U\subseteq M$, graf gładkiej funkcji $f:U\to N$
    $$\Gamma(f)=\{(x,f(x))\;:\;x\in U\}\subseteq M\times N$$
    jest $m$-podrozmaitością $M\times N$.
\end{example}

\subsection{Podrozmaitości zadane przez odwzorowanie włożenia}

\begin{definition}
  Odwzorowanie $f:N\to M$ jest \important{immersją}, gdy rząd $f$ w każdym punkcie jest równy wymiarowy $\dim N$, tzn.
  $$(\forall\;x\in N)\; rank(f,x)=\dim N$$
\end{definition}

Oczywiście, aby $f$ było immersją, musimy mieć $\dim(N)\leq\dim(M)$ oraz dla każdego $p\in N$ różniczka
$$df_p:T_pN\to T_{f(p)}M$$
musi być \acc[b]{różnowartościowa}.

\begin{definition}
  Immersję $f$ nazywamy \important{gładkim włożeniem}, jeśli jest homeomorfizmem na swój obraz.
\end{definition}

\begin{example}
  \item Wstęga Mobiusa bez brzegu $M=\R\times(-1,1)/\Z$ może być włożona w $\R^3$.

    \begin{proof}
      Działanie $\Z$ na $\R\times(-1,1)$ jest zdefiniowane jako $k(x,y)=(x+k\cdot2\pi, (-1)^k\cdot y)$. Dla wybranego $(x,y)\in M$ rozważmy funkcje
      $$N(x)=(\;\cos x,\; \sin x,\; 0\;)$$
      %$$T(x)=(\;-\sin x,\;\cos x,\;0\;)$$
      $$V(x)=(\;0,\;0,\;1\;)$$
      $$K(x)=\sin \frac{x}{2}\cdot N(x)+\cos \frac{x}{2}\cdot V(x)$$
      %\begin{illustration}
      %  \begin{axis}[axis lines=middle,
      %      xmin=-2.5,
      %      xmax=2.5,
      %      ymin=-2.5,
      %      ymax=2.5,
      %      zmin=-2.5,
      %      zmax=2.5,
      %      xlabel=$x$,
      %      ylabel=$y$,
      %      zlabel=$z$
      %    ]
      %    \addplot3[ samples=100] ({cos(deg(x))}, {sin(deg(x))}, 0);
      %    \addplot3[
      %      mark=*, 
      %      green
      %    ] coordinates {(0, 0, 1)};
      %    \addplot3[mark=none] coordinates{(}
      %  \end{axis}
      %\end{illustration}

      Rozważmy funkcję 
      $$f:\R\times(-1,1)\to\R^3$$
      zadaną przez
      $$f(x,y)=2\cdot N(x)+y\cdot K(x)=(\;2\cos x+ y\cdot\sin\frac{x}{2},\; 2\sin x+y\cdot\sin\frac{x}{2},\; y\cdot\cos\frac{x}{2}\;)$$
     % \begin{illustration}
     %   \begin{axis}[
     %       axis lines=middle,
     %       xmin=-2.5,
     %       xmax=2.5,
     %       ymin=-2.5,
     %       ymax=2.5,
     %       zmin=-2.5,
     %       zmax=2.5,
     %       xlabel=$x$,
     %       ylabel=$y$,
     %       zlabel=$z$
     %       ]
     %     \addplot3[
     %       mark=none,
     %       surf,
     %       opacity=0.8
     %       ] ({cos(deg(x))}, {sin(deg(x))}, 0);

     %     \addplot3[
     %     mark = none,
     %     surf,
     %     opacity=0.2
     %     ](
     %       {2*cos(deg(x))+y*sin(deg(x/2))},
     %       {2*sin(x)+y*sin(deg(x/2))},
     %       {y*cos(deg(x/2))}
     %     );


     %     \addplot3[
     %       mark=none,
     %       %surf,
     %       opacity=0.6,
     %       color=blue,
     %       %domain=-2:2
     %       ] ({sin(deg(x/2))*cos(deg(x))}, {sin(deg(x/2)) *sin(deg(x))}, {cos(deg(x/2))});
     %   \end{axis}
     % \end{illustration}

      $f$ jest immersją pasa $\R\times(-1,1)$ w $\R^3$. Wystarczy sprawdzić rząd $f$ w dowolnym punkcie $(x,y)$:
      \begin{align*}
        D_{(x,y)}f=&
        \begin{bmatrix} 
          \frac{1}{2}y\cos\frac{x}{2}-2\sin x & \sin\frac{x}{2}\\
          \frac{1}{2}y\cos\frac{x}{2}+2\cos x & \sin\frac{x}{2}\\
          -\frac{1}{2}y\sin\frac{x}{2} & \cos\frac{x}{2}
        \end{bmatrix}
      \end{align*}
      łatwo zauważyć, że kolumny tej macierzy są liniowo niezależne gdy $y\neq 0$, gdyż wtedy ostatnie współrzędne ($\sin\frac{x}{2}$ i $\cos\frac{x}{2}$) są liniowo niezależnymi funkcjami. Jeśli natomiast $y= 0$, to aby wektory były liniowo zależne, musiałoby istnieć $a,b$ takie, że
      $$\begin{cases}\frac{a}{2}y\cos\frac{x}{2}-2a\sin x+b\sin\frac{x}{2}=0\\
      \frac{a}{2}y\cos\frac{x}{2}+2a\sin x+b\sin\frac{x}{2}=0\end{cases}$$
      po dodaniu obu równań dostajemy, że
      $$0=ay\cos\frac{x}{2}+(2b)\sin\frac{x}{2}=(2b)\sin\frac{x}{2}$$
      dla dowolnego $x$ (bo $y=0$), ale tak się dzieje tylko jeśli $2b=0$.

      Sprawdźmy teraz, czy $f$ zachowuje działanie grupy $\Z$ na $\R\times(-1,1)$:
      \begin{align*}
        f(k(x,y))=&f(x+k2\pi,(-1)^ky)=\\
        =&\begin{bmatrix}
          2\cos(x+2k\pi)+(-1)^ky\sin(\frac{x}{2}+k\pi)\\
          2\sin(x+2k\pi)+(-1)^ky\sin(\frac{x}{2}+k\pi)\\
          (-1)^k\cos(\frac{x}{2}+k\pi)
        \end{bmatrix}
      \end{align*}
      oczywiście czynniki $\cos(x+2k\pi)=\cos(x)$ pozostają bez zmiany. Tak samo $\sin(\frac{x}{2}+k\pi)$ dla parzystego $k$. Dla $k$ nieparzystego natomiast mamy $\sin(\frac{x}{2}+k\pi)=-\sin\frac{x}{2}$, czyli 
      $$(-1)^k\sin(\frac{x}{2}+k\pi)=(-1)^k\cdot(-\sin\frac{x}{2})=\sin\frac{x}{2}$$
      tak jak chcieliśmy. Tak samo dla $\cos(\frac{x}{2}+k\pi)=-\cos\frac{x}{2}$, stąd
      $$f(k(x,y))=f(x,y)$$
      a więc istnieje funkcja indukowana przez $f$
      $$\overline{f}:\R\times(-1,1)/\Z\to\R^3$$
      o własności $rank(\overline{f},x)=rank(f,x)$.

      Nietrudno też sprawdzić, że $\overline{f}$ jest homeomorfizmem na swój obraz, co zostaje zostawione jako ćwiczenie dla czytelnika.
    \end{proof}
\end{example}
