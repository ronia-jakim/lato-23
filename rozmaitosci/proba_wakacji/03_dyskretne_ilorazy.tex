\section{Dyskretne ilorazy rozmaitości}

\subsection{Klejenie rozmaitości wzdłuż brzegu}

\begin{theorem}[otoczenie kołnierzowe] Niech $M$ będzie gładką $n$-rozmaitościa, a $B$ niech będzie kompotentą brzegu $\partial M$. Wtedy istnieje dyfeomorficzne (dyfeomorfizm na obraz) włożenie
  $$K:B\times[0, 1)\to M$$
  na otwarte otoczenie $U$ komponenty $B$ w $M$ takie, że $K(x, 0)=x$ dla $x\in B$.
\end{theorem}

\begin{proof}Dowód za kilka wykładów przy pomocy potoków wektorowych.

  \begin{illustration}
    \filldraw[orange!25] (6.5, 0) arc (0:360: 0.5cm and 1cm);
    \filldraw[orange!25] (6, 1)--(7, 1)--(7, -1)--(6, -1);
    \draw[rounded corners=35pt](7,-1)--(4.2,-1)--(2,-2)--(0,0) -- (2,2)--(4.2,1)--(7,1);
    \draw (1.5,0.2) arc (175:315:1cm and 0.5cm);
    \draw (3,-0.28) arc (-30:180:0.7cm and 0.3cm);
    \filldraw[color=black, fill=white] (7.5,0) arc (0:360:0.5cm and 1cm);
    \node (a) at (20:2.5) {$M$};
    %\node (a) at (-12:7.5) {$\partial M=N$};
    \node at (7.8, 0) {$B$};

    \node at (6, 0) {$\color{orange}U$};
  \end{illustration}
\end{proof}

Jeśli $M_1,B_1$ oraz $M_2,B_2$ są jak wyżej oraz istnieje dyfeomorfizm
$$f:B_1\to B_2$$
to możemy zdefiniować relację równoważności
$$B_1\ni x\sim f(x)\in B_2$$
oraz stworzyć rozmaitość:
$$M_1\cup_fM_2=M_1\sqcup M_2/\sim.$$

\textbf{Struktura} na $M_1\cup_fM_2$ jest częściowo odziedziczona po $M_1$ i $M_2$. Dodatkowo sklejamy zbiory $U_i$ utożsamiając je z produktami $B_i\times[0,1)$ za pomocą $B_i$:
$$K_i:B_i\otimes [0,1)\to M_i$$

  \begin{illustration}
    \filldraw[orange!25] (6.5, 0) arc (0:360: 0.5cm and 1cm);
    \filldraw[orange!25] (6, 1)--(7, 1)--(7, -1)--(6, -1);
    \draw[orange] (6, 1) arc (90:270:0.5cm and 1cm);
    \draw[orange, dashed] (6, 1) arc (90:-90:0.5cm and 1cm);
    \draw[orange] (6.98, 1) arc (90:270:0.5cm and 1cm);
    \draw[rounded corners=35pt](7,-1)--(4.2,-1)--(2,-2)--(0,0) -- (2,2)--(4.2,1)--(7,1);
    \draw (1.5,0.2) arc (175:315:1cm and 0.5cm);
    \draw (3,-0.28) arc (-30:180:0.7cm and 0.3cm);
    \filldraw[blue!25] (7.5,0) arc (0:360:0.5cm and 1cm);
    \node (a) at (20:2.5) {$M_1$};
    %\node (a) at (-12:7.5) {$\partial M=N$};
    %\node at (7.8, 0) {$B_1$};

    \node at (6, 0) {$\color{orange}U_1$};

    \filldraw[blue!25] (8,0) arc (0:360:0.5cm and 1cm);
    \filldraw[blue!25] (7.5, 1)--(7, 1)--(7, -1)--(7.5, -1);
    %\draw[blue, thick] (7.01, 1) arc (90:270:0.5cm and 1cm);
    \draw[blue] (8,0) arc (0:360:0.5cm and 1cm);
    \node at (7.5, 0) {$\color{blue}U_2$};

    \draw[rounded corners=35pt, rotate around={180:(6.9, 0)}](7,-1)--(4.2,-1)--(2,-2)--(0,0) -- (2,2)--(4.2,1)--(7,1);

    \draw (10.5,0.2) arc (175:315:1cm and 0.5cm);
    \draw (12,-0.28) arc (-30:180:0.7cm and 0.3cm);
    \node at (11.5, 0.8) {$M_2$};

    \node at (6.8, -2) {$W$};

    \path[->] (6, -0.5) edge [bend right=20] (6.75, -1.7);
    \path[->] (7.5, -0.5) edge [bend left=20] (6.85, -1.7);
  \end{illustration}

Na $M_1\cup_fM_2$ istnieją trzy rodzaje map:
\begin{enumerate}
  \item dla dowolnej mapy $(U,\phi)$ na $M_1$ rozważamy jej obcięcie do $U\setminus B_1$
  \item dla dowolnej mapy $(V,\psi)$ na $M_2$ rozważamy jej obcięcie do $V\setminus B_2$
  \item dla dowolnej mapy $(W,\xi)$ na $B_1$ i $\xi:W\to\overline{W}\subseteq\R^{n-1}$ rozważamy zbiór 
    $$[W\times[0,1)]\cup_{f\restriction W}[f(W)\times[0,1)]=\hat{W}\subseteq M_1\cup_fM_2$$
    z mapą 
    $$\hat{\xi}:\hat{W}\to\overline{\hat{W}}\subseteq\R^n$$
    $$\hat{\xi}(x, t)=\begin{cases}(\xi(x),-t)&(x,t)\in U_1\\(\xi(f^{-1}(x)), t)&(x,t)\in U_2\end{cases}$$
    Mamy $\hat{\xi}(x, 0)=\hat{\xi}(f(x), 0)$, więc $\hat{x}$ jest dobrze zdefiniowane w punktach sklejenia. 
    $$\overline{\hat{W}}=\overline{W}\times(-1,1)\subseteq\R^n\times (-1,1)\subseteq\R^{n+1}$$
    zaś $\hat{\xi}:\hat{W}\to\overline{\hat{W}}$ jest homeomorfizmem.
\end{enumerate}

Sprawdzenie gładkiej zgodności map z podpunktów 1, 2 i 3 zostanie pominięte.

Rozmaitość $M_1\cup_fM_2$ wydaje się zależeć jednocześnie od wyboru $f$ oraz otoczeń kołnierzowych $K_i$ komponent brzegów $B_i$. W rzeczywistości jednak, $M_1\cup_fM_2$ jest takie same z dokładnością do dyfeomorfizmu dla dowolnych wyborów $K_i$:

\begin{fact}$ $\newline
  \begin{enumerate}
    \item Jeśli $K_1,K_1'$ są podobnie położone w $M_1$, tzn. istnieje $h:M_1\to M_1$ dyfeomorfizm taki, że 
      $$K_1'\restriction B_1\times[0,1\frac12)=h\circ K_1\restriction B_1\times[0,\frac12),$$ 
      to wtedy 
      $$M_1\cup_{f,K_1,K_2}M_2\cong M_1\cup_{f,K_1',K_2}M_2.$$
      Analogicznie gdy weźmiemy $K_2,K_2'$. [dowód: ćwicznia] 
    \item Każde dwa otoczenia kołnierzowe komponenty $B_1$ brzegu $\partial M$ są podobnie położone. [dowód trudny] 
    \item Ustalmy otoczenia kołnierzowe $K_1,K_2$. Jeśli $f_0,f_1:B_1\to B_2$ są izotopijnymi dyfeomorfizmami, tzn. istnieje gładkie $F:[0,1]\times B_1\to B_2$ takie, że $F(0)=f_0$ a $F(1)=f_1$, wtedy 
      $$M_1\cup_{f_0,K_1,K_2}M_2\cong M_1\cup_{f_1,K_1,K_2}M_2.$$
      [dowód łatwy]
  \end{enumerate}
\end{fact}

\subsection{Suma spójna rozmaitości}

Niech $M_1,M_2$ będą rozmaitościami wymiaru $n$. Weźmy $D_i\subseteq M_i$, czyli kule $n$-wymiarowe zawarte w otoczeniach mapowych. Oznaczmy $B_i=\partial D_i\cong S^{n-1}$ jako komponenty brzegu rozmaitości $M_i\setminus Int(D_i)$. Niech
$$f:B_1\to B_2$$
będzie dyfeomorfizmem. Oznaczamy wówczas
$$[M_1\setminus Int(D_1)]\cup_f[M_2\setminus Int(D_2)]=\color{blue}M_1\#M_2$$
jako \important{sumę spójną} rozmaitości $M_1$ i $M_2$.

\begin{illustration}
  \draw[rounded corners=35pt, rotate around={180:(4, 0)}](6,-1)--(4.2,-1)--(2,-2)--(0,0) -- (2,2)--(4.2,1)--(6,1);
  \draw (3,0.2) arc (175:315:1cm and 0.5cm);
  \draw (4.5,-0.28) arc (-30:180:0.7cm and 0.3cm);
  \draw (2.5,0) arc (0:360:0.5cm and 1cm);
  \node (a) at (4.5, 0.6) {$M_1$};
  \filldraw[color=blue, fill=blue!30] (6, 0.2) circle (0.6);

  \draw (10, 0) ellipse (2cm and 1.3cm);
  \filldraw[color=blue, fill=blue!30, rotate around={30:(8.9, 0.5)}] (8.9, 0.5) ellipse (0.6 and 0.4);
  \draw (9.5, 0) arc (160:20:0.8 and 0.4);
  \draw (9.3, 0.2) arc (200:340:1 and 0.6);
  \node at (10.5, 0.9) {$M_2$};
  \node at (6, 1) {$D_1$};
  \node at (8.9, 1.2) {$D_2$};
\end{illustration}

\begin{illustration}
  \draw[rounded corners=35pt, rotate around={180:(4, 0)}](6,-1)--(4.2,-1)--(2,-2)--(0,0) -- (2,2)--(4.2,1)--(6,1);
  \draw (3,0.2) arc (175:315:1cm and 0.5cm);
  \draw (4.5,-0.28) arc (-30:180:0.7cm and 0.3cm);
  \draw (2.5,0) arc (0:360:0.5cm and 1cm);
  \node (a) at (4.5, 0.6) {$M_1$};
  %\filldraw[color=blue, fill=blue!30] (7, 0) circle (0.6);

  \draw (9, 0) ellipse (2cm and 1.3cm);
  %\filldraw[color=blue, fill=blue!30, rotate around={30:(8.9, 0.5)}] (8.9, 0.5) ellipse (0.6 and 0.4);
  \draw (8.5, 0) arc (160:20:0.8 and 0.4);
  \draw (8.3, 0.2) arc (200:340:1 and 0.6);
  \node at (10, 0.9) {$M_2$};
  %\node at (6, 1) {$D_1$};
  %\node at (8.9, 1.2) {$D_2$};
  \filldraw[color=blue, fill=blue!25] (7.3, 0) ellipse (0.4 and 0.7);
\end{illustration}

\begin{illustration}
  \draw[rounded corners=35pt, rotate around={180:(4, 0)}](6,-1)--(4.2,-1)--(2,-2)--(0,0) -- (2,2)--(4.2,1)--(6,1);
  \draw (3,0.2) arc (175:315:1cm and 0.5cm);
  \draw (4.5,-0.28) arc (-30:180:0.7cm and 0.3cm);
  \draw (2.5,0) arc (0:360:0.5cm and 1cm);
  \node (a) at (4.5, 0.6) {$M_1$};
  %\filldraw[color=blue, fill=blue!30] (7, 0) circle (0.6);

  \draw (9, 0) ellipse (2cm and 1.3cm);
  %\filldraw[color=blue, fill=blue!30, rotate around={30:(8.9, 0.5)}] (8.9, 0.5) ellipse (0.6 and 0.4);
  \draw (8.5, 0) arc (160:20:0.8 and 0.4);
  \draw (8.3, 0.2) arc (200:340:1 and 0.6);
  \node at (10, 0.9) {$M_2$};
  %\node at (6, 1) {$D_1$};
  %\node at (8.9, 1.2) {$D_2$};
  \filldraw[white] (7.3, 0) ellipse (0.5 and 1);
  \draw (7.08, 0.9) arc(250:285:0.8);

  \draw (7.08, -0.9) arc(110:75:0.8);
  \draw[dashed] (7.3, 0) ellipse (0.3 and 0.85);
\end{illustration}

\begin{remark}$ $\newline
  \begin{enumerate}
    \item Jeśli $M_i$ jest rozmaitością spójną, to $M_i\setminus Int(D_i)$, z dokładnością do dyfeomorfizmu, nie zależy od wyboru dysku $D_i$.
    \item Istnieją dokładnie $2$ klasy izotopii dyfeomorfizmów $f:S^{n-1}\to S^{n-1}$: te zachowujące orientację oraz te, które orientacji nie zachowują.
    \item Są co najwyżej dwie rozmaitości będące sumą spójną $M_1\#M_2$. W przypadku rozmaitości zorientowanych, jedna z nich jest preferowana.
  \end{enumerate}
\end{remark}

\textbf{Klasyfikacja zamkniętych powierzchni spójnych} (czyli zwarte $2$-wymiarowe rozmaitości bez brzegu):
\begin{enumerate}
  \item Powierzchnie orientowalne: $S^2, T^2, T^2\#T^2,T^2\#T^2\#T^2,...$
  \item Powierzchnie nieorientowalne $\R P^2=S^2/\Z_2,\R P^2\#\R P^2,...$
\end{enumerate}
Powierzchnie z powyższej listy są parami niedyfeomorficzne. Każda zamknięta powierzchnia jest dyfeomorficzna z jedną z tej listy.

\textbf{$3$-rozmaitości:}\marginpar{Poniżej bardzo luźne opisy z wikipedii. Dokładniejsze opisy lepiej jest doczytać w literaturze.}
\begin{itemize}
  \item[\PHtunny] \acc[b]{Dehn surgery:} niech $M$ będzie $3$-wymiarową rozmaitością $M$ z kolekcją węzłów (podrozmaitości $S^n$ dyfeomorficznych do skończonej rozłącznej sumy $S^j$) $L=L_1\cup...\cup L_k$. Rozmaitość $M$ wywiercona wzdłuż tubowego otoczeniem $L$ posiada $k$-wiele komponentów brzegu $T_1\cup...\cup T_k$. Chirurgia Dehna polega na wywierceniu z $M$ tubowego otoczenia $L$ wraz ze sklejeniem każdej z komponent brzegu $T_1\cup...\cup T_k$ w jeden torus [to jest Dehn filling i jest wiele sposobów na wytworzenie go].
  \item[\PHtunny] \acc[d]{Rozkłady Heegaarda} [Heegaard's splittings] na zorientowanej $3$-rozmaitości z brzegiem $M$ polega na na podzieleniu jej na dwa handlebody [fidget spinnery; $3$-rozmaitości oriengowalne z brzegiem zawierające parami rozłączne włożone $2$-dyski takie, że rozmaitość wzdłuż nich przecięta jest $S^3$].
    \begin{illustration}
      \filldraw[color=black, fill=white] (0, 0) ellipse (1.8 and 1.5);
      \filldraw[color=black, fill=white] (2.5, 0) ellipse (1.8 and 1.5);
      \filldraw[white] (1.25, 0) ellipse (0.8 and 1.2);
      \filldraw[color=black, fill=white] (1.25, -2.2) ellipse (1.8 and 1.5);
      \filldraw[white] (1.25, -1.5) ellipse (1.9 and 1);
      \draw (0, 0) ellipse (0.6 and 0.5);
      \draw (2.5, 0) ellipse (0.6 and 0.5);
      \draw (1.25, -2.2) ellipse (0.6 and 0.5);
      \draw (1.1, 1.2) arc (240:300:0.3);
      \draw (-0.7, -1.38) arc (70:-30:0.35);
      \draw (3.2, -1.38) arc (110:210:0.35);
      \node at (4.8, -2) {\scriptsize genus $3$ handlebody};
      \path [->] (4.8, -2.3) edge [bend left=40] (3.2, -3);
    \end{illustration}
\end{itemize}

\subsection{Działanie grupy dyfeomorfizmów}

\begin{definition}[grupa dyfeomorfizmów] Grupa $G$ dyfeomorfizmów $M$ to zbiór dyfeomorfizmów $g:M\to M$ zamknięty na składanie i branie odwrotności. Mówimy wtedy, że $G$ działa na $M$ przez dyfeomorfizmy.
\end{definition}

\begin{definition}[orbita] \important{Orbitą} punktu $x\in M$ względem działania $G$ na $M$ nazywamy zbiór
  $$\color{blue}G(x)=\{g(x)\;:\;g\in G\}$$
\end{definition}

\begin{remark} Orbity $G(x)$ i $G(y)$ są albo rozłączne, albo pokrywają się.
\end{remark}

Rodzina wszystkich orbit stanowi \acc[b]{rozbicie} rozmaitości $M$ na podzbiory.

\begin{definition}[przestrzeń ilorazowa działania $G$ na $M$] \important{Przestrzeń ilorazowa} działania $G$ na $M$ to przestrzeń, której punktami są orbity $G(x)$:
  $$\color{blue}M/G=\{G(x)\;:\;x\in M\}$$
  zaś topologia jest ilorazowa, tzn. \acc[i]{zbiór orbit jest otwarty} w $M/G$ $\iff$ suma tych orbit stanowi otwarty podzbiór w $M$.
\end{definition}

Jeśli $U\subseteq M$ jest otwartym podzbiorem, to 
$$\color{blue}G(U)/G=\{G(x)\;:\;x\in U\}$$
jest otwarty w $M/G$ i każdy zbiór otwarty w $M/G$ jest takiej postaci. Kiedy $\set{B}$ jest bazą topologii w $M$, to rodzina
$$\{G(U)/G\;:\;U\in\set{B}\}$$
jest \acc[b]{bazą topologii} w $M/G$. Z tego powodu $M/G$ \textbf{zawsze posiada przeliczalną bazę}.

\begin{definition}[działanie nakrywające] Lokalną euklidesowość $M/G$ zapewnia warunek na \important{działanie nakrywające}:
  $$(\forall\;p\in M)(\exists\;p\in U\overset{otw.}{\subseteq}M)(\forall\;g_1,g_2\in G)\;g_1(U)\cap g_2(U)=\emptyset.$$
  Przy takim działaniu $G$ na $M$ podzbiór $G(U)/G$ jest otoczeniem $G(p)$ homeomorficzny z $U$. Oznacza to lokalną euklidesowość $M/G$.
\end{definition}

\begin{fact}
  Jeśli działanie grupy$G$ przez homeomorfizmy na rozmaitości $M$ jest nakrywające, to iloraz $M/G$ jest lokalnie euklidesowy dla wymiaru $n=dim(M)$.
\end{fact}

\begin{example}
\item Działanie grupy $\Z$ na $\R^2\setminus\{(0, 0)\}$ przez potęgi przekształcenia liniowego zadanego macierzą
  $$A=\begin{pmatrix}2&0\\0&\frac{1}{2}\end{pmatrix}$$
  jest nakrywające. W takim razie iloraz $(\R^2\setminus\{(0,0)\})/\langle A\rangle$ jest lokalnie euklidesowy wymiaru $2$. Jednak iloraz ten nie jest przestrzenią Hausdorffa, bo dla punktów na osobnych osiach $p$ i $q$ zbiory otwarte:
  
  \begin{illustration}
    \draw[<-] (0, 4.5)--(0, -4.5);
    \draw[->] (-4.5, 0)--(4.5, 0);
    \filldraw (0, 3) circle (1.5pt) node [right] {$q$};
    \filldraw (3, 0) circle (1.5pt) node [below] {$p$};

    \draw[blue, thick] (-0.5, 2.5) rectangle (0.5, 3.5);
    \node at (-0.3, 3.8) {$\color{blue}V$};
    \draw[blue, thick] (-1, 1.25) rectangle (1, 1.75);
    \draw[blue, thick] (-2, 0.625) rectangle (2, 0.875);
    \draw[blue, thick] (-4, 0.3125) rectangle (4, 0.4375);

    \draw[orange, thick] (2.5, -0.5) rectangle (3.5, 0.5);
    \node at (3.3, 0.8) {$\color{orange}U$};
    \draw[orange, thick] (1.25, -1) rectangle (1.75, 1);
    \draw[orange, thick] (0.625, -2) rectangle (0.875, 2);
    \draw[orange, thick] (0.3125, -4) rectangle (0.4375, 4);
  \end{illustration}
  nigdy nie mogą być rozłączne. Stąd rozmaitość ilorazowa $M/G$ nie może być nigdy rozmaitością różniczkowalną.
\end{example}

\begin{definition}[działanie wolne, właściwie nieciągłe]
  Działanie $G$ na $M$ przez dyfeomorfizm jest:
  \begin{enumerate}
    \item \important{wolne}, gdy dla każdego $g\in G\setminus\{id\}$ i dla każdego $x\in M$ $g(x)\neq x$
    \item \important{właściwie nieciągłe} [properly discontinuous], gdy dla każdego zwartego $K\subseteq M$ zbiór $\{g\in G\;:\;g(K)\cap K\neq \emptyset\}$ jest skończony.
  \end{enumerate}
\end{definition}

\begin{definition}[stabilizator]
  Dla $x\in M$ \acc[b]{stabilizator} (nadgrupa stabilizująca) punktu $x$ względem $G$ to
  $$\color{blue} Stab(x):=\{g\in G\;:\;g(x)=x\}$$
  jest automatycznie podgrupą $G$.
\end{definition}

\begin{fact}
  Działanie $G$ jest wolne $\iff$ wszystkie stabilizatory $stab(x)$ są trywialne ($=\{id\}$).
\end{fact}

\begin{example}
  \item Działanie grupy $\Z_n$ na $\R^2$ zadane przez potęgi obrotu o kąt $\frac{2\pi}{n}$ nie jest wolne.
  \item Działanie $G$ jest wolne $\iff$ dla każdego $x\in M$ odwzorowanie $G\to G(x)$ zadane przez $g\mapsto g(x)$ jest bijekcją.
\end{example}

\begin{fact}$ $\newline
  \begin{enumerate}[leftmargin=*]
  \item Gdy działanie $G$ przez homeomorfizmy na przestrzeni topologicznej lokalnie zwartej $X$ jest właściwie nieciągłe, to każda orbita $G(x)$ jest dyskretnym podzbiorem w $X$ (tzn. każdy $x\in G(x)$ ma otwarte otocznie $U$ takie, że $U\cap G(x)=\{x\}$).
  \item Jeśli działanie $G$ na $X$ jest właściwie nieciągłe i wolne, to jest też nakrywające.

  \item Jeśli $G$ działa przez homeomorfizmy na przestrzeni lokalnie zwartej $X$ w sposób właściwie nieciągły, to iloraz $X/G$ jest przestrzenią Hausdorffa.
\end{enumerate}
\end{fact}

\begin{example}
  \item Działanie grupy $\Z$ na $S^1$ przez potęgi obrotu o kąt $\alpha$ niewspółmierny z $2\pi$ jest wolne, ale ma orbity gęste w $S^1$, a więc nie są one dyskretne. Zatem działanie nie jest ani właściwie nieciągłe, ani wolne. Iloraz $s^1/\Z$ jest wtedy przestrzenią z topologią trywialną, więc nie jest rozmaitością.
  \item Działanie $\Z$ na $\R^2\setminus\{(0,0)\}$ przez potęgi
    $$A=\begin{pmatrix}2&0\\0&\frac{1}{2}\end{pmatrix}$$
    nie może być właściwie nieciągłe. Można to zobaczyć bezpośrednio:
    \begin{illustration}
      \draw[<-] (0, 4)--(0, -3.8);
      \draw[->] (-4, 0)--(4, 0);
      \node at (0.6, 3.8) {$K$};
      \node at (1.4, 2) {$A(K)$};
      \node at (2.8, 1.1) {$A^2(K)$};
      \draw[thick] (-0.5, 3.5)--(0.8, 3.5)--(0.8, -0.5)--(0.5, -0.5)--(0.5, 3)--(-0.5, 3)--cycle;
      \draw[thick] (-1, 1.75)--(1.6, 1.75)--(1.6, -0.25)--(1, -0.25)--(1, 1.5)--(-1, 1.5)--cycle;
      \draw[thick] (-2, 0.875)--(3.2, 0.875)--(3.2, -0.125)--(2, -0.125)--(2, 0.75)--(-2, 0.75)--cycle;
    \end{illustration}
    dla każdego $n\geq 1$ mamy $A^n(K)\cap K\neq \emptyset$.

    Jednakże tak zadane działanie $\Z$ na $\R^2\setminus\{(0,0)\}$ jest wolne i ma dyskretne orbity. W takim razie warunek, by działanie było wolne i miało dyskretne orbity nie jest wystarczający do tego, by iloraz był rozmaitością. Nie musi być nawet przestrzenią Hausdorffa, jak pokazaliśmy wcześniej.
\end{example}

\begin{fact}
  Jeśli $G$ jest działaniem na $M^n$ przez dyfeomorfizmy w sposób wolny i właściwie nieciągły, to iloraz $M/G$ jest
  \begin{itemize}
    \item lokalnie euklidesowy $n$-wymiarowy
    \item Hausdorffa
    \item ma przeliczalną bazę
  \end{itemize}
  Zatem $M/G$ jest $n$-wymiarową rozmaitością topologiczną.
\end{fact}
