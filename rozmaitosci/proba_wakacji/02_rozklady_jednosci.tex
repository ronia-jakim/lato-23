\section{Rozkład jedności}

Rozważmy rozmaitość z brzegiem $M$. \marginpar{Bardziej ogólnie, możemy chcieć dla dowolnego zbioru domkniętego $D\subseteq M$ znaleźć funkcję, która dla $p\in D$ jest równa zero, a na $M\setminus D$ ma wartości ściśle dodatnie.}Chcielibyśmy mieć narzędzie, które pozwoli nam tworzyć gładkie funkcje $f:M\to\R$ takie, że $f(p)=0$ gdy $p\in\partial M$ oraz $f(p)>0$ dla dowolnego $p\in Int(M)$.

\begin{illustration}
  \filldraw[color=green, fill=green!30, rotate around={20:(7, -0.5)}] (7, -0.5) arc (300:30:0.7 and 0.5);

  \filldraw[color=green, fill=green!30] (9.5,0.5) arc (180:0:0.5);
  \draw[->] (9, 0.5)--(11, 0.5);
  \draw[->] (10, -0.5)--(10, 1.5) node [right] {$\scriptstyle x_n$};

  \draw[rounded corners=35pt](7,-1)--(4.2,-1)--(2,-2)--(0,0) -- (2,2)--(4.2,1)--(7,1);
  \draw (1.5,0.2) arc (175:315:1cm and 0.5cm);
  \draw (3,-0.28) arc (-30:180:0.7cm and 0.3cm);
  \filldraw[color=black, fill=white] (7.5,0) arc (0:360:0.5cm and 1cm);
  \node (a) at (20:2.5) {$M$};
  \node (a) at (-12:7.5) {$\partial M$};

  \draw[->] (6.3, -0.5)..controls(7, -1)and(8, -0.5)..(9.7, 0.6) node [midway, above] {$\phi$};
\end{illustration}

Lokalnie, na zbiorze mapowym $(U_\alpha,\phi)$ możemy funkcję spełniającą wymagania wyżej zadać przy pomocy funkcji wychodzącej z $\overline{U_\alpha}=\phi(U_\alpha)$
$$f_\alpha:\overline{U_\alpha}\to\R,\quad f(x_1,...,x_n)=x_n,$$
gdyż ostatnia współrzędna punktów z $\partial M$ jest zawsze zerowa (gdyż są one w $\partial H^n$). Stąd w prosty sposób dostajemy funkcję:
$$f_\alpha:U_\alpha\to\R,\quad f_\alpha=\overline{f_\alpha}\circ \phi$$
która lokalnie spełnia nasze wymagania. Nie możemy jednak w prosty sposób przełożyć lokalne $f_\alpha$ na funkcję $f:M\to\R$. 

\subsection{Lokalnie skończone rozdrobnienie}

Przypomnijmy definicje, które będą przydatne przy rozkładach jedności:

\begin{definition}[pokrycie lokalnie skończone] Pokrycie $\{A_\alpha\}$ podzbiorami przestrzeni topologicznej $X$ jest \important{lokalnie skończone}, jeśli dla każdego $p\in X$ istnieje otoczenie $U_p$ takie, że $U_p\cap A_\alpha\neq\emptyset$ tylko dla skończenie wielu $\alpha$.
\end{definition}

\begin{definition}[rozdrobnienie] Pokrycie $\{V_\beta\}$ przestrzeni $X$ zbiorami otwartymi nazywamy \important{rozdrobnieniem pokrycia} $\{U_\alpha\}$, jeśli każdy $V_\beta$ zawiera się w pewnym $U_\alpha$.
\end{definition}

Warto nadmienić, że relacja bycia rozdrobnieniem jest przechodnia.\marginpar{$\{W_\gamma\}\prec\{V_\beta\}\prec\{U_\alpha\}\implies\;\implies\{W_\gamma\}\prec\{U_\alpha\}$} Będziemy oznaczać ją przez $\{V_\beta\}\prec\{U_\alpha\}$.

\begin{definition}[przestrzeń parazwarta] Przestrzeń topologiczna $X$ jest \important{parazwarta}, jeśli każde jej pokrycie $\{U_\alpha\}$ zbiorami otwartymi posiada lokalnie skończone rozdrobnienie $\{V_\beta\}$.
\end{definition}

Warto przypomnieć, że każda rozmaitość topologiczna jest parazwarta. \marginpar{Dowód: patrz Lee strona 36-37}Dowód tego lematu wykorzystuje w istotny sposób lokalną zwartość, czyli istnienie dla każdego punktu otoczeń prezwartych (po domknięciu zwartych). Własność ta została udowodniona na ćwiczeniach.

\begin{remark}
Rozdrobnienie wynikające z parazwartości rozmaitości topologicznych można z góry uznać za składające się z prezwartych zbiorów mapowych.
\end{remark}

\begin{proof}
Niech $\{U_\alpha\}$ będzie pokryciem $M$. Łatwo jest znaleźć rozdrobnienie $\{U'_\gamma\}\prec\{U_\alpha\}$ złożone ze zbiorów prezwartych mapowych. Wystarczy obraz każdego $U_\alpha$ w $\R^n$ pokryć zbiorami prezwartymi i wrócić z nimi na $M$. Z faktu, że rozmaitości są parazwarte dostajemy lokalnie skończone rozdrobnienie $\{V_\beta\}\prec\{U'_\gamma\}$, które z przechodności $\prec$ jest też rozdrobnieniem $\{U_\alpha\}$. Dodatkowo, każdy $V_\beta$ zawiera się w pewnym $U'_\gamma$, które były mapowe i prezwarte, więc i $V_\beta$ taki jest.
\end{proof}

{\large\color{red}JESTEM NA 2 UWADZE Z ROZKLADOW JEDNOSCI}
