\section{Definiowanie rozmaitości}

\subsection{Rozmaitość topologiczna}

\begin{definition}[przestrzeń topologiczna]
  Przestrzeń topologiczna $M$ jest $n$-wymiarową rozmaitością ($n$-rozmaitością) topologiczną, jeśli:
  \begin{itemize}
    \item jest \acc{Hausdorffa}
    \item ma \acc{przeliczalną bazę} topologii
    \item jest \acc{lokalnie euklidesowa} wymiaru $n$, tzn. każdy punkt posiada otoczenie otwarte homeomorficzne z otwartym podzbiorem w $\R^n$
  \end{itemize}
\end{definition}

  Warunkiem równoważnym do lokalnej euklidesowości jest posiadanie przez każdy punkt $p\in M$ otoczenia $U$ takiego, że istnieje homeomorfizm $U\xrightarrow[]{\cong} B_r\subseteq\R^n$. [ćwiczenia]
  \bigskip

\textbf{Hausdorffowość}

Dzięki warunkowi Hausdorffowości wykluczone są np. patologie pokroju

\begin{illustration}
  \draw(0, 0)--(2, 0);
  \draw[orange] (1.2, -0.02)--(2, -0.02);
  \draw[orange] (2, -0.52)--(2.8, -0.52);

  \draw[green] (1.5, 0.02)--(2, 0.02);
  \draw[green] (2, 0.52)--(2.5, 0.52);


  \filldraw[color=black, fill=white] (2, 0) circle (1.5pt);
  \draw (2, 0.5)--(4, 0.5);
  \filldraw (2, 0.5) circle (1.5pt) node [above] {$A$};
  \draw(2, -0.5)--(4, -0.5);
  \filldraw(2, -0.5) circle (1.5pt) node [below] {$B$};
  
  \draw[green, very thick] (1.6, 0.2) arc (120:240:0.2);
  \draw[orange, very thick] (1.3, 0.2) arc (120:240:0.2);


  \draw[green, very thick] (2.4, 0.7) arc (60:-60:0.2);
  \draw[orange, very thick] (2.7, -0.3) arc (60:-60:0.2);
\end{illustration}

gdzie punktów $A$ i $B$ nie da się rozdzielić za pomocą rozłącznych zbiorów otwartych.

Ogólniej, warunek ten mówi, że lokalnie topologiczne własności z $\R^n$ przenoszą się na $M$ przez homeomorfizmy, np dla podzbioru $U\subseteq M$ i homeomorfizmu $\phi:U\to\overline{U}\subseteq\R^n$:

\begin{illustration}
%\draw[help lines,step=1] (0, 0) grid (12,4);
\draw[rounded corners=36pt](6,-1)--(4.2,-1)--(2,-2)--(0,0)--(2,2)--(4.2,1)--(7,1)--(9.2,2)--(11,0)
--(9,-2)--(6,-1);
\draw (1.5,0.2) arc (175:315:1cm and 0.5cm);
\draw (3,-0.28) arc (-30:180:0.7cm and 0.3cm);
%\draw (5.8,0) arc (0:360:0.5cm and 1cm);
\draw (7.5,0.2) arc (175:315:1cm and 0.5cm);
\draw (9,-0.28) arc (-30:180:0.7cm and 0.3cm);
%\node (a) at (-13:5.8) {$\partial M$};
%\node (a) at (26:2.5) {$\tilde{M}$};
\draw[rotate around={20:(4.5, 0)}] (4.5, 0) ellipse (0.5 and 0.8) node [below] {$U$};
\filldraw (4.3, 0.2) circle (1.5pt) node [below] {$p$};
\draw[->] (10, 2) node [right] {$\R^n$} --(10, 4);
\draw[->] (9, 3) --(11, 3);
\draw (10, 3) circle (0.7);
\node at (11, 3.8) {$\overline{U}=\phi(U)$};
\draw[smooth, ->, tension=1] plot coordinates {(5, 0.3) (6, 0.4) (8, 1.2) (9.3, 2.5)};
\end{illustration}

Dodatkowo, dla dowolnego \emph{zwartego} $\overline K\subseteq\overline{U}$ jego odpowiednik na $M$, czyli $K=\phi^{-1}(\overline{K})\subseteq U$, jest \emph{domknięty i zwarty} [ćwiczenia]. Jeśli zaś $\overline{K}$ jest zbiorem domknięty w $\overline{U}$, ale niezwartym, to nie zawsze $K$ jest domknięty w $M$. Weźmy np. $\phi:U\to\overline{U}=\R^n$ i zbiór domknięty $\overline{K}=\R^n$ (cała przestrzeń jest jednocześnie domknięta i otwarta). Wtedy $K=\phi^{-1}(\overline{K})=U$ jest otwartym podzbiorem  $M$ mimo, że $\overline{K}$ jest otwarte.

Skończone podzbiory rozmaitości będącej przestrzenią Hausdorffa są zawsze domknięte i co ważne, granice ciągów na rozmaitościach topologicznych są jednoznacznie określone.
\medskip

\textbf{Przeliczalna baza}

Warunek przeliczalnej bazy został wprowadzony, by rozmaitości nie były "zbyt duże". Nieprzeliczalna suma parami rozłącznych kopii $\R^n$ nie może być rozmaitością. Warunek ten implikuje, że każde pokrycie zbiorami otwartymi zawiera przeliczalne podpokrycie [ćwiczenia], co jest nazywane \important{warunkiem Lindel\"ofa}.

Przeliczalność bazy implikuje również, że każda rozmaitość topologiczna jest wstępującą sumą zbiorów otwartych
$$U_1\subseteq U_2\subseteq...\subseteq U_n\subseteq...,$$
które po domknięciu są nadal zawarte w niej. Pozwala ona również na włożenie $M$ do $\R^n$ dla odpowiednio dużego $n$. Czyli na przykład $S^2$, sfera, ma naturalne włożenie w $\R^3$ pomimo lokalnej euklidesowości z $\R^2$.

Rodzina $\set{X}$ podzbiorów $M$ jest \acc[i]{lokalnie skończona}, jeżeli każdy punkt $p\in M$ ma otoczenie, które przecina się co najwyżej ze skończoną liczbą zbiorów z $\set{X}$. Jeżeli $M$ ma dwa pokrycia: $\set{U}$ i $\set{V}$ takie, że dla każdego $V\in\set{V}$ znajdziemy $U\in\set{U}$ takie, że $V\subseteq U$, to $\set{V}$ jest \acc[i]{pokryciem włożonym/rozdrobnieniem} $\set{U}$. Dzięki przeliczalności bazy $M$, każda rozmaitość jest \important{parazwarta}, czyli zawiera lokalnie skończone rozdrobnienie.

\textbf{Lokalna euklidesowość}

\begin{theorem}[twierdzenie brouwer'a]\label{twierdzebie brouwer'a} \textbf{\color{orange}Twierdzenie Brouwer'a} Dla $m\neq n$ otwarty podzbiór $\R^n$ nie może być homeomorficzny z żadnym otwartym podzbiorem $\R^m$.
\end{theorem}

Z twierdzenia wyżej wynika, że liczba $n$ jest przypisana do $M$ jednoznacznie i nazywa się \important{wymiarem} $M$ ($dim(M)=n$). Jeśli wymiar rozmaitości $M$ wynosi $n$, to nazywamy ją czasem \acc[i]{$n$-rozmaitością}.

Tutaj warto zaznaczyć, że zbiór pusty zaspokaja definicję rozmaitości topologicznej dla dowolnego $n$. Wygodnie jest go jednak móc użyć, więc w definicji niepustość $M$ nie jest przez nas wymagana.

\begin{remark}
  Każdy otwarty podzbiór $n$-rozmaitości topologicznej jest $n$-rozmaitością topologiczną.
\end{remark}

\begin{proof} Ćwiczenia \end{proof}

