\documentclass{article}

\usepackage{../../../lecture_notes}

\title{ROZMAITOŚCI RÓŻNICZKOWALNE\\{\normalsize Lista 5}}
\author{}
\date{}

\begin{document}
\maketitle\thispagestyle{empty}

\begin{problem}[]{}
  Niech $\lambda\neq0$ będzie stałą rzeczywistą. Dla zbazowanej krzywej $(c, t_0)\in C_pM$ rozważmy krzywą $c_\lambda$ zadaną wzorem $c_\lambda(t)=c(\lambda t)$. Uzasadnij, że w przestrzeni wektorowej $T_pM$ zachodzi $\lambda[c, t_0]=[c_\lambda, t_o/\lambda]$.
\end{problem}

Wiem, że krzywa $(c, t_0)$ przechodziła przez punkt $p$. Biorę teraz wszystkie krzywe które są w tym punkcie styczne i chcę powiedzieć, że jeżeli przemnożę je przez $\lambda$, to jest ta sama klasa, co wszystkie krzywe styczne w punkcie $t_0$ względem $p$ do $c_\lambda$
\end{document}
