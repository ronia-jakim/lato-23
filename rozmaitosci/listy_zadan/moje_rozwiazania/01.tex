\documentclass{article}

\usepackage{../../../notatki}

\begin{document}
\subsection*{ZADANIE 1.}
\emph{\color{pink}Uzasadnij, że jeśli w definicji rozmaitości topologicznej warunek lokalnej euklidesowości zastąpimy którymkolwiek z następujących warunków:}

\emph{\color{pink}(a) każdy punkt posiada otwarte otoczenie homeomorficzne z otwartą kulą w $\R^n$,}

\emph{\color{pink}(b) każdy punkt posiada otwarte otoczenie homeomorficzne z całą przestrzenią $\R^n$}

\emph{\color{pink}to otrzymamy definicję równoważną.}
\medskip

To, że $(a)\iff(b)$ wynika z tego, że otwarta kula jest homeomorficzna z $\R^n$. Pokażemy więc, że

\emph{Lokalnie euklidesowa $\iff$ każdy punkt posiada otoczenie homeomorficzne z otwartą kulą.}

$\implies$

Ustalmy dowolne $x\in M$. Niech $x\in U\subseteq M$ będzie otwartym otoczeniem $x$ w $M$ takim, że $U\cong \overline U\subseteq\R^n$ z definicji podanej na wykładzie. Nazwijmy ten homeomorfizm $\phi:U\to\overline U$. Wiemy, że istnieje $r>0$ takie, że $B_r(\phi(x))\subseteq\overline U$. Co więcej, $\phi^{-1}[B_r(\phi(x))]$ jest otwartym podzbiorem $M$, bo $\phi$ to homeomorfizm i przeciwobraz zbioru otwartego jest przezeń otwarty. Czyli $M\supseteq \phi^{-1}[B_r(\phi(x))]\ni x$ jest otwartym podzbiorem $M$ zawierającym $x$ i homeomorficznym z otwartą kulą w $\R^n$.

$\impliedby$

Otwarta kula jest otwartym podzbiorem $\R^n$, więc mamy homeomorfizm między pewnym otwartym otoczeniem $x\in U\subseteq M$ a otwartym podzbiorem $\R^n$.

\subsection*{ZADANIE 2.}
\emph{\color{pink}Uzasadnij, że każdy otwarty podzbiór rozmaitości topologicznej jest rozmaitością topologiczną.}
\medskip

Niech $M$ będzie rozmaitością topologiczną, a $M'\subseteq M$ jej otwartym podzbiorem.

1. Hausdorffowość: 

$x,y\in M'\implies x,y\in M$, czyli istnieją $U,V\subseteq M$ otwarte podzbiory $M$ takie, że $x\in U,y\in V$ oraz $U\cap V=\emptyset$. Ponieważ $M'$ jest otwarty, to istnieją otwarte $x\in U'$ i $y\in V'$ zawarte w $M'$. Skończony przekrój zbiorów otwartych, więc $x\in U'\cap U$ i $y\in V'\cap V$ są rozłącznymi zbiorami otwartymi w $M'$.

2. Przeliczalna baza:

Niech $\{U_i\}_{i\in\N}$ będzie przeliczalną bazą $M$. Wtedy $\{U_i\cap M'\}_{i\in\N}$ jest przeliczalną rodziną zbiorów otwartych w $M'$ (przecięcie dwóch otwartych jest otwarte). Ponieważ otwarty zbiór w $M'$ jest również otwarty w $M$, to mogliśmy go wysumować za pomocą $U_i$, czyli w szczególności możemy go wysumować z $U_i\cap M'$, bo sam jest i tak zawarty z $M'$.

3. Lokalna Hausdorffowość:

Weźmy dowolny $x\in M'\subseteq M$. Ponieważ $M$ było rozmaitością topologiczną, to dla pewnego otwartego otoczenia $x\in U\subseteq M$ mieliśmy homeomorfizm $\phi:U\to\overline U\subseteq\R^n$. Znowu, $U\cap M'$ jest zbiorem otwartym, a więc $\phi\obciete (U\cap M')$ jest homeomorfizmem z otwartym podzbiorem $\R^n$ (bo $U\cap M'$ przechodzi na coś otwartego).

\subsection*{ZADANIE 3.}
\emph{\color{pink}Uzasadnij, że jeśli rozmaitość $M$ jest spójna, to jest też \text{drogowo spójna}, tzn. każde dwa punkty $p,q\in M$ można połączyć ciągłą krzywą $\gamma:[0,1]\to M$ (taką, że $\gamma(0)=p,\gamma(1)=q$). Wskazówka: dla ustalonego punktu $p$ rozważ zbiór tych punktów $q$, które można połączyć z $p$ krzywą ciągłą.}
\medskip

Spójna $\implies$ jedyne zbiory otwarto-domknięte to $\emptyset$ i $M$.

Ustalmy dowolne $p\in M$. Niech $\Sigma_p$ będzie zbiorem tych punktów $q\in M$, które można połączyć z $p$ krzywą ciągłą. 

%Zauważmy, że łączenie punktów krzywą ciągłą jest relacją równoważności: zwrotność i symetryczność są trywialne, a przechodniość wynika z faktu, że możemy łączyć dwie krzywe w jedną dłuższą. W takim razie, zbiory $\Sigma_p$ są klasami równoważności na $M$.

1. $\Sigma_p$ jest zbiorem otwartym:

Niech $q\in\Sigma_p$ i $\gamma$ będzie krzywą taką, że $\gamma(0)=p,\gamma(1)=q$. Pokażemy, że możemy na nim opisać zbiór otwarty. Niech $q\in U\subseteq M$ będzie otwartym otoczeniem $q$, a $\phi:U\to\overline U\subseteq \R^n$ będzie homeomorfizmem wynikającej z lokalnej euklidesowości $M$. Weźmy teraz dowolny $y\in U$ i pokażemy, że wówczas istnieje krzywa z $p$ do $y$.

Wiemy, że $\R^n$ jest przestrzenią łukowo spójną, niech więc $\mu:[0,1]\to\R^n$ będzie krzywą ciągłą taką, że $\mu(0)=\phi(q)$ i $\mu(1)=\phi(y)$. Rozważmy teraz krzywą
$$\gamma':[0,1]\to M$$
$$\gamma'(a)=\begin{cases}
    \gamma(2a)\quad a\leq\frac12\\
    \phi^{-1}[\mu(2a-1)]
\end{cases}$$
Mamy $\gamma'(0)=p$ i $\gamma'(1)=\phi^{-1}[\mu(1)]=\phi^{-1}[\phi(y)]=y$, czyli $y\in \Sigma_p$

2. $\Sigma_p$ jest zbiorem domkniętym:

Równoważnie, $M\setminus\Sigma_p$ jest zbiorem otwartym. Jeśli $M\setminus\Sigma_p$ nie byłoby otwarte, to dla pewnego $x\notin\Sigma_p$ mielibyśmy otoczenie z $y\in \Sigma_p$ i argument podobny jak wyżej: punkty są w jednym otoczeniu homeomorficznym z $\R^n$, więc możemy skonstruować krzywą z $p$ przez $y$ do $x$, więc $x\in\Sigma_p$ i mamy sprzeczność.

\subsection*{ZADANIE 4.}
\emph{\color{pink}Udowodnij, że jeśli $(U,\phi)$ jest mapą na rozmaitości $M$, zaś $K$ jest zwartym podzbiorem $\phi(U)$, to zbiór $\phi^{-1}(K)$ jest domknięty i zwarty w $M$. Pokaż też, że jeśli $K$ jest domknięty w $\phi(U)$, to $\phi^{-1}(K)$ nie musi być domknięty w $M$.}
\medskip

Jeśli $K$ jest zwartym podzbiorem $\phi(U)$, to z każdego pokrycia $K$ możemy wybrać podpokrycie skończone. Popatrzmy na zbiór $\phi^{-1}(K)$. Możemy go pokryć zbiorami otwartymi $\{V_i\}_{i\in I}$. Czyli $\phi(V_i)$ pokrywają $K$, a więc możemy wybrać ciąg $i_1,...,i_n\subseteq I$ taki, że $K=\bigcup\limits_{1\leq k\leq n}\phi(V_k)$. W takim razie, 
$$\bigcup\limits_{1\leq k\leq n}V_i$$ 
pokrywają $\phi^{-1}(K)$. Czyli $\phi^{-1}(K)$ jest zwarty. 

To drugie to jakiś kontrprzykład, ale mi się nie chce.

\subsection*{ZADANIE 5.}
\emph{\color{yellow}Pokaż, że jeśli przestrzeń topologiczna ma przeliczalną bazę, to z każdego jej pokrycia zbiorami otwartymi można wybrać przeliczalne podpokrycie.}
\medskip

\subsection*{ZADANIE 6.}
\emph{\color{pink}Korzystając z zadań 4 i 5 uzasadnij, że każda rozmaitość jest przeliczalną sumą otwartych podzbiorów homeomorficznych z otwartymi kulami w $\R^n$, których domknięcia w $M$ są homeomorficzne z domkniętymi kulami w $\R^n$.}
\medskip

Niech $(U_i,\phi_i)_{i\in I}$ będzie rodziną map z $M$. Na mocy zadania $5$ możemy wybrać ciąg $i_1,...,i_n,...\subseteq I$ taki, że
$$M=\bigcup\limits_{1\leq k}U_k.$$
Popatrzmy teraz, co się dzieje w środku jednej takiej mapy. To jest ustalmy dowolne $i$ z wcześniej wybranego ciągu $i_1,...,i_n,...$. 

Niech $\overline{U_i}=\phi(U_i)$. Jest to zbiór otwarty w $\R^n$, czyli na dowolnym $x\in\overline{U_i}$ możemy opisać kulę $B_r(x)$ o promieniu $r>0$. Teraz, jeśli weźmiemy $B_{r/2}(x)$, to możemy taką kulę domknąć nie wychodząc z $\overline{U_i}$ (chociażby dlatego, że to domknięcie dalej będzie się zawierało w $B_r(x)$). Teraz zbiór $F=cl(B_{r/2}(x))$ jest zwarty w $\R^n$, czyli na mocy zadania 4. mamy, że $\phi^{-1}(F)$ jest domknięty w $M$.

Mamy więc, że w każdej mapie $(U_i,\phi_i)$ możemy pokryć zbiorami otwartymi homeomorficznymi z kulami w $\R^n$ i o domknięciach homeomorficznych z domkniętymi kulami w $\R^n$. Wystarczy teraz dla każdego $(U_i,\phi_i)$ wybrać przeliczalnie wiele takich zbiorów otwartych, co możemy zrobić z ośrodkowości $\R^n$.

\subsection*{ZADANIE 7.}
\emph{\color{pink}Uzasadnij, że lokalnie wokół każdego punktu $(x,y)\neq(0, 0)$ współrzędne biegunowe na $\R^2$ są zgodne ze współrzędnymi kartezjańskimi.}
\medskip

Po pierwsze, co rozumiemy przez współrzędne? To są odwzorowania w $\R^2$, parametryzacje naszej rozmaitości. W tym przypadku kartezjańskie współrzędne to będzie dla nas tak naprawdę funkcja $id$. Popatrzymy też na $\phi$, czyli przejście ze współrzędnych biegunowych do współrzędnych kartezjańskich zadane wzorem:
$$\phi(\alpha, r)=\begin{pmatrix}
    r\cos\alpha\\r\sin\alpha
\end{pmatrix}.$$
Aby obie te współrzędne były zgodne, potrzebujemy, żeby kolorowe strzałki były funkcjami gładkimi (bo jest to odpowiedni $id\circ\phi^{-1}$ i $\phi\circ id^{-1}$).

\begin{illustration}
    \draw (0, 0) rectangle (2, 2);
    \draw (4, 0) rectangle (6, 2);
    \draw (0, -2) rectangle (2, -4);
    \draw (5, 1)--(5, 2) node [above] {$\alpha$};
    \draw (5, 1)--(6, 1) node [right] {$r$};
    \node at (1, 1) {$\circ$};
    \node at (5, 1) {$\circ$};
    \node at (1, -3) {$\circ$};
    \draw[<-] (2.2, 1)--(3.8, 1) node [midway, above] {$\phi^{-1}$};
    \node at (0.5, 2.3) {$\R^2\setminus(0, 0)$};
    \node at(6.5, 2.3) {$\R_+\times[0, 2\pi)$};
    \draw (1, -3)--(1, -2) node [above, right] {y};
    \draw (1, -3)--(2, -3) node [right] {x};
    \draw[->] (1, -0.3)--(1, -1.7) node [midway, right] {id};
    \draw[<-, blue] (2.3, -2.5)..controls (3, -2.5) and (4.5, -1.1)..(4.5, -0.3) node [midway, above, rotate=40] {$\color{blue}id\circ\phi^{-1}$};
    \draw[->, pink] (2.3, -2.8)..controls (3.2, -2.8) and (4.8, -1.2)..(4.8, -0.3) node [midway, below, rotate=40] {$\color{pink}\phi\circ id^{-1}$};
\end{illustration}

Ciągłość funkcji $\phi\circ id^{-1}$ jest jasna ze wzoru na $\phi$. Wystarczy teraz pokazać, że $\phi^{-1}$ jest gładkie. Wiemy, że jeśli Jakobian funkcji nie zeruje się w pewnym punkcie, to na jego otoczeniu funkcja jest odwracalna i ta odwrotność też będzie gładka, bo $\phi_1$ takie było.
$$D_{\phi}(\alpha,r)=\begin{bmatrix}
    \cos\alpha&-r\sin\alpha\\\sin\alpha&r\cos\alpha
\end{bmatrix}=r>0.$$
Z zadania tego możemy wyciągnąć wniosek, że mapami możemy zadać więcej niż jedną strukturę na rozmaitości.

\subsection*{ZADANIE 8.}
\emph{\color{yellow}Pokaż, że współrzędne geograficzne na sferze $S^2$ (określone na dopełnieniu biegunów i jednego z południków) są zgodne ze standardową strukturą na $S^2$. Wskazówka: skorzystaj z parametryzacji równania sfery z użyciem współrzędnych geograficznych.}
\medskip

Czy współrzędne geograficzne to to samo co współrzędne sferyczne?

To zadanie wygląda syfnie jakoś, idę dalej

\subsection*{ZADANIE 9.}
\emph{\color{pink}Uzasadnić, że zgodność atlasów jest relacją symetryczną i przechodnią.}
\medskip

Niech $\set{A}_1,\set{A}_2,\set{A}_3$ będą atlasami na rozmaitości $M$.

\textbf{Symetryczność:}

Pokazanie symetryczności relacji zgodności atlasów sprowadza się do wzięcia dwóch map: $(U_1,\phi_1)\in\set{A}_1$ i $(U_2,\phi_2)\in\set{A}_2$ i stwierdzeniu, że jeśli $(U_1,\phi_1)$ jest zgodna z $(U_2,\phi_2)$ (czyli po porównaniu wszystkich $\set{A}_1$ zgodny z $\set{A}_2$), to $\phi_1\phi_2^{-1}$ oraz $\phi_2\phi_1^{-1}$ są gładkie. No ale to samo, jeśli przestawimy indeksy, czyli $(U_2,\phi_2)$ jest zgodne z $(U_1,\phi_1)$ ($\set{A}_2$ jest zgodny z $\set{A}_1$).

\textbf{Przechodniość:}

Tutaj kusiłoby wziąć dowolne trzy mapy: $(U_1,\phi_1)$, $(U_2,\phi_2)$ i $(U_3,\phi_3)$ odpowiednio z $\set{A}_1,\set{A}_2,\set{A}_3$ i powiedzieć, że śmiga, ale w taki sposób ignorujemy dziedziny poszczególnych $\phi_i$. To znaczy, może zajść coś takiego:

\begin{illustration}
    \draw [very thick] (0, 0) ellipse (3 and 1);
    \draw [rotate=60, blue, very thick] (0.8, 1.5) ellipse (3 and 1);
    \draw [rotate=-60, pink, very thick] (-0.8, 1.5) ellipse (3 and 1);
    \node at (0, -1.5) {$U_1$};
    \node at (2.5, 1.5) {$\color{pink}U_2$};
    \node at (-2.5, 1.5) {$\color{blue}U_2$};
\end{illustration}
I wtedy dziedziny np $\phi_1\phi_2^{-1}$ i $\phi_1\phi_3^{-1}$ są rozłączne.

\subsection*{ZADANIE 10.}
\emph{\color{pink}Uzasadnij, że każdy atlas $\set{A}$ na rozmaitości $M$ zawiera się w dokładnie jednym atlasie maksymalnym (złożonym ze wszystkich map na $M$ zgodnych z $\set{A}$).}
\medskip

Z poprzedniego zadania wiemy, że relacja zgodności atlasów $\sim$ jest relacją równoważności na zbiorze wszystkich atlasów danej rozmaitości i klasami równoważności są wszystkie atlasy zgodne z reprezentantem. Chcę pokazać, że dla każdej klasy istnieje atlas, który zawiera wszystkie pozostałe. 

Niech $\set{A}$ będzie atlasem na $M$ i popatrzmy na $[\set{A}]=\Sigma$, czyli wszystkie atlasy z nim zgodne. Postuluję, że zbiór 
$$A=\bigcup\limits_{b\in\Sigma}b$$
jest atlasem maksymalnym z $\Sigma$ zawierającym $\set{A}$.

To, że $\set{A}\subseteq A$ jest oczywiste: $\set{A}$ pojawia się jako element sumy którą jest $A$. To, że $A$ jest atlasem też jest jasne: każdy atlas z sumy pokrywa nam całe $M$, a ponieważ wszystkie atlasy z $\Sigma$ są zgodne, to mamy, że jeśli wszystkie wsadzimy w jeden worek, to też dostaniemy atlas złożony z map zgodnych.

$A$ jest jedynym atlasem maksymalnym, bo wyjęcie z niego jakiejkolwiek mapy (czyli wyjęcie atlasów, które ją zawierają), będzie równoznaczne z niezawieraniem przez $A$ wszystkich zgodnych map.

\subsection*{ZADANIE 11.}
\emph{\color{pink}Uzasadnij, że produkt $M\times N$ rozmaitości topologicznych jest rozmaitością topologiczną. Zakładając, że $M$ i $N$ są rozmaitościami gładkimi, opisz naturalny atlas definiujący strukturę gładką na produkcie (i sprawdź, że mapy są gładko zgodne).}
\medskip

\indent 1. Hausdorffowość

Trywialne, bo jeśli mam dwa punkty $(x_1,y_1), (x_2,y_2)$, $x_i\in M,y_i\in N$, to mam jakieś zbiory otwarte $x_i\in U_i,y_i\in V_i$ takie, że $U_1\cap U_2=\emptyset,V_1\cap V_2=\emptyset$. Mam, że $U_i\times V_i$ jest zbiorem otwartym i
$$(x_i,y_i)\in U_i\times V_i$$
$$U_1\times V_1\cap U_2\times V_2=\emptyset.$$

\indent 2. Ma przeliczalną bazę

Odmawiam. Trywialne.

\indent 3. Lokalna euklidesowość

Weźmy punkcik $(x,y)\in M\times N$. Wiem, że $x\in M$ ma otoczenie $x\in U$ takie, że $\phi:U\to \overline U\subseteq \R^n$ jest homeomorfizmem. Tak samo dla $y\in N$ jest $\psi:V\to \overline V\subseteq \R^n$. Rozważmy teraz odwzorowanie $\heartsuit:U\times V\to \R^{n+m}$ dane wzorem:
$$\heartsuit(a, b)=\begin{pmatrix}
    \phi(a)\\
    \psi(b)
\end{pmatrix}$$
to znaczy pierwsze $n$ współrzędnych jest zarezerwowanych dla współrzędnych $\phi(a)$, a później do samego dołu mamy $\psi(b)$.

Ciągłość $\heartsuit$ jest trywialna. Wiem, że $\phi,\psi$ mają ciągłe funkcje odwrotne, jak to jest z $\heartsuit$?
$$\heartsuit^{-1}(a_1,...,a_n,a_{n+1},...,a_{n+m})=(\phi^{-1}(a_1,...,a_n), \psi^{-1}(a_{n+1},...,a_{n+m}))$$
wygląda jak dobrze zdefiniowana, ciągła funkcja odwrotna. Hence $\heartsuit$ jest homeomorfizmem.

\indent Szukanie atlasu

Niech $\set{M}$ będzie atlasem na $M$, a $\set{N}$ będzie atlasem na $N$. Twierdzę, że na $M\times N$ mogę opisać atlas 
$$\set{A}=\{(U\times V, \phi\times\psi)\;:\;(U, \phi)\in\set{M}, (V,\psi)\in\set{N}\},$$
gdzie $\phi\times\psi$ to funkcja jak $\heartsuit$ wyżej.

ONI TUTAJ JAKIEŚ JAKOBIANY SZUKAJA
$$D(\phi_1\phi_2^{-1}\times\psi_1\psi_2^{-1})=\begin{bmatrix}
    D(\phi_1\phi_2^{-1}&0\\0&D\psi_1\psi_2^{-1})
\end{bmatrix}$$

\subsection*{ZADANIE 12.}
\emph{\color{pink}Znajdź gładki atlas na $\R^1$ niezgodny ze standardowym. Zrób to samo dla $S^1$.}
\medskip

Mamy $\R^1$ jako potęgę rozmaitość i $\R$ jako jej model. Funkcja z $\phi_1:\R^1\to\R$ będzie mapą taką, że dla $x\in\R^1$ $\phi_1(x)=\sqrt[3]{x}\in\R$. Nie jest to zgodne, bo dla $\phi_2=id$ mapy nie są zgodne.

Warto sobie sprawdzić, że $\sqrt[3]{x}$ jest dyfeomorfizmem.

Całkiem ściśle wypadałoby jeszcze pokazać, że ta mapa sama w sobie zadaje atlas, ale nam się nie chce.

Dla $S^1$ chcemy ten kawałek nieróżniczkowalny mapy $\sqrt[3]{x}$ wkleić w $S^1$. Bierzemy wykres funkcji biegunowych dla $S^1$.

\subsection*{ZADANIE 13.}
\emph{\color{pink}Uzasadnij, że dla $k\geq1$ nie istnieje $c^k$-dyfeomorfizm pomiędzy otwartymi podzbiorami w $\R^n$ i $\R^m$ gdy $n\neq m$. Pozwoli to określić pojęcie wymiaru gładkiej rozmaitości w sposób niezależny od topologicznego (znacznie trudniejszego) twierdzenia o nieistnieniu homeomorfizmu pomiędzy otwartymi podzbiorami w $\R^n$ i $\R^m$.}

Pokażemy, że nie istnieje $C^1$ dyfeomorfizm $f:O\subseteq\R^n\to\Omega\subseteq\R^m$ dla $n\neq m$.

Ponieważ dyfeomorfizm musi mieć odwrotność, to $f^{-1}:\Omega\to O$. Możemy założyć, że $m<n$. Pokażemy, że funkcja $C^1$ i "na" nie jest $1-1$. Weźmy $x\in O$, wtedy wiemy, że $\ker(Df(x))\geq1$, bo jeśli mamy operator liniowy $T:V\to W$ i $dim(V)=dim(ker T)+dim(Im T)$.

Zdefiniujmy $F:\R^m\to\R^m$ i $F(x)=(f(x_1,...,x_n), x_{n+1},...,x_{m})$. Z twierdzenia o funkcji odwrotnej wiemy, że jeśli $F$ jest bijekcją na otoczeniu $x$, to dla małego $h$
$$f(x_1,...,f_m,x_{m+1})$$

Rozwiązanie dużo prostsze: $f^{-1}\circ f=id_{\R^n}$, czyli stosując chain rule
$$Did_{\R^n}=D(f^{-1}\circ f)=Df^{-1}\cdot Df$$
stąd $m\geq n$
$$f\circ f^{-1}=id_{\R^m}$$
a tutaj $n\geq m$.

\subsection*{ZADANIE 15.}
\emph{\color{yellow}Niech $M$ będzie rozmaitością gładką, $p\in M$ ustalonym punktem zaś $f:M\to\R$ funkcją rzeczywistą na $M$. Uzasadnij, że jednokrotna różniczkowalność funkcji $f\circ\phi^{-1}$ w punkcie $\phi(p)$ nie zależy od wyboru mapy $(U, \phi)$ zawierającej $p$ (tzn. takiej, że $p\in U$). Oznacza to, że jednokrotna różniczkowalność w punkcie jest dobrze określonym pojęciem dla funkcji rzeczywistych na rozmaitości gładkiej.}

$$\lim\limits_{h\to0}{f\phi_1^{-1}(\phi_1(p)+h)-f\phi_1^{-1}(p)\over h}=c\text{ istnieje}$$

Rysuneczek się przydałby i powiedzieć, że złożenie funkcji gładkich jest różniczkowalne.

\subsection*{ZADANIE 17.}
\emph{\color{pink}Mówimy, że funkcja wielu zmiennych ma w pewnym punkcie pochodną zerową, gdy odpowiedni funkcjonał liniowy przybliżający funkcję na otoczeniu tego punktu zadany przez pochodne cząstkowe jest zerowy. Pokaż, że zerowość i niezerowość pochodnej funkcji $f:M\to\R$ w punkcie $p\in M$ nie zależy od wyboru mapy. Pokaż też, że w każdym punkcie $p$ rozmaitości $M$, w którym funkcja gładka $f:M\to\R$ osiąga ekstremum lokalne, pochodna tej funkcji jest zerowa.}

\end{document}