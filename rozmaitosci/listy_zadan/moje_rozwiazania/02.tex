\documentclass{article}

\usepackage[polish]{../../../lecture_notes}

\begin{document}
\begin{problem}{d}
{Dla ciągłych funkcji rzeczywistych $f,g:M\to\R$ na rozmaitości gładkiej $M$, oraz dla $\varepsilon > 0$ mówimy, że $g$ jest $\varepsilon$-aproskymacją $f$, jeśli $\|f-g\|<\varepsilon$ (tzn. dla każdego $x\in M$ mamy $|f(x)-g(x)|<\varepsilon$).
\begin{enumerate}[label=(\alph*)]
    \item Uzasadnij, że dla każdego $\varepsilon>0$ każda ciągła funkcja $F:M\to\R$ posiada gładką $\varepsilon$-aproksymację.
    \item Rozszerz ten wynik do sytuacji, gdy $\varepsilon:M\to\R$ jest dowolną ciągłą dodatnią funkcją rzeczywistą, zaś $\varepsilon$-aproksymacja funkcji $f$ to dowolna taka funkcja $g$, że dla każdego $x\in M$ mamy $|f(x)-g(x)|<\varepsilon(x)$.
    \item Niech $D\subseteq M$ będzie dowolnym domkniętym podzbiorem. Dla dowolnego $\varepsilon$ jak w punkcie (b) uzasadnij, że dowolna funkcja ciągła $f:M\to\R$, która jest gładka na pewnym otwartym otoczeniu zbioru $D$, posiada gładką $\varepsilon$-aproksymację $g:M\to\R$ taką, że $g\restriction D=f\restriction D$.
\end{enumerate}
} 
\end{problem}

\emph{(a)}

Daną mam ciągłą funkcję $F:M\to\R$. Wiem, że każdą funkcję mogę dowolnie dokładnie aproksymować za pomocą wielomianu o współczynnikach wymiernych.

Weźmy sobie jakiś atlas na $M$ zawierający mapy $(U_\alpha,\phi_\alpha)$. Wiem, że skoro $f$ było ciągłe, a $\phi$ nic nie psuje, to pewnie i $f\circ\phi^{-1}$ jest ciągłe. Czyli w ten sposób dostaję funkcję $\R\to\R$ i na tym już chyba umiem pracować jakoś po ludzku.

Ten obrazek średnio cokolwiek daje, ale zrobiłam go i nie zamierzam usuwać, bo wygląda zajebiście.

\begin{illustration}
    \draw[very thick] (0, 0) ellipse (2 and 1);
    \draw (0.9, 0) arc (60:120:1.8);
    \draw[thick] (1.15, 0.18) arc (-50:-130:1.8);
    \filldraw (-1.5, -0.3) circle (1pt) node [above] {$p$};
    \draw[thick] (2, -4) rectangle (5, -1.5);
    \draw[thick] (-2, -4) rectangle (-5, -1.5);
    \draw[thick, ->, green] (2, -0.5)..controls(2.5, -0.1) and (3.5, -1)..(3.5,-1.4) node [midway, right] {$f$};
    \draw[green, dashed] (2, -2.6)..controls(2.5, -1.6) and(4, -4.6)..(5, -1.6) node [right] {$f+\varepsilon$};
    \draw[green, dashed] (2, -3.4)..controls(2.5, -2.4) and(4, -5.4)..(5, -2.4) node [right] {$f-\varepsilon$};
    \draw[green] (2, -3)..controls(2.5, -2) and(4, -5)..(5, -2) node [right] {$f$};
    \draw[purple] (2, -3.2)..controls(2.7, -1) and (4, -6)..(5, -2);
    \node at (1.5, -3.2) {$\color{purple}w(x)$};
    \draw[thick, <-, yellow] (-2, -0.5)..controls(-2.5, -0.1) and (-3.5, -1)..(-3.5,-1.4) node [midway, left] {$\phi^{-1}$};
    \filldraw (-4, -2.5) circle (1pt) node [right] {$\phi_\alpha(p)$};
    \draw[->] (-1.7, -3)..controls(-1.2, -2)and(1.2, -2)..(1.7, -3) node [midway, above] {$f\circ\phi_\alpha^{-1}(\phi_\alpha(p))$};
\end{illustration}

Chyba mogę znaleźć sobie wielomian $w\in\R[X^n]$ taki, że siedzi w kulce nałożonej na $f$ w przestrzeni funkcji ciągłych. Ewentualnie mogę powiedzieć, że $w$ to jest po prostu gładka funkcja blisko $f$ określona na $\phi_\alpha(U_\alpha)$, bo chyba funkcje gładkie są gęste w zbiorze funkcji ciągłych czy jakoś tak. Teraz chcę sobie produkować $g=w(\phi(p))$, ale wtedy to nie wyśmignie się chyba tak od razu?

Może wyprodukujmy sobie rozkład jedności $\psi_\alpha$ taki, że $\psi_\alpha\equiv0$ poza $U_\alpha$ i dowolny punkt $p\in M$ jest $\psi_\alpha(p)>0$ dla skończenie wielu $\alpha$. No i jeszcze ten $\sum\psi_\alpha(p)=1$ dla każdego $p\in M$. Czyli mam pysia będącego rozkładem jedności. Czyli mogę go chyba użyć do wytworzenia w końcu tego $g$? Bo jak $\psi_\alpha$ jest gładkie, to ten
$$g(p)=\sum w(\phi_\alpha(p))\psi_\alpha(p)$$
jest nadal gładkie? Znaczy tutaj jest nieścisłość, bo powinnam pisać, że tak jest dla $p\in U_\alpha$, a jeśli $p\notin U_\alpha$, to po prostu $0$, ale to i tak na jedno wychodzi, bo wtedy $\psi_\alpha(p)$ się zeruje. No i jakaś suma skończenie lokalnych gładkich funkcji bla bla bla bla bla

Teraz muszę się upewnić, że to faktycznie jest ograniczeniem moim?
\begin{align*}
   \left\|f(x)-g(x)\right\|&=\left\|f(x)-\sum w(\phi_\alpha(x))\psi_\alpha(x)\right\|\leq\\
   &\leq\left\|\sum(\psi_\alpha(x))[f(x)-w(\phi_\alpha(x))]\right\|<\\
   &<\left|\sum(\psi_\alpha(x))\varepsilon\right|=1\cdot\varepsilon=\varepsilon
\end{align*}

\emph{(b)}

Teraz zamiast ładnej kulki mam troszkę brzydszą kulkę bo $\{w\;:\;f(x)-\varepsilon(x)<w(x)<f(x)+\varepsilon(x)\}$, ale nadal mogę znaleźć jakieś gładkie $w$ i postąpić analogicznie jak wyżej.

Mamy $\{(U_\alpha,\phi_\alpha)\}$ - atlas złożony z prezwartych $U_\alpha$. Niech $\{\psi_\alpha\}$ będzie rozkładem jedności wpisanym w $U_\alpha$. Ustalmy $\alpha$. Wtedy $\varepsilon\circ\phi_\alpha^{-1}$ przyjmuje na $\overline{\phi_\alpha(U_\alpha)}$ minimum $m_\alpha>0$. Istnieje gładka $g_\alpha:U_\alpha\to\R$ takie, że $\|F\restriction U_\alpha-g_\alpha\|<m_{\alpha}$. Niech 
$$\overline {g_\alpha}(p)=\begin{cases}g_\alpha(p)&\in U_\alpha\\
0&p\notin U_\alpha\end{cases},$$
wtedy $m_\alpha\circ \psi_\alpha(p)\leq\varepsilon(p)$. Niech $G=\sum_\alpha\psi_\alpha\overline{g_\alpha}$.
\begin{illustration}
    \draw (-3, 0)--(3, 0);
    \draw[ultra thick] (-3, 0)--(-1, 0);
    \draw[ultra thick] (1, 0)--(3, 0);
    \draw (-2, 1)..controls (-0.5,2)and(1, 0)..(2, 1.5) node [above] {$\overline{g_\alpha}$};
    \draw[thin, blue](-3, -0.3)--(-0.8, -0.3);
    \draw[thin, blue](-0.8, -0.3)..controls (-0.5, 0.4)and (0.5,0.4)..(0.8, -0.3);
    \draw[thin, blue](0.8, -0.3)--(3, -0.3);
    \node at (0, -0.5) {\color{blue}$U_\alpha$};
    \node at (0, 0.5) {$\color{blue}supp(\psi_\alpha)$};
\end{illustration}

\begin{align*}
    G(p)&\leq\sum\psi_\alpha(F(p)+\varepsilon(p))=F(p)+\varepsilon(p)\\
    G(p)&\geq\sum\psi_\alpha(f(p)-\varepsilon(p))=F(p)-\varepsilon(p)
\end{align*}

\emph{(c)}

Czyli robię jakieś bump function?
Czyli biorę sobie atlas $(U_1,\phi_1),(U_2,\phi_2)$ taki, że $U_1=M\setminus D$, a $U_2$ jest otwartym podzbiorem zawierającym $D\subseteq U_2$. Niech wtedy $\psi_1,\psi_2$ będzie gładkim rozkładem jedności takim, że $\psi_1\equiv 0$ na $D$. Wtedy $\phi_2$ na $D$ się nie zeruje, a sumuje do $1$, a na okolicy $D$ musi stopniowo schodzić do $0$, czyli wyśmignie. To teraz wystarczy znaleźć funkcję, która na $f(\phi_2(D))$ jest identyczna, a na pozostałej części troszkę odbiega, ale to też się da zrobić, taka funkcja to może być $w$ i wtedy
$$g(x)=\begin{cases}
    w(\phi_2(x)) & x\in D\\
    \sum w(\phi_\alpha(x))\psi_\alpha(x)&wpp
\end{cases}$$

Niech $\{\psi_1,\psi_2\}$ będzie rozkładem jedności wpisanym w $\{U, M\setminus D\}$, gdzie $U$ jest otoczeniem otwartym $D$, na którym $f$ jest gładka. Niech 
$$G=f\circ \phi_1+g\circ \phi_2,$$
gdzie $g$ to jest $\varepsilon$-aproksymacja $f$ z punktu $(b)$.

\begin{illustration}
    \node at (0, 0) {$D$};
    \node at (1.5, 1.3) {$U$};
    \node at (3, -2.5) {$\color{yellow}M\setminus D$};
    \draw[thick] (0, 0) circle (1.5);
    \node at (0, -1.75) {$\color{blue}U$};
    \draw[rotate=30, thick] (-0.8, -0.8) rectangle (0.8, 0.8);
    \draw (-4, -3)--(4, -3);
    \draw[very thick, blue] (-4, -3)--(-1.2, -3);
    \draw[very thick, blue] (1.2, -3)--(4,-3);
    %\draw[very thick, blue] (-1.2, -3)..controls(-0.8, -2)and(0.8, -2)..(1.2, -3);
    \draw[very thick, blue] (1.2, -3)--(0.9,-2)--(-0.9, -2)--(-1.2,-3); 
    \draw[very thick, yellow] (-4, -2)--(-1.2, -2);
    \draw[very thick, yellow] (1.2, -2)--(4, -2);
    \draw[very thick, yellow] (1.2, -2)--(0.8,-3)--(-0.8, -3)--(-1.2,-2);
\end{illustration}


\begin{problem}{u}
{Dla niezwartej rozmaitości gładkiej $M$ skonstruuj gładką funkcję $f:M\to\R$ taką, że dla każdego naturalnego $n$ przeciwobraz $f^{-1}([-n,n])$ jest zwartym podzbiorem w $M$. Funkcje o tej własności nazywają się funkcjami właściwymi. Wskazówka: wykorzystaj zadanie $6$ z listy $1$: uzasadnij też najpierw następujący fakt pomocniczy: istnieje ciąg otwartych zbiorów $V_i$ takich, że $\bigcup\limits_{i\in\N}V_i=M$, oraz dla każdego $i$ domknięcie $cl(V_i)$ w $M$ jest zwarte i zawarte w $V_{i+1}$.}
\end{problem}

Zadanie $6$ w liście $1$ mówi, że każda rozmaitość $M$ jest przeliczalną sumą otwartych podzbiorów homeomorficznych z otwartymi kulami w $\R^n$, których domknięcia w $M$ sa homeomorficzne z domkniętymi kulami w $\R^n$.

Myślę, że wystarczy wziąć ten ciąg jak z faktu pomocniczego i troszkę go podciąć tak, żeby był homeomorficzny z otwartymi kulami. To już robiliśmy. Potem wiem, że domknięcie otwartej kuli jest zbiorem zwartym w $\R^n$, czyli jego przeciwobraz przez funkcje z atlasu też jest zbiorem zwartym. Mogę więc funkcją $f$ skalować promień na kolejne liczby naturalne, a odległość od kuli i położenie w odpowiedniej półkuli skalować na cały taki odcinek. Taki mam chwilowo pomysł.

To może teraz uzasadnienie faktu pomocniczego?
Niech $\bigcup U_i=M$ będzie przeliczalnym pokryciem $M$.
To suniemy z tworzeniem ciągu $V_i$? Niech $V_0=U_0$. Czy mogę powiedzieć, że jeśli zrobię $V_1=U_1\cup cl(V_0)$ to jeśli rzucę to na $\R^n$ i znajdę tam otwarty podzbiór niebędący całym obrazem $M$, który to zawiera, to jestem w domu? Raczej tak. Czyli takie coś powtarzam dla każdego $i$ i jestem w domu.

\sep{yellow}

Robimy zadanie 3 w inny sposób XD

Istnieją $V_i$ - otwarte prezwarte takie, że $cl(V_i)\subseteq V_{i+1}$ i $\abigcup V_i=M$. Z zadania 6 z poprzedniej listy wiemy, że istnieją $B_k$ takie, że $M=\bigcup B_k$ i $B_k\cong B(0,1)$, wtedy $cl(B_k)\cong cl(B(0,1))$. 

Możemy więc robić, żeby
$$B\cong B(0,1)=\abigcup B(0,1-\frac1{s})$$
To teraz niech $B_k=\abigcup B_{k,s}$ i wtedy mamy to z domknięciem. Teraz możemy zdefiniować $V_i$ jako:
$$V_i=\bigcup\limits_{k=1}^i\abigcup\limits_{s=1}^iB_{k,i}=\bigcup\limits_{k=1}^iB_{k,i}$$

Wtedy 
$$cl(V_i)=\bigcup cl(B_{k,i})\subseteq \bigcup B_{k,i+1}\subseteq V_{i+1}$$
Będziemy mieli pary $\{V_{i+1},M\setminus cl(V_i)\}$, czyli to jest takie dwuelementowe pokrycie rozmaitości. Teraz chcemy funkcję, która jest $1$ na $cl(V_{i+1})$, a zerem na tym drugim. Rozważmy rozkład jedności $\{\psi_1,1-\psi_i\}$ taki, że $\psi_i(M\setminus V_{i+1})\equiv 1$ i $\psi_i(V_i)\equiv 0$.

Niech $f(x)=\sum\psi_i(x)$. Wystarczy pokazać, że jest to funkcja właściwa. Support to dopełnienie $cl(V_i)$ i dla dużych $i$ będzie zerem? Nieee wiem, chce spać.

Jeśli $x\in V_i$, to $f(x)\leq i-1$. Z drugiej strony, jeśli $x\in M\setminus V_i$, to $f(x)\geq i$.

Teraz potrzebujemy, żeby $f^{-1}[-n,n]$ było zwarte. Jak to się ma dla $f^{-1}[0,n]$? jest on zawarty w $cl(V_{n+1})$











\begin{problem}{u}
{Niech $\set{U}$ będzie dowolnym pokryciem rozmaitości $M$ zbiorami prezwartymi i niech $\{f_j\}$ będzie gładkim rozkładem jedności wpisanym w $\set{U}$. Uzasadnij, że funkcja $h=\sum j\circ f_j$ jest gładką funkcją właściwą o dodatnich wartościach. Uzasadnij, że funkcja ta, jak każda rzeczywista funkcja właściwa ograniczona od dołu, posiada globalne minimum (czyli taki punkt $p\in M$ że dla każdego $x\in M$ zachodzi $f(x)\geq f(p)$.}
\end{problem}

Mamy funkcję $h=\sum j\circ f_j$. Jest gładka i tnie się niepusto tylko ze skończenie wieloma zbiorami z pokrycia. Chcemy pokazać, że to jest ograniczone od dołu. Jest właściwa, bo wiemy, że poza sumą zbiorów $U_1,...,U_n$ to jest suma większa niż $n$. Zbiory te są prezwarte, suma skończenie wielu zbiorów prezwartych jest prezwarty. Czyli 
$$h^{-1}[-n,n]\subseteq\bigcup\limits_{i\leq n}U_i.$$
Mamy pokazać, że istnieje minimum, czyli cośtam gdzie cośtam jest realizowane.

Weźmy sobie punkt $p$ taki, że $n\leq h(p)< n+1$. Teraz chcemy dzielić sobie rozmaitość na sumę od $U_1,...,U_{n+1}$ i na resztę. I teraz na dopełnienu sumy $U_1,...,U_{n+1}$ na pewno nie znajdę minimum, bo na tym dopełnieniu ta suma jest większa niż $n+1$. Ni cholery nie zrozumiałam.

\begin{problem}{u}
{Dla rozmaitości $M$ z brzegiem skonstruuj taką gładką funkcję $f:M\to[0,\infty)$ taką, że $\partial M=f^{-1}(0)$ oraz rząd $f$ w dowolnym punkcie brzegowym wynosi $1$.}
\end{problem}

Rząd funkcji 

Idea:
Ustalmy rozkład jedności dla zbiorów mapowych $\{h_i\}$. Nie rozumiem tego, co szachu mówi.

$$f(\underset{\in M}{x})=\sum g_j(\psi_j(x))\cdot h_j(x)$$

Teraz wypadałoby sprawdzić, że $h_j(x)$ śmiga. Skoro cośtam zeruje się na brzegu, to ma rząd $1$. Teraz trzeba policzyć pochodną. Kurwa zgubiłam się. Ale to, że rząd jest co najwyżej $1$ to też wynika z tego, że idziemy w $\R$. Teraz chcemy liczyć pochodną i policzyć, że się nie zeruje.

Pochodna to będzie suma pochodnych czyli będą kontrybucje z różnych map. Teraz musimy wziąc jedną konkretną mapę i policzyć jej pochodną i pokazać, że ta pochodna jest dodatnia, czyli i cała suma też będzie sumą dodatnich i będzie dodatnia.

Ustalmy dowolne $p\in\partial M$. Zauwazmy, że $f(p)$ jest skończenie wielu obrazów i skupmy się na jednym, nazwijmy go indeksem $\beta$.
$$g_\beta(\psi_\beta(p))\circ h_\beta(p)=\xi(p)$$
i chcemy pokazać, że ${\partial \xi(p)}> 0$

{\large\color{orange}DUUUŻO DZIWNYCH RZECZY }

Trzeba udwodonić lemacik, że $f:M\to\R$ i nie wiem co sie dzieje hyhy

Idea jest taka, że patrzymy na pochodną funkcji $D(f\circ\psi^{-1})$ to gradient. To jest macierz $1\times n$. Wiemy, że jest to coś dodatniego i potem coś. 
AAAAA DUPA DUPA DUPA DUPA DUPA DUPA DUPA DUPA DUPA DUPA DUPA DUPA DUPA

Ten lemat to w sumie trzeba zmienic, ale to i tak zdjęcia zrobię

\begin{problem}{}
Uzasadnij, że naturalne włożenie $i:S^n\to\R^{n+1}$ jest gładkie.
\end{problem}

To nie tak, że po prostu mamy identyczność?

\begin{problem}{d}
Niech $M,N$ będą rozmaitościami różniczkowalnymi i niech $f:M\to N$ będzie przekształceniem gładkim, zaś $g:N\to\R$ gładką funkcją rzeczywistą. Uzasadnij z definicji, że złożenie $g\circ f:M\to \R$ jest funkcją gładką.
\end{problem}

Niech $p\in M$, $q=f(p)\in N$ i $s=g(q)=g(f(p))\in\R$. Teraz ustalamy sobie mapy $(U,\phi),(V,\psi)$ wokół $p$ i $q$. Wiem, że $\psi\circ f\circ\phi^{-1}$ i $g\circ\psi^{-1}$ są gładkie i obie przyjmują i oddają wartości rzeczywiste, czyli ich złożenie na pewno będzie gładkie:
$$(g\psi^{-1})(\psi f\phi^{-1})=g\psi^{-1}\psi f\phi^{-1}=g f\phi^{-1}$$
jest gładkie, to jak włożymy do tego $\phi(p)$ też będzie gładkie. Koniec. 

\begin{illustration}
    \draw (0, 0) ellipse (1 and 1.5);
    \draw (3, 0) ellipse (1 and 1.5);
    \draw (0.2, -2) rectangle (2.8, -5);
    \node (p) at (0, 0.5) {$p$};
    \node (q) at (2.8, 0.3) {$q$};
    \node (s) at (1.5, -3) {$s$};
    \draw[->] (p)--(q) node[midway, above] {$f$};
    \draw[->] (q)--(s) node[midway, above] {$g$};
    \draw[->] (p)..controls(-1, -0.5)and(-1, -2)..(0, -3) node[midway, left] {$\phi$};
    \draw[->] (q)..controls(4, -0.5) and (4, -2)..(3.2,-3) node[midway, right] {$\psi$};
\end{illustration}

\begin{problem}{d}
Sprawdź, że dla naturalnej struktury rozmaitości gładkiej na produkcie $M\times N$ dwóch rozmaitości gładkich rzutowania $M\times N\to M$ i $M\times N\to N$ są odwzorowaniami gładkimi.
\end{problem}

Na poprzedniej liście już pokazaliśmy, że jest to ziomek gładki i chyba, że $(U_\alpha\times V_\beta, \phi_\alpha\times \psi_\beta)$ to jest atlas na produkcie kartezjańskim.

Weźmy sobie funkcję $f:M\times N\to M$ taką, że $f(p,q)=p$.

Niech $(p,q)\in M\times N$ i bierzemy mapy $(p,q)\in (U\times V,\phi\times \psi)$, $p\in(U,\phi)$. Chcę, żeby
$$\phi f(\phi\times\psi)^{-1}$$
było gładkie w punkcie $(\phi\times\psi)(p,q)$.
\begin{align*}
    \phi f(\phi\times\psi)^{-1}(\phi\times\psi)(p,q)=\phi(f(p,q))=\phi(p)
\end{align*}
no a to jest bardzo gładki?

\begin{problem}{}
Niech $\mathfrak{L}$ będzie rozmaitością prostych na płaszczyźnie.
\begin{enumerate}[label=(\alph*)]
    \item Zdefiniuj rozmaitość kierunków prostych na płaszczyźnie i pokaż, że odwzorowanie przyporządkowujące prostej z $\mathfrak{L}$ jej kierunek jest gładkie.
    \item Pokaż, że odwzorowanie $\mathfrak{L}\to\mathfrak{L}$ przyporządkowujące każdej prostej prostopadłą do niej przechodzącą przez punkt $(0,0)$ jest gładkie.
    \item Dane są gładkie funkcje $p:\R\to\R^2$ i $\theta:\R\to\R$. Niech $L:\R\to\mathfrak{L}$ będzie odwzorowaniem w którym $L(t)$ jest prostą przechodzącą przez punkt $p(t)$ i mającą kierunek $\theta(t)$ (liczony w mierze łukowej, tak, że wartości różniące się o $\pi$ oznaczają ten sam kierunek). Wykaż za pomocą map dla $\mathfrak{L}$, że $L$ jest odwzorowaniem gładkim (gładką krzywą w $\mathfrak{L}$).
\end{enumerate}
\end{problem}

Nie rozumiem czego ode mnie wymagają.

\begin{problem}{}
Odwzorowanie $F:S^3\to S^2$ zadane jest wzorem $F(z,w)=(z\overline{w}+w\overline{z},iw\overline{z}-iz\overline{w},z\overline{z}-w\overline{w})$, gdzie $S^3$ traktujemy jako podzbiór w $\C^2$ zadany równaniem $|z|^2+|w|^2=1$. Uzasadnij za pomocą wyliczenia w mapach, że $F$ jest odwzorowaniem gładkim. Wcześniej uzasadnij, że jest ono dobrze określone.
\end{problem}

$S^3$ mogę zanurzyć w $\R^4$, ale ja mam myśleć w terminach $\C^2$, czyli moje zbiory $U_1^\pm,U_2^\pm,U_3^\pm,U_4^\pm$ będą zbiorami które mają dla nieparzystych indeksów $Re(z)$ jest większe lub mniejsze niż $0$, a dla parzystych to $Im(z)$ jest warunkowane.

To jak wyglądałaby mi mapa $\phi_1^+:\C^2\to\R^3$ na
$$U_1^+=\{(z,w)\;:\;Re(z)>0\}?$$
$$\phi_1^+(z,w)=(Im(z), Re(w),Im(w))$$

Aby pokazać gładkość $F$, muszę brać dowolny $(p,q)\in S^3$ i mapę $(U,\phi)$ zawierającą $(p,q)$. Potem dobieram mapę $(V,\psi)$ zawierającą $F(p,q)$ i sprawdzam, że $\psi\circ F\circ\phi^{-1}$ jest gładkie.

Dobra, bierem $(p,q)\in S^3$ i żeby mi było dobrze w życiu, to chcę, aby $(p,q)\in U_1^+$, czyli $Re(p)>0$. Wtedy
\begin{align*}
    F(p,q)=(p\overline{q}+q\overline{p}, i[q\overline{p}-p\overline{q}], |p|-|q|)
\end{align*}
Na $S^2$ ustalam mapy $V_1^\pm,V_2^\pm,V_3^\pm$ i ten obraz musi wpadać w jedną z nich. Chyba mnie nie obchodzi w którą, bo ma szansę wpaść w każdą z nich, w zależności od tego jak będzie wyglądać $q$?


Niech $(x,y,z)=\phi(p,q)$, tzn. $p=w+ix,q=y+iz$.
\begin{align*}
    \psi\circ F\circ\phi^{-1}(x, y, z)=\psi(2(wy+xz),2(wz+xy),w^2+x^2-y^2-z^2)
\end{align*}
Nie ważne które usuniemy, zawsze dostajemy funkcję gładką, więc i całość jest gładka?














\end{document}
