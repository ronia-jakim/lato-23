\section{Wstęp}

Zanim podany dokładną definicję, możemy rozważyć kilka przykładów rozmaitości różniczkowalnych:

\indent \point powierzchnia, domknięta lub nie,

\indent \point przestrzeniach opisanych (lokalnie) skończoną liczbą parametrów,

\indent \point podzbiory $\R^n$ lub $\C^n$ zapisywalne równaniami algebraicznymi (np. $z_1^2+z_2^2+z_3^1$ w $\C^3$).

Cały wykład będzie wstępnym słownikiem wokół pojęcia rozmaitości różniczkowalnej.

\subsection{Rozmaitości topologiczne}

Przestrzeń topologiczna $M$ jest $n$-wymiarową \deff{rozmaitością topologiczną} [$n$-rozmaitością], jeżeli spełnia:

\indent 1. jest Hausdorffa,

\indent 2. ma przeliczalną bazę,

\indent 3. jest \acc{lokalnie euklidesowa} wymiaru $n$, czyli każdy punkt z $M$ posiada otwarte otoczenie w $M$ homeomorficzne z otwartym podzbiorem w $\R^n$.
\medskip

\deff{\large Konsekwencje Hausdorffowości:}
\smallskip

\indent \point Mamy wykluczone pewne patologie, na przykład przestrzeń

\begin{illustration}
    \draw[very thick] (-3, 0)--(0, 0);
    \draw[very thick] (0, 0.5)--(3, 0.5);
    \draw[very thick] (0, -0.5)--(3, -0.5);
    \filldraw[color=fore, fill=back, very thick] (0, 0) circle (0.1);
\end{illustration}

nie jest rozmaitością topologiczną.

\indent \point Pewne własności otoczeń punktów są zachowywane. To znaczy, dla dowolnego zwartego podzbioru otoczenia punktu $x\in U\subseteq \R^n$ $K\subseteq U$ jego odpowiednik $\overline K=\phi^{-1}(K)\subseteq \overline U\subseteq M$ jest domknięty i zwarty w $M$. [ćwiczenia]
\medskip

\deff{\large Konsekwencje przeliczalności bazy:}
\smallskip

\indent \point Spełniany jest warunek Lindel\"ofa: każde pokrycie rozmaitości zbiorami otwartymi zawiera przeliczalne podpokrycie. [ćwiczenia]

\indent \point Każda rozmaitość jest wstępującą sumą otwartych podzbiorów
$$U_1\subseteq U_2\subseteq ...\subseteq U_n\subseteq...$$
które są po domknięciu w $M$ zwarte. Czyli możemy ją wyczerpać za pomocą zbiorów, które są małe.

\indent \point \acc{Parazwartość}, czyli każde zwarte pokrycie $M$ posiada lokalnie skończone rozdrobnienie.

\indent \point Każdą rozmaitość jesteśmy w stanie zanurzyć w $\R^n$ dla odpowiednio dużego $n$.
\medskip

\deff{\large Konsekwencje lokalnej euklidesowości:}
\smallskip

\indent \point \acc{Twierdzenie Brouwer'a}: dla $n\neq m$ niepusty otwarty podzbiór $\R^n$ nie jest homeomorficzny z jakimkolwiek otwartym podzbiorem w $\R^m$. 

\indent \point Czyli liczba $n$ w definicji jest jednoznaczna dla danej rozmaitości. Określamy \deff{wymiar rozmaitości} $\dim M=n$.

\subsection{Mapy, lokalne współrzędne}

\deff{Mapą} na rozmaitości topologicznej $M$ nazywamy parę $(U, \phi)$, gdzie $U$ to otwarty podzbiór w $M$, a $\phi$ to homeomorfizm $\phi:U\to \overline U\subseteq \R^n$. Mapa to jest jakiś homeomorfizm między rozmaitością a pewnym podzbiorem $\R^n$. Zbiór $U$ nazywamy \acc{zbiorem mapowym}. \textbf{Przez lokalną euklidesowość wiemy, że pokrywają one całą rozmaitość.} 

Parę $(U, \phi)$ nazywamy też \deff{lokalnymi współrzędnymi} na $M$ albo \emph{lokalną parametryzacją} $M$.
\smallskip

\textbf{Fakt}: Hausdorffowska przestrzeń $X$ o przeliczalnej bazie jest $n$-rozmaitością $\iff$ posiada rodzinę map $n$-wymiarowych dla której zbiory mapowe pokrywają cały $X$.
\medskip

\textbf{\large Przykład:} Rozważmy $S^n=\{(x_1,...,x_n)\in\R^{n+1}\;:\;\sum x_i^2=1\}\subseteq \R^{n+1}$ z dziedziczoną topologią. Z racji, że $\R^{n+1}$ jest Hausdorffa i ma przeliczalną bazę, to $S^n$ tęż spełnia te dwa warunki. Wystarczy teraz wskazać odpowiednią rodzinę map, która pokryje całe $S^n$. Dla $i=1,..., n+1$ określmy otwarte podzbiory w $S^n$
$$U_i^+=\{x\in S^n\;:\;x_i>0\}$$
$$U_i^-=\{x\in S^n\;:\;x_i<0\}$$

\emph{\Large\color{red}RYSUNEK DLA $S^3$}

Określmy odwzorowania $\phi^\pm_i\;:\;U_i^\pm\to\R^n$
$$\phi_i^\pm(x)=(x_1,...,x_{i-1},\hat{x_i},x_{i+1},...,x_n).$$
Obraz tego odwzorowania to
$$\overline U_i^\pm=\phi_i^\pm(U_i^\pm)=\{(x_1,...,x_n)\in \R^n\;:\;\sum x_i^2<1\}.$$
Odwzorowanie $\phi_i^\pm: U_i^\pm\to\overline U_i^\pm$ jest wzajemnie jednoznaczne [bijekcja], bo
$$(\phi_i^\pm)^{-1}(x_1,...,x_n)=(x_1,...,x_{i-1}, \pm\sqrt{1-\sum x_j^2}, x_{i+1},...,x_n).$$
Mamy w obie strony odwzorowanie ciągłe, więc jest to homeomorfizmy z odpowiednimi zbiorami $\R^n$.

\subsection{Rozmaitości gładkie [różniczkowalne]}

Na tym wykładzie nie będziemy poświęcać dużej uwagi rozmaitościom różniczkowalnym nie nieskończenie razy, więc pomimo lekkich niuansów między tymi dwoma słowami, dla nas zwykle one znaczą to samo.
\medskip

Dla funkcji $f:M\to\R$ chcemy określić, co znaczy, że \dyg{$f$ jest różniczkowalna}? Będziemy to robić za pomocą wcześniej zdefiniowanych map:

\indent \point Funkcja $f$ \acc{wyrażona} w mapie $(U, \phi)$ to nic innego jak złożenie $f\circ \phi^{-1}:\overline U\to \R$. Teraz $f\circ\phi^{-1}$ jest funkcją zależącą od $n$ zmiennych rzeczywistych.

\indent \point Chciałoby się powiedzieć, że funkcja $f: M\to\R$ jest gładka, jeśli dla każdej mapy $(U, \phi)$ na $M$, ten fragment wyrażony w tej mapie $f\circ\phi^{-1}$ jest gładki. Niestety, tych map może być nieco za dużo.

\indent \point \acc{odwzorowanie przejścia między dwoma mapami}
\medskip

\deff{Mapy} $(U, \phi_1)$ oraz $(U, \phi_2)$ są \deff{zgodne} (gładko-zgodne), gdy odwzorowanie przejścia $\phi_1\phi_2^{-1}$ jest gładkie. Dla map \acc{$(U, \phi)$ i $(V, \psi)$} mówimy, że są one zgodne, jeśli 

\indent \point $U\cap V=\emptyset$, albo

\indent \point $\phi\psi^{-1}:\psi(U\cap V)\to \phi(U\cap V)$ i $\psi\phi^{-1}(U\cap V)\to \psi(U\cap V)$ są gładkie.

Warto zauważyć, że jeśli $(U, \phi)$ i $(V, \psi)$ są zgodne, to $f\circ\phi^{-1}\obciete (\phi(U\cap V))$ jest gładkie $\iff$

Odwzorowania przejściowe map są automatycznie \dyg{dyfeomorfizmami}.
\medskip

\deff{Gładkim atlasem} $\set{A}$ na topologicznej rozmaitości $M$ nazywamy dowolny taki zbiór map $\{(U_\alpha,\phi_\alpha)\}$ taki, że:
\smallskip

\indent 1. zbiory mapowe $U_\alpha$ pokrywają całe $M$

\indent 2. każde dwie mapy z tego zbioru są zgodne.
\medskip

\textbf{Przykład:} Rodzina map $\{(U_i^\pm,\phi_i^\pm)\;:\;i=1,2,...,n+1\}$ jak wcześniej na sferze $S^n\subseteq R^{n+1}$ tworzy gładki atlas. Wystarczy zbadać gładką zgodność tych map. Rozpatrzmy jeden przypadek: $(U_i^+,\phi_i^+),(U_j^+,\phi_j^+),i<j$. Po pierwsze, jak wygląda przekrój tych zbiorów?
$$U_i\cap U_j=\{x\in S^n\;:\;x_i>0,x_j>0\}$$
Dalej, jak wyglądają obrazy tego przekroju przez poszczególne mapy?
$$\phi_i^+(U_i\cap U_j)=\{x\in\R^n\;:\;|x|<1, x_{j-1}>0\}$$
$$\phi_j^+(U_i\cap U_j)=\{x\in\R^n\;:\;|x|<1,x_i<0\}$$
Odwzorowania przejścia to:
\begin{illustration}
    \node (L) at (0, 0) {$\phi_j^+(U_i^+\cap U_j^+)\ni(x_1,...,x_n)$};
    \node (C) at (4, -1) {$(x_1,...,x_{j-1},\sqrt{1-|x|^2},x_{j},...x_n)$};
\end{illustration}
$$\phi_i^+(\phi_j^+)^{-1}(x_1,...,x_n)=(x_1,...,x_{i-1}, x_{i+1},...,x_{j-1},\sqrt{1-|x|^2},x_j,...,x_n)$$
jest przekształceniem gładkim. Analogicznie dla drugiego odwzorowania przejścia.

\deff{Rozmaitość gładka} to para $(M, \set{A})$ złożona z rozmaitości $M$ i gładkiego atlasu $\set{A}$ na $M$.

\acc{Uściślenie:} Często $(M,\set{A}_1)$ i $(M, \set{A}_2)$ będące rozmaitościami gładkimi określają tę samą rozmaitość.
\medskip

\begin{important}
Niech $\set{A}$ będzie gładkim atlasem na $M$.

\indent 1. Mapa $(U,\phi)$ jest \deff{zgodna z atlasem $\set{A}$}, jeśli jest zgodna z każdą mapą z $\set{A}$.

\indent 2. Dwa \deff{atlasy $\set{A}_1,\set{A}_2$ na $M$ są zgodne}, jeśli każda mapa z $\set{A}_1$ jest zgodna z atlasem $\set{A}_2$.
\end{important}

\deff{\large Twierdzenie:} relacja atlasów jest relacją równoważności.

\textbf{Dowód:} Ćwiczenia.

Konwencja jest wtedy taka, że zgodne atlasy zadają tą samą strukturę gładką na $M$.

Zgodne atlasy można zsumować do jednego większego atlasu.
\medskip

$\set{A}$ jest \deff{atlasem maksymalnym} na $M$, jeśli każda mapa na $M$ zgodna z $\set{A}$ należy do $\set{A}$. 

\acc{Fakt} Każdy atlas $\set{A}$ na $M$ zawiera się w dokładnie jednym atlasie maksymalnym na $M$. Zaś ten atlas maksymalny to zbiór wszystkich map na $M$ zgodnych z $\set{A}$.
\medskip

\dyg{Rozmaitość gładką równoważnie definiuje się jako parę $(M, \set{A})$}, gdzie $M$ to rozmaitość topologiczna, zaś $\set{A}$ to pewien atlas maksymalny.

\subsection{Dopowiedzenie o funkcjach gładkich}

Funkcja $f:M\to \R$ jest \deff{gładka względem atlasu} $\set{A}$ na $M$, jeśli
$$(\forall\;(U,\phi)\in\set{A})\;f\circ\phi^{-1}:\overline U\to \R\text{ jest gładka.}$$
To znaczy po wyrażeniu w dowolnej mapie atlasu jest nadal funkcją gładką.

\acc{\large Fakt:}

\indent 1. Jeśli $f:M\to\R$ jest gładka względem $\set{A}$, zaś $(U,\phi)$ jest zgodna z $\set{A}$, to wówczas funkcja $f$ wyrażona w tej nowej mapie (czyli $f\circ\phi^{-1}$) też jest gładka.

\indent 2. Jeśli $\set{A}_1,\set{A}_2$ są zgodnymi atlasami, wówczas taka funkcja $f:M\to\R$ jest gładka względem $\set{A}_1$ $\iff$ jest gładka względem $\set{A}_2$ $\iff$ jest gładka względem atlasu maksymalnego $\set{A}\supseteq \set{A}_1,\set{A}_2$ zawierającego $\set{A}_1$ (oraz $\set{A_2}$).
\medskip

Niech $M$ będzie gładką rozmaitością. Wówczas $f:M\to\R$ \deff{jest gładka} jeśli $f$ jest gładka względem każdego (dowolnego) atlasu $\set{A}$ wyznaczającego na $M$ daną gładką strukturę.
\medskip

\podz{dark-blue}
\medskip

Dwie mapy $(U,\phi)$ i $(V,\psi)$ są \deff{$C^k$-zgodne}, jeśli $\phi\psi^{-1}$ oraz $\psi\phi^{-1}$ są funkcjami klasy $C^k$.

\deff{$C^k$-atlas} to atlas składający się z map, które są $C^k$-zgodne. Taki atlas określa strukturę $C^k$-rozmaitości na $M$. Jest to coś słabszego niż struktura rozmaitości gładkiej.

$C^0$ tutaj to jest rozmaitość topologiczna, a $C^\infty$ to często jest rozmaitość gładka.

Na $C^k$-rozmaitości nie da się sensownie określić funkcji klasy $C^m$ dla $m>k$.

\deff{Rozmaitość analityczna} [$C^\omega$] to rozmaitość, dla której atlas składa się z map analitycznie zgodnych (czyli wyrażają się za pomocą szeregów potęgowych).

\deff{Rozmaitość zespolona} ma mapy jako funkcje w $\C^n$ zamiast w $\R^n$

Rozmaitość konformena - zachowuje kąty.

kawałkami liniowe

Dychotomia pomiędzy sytuacją $C^0$ a sytuacją $C^k$ dla $k>0$:

\indent \point Z każdego maksymalnego atlasu $C^k$-rozmaitości można wybrać atlas złożony z map $C^\infty$-zgodnych. A zatem, każda $C^k$-rozmaitość posiada $C^k$-zgodną strukturę $C^\infty$-rozmaitości.

\indent \point Istnieją $C^0$-rozmaitości niedopuszczające żadnej struktury gładkiej.

Definiowanie rozmaitości gładkiej za pomocą samego atlasu (bez odwołań do topologii).

\acc{Lemat:} Niech $X$ będzie zbiorem (bez topologii). Niech $\{U_\alpha\}$ będzie kolekcją podzbiorów $X$ i dla każdego $\alpha$ mamy $\phi_\alpha:U_\alpha\to\R^n$ różnowartościowe ($n$ jest ustalone dla całego $X$). Ta trójka obiektów ma spełniać następujące warunki:

\indent 1. Dla każdego $\alpha$ $\phi_\alpha(U_\alpha)$ jest otwarty w $\R^n$

\indent 2. Dla każdych $\alpha,\beta$ $\phi_\alpha(U_\alpha\cap U_\beta)$ oraz $\phi_\beta(U_\alpha\cap U_\beta)$ są otwarte w $\R^n$

\indent 3. Gdy $U_\alpha\cap U_\beta\neq\emptyset$, to $\phi_\alpha\circ\phi_\beta^{-1}:\phi(U_\alpha\cap U_\beta)\to\phi(U_\alpha\cap U_\beta)$ jest odwzorowaniem gładkim. Są to dyfeomorfizmy (gładkie i odwracalne).

\indent 4. Przeliczalnie wiele spośród zbiorów $U_\alpha$ pokrywa całe $X$.

\indent 5. Dla dowolnych punktów $p,q\in X,p\neq q$ istnieją $\alpha,\beta$ oraz otwarte podzbiory $V_p\subseteq\phi_\alpha(U_\alpha)$, $V_q\subseteq\phi_\beta(U_\beta)$ takie, że $p\in\phi_\alpha^{-1}[V_p]$, $q\in\phi_\beta^{-1}[V_q]$ oraz $\phi_\alpha^{-1}[V_p]\cap\phi_\beta^{-1}[V_q]=\emptyset$. Czyli możemy rozdzielić dwa dowolne różne punkty za pomocą zbiorów otwartych w $\R^n$.

Wówczas na $X$ istnieje struktura rozmaitości topologicznej dla której $U_\alpha$ są otwarte. Ponadto rodzina $(U_\alpha,\phi_\alpha)$ tworzy gładki atlas na $X$.

\textbf{Szkic dowodu:} Topologię produkujemy jako bazę topologii na $X$ bierzemy przeciwobrazy przez poszczególne $\phi_\alpha$ otwartych podzbiorów w zbiorach $\phi_\alpha(U_\alpha)\subseteq\R^n$.

Lokalna $n$-euklidesowość $X$ względem takiej topologii jest oczywista. Nietrudno jest też wybrać mniejszą bazę przeliczalną [ćwiczenia]. Hausdorffowość tak określonej topologii wynika z warunku 5.

\textbf{Przykład:} Niech $\set{L}$ będzie zbiorem wszystkich prostych na płaszczyźnie. Nie ma na tym zbiorze wygodnej do opisania topologii, ale możemy skorzystać z lematu wyżej.

Zacznijmy od opisania podzbiorów 
$$U_h=\{\text{proste niepionowe}\}$$
$$U_v=\{\text{proste niepoziome}\}$$
Jeśli $U_h\ni L$, to wtedy $L=\{y=ax+b\}$ i wtedy $\phi_h$ będzie przypisywać takiej prostej parę $(a, b)$. Jeśli zaś $U_v\ni L$, to wtedy $L=\{x=yc+d\}$ i wtedy $\phi_v$ przypisze jej $(c, d)$. To, że $\phi_h(U_h)$ i $\phi_v(U_v)$ są różnowartościowe widać. Przyjrzyjmy się teraz przekrojowi naszych zbiorków:
$$U_h\cap U_v=\{\text{proste niepoziomie i niepionowe}\}=\{y=ax+b\;:\;a\neq 0\}=\{x=cd+d\;:\;c\neq0\}$$
$$\phi_h(U_h\cap U_v)=\{(a,b)\in\R^2\;:\;a\neq 0\}$$
$$\phi_v(U_h\cap U_v)=\{(c,d)\in\R^2\;:\;c\neq0\}$$
i są to zbiory otwarte, więc warunek 3. jest spełniony. Warunek 4. jest tutaj trywialny.

Niech 

To jest homeomorficzne z wnętrzem wstęgi Moejashdfkjasd