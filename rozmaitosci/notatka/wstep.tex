\section{Wstęp}

Zanim podany dokładną definicję, możemy rozważyć kilka przykładów rozmaitości różniczkowalnych:

\indent \point powierzchnia, domknięta lub nie,

\indent \point przestrzeniach opisanych (lokalnie) skończoną liczbą parametrów,

\indent \point podzbiory $\R^n$ lub $\C^n$ zapisywalne równaniami algebraicznymi (np. $z_1^2+z_2^2+z_3^1$ w $\C^3$).

Cały wykład będzie wstępnym słownikiem wokół pojęcia rozmaitości różniczkowalnej.

\subsection{Rozmaitości topologiczne}

Przestrzeń topologiczna $M$ jest $n$-wymiarową \deff{rozmaitością topologiczną} [$n$-rozmaitością], jeżeli spełnia:

\indent 1. jest Hausdorffa,

\indent 2. ma przeliczalną bazę,

\indent 3. jest \acc{lokalnie euklidesowa} wymiaru $n$, czyli każdy punkt z $M$ posiada otwarte otoczenie w $M$ homeomorficzne z otwartym podzbiorem w $\R^n$.
\medskip

\deff{\large Konsekwencje Hausdorffowości:}
\smallskip

\indent \point Mamy wykluczone pewne patologie, na przykład przestrzeń

\begin{illustration}
    \draw[very thick] (-3, 0)--(0, 0);
    \draw[very thick] (0, 0.5)--(3, 0.5);
    \draw[very thick] (0, -0.5)--(3, -0.5);
    \filldraw[color=fore, fill=back, very thick] (0, 0) circle (0.1);
\end{illustration}

nie jest rozmaitością topologiczną.

\indent \point Pewne własności otoczeń punktów są zachowywane. To znaczy, dla dowolnego zwartego podzbioru otoczenia punktu $x\in U\subseteq \R^n$ $K\subseteq U$ jego odpowiednik $\overline K=\phi^{-1}(K)\subseteq \overline U\subseteq M$ jest domknięty i zwarty w $M$. [ćwiczenia]
\medskip

\deff{\large Konsekwencje przeliczalności bazy:}
\smallskip

\indent \point Spełniany jest warunek Lindel\"ofa: każde pokrycie rozmaitości zbiorami otwartymi zawiera przeliczalne podpokrycie. [ćwiczenia]

\indent \point Każda rozmaitość jest wstępującą sumą otwartych podzbiorów
$$U_1\subseteq U_2\subseteq ...\subseteq U_n\subseteq...$$
które są po domknięciu w $M$ zwarte. Czyli możemy ją wyczerpać za pomocą zbiorów, które są małe.

\indent \point \acc{Parazwartość}, czyli każde zwarte pokrycie $M$ posiada lokalnie skończone rozdrobnienie.

\indent \point Każdą rozmaitość jesteśmy w stanie zanurzyć w $\R^n$ dla odpowiednio dużego $n$.
\medskip

\deff{\large Konsekwencje lokalnej euklidesowości:}
\smallskip

\indent \point \acc{Twierdzenie Brouwer'a}: dla $n\neq m$ niepusty otwarty podzbiór $\R^n$ nie jest homeomorficzny z jakimkolwiek otwartym podzbiorem w $\R^m$. 

\indent \point Czyli liczba $n$ w definicji jest jednoznaczna dla danej rozmaitości. Określamy \deff{wymiar rozmaitości} $\dim M=n$.

\subsection{Mapy, lokalne współrzędne}

\deff{Mapą} na rozmaitości topologicznej $M$ nazywamy parę $(U, \phi)$, gdzie $U$ to otwarty podzbiór w $M$, a $\phi$ to homeomorfizm $\phi:U\to \overline U\subseteq \R^n$. Mapa to jest jakiś homeomorfizm między rozmaitością a pewnym podzbiorem $\R^n$. Zbiór $U$ nazywamy \acc{zbiorem mapowym}. \textbf{Przez lokalną euklidesowość wiemy, że pokrywają one całą rozmaitość.} 

Parę $(U, \phi)$ nazywamy też \deff{lokalnymi współrzędnymi} na $M$ albo \emph{lokalną parametryzacją} $M$.
\smallskip

\textbf{Fakt}: Hausdorffowska przestrzeń $X$ o przeliczalnej bazie jest $n$-rozmaitością $\iff$ posiada rodzinę map $n$-wymiarowych dla której zbiory mapowe pokrywają cały $X$.
\medskip

\textbf{\large Przykład:} Rozważmy $S^n=\{(x_1,...,x_n)\in\R^{n+1}\;:\;\sum x_i^2=1\}\subseteq \R^{n+1}$ z dziedziczoną topologią. Z racji, że $\R^{n+1}$ jest Hausdorffa i ma przeliczalną bazę, to $S^n$ tęż spełnia te dwa warunki. Wystarczy teraz wskazać odpowiednią rodzinę map, która pokryje całe $S^n$. Dla $i=1,..., n+1$ określmy otwarte podzbiory w $S^n$
$$U_i^+=\{x\in S^n\;:\;x_i>0\}$$
$$U_i^-=\{x\in S^n\;:\;x_i<0\}$$

\emph{\Large\color{red}RYSUNEK DLA $S^3$}

Określmy odwzorowania $\phi^\pm_i\;:\;U_i^\pm\to\R^n$
$$\phi_i^\pm(x)=(x_1,...,x_{i-1},\hat{x_i},x_{i+1},...,x_n).$$
Obraz tego odwzorowania to
$$\overline U_i^\pm=\phi_i^\pm(U_i^\pm)=\{(x_1,...,x_n)\in \R^n\;:\;\sum x_i^2<1\}.$$
Odwzorowanie $\phi_i^\pm: U_i^\pm\to\overline U_i^\pm$ jest wzajemnie jednoznaczne [bijekcja], bo
$$(\phi_i^\pm)^{-1}(x_1,...,x_n)=(x_1,...,x_{i-1}, \pm\sqrt{1-\sum x_j^2}, x_{i+1},...,x_n).$$
Mamy w obie strony odwzorowanie ciągłe, więc jest to homeomorfizmy z odpowiednimi zbiorami $\R^n$.

\subsection{Rozmaitości gładkie [różniczkowalne]}

Na tym wykładzie nie będziemy poświęcać dużej uwagi rozmaitościom różniczkowalnym nie nieskończenie razy, więc pomimo lekkich niuansów między tymi dwoma słowami, dla nas zwykle one znaczą to samo.
\medskip

Dla funkcji $f:M\to\R$ chcemy określić, co znaczy, że \dyg{$f$ jest różniczkowalna}? Będziemy to robić za pomocą wcześniej zdefiniowanych map:

\indent \point Funkcja $f$ \acc{wyrażona} w mapie $(U, \phi)$ to nic innego jak złożenie $f\circ \phi^{-1}:\overline U\to \R$. Teraz $f\circ\phi^{-1}$ jest funkcją zależącą od $n$ zmiennych rzeczywistych.

\indent \point Chciałoby się powiedzieć, że funkcja $f: M\to\R$ jest gładka, jeśli dla każdej mapy $(U, \phi)$ na $M$, ten fragment wyrażony w tej mapie $f\circ\phi^{-1}$ jest gładki. Niestety, tych map może być nieco za dużo.

\indent \point \acc{odwzorowanie przejścia między dwoma mapami}
\medskip

\deff{Mapy} $(U, \phi_1)$ oraz $(U, \phi_2)$ są \deff{zgodne} (gładko-zgodne), gdy odwzorowanie przejścia $\phi_1\phi_2^{-1}$ jest gładkie. Dla map \acc{$(U, \phi)$ i $(V, \psi)$} mówimy, że są one zgodne, jeśli 

\indent \point $U\cap V=\emptyset$, albo

\indent \point $\phi\psi^{-1}:\psi(U\cap V)\to \phi(U\cap V)$ i $\psi\phi^{-1}(U\cap V)\to \psi(U\cap V)$ są gładkie.

Warto zauważyć, że jeśli $(U, \phi)$ i $(V, \psi)$ są zgodne, to $f\circ\phi^{-1}\obciete (\phi(U\cap V))$ jest gładkie $\iff$

Odwzorowania przejściowe map są automatycznie \dyg{dyfeomorfizmami}.
\medskip

\deff{Gładkim atlasem} $\set{A}$ na topologicznej rozmaitości $M$ nazywamy dowolny taki zbiór map $\{(U_\alpha,\phi_\alpha)\}$ taki, że:
\smallskip

\indent 1. zbiory mapowe $U_\alpha$ pokrywają całe $M$

\indent 2. każde dwie mapy z tego zbioru są zgodne.
\medskip

\deff{Rozmaitość gładka} to para $(M, \set{A})$ złożona z rozmaitości $M$ i gładkiego atlasu $\set{A}$.