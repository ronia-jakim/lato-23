\section{Rozkłady jedności}
\emph{Motywacja: jak sklejać funkcje?} W szczególności, jak uzasadnić, że na każdej rozmaitości z brzegiem $M$ istnieje gładka funkcja $f:M\to\R^n$ taka, że:
$$
\begin{matrix}
f(p)=0 & p\in\partial M\\
f(p)>0 & p\in Int(M)?
\end{matrix}
$$

\begin{definition}[rodzina lokalnie skończona]
Rodzina $\{A_i\}$ podzbiorów przestrzeni topologicznej $X$ jest \important{lokalnie skończona}, jeżeli dla każdego $p\in X$ istnieje otwarte otoczenie $p\in U_p$ w $X$ takie, że $U_p\cap A_\alpha\neq\emptyset$ tylko dla skończenie wielu $\alpha$.
\end{definition}

\begin{definition}[nośnik funkcji]
Dla funkcji rzeczywistej $f:X\to\R$ jej \important{nośnik} to
$$supp(f)=cl(\{x\in X\;:\;f(x)\neq0\})$$
\end{definition}
