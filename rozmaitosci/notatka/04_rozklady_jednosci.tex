\section{Rozkłady jedności}
\emph{Motywacja: jak sklejać funkcje?} W szczególności, jak uzasadnić, że na każdej rozmaitości z brzegiem $M$ istnieje gładka funkcja $f:M\to\R^n$ taka, że:
$$
\begin{matrix}
f(p)=0 & p\in\partial M\\
f(p)>0 & p\in Int(M)?
\end{matrix}
$$

\subsection{Nośnik funkcji}

\begin{definition}[rodzina lokalnie skończona]
Rodzina $\{A_i\}$ podzbiorów przestrzeni topologicznej $X$ jest \important{lokalnie skończona}, jeżeli dla każdego $p\in X$ istnieje otwarte otoczenie $p\in U_p$ w $X$ takie, że $U_p\cap A_\alpha\neq\emptyset$ tylko dla skończenie wielu $\alpha$.
\end{definition}

\begin{definition}[nośnik funkcji]
Dla funkcji rzeczywistej $f:X\to\R$ jej \important{nośnik} to
$$supp(f)=cl(\{x\in X\;:\;f(x)\neq0\})$$
\end{definition}


\begin{theorem}[o rozkładzie jedności]
\emph{[Twierdzenie o rozkładzie jedności]} Dla każdego otwartego pokrycia $\{U_\alpha\}$ rozmaitości gładkiej $M$ istnieje rodzina $\{f_j\}_{j\in J}$ gładkich funkcji $f_j:M\to\R$ takich, że
\begin{itemize}
    \item $f_j\geq0$
    \item każdy nośnik $supp(f_j)$ zawiera się w pewnym $U_\alpha$ z pokrycia
    \item nośniki $\{supp(f_j)\}_{j\in J}$ tworzą lokalnie skończoną rodzinę podzbiorów w $M$.
    \item dla każdego $x\in M\;\sum\limits_{j\in J}f_j(x)=1$
\end{itemize}
Jest to \important{rozkład jedności wpisany w pokrycie $\{U_\alpha\}$}
\end{theorem}
