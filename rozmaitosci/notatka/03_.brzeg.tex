\section{Rozmaitość z brzegiem}

Lokalnie wygląda jak $\R^n$ albo jak półprzestrzeń $n$-wymiarowa:
$$\color{blue}H^n=\{(x_1,...,x_n)\in\R^n\;:\;x_n\geq 0\}$$
\deff{brzegiem} takiej półprzestrzeni nazywamy zbiór:
$$\partial H^n=\{x\in\R^n\;:\;x_n=0\}$$
definiuje się też wnętrze takiej półprzestrzeni:
$$int(H^n)=\{x\in\R^n\;:\;x_n>0\}$$

\begin{definicja}[brzeg, wnętrze zbioru otwartego, gładka funkcja ze zbioru]
Dla otwartego zbioru $U\subseteq H^n$ określamy
\begin{itemize}
    \item[\point] brzeg zbioru: $\color{blue}\partial U=U\cap \partial H^n$
    \item[\point] wnętrze zbioru: $\color{blue}int(U)=U\cap int(H^n)$ 
    \item[\point] Jeżeli mamy zadane $f:U\to\R^m$, to jest ono \deff{gładkie}, gdy jest obcięciem do $U$ pewnej gładkiej funkcji $\overline f:\overline U\to \R^m$, gdzie $\overline U$ jest otwartym podzbiorem $\R^n$ taki, że $U\subseteq\overline U$.
\end{itemize}
\end{definicja}

Jeśli $f:U\to\R^m$ jest gładka, to wówczas pochodne cząstkowe $f$ są dobrze określone w punktach $int(U)$. Ze względu na ciągłość pochodnych cząstkowych dowolnego rozszerzenia $\overline f$, \acc{pochodne cząstkowe $f$ są również dobrze określone w punktach $\partial U$}. 

\begin{fakt}[o istnieniu rozszerzenia funkcji]
Z analizy: rozszerzenie $\overline f$ istnieje $\iff$ $f$ jest gładka na $int(U)$ oraz pochodne cząstkowe tego $f$ obciętego do $int(U)$ w sposób ciągły rozszerzają się na $\partial U$.
\end{fakt}

\begin{definicja}[gładka rozmaitość z brzegiem]
$M$ jest \deff{gładką rozmaitością z brzegiem}, jeśli posiada atlas $\{(U_\alpha,\phi_\alpha)\}$ taki, że 
\begin{itemize}
    \item[\point] $U_\alpha$ jest otwartym podzbiorem $M$ 
    \item[\point] oraz $\phi_{\alpha}:U_\alpha\to H^n$ jest homeomorfizmem na swój obraz, 
    \item[\point] $\overline U_\alpha=\phi(U_\alpha)\subseteq H^n$ jest otwarty,
    \item[\point] odwzorowania przejścia $\phi_\alpha\phi_\beta^{-1}:\phi_\beta(U_\alpha\cap U_\beta)\to\phi_\alpha(U_\alpha\cap U_\beta)$ są gładkie [$U_\alpha\cap U_\beta\subseteq H^n$ otwarte].
\end{itemize}
\end{definicja}

\begin{fakt}[jeśli obraz punktu jest w rzegu w jednej mapie, to jest w brzegu w każdej]
Jeśli w pewnej mapie $(U_\alpha,\phi_\alpha)$ $\phi_\alpha(p)\in\partial H^n$, to w każdej innej mapie $(U_\beta,\phi_\beta)$ zawierającej punkt $p$ również obraz punktu $p$ należy do brzegu $H^n$.
\end{fakt}

\textbf{Dowód:}

Odwzorowania przejścia są gładkie, ale gładkie są też odwzorowania odwrotne, czyli $\phi_\alpha\phi_\beta^{-1}$ są gładkie i gładko odwracalne.

Twierdzenie o odwzorowaniu otwartym z analizy wielu zmiennych

Odwzorowania przejścia mają nieosobliwe macierze pierwszych pochodnych cząstkowych we wszystkich punktach.
\proofend
\medskip

\begin{uwaga}[fakt wyżej jest prawdziwy dla rozmaitości topologicznych z brzegiem]
    Dla rozmaitości topologicznych z brzegiem (ta sama definicja, tylko odwzorowania przejścia nie muszą być gładkie, a wystarczy homeomorfizmy) dowód wyżej nie śmignie, ale \dyg{analogiczny fakt również zachodzi}, tylko dowód jest trudniejszy i opiera się na twierdzeniu Brouwera o niezmienniczości obszaru (analog twierdzenia o odwzorowaniach otwartych dla ciągłych $fLR\to\R^n$)
\end{uwaga}

Dzięki twierdzeniom powyżej następujące definicje mają sens:
$$\color{blue}\partial M=\{p\in M\;:\;\text{w pewnej mapie (każdej) }\phi_\alpha(p)\in\partial H^n\}$$ 
$$\color{blue}int(M)=\{p\in M\;:\;\text{dla pewnej mapy }(U_\alpha,\phi_\alpha),\;\phi_\alpha(p)\in int(H^n)\}$$

\subsection{O brzegu i wnętrzu}

\begin{fakt}[wnętrze rozmaitości jest rozmaitością]
    Wnętrze $int(M)$ $n$-rozmaitości gładkiej $M$ jest $n$-rozmaitością gładką bez brzegu.
\end{fakt}

\textbf{Dowód:} 

Pokażemy atlas, który działa dla $int(M)$. Weźmy $\{(U_\alpha',\phi_\alpha')\}$, gdzie
$$U_\alpha'=U_\alpha\cap int(M),\quad \phi_\alpha'=\phi_\alpha\obciete U_\alpha$$
a $(U_\alpha,\phi_\alpha)$ było atlasem na $M$.
\proofend

\begin{fakt}[brzeg rozmaitości jest rozmaitością]
    Brzeg $\partial M$ $n$-rozmaitości $M$ z brzegiem jest $(n-1)$ wymiarową rozmaitością gładką bez brzegu.
\end{fakt}

\textbf{Dowód:}

Jako atlas na $\partial M$ bierzemy $\{(U_\alpha',\phi_\alpha')\}$, gdzie 
$$U_\alpha'=U_\alpha\cap\partial U_\partial M$$
$$\phi_\alpha':U_\alpha'\to\R^{n-1}=\partial H^n\quad \phi_\alpha'=\phi_\alpha\obciete U_\alpha'$$
\proofend

\textbf{Przykład:} Dysk $D^n=\{x\in\R^n\;:\;|x|\leq 1\}$ jest rozmaitością gładką z brzegiem $\partial D^n=\{x\in\R^n\;:\;|x|=1\}$. Pokażemy mapy, ale uzasadnienie ich gładkiej zgodności pominiemy.
$$(U_0,\phi_0)\quad:\quad U_0=\{x\;:\;|x|<1\},\quad\phi_0:U_0\to H^n,\;\phi_0(x_1,...,x_n)=(x_1,...,x_{n-1},x_n+2)$$

\begin{illustration}
    \filldraw[fill=black!70!green!80] (0, 0) circle (1);
    \filldraw[fill=black!70!green!80] (5, 1) circle (1);
    \draw[->] (0, -2)--(0, 3);
    \draw[->] (-2, 0)--(2.5, 0);
    \draw[->] (5, -2)--(5, 3);
    \draw[->] (3, 0)--(7.5, 0);
    \draw[->] (1, 0.5)..controls (1.5, 1.5) and (3, 1.8)..(3.8, 1) node [midway, above] {$\phi_0$};
\end{illustration}

$$(U_i^\pm,\phi_i^\pm)\quad:\quad U_i^\pm=\{x\in D^n\;:\;\pm x_i>0\},\quad\phi_1:U_1\to H^n$$

\begin{illustration}
    \draw[->] (-2, 0)--(3, 0);
    \draw[->] (0, -3)--(0, 3);
    \draw (0, 0) circle (1.5); 
    \draw[very thick] (1.5, -3)--(1.5, 3);
    \node at (2.1, -1.9) {$\R^{n-1}$};
    \node at (2.4, -2.5) {$\{x_i=1\}$};
    \draw (0, 0)--(1.5, 2);
    \draw[very thick] (0, 0)--(0.5, 0.66) node [midway, left] {$r$};
    \node at (0.5, 0.66) {$\bullet$};
    \node at (0.5, 1) {$p$};
    \node at (1.5, 2) {$\bullet$};
    \node at (2, 2) {$\pi(p)$};
\end{illustration}
Czyli w punkcie opisujemy $n-1$ wymiarową płaszczyznę styczną i rzucamy punkty $p\in D^n$ przez rzut odśrodkowy $\pi$ na tę płaszczyznę. Funkcje $\phi_i^\pm$ opisują się wtedy wzorem:
$$\phi_i^\pm(p)=(\pi(p), 1-r^2)$$
lub konkurencyjnie
$$\phi_i^\pm(x_1,...,x_n)=(\frac {x_1}{x_i},...,\frac {x_{i-1}}{x_i},\frac {x_{i+1}}{x_i},...,\frac {x_n}{x_i}, 1-\sum\limits_{i=1}^nx_i^2)$$

Inny atlas gładki na dysku $D^n$ (zgodny z poprzednim)
\begin{illustration}
    \filldraw[color=back, pattern={Lines[distance=5pt,line width=.8pt,angle=40]}, pattern color=black!70!green!80] (-1.5, -3) rectangle (-8, 3);
    \filldraw[color=back, pattern={Lines[distance=5pt,line width=.8pt,angle=40]}, pattern color=black!70!green!80] (1.5, -3) rectangle (8, 3);
    \draw[black!70!green!80](-1.5, -3)--(-1.5, 3);
    \draw[black!70!green!80](1.5, -3)--(1.5, 3);
    \draw (0, 0) circle (1.5);
    \draw[->](0, 0)--(3.3, 0);
    \draw[->](0, 0)--(0, 3.3);
    \filldraw (1.5, 0) circle (1pt) node [above right] {B};
    \filldraw (-1.5, 0) circle (1pt) node [above left] {A};
    \node at (3, -2.5) {$\{x_1\geq1\}=H_A^n$};
    \node at (-3, -2.5) {$\{x_1\leq-1\}=H_B^n$};
\end{illustration}
$$U_A=D^n\setminus\{A\}$$
$$U_B=D^n\setminus\{B\}$$
$$\phi_A:U_A\to H_A^n\leftarrow\text{inwersja względem sfery o środku A i }r=2$$

\subsection{Rozkłady jedności}

\emph{Motywacja: jak uzasadnić, że na każdej rozmaitości z brzegiem $M$ istnieje gładka funkcja $f$ taka, że $f:M\to\R^n$ taka, że}
$$\begin{matrix}
    f(p)=0&p\in \partial M\\
    f(p)>0&p\in Int(M)?
\end{matrix}$$

Na zbiorze mapowym możemy taką funkcję zadać przez:
$$\overline f_\alpha:\overline U_\alpha\to\R$$
$$\overline f_\alpha(x_1,...,x_n)=x_n$$
$$f_\alpha:U_\alpha\to\R$$
$$f_\alpha=\overline f_\alpha\circ \phi_\alpha$$
Czyli zmuszamy funkcję do bycia gładką.

{\large\color{orange}TU PEWNIE JAKIŚ BULLSHIT PISZĘ, DOCZYTAĆ I POPRAWIĆ.}

% \begin{definicja}[rodzina lokalnie skończona]
%     Rodzina $\{A_\alpha\}$ podzbiorów przestrzeni topologicznej $X$ jest \deff{lokalnie skończona}, jeśli dla każdego $p\in X$ istnieje otwarte otocznie $p\in U_p\subseteq X$ takie, że $U_p\cap A_\alpha\neq\emptyset$ tylko dla skończenie wielu zbiorów. 
% \end{definicja}

\begin{definicja}[rodzina lokalnie skończona]
Rodzina $\{A_i\}$ podzbiorów przestrzeni topologicznej $X$ jest \deff{lokalnie skończona}, jeśli dla każdego $p\in X$ istnieje otwarte otoczenie $p\in U_p$ w $X$ takie, że $U_p\cap A_\alpha\neq\emptyset$ tylko dla skończenie wielu $\alpha$.
\end{definicja}

\begin{definicja}[nośnik funkcji]
Dla funkcji rzeczywistej $f:X\to\R$ jej \deff{nośnik} $supp(f)=cl(\{x\in X\;:\;f(x)\neq0\})$
\end{definicja}

\begin{tw}[o rozkładzie jedności]
[\dyg{Twierdzenie o rozkładzie jedności}] Dla każdego otwartego pokrycia $\{U_\alpha\}$ rozmaitości gładkiej $M$ (może być z brzegiem) istnieje rodzina $\{f_j\}_{j\in J}$ gładkich funkcji $f_j:M\to\R$ takich, że
\begin{itemize}
    \item $f_j\geq0$
    \item każdy nośnik $supp(f_j)$ zaiwera się w pewnym $U_\alpha$ z pokrycia
    \item nośniki $\{supp(f_j)\}_{j\in J}$ tworzą lokalnie skończoną rodzinę podzbiorów w $M$
    \item dla każdego $x\in M$ $\sum\limits_{j\in J}f_j(x)=1$
\end{itemize}
Jest to \deff{rozkład jedności wpisany w pokrycie $\{U_\alpha\}$}
\end{tw}

Wracamy do pytania o istnienie $f:M\to\R$ takiego, że $f\obciete\partial M\equiv 0$ i $f\obciete int(M)>0$. 

Niech $\{U_\alpha\}$ będzie dowolnym pokryciem rozmaitości $M$ zbiorami mapowymi. Wtedy $f_\alpha:U_\alpha\to\R$ jest gładka, jeśli
\begin{itemize}
    \item $U_\alpha\cap\partial M\neq\emptyset\implies f_\alpha=\overline f_\alpha\phi_\alpha$, gdzie $\overline f_\alpha:\overline U_\alpha\to\R,\; \overline f_\alpha(x_1,...,x_n)=x_n$
    \item $U_\alpha\cap\partial M=\emptyset\implies f_\alpha=1$
\end{itemize}
Niech $\{h_j\}$ będzie rozkładem jedności wpisanym w $\{U_\alpha\}$. Dla każdego $j\in J$ wybieramy $\alpha(j)$ takie, że $supp(h_j)\subseteq U_{\alpha(j)}$. Definiujemy wtedy $h_j'=h_j\cdot f_{\alpha(j)}:M\to \R$ takie, że
$$h_j'(p)=\begin{cases}
    h(p)f_{\alpha(j)}(p)\quad p\in U_{\alpha(j)}\\
    0
\end{cases}$$
taka funkcja jest gładka, bo $supp(h_j)\subseteq U_{\alpha(j)}$.