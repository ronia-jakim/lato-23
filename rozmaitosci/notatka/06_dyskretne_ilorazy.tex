\section{Dyskretne ilorazy rozmaitości gładkich przez grupy dyffeomorfizmów}

\begin{definition}[grupa dyfeomorfizmów]
    \acc{Grupa $G$ dyfeomorfizmów rozmaitości $M$} to dowolny niepusty zbiór dyfeomorfizmów $g:M\to M$, który jest zamknięty na operację składania i brania odwrotności. Elementem identycznym jest $id_M$, a odwrotne to dyfeomorfizmy odwrotne. Grupa $G$ działa przez dyfeomorfizmy na rozmaitość $M$.
\end{definition}

\begin{definition}[orbita, rozbicie]
    \acc{Orbitą} punktu $x\in M$ względem działania $G$ na $M$ nazywamy zbiór
    $$\color{blue}G(x)=\{g(x)\;:\;g\in G\}$$
    Rodzina wszystkich orbit tworzy \acc[b]{rozbicie rozmaitości $M$} na podzbiory.
\end{definition}
Dwie orbity są albo całkiem rozłączne, albo pokrywają się.

\begin{definition}
    Zbiór orbit to $M/G$. $M/G$ tak naprawdę oznacze przestrzeń ilorazową działania $G$ na $M$, czyli przestrzeń topologiczną której elementami są orbity działania $G$ na $M$, zaś topologia jest \acc{ilorazowa}. To znaczy, że zbiór orbit jest otwarty w tym ilorazie $\iff$ suma tych orbit tworzy otwarty zbiór w $M$.
\end{definition}

Na przykład, jeśli $U\subseteq M$ jest otwarty, to $G(U)/G:=\{G(x)\;:\;x\in U\}$, to ten zbiór jest zbiorem otwarty w $M/G$. Co więcej, każdy otwarty zbiór w $M/G$ ma postać $G(U)/G$ jak wyżej. Czyli jeśli $\set{B}$ jest bazą na $M$, to wtedy 
$$\{G(U)/G\;:\;U\in\set{B}\}$$
jest bazą w $M/G$ [ćwiczenia].

\begin{conclusion}
    Iloraz $M/G$ zawsze posiada przeliczalną bazę na topologii.
\end{conclusion}

\textbf{Przykład}: Działanie grupy $\Z$ na $\R$ określone przez: dla $k\in\Z$ $k(x)=x+k$. Wtedy 
$$\R/\Z\cong S^1$$
(sklejamy odcinki długości $1$).

\begin{definition}[działanie nakrywające]
$G$ działa na $M$ \acc{nakrywająco}, jeśli dla każdego $p\in M$ istnieje otwarte  otoczenie $p\in U\subseteq M$ takie, że rodzina obrazów $g(U)$ po $g\in G$ jest parami rozłączna $g_1(U)\cap g_2(U)=\emptyset$ dla $g_1\neq g_2$.
\end{definition}

Dla $U$ jak wyżej odwzorowanie $U\to G(U)/G$ zadane przez $x\mapsto G(x)$ jest homeomorfizmem. Z tego wynika, że dla działania nakrywającego rozmaitość $M$, \acc[i]{iloraz $M/G$ jest przestrzenią lokalnie euklidesową} tego samego wymiaru co wymiar rozmaitości $M$.

Iloraz zadany przez działanie nakrywające niekoniecznie jest rozmaitością topologiczną:

\textbf{Przykład:} Działanie $\Z$ na $\R^2\setminus\{(0,0)\}$ przez potęgi przekształcenia liniowego zadanego macierzą
$$\begin{pmatrix}2&0\\0&\frac{1}{2}\end{pmatrix}$$
jest nakrywające. Orbity wyglądają
\begin{illustration}
\draw(-0.3, 0)--(5.2, 0);
\draw(0,-0.3)--(0,5.2);
\end{illustration}
Natomiast taka przestrzeń nie jest przestrzenią Hausdorffa.

\begin{definition}
Działanie $G$ na $M$ przez dyfeomorfizmy jest
\begin{itemize}
    \item[\PHtunny] \importnat{wolne}, gdy dla każdego $g\in G\setminus\{id\}$ i dla każdego $x\in M$ jest $g(x)\neq x$,
    \item[\PHtunny] \important{właściwie nieciągłe} [properly dicontinuous], gdy dla każdego zwartego $K\subseteq M$ zbiór $g\in G$, że $g(K)\cap K\neq\emptyset$ jest skończony.
\end{itemize}
\end{definition}

\subsection{Kilka szybkich własności}

\begin{definition}
Dla $x\in M$ \acc{stabilizator} (podgrupa stabilizująca) to
$$Stab(x)=\{g\in G\;:\;g(x)=x\}.$$
\end{definition}

\begin{remark}
Działanie $G$ jest wolne $\iff$ dla każdego $x\in M$ $Stab(x)=\{id\}$.
\end{remark}

\textbf{Przykład:} Działanie $\Z$ na $\R^2$ przez potęgi orotów o $\frac{\pi}{n}$ nie jest wolne, bo $(0,0)$ zostaje na swoim miejscu. Natomiast to samo działanie na $\R^2$ jest już wolne.

\begin{fact}
Działanie grupy $G$ jest wolne $\iff$ dla każdego $x\in M$ odwzorowanie $G\to G(x)$ zadane przez $g\mapsto g(x)$ jest bijekcją.
\end{fact}

\begin{fact}
Gdy działanie $G$ przez homeomorfizmy na lokalnie zwartej przestrzeni topologicznej $X$ jest właściwie nieciągłe, to wówczas każda orbita $G(x)$ jest dyskretnym podzbiorem $X$ (każdy punkt z orbity posiada otoczenie otwarte $z\in U$ takie, że $U\cap G(x)=\{x\}$)
\end{fact}

Jeśli ponadto działanie to jest wolne, to jest ono nakrywające.

\begin{fact}(ważny i trudny) 
Jeśli $G$ działa przez homeomorfizmy na lokalnie zwartej przestrzeni $X$ w sposób właściwie nieciągły, to iloraz $X/G$ jest przestrzenią Hausdorffa.
\end{fact}

\textbf{Uwaga:} Działanie $\Z$ na $\R^2\setminus\{(0,0)\}$ przez potęgi $\begin{pmatrix}2&0\\0&\frac{1}{2}\end{pmatrix}$ nie jest właściwie nieciągłe. 

\begin{fact}
Jeśli $G$ działa na $M^n$ przez dyfeomorfizmy w sposób wolny, włąściwie nieciągły, to iloraz $M/G$ jest $n$-wymiarową rozmaitością topologiczną.
\end{fact}

Oznaczenie $M^n$ mówi, że $M$ jest rozmaitością $n$-wymiarową.

\subsection{Gładki atlas na $M/G$}

\phantomsection\label{gw:5:2:1}
\begin{bbox}
(\Coffeecup) $U$ jest otwarty i mapowy oraz dla każdych $g_1,g_2\in G$ różnych $g_1(U)\cap g_2(U)=\emptyset$.
\end{bbox}
\begin{itemize}
    \item[\PHtunny] Każdy $x\in M$ ma otoczenie $U$ spełniające (\Coffeecup), a stąd każda orbita $G(p)\in M/G$ ma otoczenie otwarte postaci $G(U)/G$ ze zbiorem $U$ spełniającym \hyperref[gw:5:2:1]{(\Coffeecup)}
    \item[\PHtunny] Jeśli $U$ spełnia \hyperref[gw:5:2:1]{(\Coffeecup)}, to odwzorowanie $i_u:U\to G(U)/G$ $p\mapsto G(p)$ jest homeomorfizmem. Wtedy $\phi_G:G(U)/G\to\overline{U}\subseteq\R^n$ określone przez $\phi_G=\phi\circ i_U^{-1}$ jest kandydatem na mapę na $M/G$.
\end{itemize}

\phantomsection\label{atlas:gladki:iloraz}
Niech $\set{A}$ będzie atlasem na $M$. Rozważmy rodzinę
$$A_G:=\{(G(U)/G,\phi_G)\;:\;U\text{ spełnia (\Coffeecup)},\;(U,\phi)\in\set{A}\}$$


\begin{itemize}
    \item[\PHtunny] $A_G$ jest gładko zgodny, więc jest gładkim atlasem na $M/G$
    \item[\PHtunny] odwzorowanie ilorazowe $q_G:M\to M/G$ zadane przez $q_G(x)=G(x)\in M/G$ jest gładkie
\end{itemize}

\begin{definition}[DUPAAA]
$f:M\to N$ jest \acc{lokalnym dyfeomorfizmem}, gdy każdy $x\in M$ posiada otwarte otoczenie $x\in U\subseteq M$ takie, że $f\restriction U:U\to f(U)$ jest dyfeomorfizmem na otwarty podzbiór $f(U)$.

W szczególności wymiary tych dwóch rozmaitości muszą się zgadzać.
\end{definition}

Gładka zgodność map z \hyperref[atlas:gladki:iloraz]{atlasem}. Niech $(G(U)/G,\phi_G)$ oraz $(G(V)/G,\psi_G)$ będą mapami zgodnymi z mapami $(U,\phi)$ oraz $(V,\psi)$ na $M$. Wtedy
$$\phi_G=\phi\circ i_U^{-1},\quad\psi_G=\psi\circ i_V^{-1}$$
i odwzorowanie przejścia to
$$\psi_G\circ\phi_G^{-1}:\phi_(G(U)/G\cap G(V)/G)\to\psi_G(G(U)/G\cap G(V)/G)$$
i zachodzi
$$\psi_G\psi_G^{-1}=\psi\circ i_V^{-1}\circ(\phi\circ i_u^{-1})^{-1}=\psi\circ i_V^{-1}\circ i_u\circ\phi^{-1}$$
Przyglądamy się przekształceniu
$$i_V^{-1}\circ i_U:U\cap i_U^{-1}(G(V)/G)\to V\cap i_v^{-1} (G(U)/G)$$
dla $y=i_V^{-1}\circ i_U(x)$ zachodzi $G(x)=i_U(x)=i_V(y)=G(y)$. Zatem $y=g_x(x)$ dla pewnego (jedynego, bo wolne) $g_x\in G$. 

Wiemy też, że $i_V^{-1}i_U$ jest homeomorfizmem, więc w szczególności jest ciągłe.
Z tej ciągłości wynika, że $x\mapsto g_x$ musi być stałe na komponentach spójności. Komponenty spójności $U\cap i_U^{-1}(G(V)/G)$ są otwarte w $M$. Na każdej z nich mamy $i_V^{-1}\circ i_U(x)=g(x)$ dla ustalonego $g\in G$ zależnego od komponentu.

Zatem $\psi_G\circ\phi_G^{-1}(x)=\psi\circ g\circ\phi^{-1}(x)$ dla $x$ z komponenty. Więc na tym zbiorze otwartym $\psi_G\circ\phi_G^{-1}$ jest gładkie.

Sprawdzamy własności $q_G$ w mapach $(U,\phi)$ na $M$ oraz $(G(U)/G,\phi_G)$ na $M/G$.

\textbf{Przykłady:} 
\begin{enumerate}
\item $\Z^n$ na $\R^n$ przez przesunięcia: $(k_1,...,k_n)(x_1,...,x_n)=(x_1+k_1,...,x_n+k_n)$. Można sprawdzić, że jest to działanie wolne i właściwie nieciągłe. Iloraz $\R^n/\Z^n$ jest nazywane $n$-wymiarowym torusem i jest homeomorficzne z $S^1\times...\times S^1$.
\item $\Z$ na $S^1\times\R$ (współrzędne $\theta$ na $S^1$, $n$ na $\R$. $k\in\Z$ działa przez $k(\theta,t)=((-1)^k\theta,t+k)$. To jest nic innego jak butelka Kleina.
\item $\Z$ działa na $[-1,1]\times\R$ przez $k(x,y)=((-1)^k,y+k)$ [wstęga mobiusa z brzegiem]
\end{enumerate}























