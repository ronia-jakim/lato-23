\section{Dyskretne ilorazy rozmaitości gładkich przez grupy dyffeomorfizmów}

\begin{definition}[grupa dyfeomorfizmów]
    \acc{Grupa $G$ dyfeomorfizmów rozmaitości $M$} to dowolny niepusty zbiór dyfeomorfizmów $g:M\to M$, który jest zamknięty na operację składania i brania odwrotności. Elementem identycznym jest $id_M$, a odwrotne to dyfeomorfizmy odwrotne. Grupa $G$ działa przez dyfeomorfizmy na rozmaitość $M$.
\end{definition}

\begin{definition}[orbita, rozbicie]
    \acc{Orbitą} punktu $x\in M$ względem działania $G$ na $M$ nazywamy zbiór
    $$\color{blue}G(x)=\{g(x)\;:\;g\in G\}$$
    Rodzina wszystkich orbit tworzy \acc[b]{rozbicie rozmaitości $M$} na podzbiory.
\end{definition}
Dwie orbity są albo całkiem rozłączne, albo pokrywają się.

\begin{definition}
    Zbiór orbit to $M/G$. $M/G$ tak naprawdę oznacze przestrzeń ilorazową działania $G$ na $M$, czyli przestrzeń topologiczną której elementami są orbity działania $G$ na $M$, zaś topologia jest \acc{ilorazowa}. To znaczy, że zbiór orbit jest otwarty w tym ilorazie $\iff$ suma tych orbit tworzy otwarty zbiór w $M$.
\end{definition}

Na przykład, jeśli $U\subseteq M$ jest otwarty, to $G(U)/G:=\{G(x)\;:\;x\in U\}$, to ten zbiór jest zbiorem otwarty w $M/G$. Co więcej, każdy otwarty zbiór w $M/G$ ma postać $G(U)/G$ jak wyżej. Czyli jeśli $\set{B}$ jest bazą na $M$, to wtedy 
$$\{G(U)/G\;:\;U\in\set{B}\}$$
jest bazą w $M/G$ [ćwiczenia].

\begin{conclusion}
    Iloraz $M/G$ zawsze posiada przeliczalną bazę na topologii.
\end{conclusion}









