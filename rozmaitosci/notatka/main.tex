\documentclass{article}

\usepackage[polish]{../../lecture_notes}
\usepackage{longtable}

\title{Rozmaitości różniczkowalne}
\author{elo}
\date{---}

\begin{document}
\maketitle
\thispagestyle{empty}
\newpage

\tableofcontents
\newpage

\section{Definicja rozmaitości}

%Zanim podany dokładną definicję, możemy rozważyć kilka przykładów rozmaitości różniczkowalnych:
%
%\indent \point powierzchnia, domknięta lub nie,
%
%\indent \point przestrzeniach opisanych (lokalnie) skończoną liczbą parametrów,
%
%\indent \point podzbiory $\R^n$ lub $\C^n$ zapisywalne równaniami algebraicznymi (np. $z_1^2+z_2^2+z_3^1$ w $\C^3$).
%
%Cały wykład będzie wstępnym słownikiem wokół pojęcia rozmaitości różniczkowalnej.
%

Definicję rozmaitości będziemy budowali warstwami: najpierw położymy fundamenty topologiczne, potem naniesiemy na to strukturę gładką, a na koniec rozszerzymy do pojęcia rozmaitości z brzegiem.

Zanim zajmiemy się konkretnymi definicjami, popatrzmy na kilka prostych przykładów rozmaitości:
\begin{itemize}
  \item powierzchnia, domknięta lub nie,
  \item przestrzenie opisane (lokalnie) skończoną liczbą parametrów,
  \item podzbiory $\R^n$ lub $\C^n$ zapisywane równaniami algebraicznymi (np. $z_1^2+z_2^2+z_3^2$ w $\C^3$).
\end{itemize}

\subsection{Rozmaitości topologiczne}

\begin{definition}[rozmaitość topologiczna]
Przestrzeń topologiczna $M$ jest $n$-wymiarową \important{rozmaitością topologiczną} [$n$-rozmaitością], jeżeli spełnia:
\begin{enumerate}
    \item jest Hausdorffa
    \item ma przeliczalną bazę
    \item jest \acc{lokalnie euklidesowa} wymiaru $n$, czyli każdy punkt z $M$ posiada otwarte otoczenie w $M$ homeomorficzne z otwartym podzbiorem w $\R^n$.
\end{enumerate}
\end{definition}

Warunkiem równoważnym do lokalnej euklidesowości jest istnienie otwartego otoczenia dla każdego punktu $p\in U\subseteq M$ takiego, że istnieje homeomorfizm $U\isomorphism B_r\subseteq\R^n$ [ćwiczenia].
\medskip

{\acc{\large Konsekwencje Hausdorffowości:}}
\begin{itemize}
    \item Mamy wykluczone pewne patologie, na przykład przestrzeń
\begin{illustration}
    \draw[very thick] (-3, 0)--(0, 0);
    \draw[very thick] (0, 0.5)--(3, 0.5);
    \draw[very thick] (0, -0.5)--(3, -0.5);
    \filldraw[very thick] (0, 0) circle (2pt);
\end{illustration}
nie jest rozmaitością topologiczną.
    \item Dla dowolnego punktu $p\in U\subseteq M$ i homeomorfizmu $\phi:U\to\overline U\subseteq\R^n$, jeśli $\overline{K}\subseteq\overline{U}$ jest zwartym podzbiorem $\R^n$, to $K=\phi^{-1}[\overline{K}]\subseteq M$ jest domknięty i zawarty w $M$ [ćwiczenia].
    \item Skończone podzbiory są zamknięte, a granice zbieżnych ciągów są jednoznacznie określone.
\end{itemize}

{\acc{\large Konsekwencje przeliczalności bazy:}}
\begin{itemize}
    \item \important{Warunek Lindel\"ofa:} każde pokrycie rozmaitości zbiorami otwartymi zawiera przeliczalne podpokrycie [ćwiczenia].
    \item Każda rozmaitość jest wstępującą sumą otwartych podzbiorów
    $$U_1\subseteq U_2\subseteq...\subseteq U_n\subseteq...$$
    które są po domknięciu zawarte w $M$.
    \item \acc{Parazwartość}, czyli każde pokrycie $M$ posiada lokalnie skończone rozdrobnienie.
    \begin{itemize}
        \item Rodzina $\set{X}$ podzbiorów $M$ jest \emph{lokalnie skończona} [locally finite], jeżeli każdy punkt $p\in M$ ma otoczenie, które przecina się co najwyżej ze skończenie wieloma elementami $\set{X}$.
        \item Jeśli mamy pokrycie $M$ zbiorami $\set{U}$ i bierzemy drugie pokrycie $\set{V}$ takie, że dla każdego $V\in\set{V}$ znajdziemy $U\in\set{U}$ takie, że $V\subseteq U$, to $\set{U}$ jest \acc{pokryciem włożonym/rozdrobnieniem}
    \end{itemize}
    \item Każdą rozmaitość jesteśmy w stanie zanurzyć w $\R^n$ dla odpowiednio dużego $n$.
\end{itemize}

{\acc{\large Konsekwencje lokalnej euklidesowości:}}
\begin{itemize}
    \item \ul{Twierdzenie Brouwer'a}: niepusta $n$ wymiarowa rozmaitość topologiczna nie może być homeomorficzna z żadną $m$ wymiarową rozmaitością gdy $m\neq n$.
    \item Liczba $n$ w definicji jest jednoznaczna, możemy więc określić \important{wymiar rozmaitości} jako $\dim M=n$.
\end{itemize}

Tutaj warto zaznaczyć, że zbiór pusty zaspokaja definicję rozmaitości topologicznej dla dowolnego $n$. Wygodnie jest jednak móc go czasem użyć, więc w definicji niepustość $M$ nie jest przez nas wymagana.

\begin{remark}[podzbiory to też rozmaitości]
    Każdy otwarty podzbiór $n$-rozmaitości topologicznej jest $n$-rozmaitością topologiczną [ćwiczenia].
\end{remark}

\subsection{Mapy, lokalne współrzędne}

\begin{definition}[mapa]
    Parę $(U,\phi)$, gdzie $U$ jest otwartym podzbiorem $M$, a $\phi$ to homeomorfizm
    $$\phi:U\to\overline{U}\subseteq\R^n.$$
    nazywamy \important{mapą} lub \important{lokalną parametryzacją} [coordinate chart] na rozmaitości $M$. Zbiór $U$ taki jak wyżej nazywamy \acc{zbiorem mapowym} [coordinate domain/neighborhood]. Z lokalnej euklidesowości wiemy, że \textbf{zbiory mapowe pokrywają całą rozmaitość}.
\end{definition}

Jeżeli $(U,\phi)$ jest mapą i dla $p\in M$ mamy $\phi(p)=0$, to mówimy, że mapa jest \emph{wyśrodkowana na $p$} [centered at $p$].

\begin{fact}[$n$-rozmaitość $\iff$ rodzina map pokrywających]
    Hausdorffowska przestrzeń $X$ o przeliczalnej bazie jest $n$-rozmaitością $\iff$ posiada rodzinę map $n$-wymiarowych dla której zbiory mapowe pokrywają cały $X$.
\end{fact}

\textbf{Przykład:} 

Rozważmy $S^n=\{(x_1,...,x_n)\in\R^{n+1}\;:\;\sum x_i^2=1\}\subseteq \R^{n+1}$ z dziedziczoną topologią. Z racji, że $\R^{n+1}$ jest Hausdorffa i ma przeliczalną bazę, to $S^n$ tęż spełnia te dwa warunki. Wystarczy teraz wskazać odpowiednią rodzinę map, która pokryje całe $S^n$. Dla $i=1,..., n+1$ określmy otwarte podzbiory w $S^n$
$$U_i^+=\{x\in S^n\;:\;x_i>0\}$$
$$U_i^-=\{x\in S^n\;:\;x_i<0\}$$

\begin{illustration}
    \shade[ball color=yellow, opacity=0.3] (2.3,0.3) arc (0:-180:2.3 and 0.6) arc (180:0:2.3 and 2.3);
    \shade[ball color=green, opacity=0.3] (2.3, -0.3) arc (0:-180:2.3 and 2) arc (180:0:2.3 and 0.6);
    \draw[color=yellow, opacity=0.5] (2.3,0.3) arc (0:-180:2.3 and 0.6) arc (180:0:2.3 and 2.3);
    \draw[color=green, opacity=0.5] (2.3, -0.3) arc (0:-180:2.3 and 2);
    \draw[color=green, opacity=0.5] (2.3, -0.3) arc (0:-180:2.3 and 0.6);
    \draw (0,0) circle (2);
    \draw (-2,0) arc (180:360:2 and 0.6);
    \draw[dashed] (2,0) arc (0:180:2 and 0.6);
    \node at (2.5, 2) {$\color{yellow}U_i^+\cap S^n$};
    \node at (2.5, -2) {$\color{green}U_i^-\cap S^n$};

    \draw (-2,-4) arc (180:360:2 and 0.6);
    \draw[dashed] (2,-4) arc (0:180:2 and 0.6);
    %\draw (-3, -5)--(3, -3);
    %\draw (-3, -4)--(3, -4);
    \draw[->] (0, -2.5)--(0, -3) node [midway, left] {$\phi_i^\pm$};
\end{illustration}

Określmy odwzorowania $\phi^\pm_i\;:\;U_i^\pm\to\R^n$
$$\phi_i^\pm(x)=(x_1,...,x_{i-1},\hat{x_i},x_{i+1},...,x_n).$$
Obraz tego odwzorowania to
$$\overline U_i^\pm=\phi_i^\pm(U_i^\pm)=\{(x_1,...,x_n)\in \R^n\;:\;\sum x_i^2<1\}.$$
Odwzorowanie $\phi_i^\pm: U_i^\pm\to\overline U_i^\pm$ jest wzajemnie jednoznaczne [bijekcja], bo
$$(\phi_i^\pm)^{-1}(x_1,...,x_n)=(x_1,...,x_{i-1}, \pm\sqrt{1-\sum x_j^2}, x_{i+1},...,x_n).$$
Mamy w obie strony odwzorowanie ciągłe, więc jest to homeomorfizmy z odpowiednimi zbiorami $\R^n$.

{\Large\color{orange} PRZYKŁADY Z LEE}

\subsection{Własności rozmaitości topologicznych}

Przypomnijmy najpierw kilka definicji z topologii i je poszerzmy. Mówimy, że przestrzeń topologiczna $X$ jest
\begin{itemize}
    \item \acc{spójna}, gdy nie można jej rozłożyć na sumę dwóch rozłącznych, otwartych i niepustych podzbiorów,
    \item \acc{drogowo spójna}, gdy każde dwa punkty można połączyć ciągłą ścieżką,
    \item \acc{lokalnie drogowo spójna}, gdy ma bazę zbiorów spójnych drogowo.
\end{itemize}

\begin{remark}[spójność rozmaitości topologicznych]
Jeśli przestrzeń $M$ jest rozmaitością topologiczną, to
\begin{enumerate}
    \item $M$ jest lokalnie spójna drogowo,
    \item $M$ jest spójna $\iff$ jest drogowo spójna,
    \item spójne składowe $M$ są takie same jak dorogowe spójne składowe,
    \item $M$ ma przeliczalnie wiele składowych, każda będąca otwartym podbiorem $M$ (a więc i spójną rozmaitością)
\end{enumerate}
\begin{proof}
    Punkt (1) jest prostą konsekwencją tego, że otwarte kule są spójne łukowo w $\R^n$ [ćwiczenia]. Punkty (2) i (3) wynikają w prosty sposób z (1). Punkt (4) jest powodowany punktami poprzednimi i tym, że baza $M$ jest przeliczalna. 
\end{proof}
\end{remark}

Przestrzeń topologiczna $X$ jest \important{lokalnie zwarta}, jeżeli każdy punkt ma bazę otoczeń których domknięcia są zwarte.

\begin{remark}[rozmaitości są lokalnie zwarte]
Każda rozmaitość topologiczna jest lokalnie zwarta.
\begin{proof}
    Zadanie na liście 1.
\end{proof}
\end{remark}

Przestrzeń zawierająca wszystkie homotopijne pętle zaczepione w $q\in X$ jest nazywana \acc{fundamentalną grupą} $X$ w $q$. Elementem neutralnym tej grupy jest funkcja stała $c_q(s)=q$. Dla rozmaitości topologicznych \emph{fundamentalne grupy} są przeliczalne.


\newpage

\section{Rozmaitości gładkie}
%Na tym wykładzie nie będziemy poświęcać dużej uwagi rozmaitościom różniczkowalnym nie nieskończenie razy, więc pomimo lekkich niuansów między tymi dwoma słowami, dla nas zwykle znaczą one to samo.
%\medskip

Na wykładzie nie będą nas zbytnio interesować rzeczy różniczkowalne tylko skończenie wiele razy. Z tego też powodu lekkie niuanse między słowami gładkie a różniczkowalne będą często pomijalne, a słowa te staną się izomorficzne. Teraz postaramy się określić, \emph{co to znaczy, że funkcja $f:M\to\R$ jest różniczkowalna?}

%
%Dla funkcji $f:M\to\R$ chcemy określić, co znaczy, że \emph{$f$ jest różniczkowalna}? Będziemy to robić za pomocą wcześniej zdefiniowanych map:
%\begin{itemize}
%    \item Funkcja $f$ \acc{wyrażona} w mapie $(U, \phi)$ to nic innego jak złożenie $f\circ \phi^{-1}:\overline U\to \R$. Teraz $f\circ\phi^{-1}$ jest funkcją zależącą od $n$ zmiennych rzeczywistych.
%    \item Chciałoby się powiedzieć, że funkcja $f: M\to\R$ jest gładka, jeśli dla każdej mapy $(U, \phi)$ na $M$, ten fragment wyrażony w tej mapie $f\circ\phi^{-1}$ jest gładki. Niestety, tych map może być nieco za dużo.
%    \item \important{Odwzorowanie przejścia między dwoma mapami} $(U_1,\phi_1)$ i $(U_2,\phi_2)$ to funkcje $\phi_1\phi_2^{-1}$ i $\phi_2\phi_1^{-1}$ określone na $U_1\cap U_2$.
%\end{itemize}

\subsection{Atlas rozmaitości}
Pojęcie różniczkowalności funkcji $f:M\to\R$ będziemy określać za pomocą \emph{map}:
\begin{itemize}
    \item Funkcja $f$ wyrażona w mapie $(U,\phi)$ to nic innego jak \acc{$f\circ\phi^{-1}:\overline{U}\to\R$}. W ten sposób dostajemy funkcję wyrażoną w zmiennych rzeczywistych.
    \item W pierwszym instynkcie możemy chcieć powiedzieć, że $f:M\to\R$ jest gładka, jeśli dla każdej mapy taka jest. Niestety, map może być bardzo dużo i może się okazać, że żadna funkcja nie jest gładka.
    \item \important{Odwzorowanie przejścia} między dwoma mapami $(U_1,\phi_1),(U_2,\phi_2)$ to funkcje $\phi_1\phi_2^{-1}$ i $\phi_2\phi_1^{-1}$ określone na $U_1\cap U_2$.
\end{itemize}


\begin{definition}[zgodność map]
Mapy $(U, \phi_1)$ oraz $(U, \phi_2)$ są \important{zgodne} (gładko-zgodne), gdy odwzorowanie przejścia $\phi_1\phi_2^{-1}$ jest gładkie. Dla map \acc{$(U, \phi)$ i $(V, \psi)$} mówimy, że są one zgodne, jeśli 
\begin{itemize}
    \item $U\cap V=\emptyset$, albo
    \item $\phi\psi^{-1}:\psi(U\cap V)\to \phi(U\cap V)$ i $\psi\phi^{-1}(U\cap V)\to \psi(U\cap V)$ są gładkie.
\end{itemize}
\end{definition}
%
%Warto zauważyć, że jeśli $(U, \phi)$ i $(V, \psi)$ są zgodne, to $f\circ\phi^{-1}\obciete (\phi(U\cap V))$ jest gładkie $\iff$
%
%Odwzorowania przejściowe map są automatycznie \emph{dyfeomorfizmami}.
%\medskip

\begin{definition}[dyfeomorfizm]
Mając dane dwie rozmaitości, $M$ i $N$, mówimy, że funkcja $f:M\to N$ jest \important{dyfeomorfizmem}, jeżeli
\begin{itemize}
    \item jest różniczkowalna
    \item jest bijekcją
    \item funkcja odwrotna $f^{-1}$ też jest różniczkowalna
\end{itemize}
\end{definition}


\begin{definition}[atlas gładki]
    \acc{Gładkim atlasem} $\set{A}$ na topologicznej rozmaitości $M$ nazywamy dowolny taki zbiór map $\{(U_\alpha,\phi_\alpha)\}$ taki, że:
    \begin{enumerate}
        \item 1. zbiory mapowe $U_\alpha$ pokrywają całe $M$
        \item 2. każde dwie mapy z tego zbioru są zgodne.
    \end{enumerate}
\end{definition}


\textbf{Przykład:}  Rodzina map $\{(U_i^\pm,\phi_i^\pm)\;:\;i=1,2,...,n+1\}$ jak wcześniej na sferze $S^n\subseteq R^{n+1}$ tworzy gładki atlas. Wystarczy zbadać gładką zgodność tych map. Rozpatrzmy jeden przypadek: $(U_i^+,\phi_i^+),(U_j^+,\phi_j^+),i<j$. Po pierwsze, jak wygląda przekrój tych zbiorów?
$$U_i\cap U_j=\{x\in S^n\;:\;x_i>0,x_j>0\}$$
Dalej, jak wyglądają obrazy tego przekroju przez poszczególne mapy?
$$\phi_i^+(U_i\cap U_j)=\{x\in\R^n\;:\;|x|<1, x_{j-1}>0\}$$
$$\phi_j^+(U_i\cap U_j)=\{x\in\R^n\;:\;|x|<1,x_i<0\}$$
Odwzorowania przejścia to:

\begin{center}\begin{tikzcd}
    \phi_j^+(U_i^+\cap U_j^+)\ni(x_1,...,x_n) \arrow[r, "\phi_j^+)^{-1}"] & (x_1,...,x_{j-1},\sqrt{1-|x|^2},x_{j},...x_n)\arrow[d,"\phi_i^+"]\\
    & (x_1,...,x_{i-1},x_{i+1},...,x_{j-1},\sqrt{1-|x|^2},x_j,...,x_n)
\end{tikzcd}\end{center}

$$\phi_i^+(\phi_j^+)^{-1}(x_1,...,x_n)=(x_1,...,x_{i-1}, x_{i+1},...,x_{j-1},\sqrt{1-|x|^2},x_j,...,x_n)$$
jest przekształceniem gładkim. Analogicznie dla drugiego odwzorowania przejścia.

\subsection{Zgodność map}

\begin{definition}[rozmaitość gładka]
    \acc{Rozmaitość gładka} to para $(M, \set{A})$ złożona z rozmaitości $M$ i gładkiego atlasu $\set{A}$ opisanego na $M$.
\end{definition}

\begin{definition}[zgodność map, atlasów]
    Niech $\set{A}_1,\set{A}_2$ będą gładkimi atlasami na $M$.
    Mówimy, że mapa $(U, \phi)$ jest zgodna z atlasem $\set{A}$, jeżeli jest zgodna z każdą mapą z $\set{A}_1$. Dalej, mówimy, że atlas $\set{A}_2$ jest zgodny z altasem $\set{A}_1$, jeżeli każda mapa z $\set{A}_1$ jest zgodna z każdą mapą z atlasu $\set{A}_2$.
\end{definition}

%
%\acc{Uściślenie:} Często $(M,\set{A}_1)$ i $(M, \set{A}_2)$ będące rozmaitościami gładkimi określają tę samą rozmaitość.
%\medskip
%
%\begin{definicja}[zgodność map, atlasów]
%Niech $\set{A}$ będzie gładkim atlasem na $M$.
%\begin{enumerate}
%    \item Mapa $(U,\phi)$ jest \deff{zgodna z atlasem $\set{A}$}, jeśli jest zgodna z każdą mapą z $\set{A}$.
%    \item Dwa \deff{atlasy $\set{A}_1,\set{A}_2$ na $M$ są zgodne}, jeśli każda mapa z $\set{A}_1$ jest zgodna z atlasem $\set{A}_2$.
%\end{enumerate}
%\end{definicja}

\begin{theorem}[zgodność to relacja równoważności]
    Relacja zgodnośc atlasów jest relacją równoważności.
\begin{proof}
Ćwiczenia
\end{proof}
\end{theorem}

\subsection{Atlas [maksymalny]}

Zgodne atlasy określają tę samą strukturę gładką na $M$. W takim razie, wygodnym będzie móc zawerzeć wszystkie zgodne atlasy w czymś większym. Z pomocą przychodzi nam pojęcie \acc{atlasu maksymalnego}.
\begin{definition}[atlas maksymalny]
    Atlas $\set{A}$ jest \important{atlasem maksymalnym}, jeżeli każda mapa $(U,\phi)$ z nim zgodna jest w nim zawarta.
\end{definition}

%\begin{tw}[zgodność to relacja równoważnośći]
%    Relacja zgodności atlasów jest relacją równoważności.
%\end{tw}
%
%\textbf{Dowód:} Ćwiczenia.
%
%Konwencja jest wtedy taka, że zgodne atlasy zadają tą samą strukturę gładką na $M$. W takim razie, zgodne atlasy można wysumować do jednego większego atlasu.
%
%\begin{definicja}[atlas maksymalny]
%$\set{A}$ jest \deff{atlasem maksymalnym} na $M$, jeśli każda mapa na $M$ z nim zgodna jest w nim zawarta. 
%\end{definicja}
%
%\begin{fakt}[dla każdego atlasu istnieje jedyny atlas maksymalny]
%    Każdy atlas $\set{A}$ na $M$ zawiera się w dokładnie jednym atlasie maksymalnym na $M$. Zaś ten atlas maksymalny to zbiór wszystkich map na $M$ zgodnych z $\set{A}$.
%\end{fakt}

\begin{fact}[każdy atlas jest zawarty w unikalnym atlasie maksymalnym]
    Każdy atlas $\set{A}$ na $M$ zawiera się w dokładnie jednym atlasie maksymalnym na $M$, który jest zbiorem wszystkich map na $M$ zgodnych z $\set{A}$.
\begin{proof}
Ćwiczenia. Korzystamy z lematu Zorna.
\end{proof}
\end{fact}

W takim razie, równoważnie do pary $(M, \set{A})$, gdzie $\set{A}$ jest dowolnym zgodnym atlasem na $M$, możemy wymóc w definicji, aby \important{$\set{A}$ był atlasem maksymalnym}.

\subsection{Funkcje gładkie}

\begin{definition}[gładkość względem atlasu]
   Funkcja $f:M\to\R$ określona na rozmaitości gładkiej $(M, \set{A})$ jest gładka, jeżeli po wyrażeniu w każdej mapie z tego atlasu jest gładka:
   $$(\forall\;(U,\phi)\in\set{A})\quad f\circ\phi^{-1}\text{ jest gładka}$$
\end{definition}

\begin{fact}
    Niech $(M,\set{A})$ będzie rozmaitością gładką, a $f:M\to\R$ będzie funkcją gładką na $M$.
    \begin{enumerate}
        \item Jeżeli $(U, \phi)$ jest mapą zgodną z $\set{A}$, to $f$ wyrażone w $(U, \phi)$, czyli $f\circ\phi^{-1}$ też jest funkcją gładką.
        \item Niech $\set{A}'$ będzie atlasem zgodnym z $\set{A}$. Wówczas funkcja $f$ jest zgodna względem $\set{A}'$ $\iff$ jest gładka względem $\set{A}'$ $\iff$ jest zgodna względem atlasu maksymalnego zawierającego oba te atlasy. 
    \end{enumerate}
\end{fact}

Co więcej, możemy powiedzieć, że $f:M\to\R$ jest gładka $\iff$ jest gładka względem każdego atlasu $\set{A}$ wyznaczającego na $M$ gładką strukturę. [Ćwiczenia]

\begin{definition}[mapa $C^k$-zgodna, $C^k$-atlas]
$ $\newline
\begin{itemize}
    \item Dwie mapy $(U, \phi)$ i $(V,\psi)$ są \important{$C^k$-zgodne}, jeżeli $\phi\psi^{-1}$ oraz $\psi\phi^{-1}$ są funkcjami klasy $C^k$.
    \item \important{$C^k$-atlas} to atlas składający się z map, które są $C^k$-zgodne
    \begin{itemize}
        \item Taki atlas określa strukturę $C^k$-rozmaitości na $M$
        \item Jest to struktura słabsza niż struktura rozmaitości gdładkiej
    \end{itemize}
\end{itemize}
\end{definition}

$C^0$ zwykle oznacza rozmaitość topologiczną, a $C^\infty$ to rozmaitość gładka.

\begin{fact}[nie można skakać $C^m\to C^k$]
Na $C^k$ rozmaitości nie można sensownie określić funkcji klasy $C^m$ dla $m>k$.
\end{fact}

Rozmaitość można definiować na różne sposoby niewymagające użycia definicji i własności topologicznych, na przykład:
\begin{itemize}
    \item \important{Rozmaitość analityczna} [$C^\omega$] to rozmaitość, dla której atlas składa się z map analitycznie zgodnych (czyli wyrażają się za pomocą szeregów potęgowych).
    \item \important{Rozmaitość zespolona} ma mapy jako funkcje w $\C^n$ zamiast w $\R^n$
    \item Rozmaitość konforemna - zachowuje kąty
    \item Rozmaitość kawałkami liniowa
\end{itemize}

Istnieją rozmaitości topologiczne, które nie dopuszczają żadnej struktury gładkiej (pierwszym takim przykładem była zwarta $10$-rozmaitość odkryta przez M. Kervaire). Z drugiej strony, z każdego maksymalnego atlasu $C^k$ rozmaitości można wybrać atlas złożony z map $C^\infty$-zgodnych, czyli na każdej $C^k$ istnieje struktura $C^\infty$ rozmaitości.

\begin{lemma}[rozmaitość gładka bez topologii]
    Niech $X$ będzie zbiorem (bez topologii). Niech $\{U_\alpha\}$ będzie kolekcją podzbiorów $X$ takich, że istnieje $n\in\N$ takie, że dla każdego $\alpha$ istnieje $\phi_\alpha:U_\alpha\to\R^n$ które jest różnowartościowe. Załóżmy, że takie $M,\{U_\alpha\},\{\phi_\alpha\}$ spełniają:
    \begin{enumerate}
        \item Dla każdego $\alpha$ $\phi_\alpha(U_\alpha)$ jest otwartym podzbiorem $\R^n$
        \item Dla każdych $\alpha,\beta$ $\phi_\alpha(U_\alpha\cap U_\beta)$ oraz $\phi_\beta(U_\alpha\cap U_\beta)$ są otwarte w $\R^n$
        \item Gdy $U_\alpha\cap U_\beta\neq\emptyset$, to 
        $$\phi_\alpha\circ\phi_\beta^{-1}:\phi_\beta(U_\alpha\cap U_\beta)\to\phi_\alpha(U_\alpha\cap U_\beta)$$
        jest gładkim \acc{dyfeomorfizmem} (gładkie i odwracalne)
        \item Przeliczalnie wiele spośród $U_\alpha$ pokrywa $X$
        \item Dla dowolnych $p,q\in X,p\neq q$ istnieją $\alpha,\beta$ oraz otwarte podzbiory $V_p\subseteq\phi_\alpha(U_\alpha), V_q\subseteq\phi_\beta(U_\beta)$ takie, że $p\in\phi_\alpha^{-1}(V_p),q\in\phi_\beta^{-1}(V_q)$ oraz $\phi_\alpha^{-1}(V_p)\cap\phi_\beta^{-1}(V_q)\emptyset$, czyli możemy \emph{rozdzielić dwa dowolne różne punkty za pomocą zbiorów otwartych w $\R^n$}.
    \end{enumerate}
    Wówczas na $X$ istnieje \important{struktura rozmaitości topologicznej na $X$}, dla której $U_\alpha$ są zbiorami otwartymi. Ponadtwo, $(U_\alpha,\phi_\alpha)$ tworzy \acc{gładki atlas na $X$}.
\end{lemma}
\begin{proof}Prezentujemy szkic dowodu:
\begin{itemize}
    \item Topologia jest produkowana jako przeciwobrazy przez poszczeglne $\phi_\alpha$
    \item Lokalna euklidesowość jest oczywista
    \item Mniejsza baza przeliczalna też śmignie [ćwiczenia]
    \item Hausdorffowość wynika z warunku 5.
\end{itemize}
\end{proof}

\textbf{PRZYKŁAD} - linie na prostej, ale nie chce już dzisiaj









\newpage

%
\section{Funkcje różniczkowalne}
\subsection{Dopowiedzenie o funkcjach gładkich}

\begin{definicja}[gładkość względem atlasu]
Funkcja $f:M\to \R$ jest \deff{gładka względem atlasu} $\set{A}$ na $M$, jeśli
$$(\forall\;(U,\phi)\in\set{A})\;f\circ\phi^{-1}:\overline U\to \R\text{ jest gładka.}$$
To znaczy po wyrażeniu w dowolnej mapie atlasu jest nadal funkcją gładką.
\end{definicja}

\begin{fakt}[funkcja gładka względem atlasu]{\color{back}DUPA}
    \begin{enumerate}
        \item Jeśli $f:M\to\R$ jest gładka względem $\set{A}$, zaś $(U,\phi)$ jest zgodna z $\set{A}$, to wówczas funkcja $f$ wyrażona w tej nowej mapie (czyli $f\circ\phi^{-1}$) też jest gładka.
        \item Jeśli $\set{A}_1,\set{A}_2$ są zgodnymi atlasami, wówczas taka funkcja $f:M\to\R$ jest gładka względem $\set{A}_1$ $\iff$ jest gładka względem $\set{A}_2$ $\iff$ jest gładka względem atlasu maksymalnego $\set{A}\supseteq \set{A}_1,\set{A}_2$ zawierającego $\set{A}_1$ (oraz $\set{A_2}$).
    \end{enumerate}
\end{fakt}

Niech $M$ będzie gładką rozmaitością. Wówczas $f:M\to\R$ \deff{jest gładka} jeśli $f$ jest gładka względem każdego (dowolnego) atlasu $\set{A}$ wyznaczającego na $M$ daną gładką strukturę.

\subsection{Atlasy $C^k$}

\begin{definicja}[mapa $C^k$-zgodna, $C^k$-atlas]{\color{back}dupaaa}
    \begin{itemize}
        \item Dwie mapy $(U,\phi)$ i $(V,\psi)$ są \deff{$C^k$-zgodne}, jeśli $\phi\psi^{-1}$ oraz $\psi\phi^{-1}$ są funkcjami klasy $C^k$.
        \item \deff{$C^k$-atlas} to atlas składający się z map, które są $C^k$-zgodne. 
        \begin{itemize}
            \item Taki atlas określa strukturę $C^k$-rozmaitości na $M$.
            \item Jest ona słabsza niż struktura rozmaitości gładkiej.
        \end{itemize}
    \end{itemize}
\end{definicja}

$C^0$ w tej konwencji to rozmaitość topologiczna, a $C^\infty$ to często jest rozmaitość gładka.

Na $C^k$-rozmaitości nie da się sensownie określić funkcji klasy $C^m$ dla $m>k$.
\medskip

Rozmaitość można definiować na różne sposoby niewymagające użycia definicji i własności topologicznych. Przykłady to:
\begin{itemize}
    \item[\point] \acc{Rozmaitość analityczna} [$C^\omega$] to rozmaitość, dla której atlas składa się z map analitycznie zgodnych (czyli wyrażają się za pomocą szeregów potęgowych).
    \item[\point] \acc{Rozmaitość zespolona} ma mapy jako funkcje w $\C^n$ zamiast w $\R^n$.
    \item[\point] Rozmaitość konforemna - zachowuje kąty.
    \item[\point] Rozmaitość kawałkami liniowa
\end{itemize}

\subsection{Rozmaitość gładka bez topologii}

Dychotomia pomiędzy sytuacją $C^0$ a sytuacją $C^k$ dla $k>0$:
\begin{itemize}
    \item Z każdego maksymalnego atlasu $C^k$-rozmaitości można wybrać atlas złożony z map $C^\infty$-zgodnych. A zatem, każda $C^k$-rozmaitość posiada $C^k$-zgodną strukturę $C^\infty$-rozmaitości.
    \item Istnieją $C^0$-rozmaitości niedopuszczające żadnej struktury gładkiej.
\end{itemize}

\begin{lemat}[rozmaitość gładka bez topologii]
    Niech $X$ będzie zbiorem (bez topologii). Niech $\{U_\alpha\}$ będzie kolekcją podzbiorów $X$ i dla każdego $\alpha$ mamy $\phi_\alpha:U_\alpha\to\R^n$ różnowartościowe ($n$ jest ustalone dla całego $X$). Ta trójka obiektów ma spełniać następujące warunki:
    \begin{enumerate}
        \item  Dla każdego $\alpha$ $\phi_\alpha(U_\alpha)$ jest otwarty w $\R^n$.
        \item Dla każdych $\alpha,\beta$ $\phi_\alpha(U_\alpha\cap U_\beta)$ oraz $\phi_\beta(U_\alpha\cap U_\beta)$ są otwarte w $\R^n$.
        \item Gdy $U_\alpha\cap U_\beta\neq\emptyset$, to $\phi_\alpha\circ\phi_\beta^{-1}:\phi(U_\alpha\cap U_\beta)\to\phi(U_\alpha\cap U_\beta)$ jest odwzorowaniem gładkim. Są to dyfeomorfizmy (gładkie i odwracalne).
        \item Przeliczalnie wiele spośród zbiorów $U_\alpha$ pokrywa całe $X$.
        \item Dla dowolnych punktów $p,q\in X,p\neq q$ istnieją $\alpha,\beta$ oraz otwarte podzbiory $V_p\subseteq\phi_\alpha(U_\alpha)$, $V_q\subseteq\phi_\beta(U_\beta)$ takie, że $p\in\phi_\alpha^{-1}[V_p]$, $q\in\phi_\beta^{-1}[V_q]$ oraz $\phi_\alpha^{-1}[V_p]\cap\phi_\beta^{-1}[V_q]=\emptyset$. Czyli możemy rozdzielić dwa dowolne różne punkty za pomocą zbiorów otwartych w $\R^n$.
    \end{enumerate}
\emph{Wówczas na $X$ istnieje \deff{struktura rozmaitości topologicznej} dla której $U_\alpha$ są otwarte.} Ponadto rodzina $(U_\alpha,\phi_\alpha)$ tworzy \acc{gładki atlas na $X$}.
\end{lemat}

\textbf{Szkic dowodu:} 
\begin{itemize}
    \item Topologię produkujemy jako bazę topologii na $X$: bierzemy przeciwobrazy przez poszczególne $\phi_\alpha$ otwartych podzbiorów w zbiorach $\phi_\alpha(U_\alpha)\subseteq\R^n$.
    \item Lokalna $n$-euklidesowość $X$ względem takiej topologii jest oczywista.
    \item Nietrudno jest też wybrać mniejszą bazę przeliczalną [ćwiczenia]. 
    \item Hausdorffowość tak określonej topologii wynika z warunku 5.
\end{itemize}
\proofend
\medskip

\textbf{Przykład:} Niech $\set{L}$ będzie zbiorem wszystkich prostych na płaszczyźnie. Nie ma na tym zbiorze wygodnej do opisania topologii, ale możemy skorzystać z lematu wyżej.

Zacznijmy od opisania podzbiorów 
$$U_h=\{\text{proste niepionowe}\}$$
$$U_v=\{\text{proste niepoziome}\}$$
Jeśli $U_h\ni L$, to wtedy $L=\{y=ax+b\}$ i wtedy $\phi_h$ będzie przypisywać takiej prostej parę $(a, b)$. Jeśli zaś $U_v\ni L$, to wtedy $L=\{x=yc+d\}$ i wtedy $\phi_v$ przypisze jej $(c, d)$. To, że $\phi_h(U_h)$ i $\phi_v(U_v)$ są różnowartościowe widać. Przyjrzyjmy się teraz przekrojowi naszych zbiorków:
$$U_h\cap U_v=\{\text{proste niepoziomie i niepionowe}\}=\{y=ax+b\;:\;a\neq 0\}=\{x=cd+d\;:\;c\neq0\}$$
$$\phi_h(U_h\cap U_v)=\{(a,b)\in\R^2\;:\;a\neq 0\}$$
$$\phi_v(U_h\cap U_v)=\{(c,d)\in\R^2\;:\;c\neq0\}$$
i są to zbiory otwarte, więc warunek 3. jest spełniony. Warunek 4. jest tutaj trywialny.

Niech {\large\color{red}COŚ TUTAJ SIĘ URWAŁO}

To jest homeomorficzne z wnętrzem wstęgi Mobiusa.
%\newpage

%\section{Rozmaitość z brzegiem}

Lokalnie wygląda jak $\R^n$ albo jak półprzestrzeń $n$-wymiarowa:
$$\color{blue}H^n=\{(x_1,...,x_n)\in\R^n\;:\;x_n\geq 0\}$$
\deff{brzegiem} takiej półprzestrzeni nazywamy zbiór:
$$\partial H^n=\{x\in\R^n\;:\;x_n=0\}$$
definiuje się też wnętrze takiej półprzestrzeni:
$$int(H^n)=\{x\in\R^n\;:\;x_n>0\}$$

\begin{definicja}[brzeg, wnętrze zbioru otwartego, gładka funkcja ze zbioru]
Dla otwartego zbioru $U\subseteq H^n$ określamy
\begin{itemize}
    \item[\point] brzeg zbioru: $\color{blue}\partial U=U\cap \partial H^n$
    \item[\point] wnętrze zbioru: $\color{blue}int(U)=U\cap int(H^n)$ 
    \item[\point] Jeżeli mamy zadane $f:U\to\R^m$, to jest ono \deff{gładkie}, gdy jest obcięciem do $U$ pewnej gładkiej funkcji $\overline f:\overline U\to \R^m$, gdzie $\overline U$ jest otwartym podzbiorem $\R^n$ taki, że $U\subseteq\overline U$.
\end{itemize}
\end{definicja}

Jeśli $f:U\to\R^m$ jest gładka, to wówczas pochodne cząstkowe $f$ są dobrze określone w punktach $int(U)$. Ze względu na ciągłość pochodnych cząstkowych dowolnego rozszerzenia $\overline f$, \acc{pochodne cząstkowe $f$ są również dobrze określone w punktach $\partial U$}. 

\begin{fakt}[o istnieniu rozszerzenia funkcji]
Z analizy: rozszerzenie $\overline f$ istnieje $\iff$ $f$ jest gładka na $int(U)$ oraz pochodne cząstkowe tego $f$ obciętego do $int(U)$ w sposób ciągły rozszerzają się na $\partial U$.
\end{fakt}

\begin{definicja}[gładka rozmaitość z brzegiem]
$M$ jest \deff{gładką rozmaitością z brzegiem}, jeśli posiada atlas $\{(U_\alpha,\phi_\alpha)\}$ taki, że 
\begin{itemize}
    \item[\point] $U_\alpha$ jest otwartym podzbiorem $M$ 
    \item[\point] oraz $\phi_{\alpha}:U_\alpha\to H^n$ jest homeomorfizmem na swój obraz, 
    \item[\point] $\overline U_\alpha=\phi(U_\alpha)\subseteq H^n$ jest otwarty,
    \item[\point] odwzorowania przejścia $\phi_\alpha\phi_\beta^{-1}:\phi_\beta(U_\alpha\cap U_\beta)\to\phi_\alpha(U_\alpha\cap U_\beta)$ są gładkie [$U_\alpha\cap U_\beta\subseteq H^n$ otwarte].
\end{itemize}
\end{definicja}

\begin{fakt}[jeśli obraz punktu jest w rzegu w jednej mapie, to jest w brzegu w każdej]
Jeśli w pewnej mapie $(U_\alpha,\phi_\alpha)$ $\phi_\alpha(p)\in\partial H^n$, to w każdej innej mapie $(U_\beta,\phi_\beta)$ zawierającej punkt $p$ również obraz punktu $p$ należy do brzegu $H^n$.
\end{fakt}

\textbf{Dowód:}

Odwzorowania przejścia są gładkie, ale gładkie są też odwzorowania odwrotne, czyli $\phi_\alpha\phi_\beta^{-1}$ są gładkie i gładko odwracalne.

Twierdzenie o odwzorowaniu otwartym z analizy wielu zmiennych

Odwzorowania przejścia mają nieosobliwe macierze pierwszych pochodnych cząstkowych we wszystkich punktach.
\proofend
\medskip

\begin{uwaga}[fakt wyżej jest prawdziwy dla rozmaitości topologicznych z brzegiem]
    Dla rozmaitości topologicznych z brzegiem (ta sama definicja, tylko odwzorowania przejścia nie muszą być gładkie, a wystarczy homeomorfizmy) dowód wyżej nie śmignie, ale \dyg{analogiczny fakt również zachodzi}, tylko dowód jest trudniejszy i opiera się na twierdzeniu Brouwera o niezmienniczości obszaru (analog twierdzenia o odwzorowaniach otwartych dla ciągłych $fLR\to\R^n$)
\end{uwaga}

Dzięki twierdzeniom powyżej następujące definicje mają sens:
$$\color{blue}\partial M=\{p\in M\;:\;\text{w pewnej mapie (każdej) }\phi_\alpha(p)\in\partial H^n\}$$ 
$$\color{blue}int(M)=\{p\in M\;:\;\text{dla pewnej mapy }(U_\alpha,\phi_\alpha),\;\phi_\alpha(p)\in int(H^n)\}$$

\subsection{O brzegu i wnętrzu}

\begin{fakt}[wnętrze rozmaitości jest rozmaitością]
    Wnętrze $int(M)$ $n$-rozmaitości gładkiej $M$ jest $n$-rozmaitością gładką bez brzegu.
\end{fakt}

\textbf{Dowód:} 

Pokażemy atlas, który działa dla $int(M)$. Weźmy $\{(U_\alpha',\phi_\alpha')\}$, gdzie
$$U_\alpha'=U_\alpha\cap int(M),\quad \phi_\alpha'=\phi_\alpha\obciete U_\alpha$$
a $(U_\alpha,\phi_\alpha)$ było atlasem na $M$.
\proofend

\begin{fakt}[brzeg rozmaitości jest rozmaitością]
    Brzeg $\partial M$ $n$-rozmaitości $M$ z brzegiem jest $(n-1)$ wymiarową rozmaitością gładką bez brzegu.
\end{fakt}

\textbf{Dowód:}

Jako atlas na $\partial M$ bierzemy $\{(U_\alpha',\phi_\alpha')\}$, gdzie 
$$U_\alpha'=U_\alpha\cap\partial U_\partial M$$
$$\phi_\alpha':U_\alpha'\to\R^{n-1}=\partial H^n\quad \phi_\alpha'=\phi_\alpha\obciete U_\alpha'$$
\proofend

\textbf{Przykład:} Dysk $D^n=\{x\in\R^n\;:\;|x|\leq 1\}$ jest rozmaitością gładką z brzegiem $\partial D^n=\{x\in\R^n\;:\;|x|=1\}$. Pokażemy mapy, ale uzasadnienie ich gładkiej zgodności pominiemy.
$$(U_0,\phi_0)\quad:\quad U_0=\{x\;:\;|x|<1\},\quad\phi_0:U_0\to H^n,\;\phi_0(x_1,...,x_n)=(x_1,...,x_{n-1},x_n+2)$$

\begin{illustration}
    \filldraw[fill=black!70!green!80] (0, 0) circle (1);
    \filldraw[fill=black!70!green!80] (5, 1) circle (1);
    \draw[->] (0, -2)--(0, 3);
    \draw[->] (-2, 0)--(2.5, 0);
    \draw[->] (5, -2)--(5, 3);
    \draw[->] (3, 0)--(7.5, 0);
    \draw[->] (1, 0.5)..controls (1.5, 1.5) and (3, 1.8)..(3.8, 1) node [midway, above] {$\phi_0$};
\end{illustration}

$$(U_i^\pm,\phi_i^\pm)\quad:\quad U_i^\pm=\{x\in D^n\;:\;\pm x_i>0\},\quad\phi_1:U_1\to H^n$$

\begin{illustration}
    \draw[->] (-2, 0)--(3, 0);
    \draw[->] (0, -3)--(0, 3);
    \draw (0, 0) circle (1.5); 
    \draw[very thick] (1.5, -3)--(1.5, 3);
    \node at (2.1, -1.9) {$\R^{n-1}$};
    \node at (2.4, -2.5) {$\{x_i=1\}$};
    \draw (0, 0)--(1.5, 2);
    \draw[very thick] (0, 0)--(0.5, 0.66) node [midway, left] {$r$};
    \node at (0.5, 0.66) {$\bullet$};
    \node at (0.5, 1) {$p$};
    \node at (1.5, 2) {$\bullet$};
    \node at (2, 2) {$\pi(p)$};
\end{illustration}
Czyli w punkcie opisujemy $n-1$ wymiarową płaszczyznę styczną i rzucamy punkty $p\in D^n$ przez rzut odśrodkowy $\pi$ na tę płaszczyznę. Funkcje $\phi_i^\pm$ opisują się wtedy wzorem:
$$\phi_i^\pm(p)=(\pi(p), 1-r^2)$$
lub konkurencyjnie
$$\phi_i^\pm(x_1,...,x_n)=(\frac {x_1}{x_i},...,\frac {x_{i-1}}{x_i},\frac {x_{i+1}}{x_i},...,\frac {x_n}{x_i}, 1-\sum\limits_{i=1}^nx_i^2)$$

Inny atlas gładki na dysku $D^n$ (zgodny z poprzednim)
\begin{illustration}
    \filldraw[color=back, pattern={Lines[distance=5pt,line width=.8pt,angle=40]}, pattern color=black!70!green!80] (-1.5, -3) rectangle (-8, 3);
    \filldraw[color=back, pattern={Lines[distance=5pt,line width=.8pt,angle=40]}, pattern color=black!70!green!80] (1.5, -3) rectangle (8, 3);
    \draw[black!70!green!80](-1.5, -3)--(-1.5, 3);
    \draw[black!70!green!80](1.5, -3)--(1.5, 3);
    \draw (0, 0) circle (1.5);
    \draw[->](0, 0)--(3.3, 0);
    \draw[->](0, 0)--(0, 3.3);
    \filldraw (1.5, 0) circle (1pt) node [above right] {B};
    \filldraw (-1.5, 0) circle (1pt) node [above left] {A};
    \node at (3, -2.5) {$\{x_1\geq1\}=H_A^n$};
    \node at (-3, -2.5) {$\{x_1\leq-1\}=H_B^n$};
\end{illustration}
$$U_A=D^n\setminus\{A\}$$
$$U_B=D^n\setminus\{B\}$$
$$\phi_A:U_A\to H_A^n\leftarrow\text{inwersja względem sfery o środku A i }r=2$$

\subsection{Rozkłady jedności}

\emph{Motywacja: jak uzasadnić, że na każdej rozmaitości z brzegiem $M$ istnieje gładka funkcja $f$ taka, że $f:M\to\R^n$ taka, że}
$$\begin{matrix}
    f(p)=0&p\in \partial M\\
    f(p)>0&p\in Int(M)?
\end{matrix}$$

Na zbiorze mapowym możemy taką funkcję zadać przez:
$$\overline f_\alpha:\overline U_\alpha\to\R$$
$$\overline f_\alpha(x_1,...,x_n)=x_n$$
$$f_\alpha:U_\alpha\to\R$$
$$f_\alpha=\overline f_\alpha\circ \phi_\alpha$$
Czyli zmuszamy funkcję do bycia gładką.

{\large\color{orange}TU PEWNIE JAKIŚ BULLSHIT PISZĘ, DOCZYTAĆ I POPRAWIĆ.}

% \begin{definicja}[rodzina lokalnie skończona]
%     Rodzina $\{A_\alpha\}$ podzbiorów przestrzeni topologicznej $X$ jest \deff{lokalnie skończona}, jeśli dla każdego $p\in X$ istnieje otwarte otocznie $p\in U_p\subseteq X$ takie, że $U_p\cap A_\alpha\neq\emptyset$ tylko dla skończenie wielu zbiorów. 
% \end{definicja}

\begin{definicja}[rodzina lokalnie skończona]
Rodzina $\{A_i\}$ podzbiorów przestrzeni topologicznej $X$ jest \deff{lokalnie skończona}, jeśli dla każdego $p\in X$ istnieje otwarte otoczenie $p\in U_p$ w $X$ takie, że $U_p\cap A_\alpha\neq\emptyset$ tylko dla skończenie wielu $\alpha$.
\end{definicja}

\begin{definicja}[nośnik funkcji]
Dla funkcji rzeczywistej $f:X\to\R$ jej \deff{nośnik} $supp(f)=cl(\{x\in X\;:\;f(x)\neq0\})$
\end{definicja}

\begin{tw}[o rozkładzie jedności]
[\dyg{Twierdzenie o rozkładzie jedności}] Dla każdego otwartego pokrycia $\{U_\alpha\}$ rozmaitości gładkiej $M$ (może być z brzegiem) istnieje rodzina $\{f_j\}_{j\in J}$ gładkich funkcji $f_j:M\to\R$ takich, że
\begin{itemize}
    \item $f_j\geq0$
    \item każdy nośnik $supp(f_j)$ zaiwera się w pewnym $U_\alpha$ z pokrycia
    \item nośniki $\{supp(f_j)\}_{j\in J}$ tworzą lokalnie skończoną rodzinę podzbiorów w $M$
    \item dla każdego $x\in M$ $\sum\limits_{j\in J}f_j(x)=1$
\end{itemize}
Jest to \deff{rozkład jedności wpisany w pokrycie $\{U_\alpha\}$}
\end{tw}

Wracamy do pytania o istnienie $f:M\to\R$ takiego, że $f\obciete\partial M\equiv 0$ i $f\obciete int(M)>0$. 

Niech $\{U_\alpha\}$ będzie dowolnym pokryciem rozmaitości $M$ zbiorami mapowymi. Wtedy $f_\alpha:U_\alpha\to\R$ jest gładka, jeśli
\begin{itemize}
    \item $U_\alpha\cap\partial M\neq\emptyset\implies f_\alpha=\overline f_\alpha\phi_\alpha$, gdzie $\overline f_\alpha:\overline U_\alpha\to\R,\; \overline f_\alpha(x_1,...,x_n)=x_n$
    \item $U_\alpha\cap\partial M=\emptyset\implies f_\alpha=1$
\end{itemize}
Niech $\{h_j\}$ będzie rozkładem jedności wpisanym w $\{U_\alpha\}$. Dla każdego $j\in J$ wybieramy $\alpha(j)$ takie, że $supp(h_j)\subseteq U_{\alpha(j)}$. Definiujemy wtedy $h_j'=h_j\cdot f_{\alpha(j)}:M\to \R$ takie, że
$$h_j'(p)=\begin{cases}
    h(p)f_{\alpha(j)}(p)\quad p\in U_{\alpha(j)}\\
    0
\end{cases}$$
taka funkcja jest gładka, bo $supp(h_j)\subseteq U_{\alpha(j)}$.
%\newpage

\section{Rozkłady jedności}
\emph{Motywacja: jak sklejać funkcje?} W szczególności, jak uzasadnić, że na każdej rozmaitości z brzegiem $M$ istnieje gładka funkcja $f:M\to\R^n$ taka, że:
$$
\begin{matrix}
f(p)=0 & p\in\partial M\\
f(p)>0 & p\in Int(M)?
\end{matrix}
$$

\subsection{Nośnik funkcji}

\begin{definition}[rodzina lokalnie skończona]
Rodzina $\{A_i\}$ podzbiorów przestrzeni topologicznej $X$ jest \important{lokalnie skończona}, jeżeli dla każdego $p\in X$ istnieje otwarte otoczenie $p\in U_p$ w $X$ takie, że $U_p\cap A_\alpha\neq\emptyset$ tylko dla skończenie wielu $\alpha$.
\end{definition}

\begin{definition}[nośnik funkcji]
Dla funkcji rzeczywistej $f:X\to\R$ jej \important{nośnik} to
$$supp(f)=cl(\{x\in X\;:\;f(x)\neq0\})$$
\end{definition}


\begin{theorem}[o rozkładzie jedności]
\emph{[Twierdzenie o rozkładzie jedności]} Dla każdego otwartego pokrycia $\{U_\alpha\}$ rozmaitości gładkiej $M$ istnieje rodzina $\{f_j\}_{j\in J}$ gładkich funkcji $f_j:M\to\R$ takich, że
\begin{itemize}
    \item $f_j\geq0$
    \item każdy nośnik $supp(f_j)$ zawiera się w pewnym $U_\alpha$ z pokrycia
    \item nośniki $\{supp(f_j)\}_{j\in J}$ tworzą lokalnie skończoną rodzinę podzbiorów w $M$.
    \item dla każdego $x\in M\;\sum\limits_{j\in J}f_j(x)=1$
\end{itemize}
Jest to \important{rozkład jedności wpisany w pokrycie $\{U_\alpha\}$}
\end{theorem}
\begin{proof}
Dla ułatwienia sprawy pokażemy prawdziwość tego twierdzenia dla rozmaitości gładkich bez brzegu. Ale to dopiero w przyszłości, bo aktualnie mi się nie chc
\end{proof}

\subsection{Rozkłady jedności w akcji}

\emph{Czy istnieje $f:M\to\R$ takie, że $f\restriction\partial M\equiv 0$ oraz $f\restriction Int(M)>0$?}

Niech $\{U_\alpha\}$ będzie dowolnym pokryciem rozmaitości $M$ zbiorami mapowymi. Wtedy $f_\alpha:U_\alpha\to\R$ jest funkcją gładką, jeżeli
\begin{itemize}
    \item $U_\alpha\cap\partial M\neq\emptyset\implies$ $f_\alpha=\hat{f}_\alpha\phi_\alpha$, gdzie $\hat{f}_\alpha:\overline U_\alpha\to \R$ i $\hat{f}_\alpha(x_1,...,x_n)=x_n$.
    \item $U_\alpha\cap\partial M=\emptyset\implies f_\alpha=1$
\end{itemize}
Niech $\{h_j\}$ będzie rozkładem jedności wpisanym w $\{U_\alpha\}$. Dla każdego $j\in J$ wybieramy $\alpha(j)$ takie, że $supp(h_j)\subseteq U_{\alpha(j)}$. Definiujemy wtedy $h_j'=h_j\circ f_{\alpha(j)}:M\to\R$ takie, że
$$h_j'(p)=
\left\{\begin{matrix}
h(p)f_{\alpha(j)}(p) & p\in U_{\alpha(j)}\\
0 & p\notin U_{\alpha(j)}
\end{matrix}\right.
$$
Taka funkcja jest gładka, bo $supp(h_j)\subseteq U_{\alpha(j)}$.

\emph{Bump function} dla domkniętego $A\subseteq M$ z nośnikiem w otwartym $U\subseteq M$ to ciągła funkcja $\psi:M\to\R$ taka, że $0\leq\psi\leq1$ na $M$ i $\psi\equiv 1$ w $A$ oraz $supp(\psi)\subseteq U$.



















\newpage

\section{Różniczkowalność odwzorowań pomiędzy rozmaitościami}

\subsection{Podstawowe definicje}
\begin{definition}[odwzorowanie różniczkowalne w punkcie]
Niech $M,N$ będą gładkimi rozmaitościami i niech $f:M\to N$ będzie ciągłe. Niech $p\in M$ i $q=f(p)$. 
\begin{enumerate}
    \item Takie $f$ jest \important{$C^r$-różniczkowalne} ($r\in\N\cup\{\infty\}$) w punkcie $p$, jeśli mapa $(U,\phi)$ wokół $p$ i $(V,\psi)$ wokół $q$ złożenie 
$$\psi\circ f\circ\phi^{-1}:\phi(U\cap f^{-1}(V))\to \psi(V)$$
jest $C^r$-różniczkowalne w punkcie $\phi(p)$. Złożenie jak wyżej oznaczamy $\color{blue}\hat{f}=\psi\circ f\circ\phi^{-1}$ nazywamy \acc{wyrażeniem $f$ w mapach $(U,\phi),(V,\psi)$}
{\large\color{orange}TUTAJ OBRAZEK}
    \item $f$ jest \important{$C^r$ na otoczeniu $p$} jeśli dla dowolnych $(U,\phi),(V,\psi)$ jak wyżej $\hat{f}$ posiada ciągłe pochodne cząstkowe rzędu $\leq r$ na pewnym otwartym otoczeniu $\phi(p)$.
\end{enumerate}
\end{definition}

\begin{fact}[różniczkowalność dla dowolnej $\iff$ dla jednej]
Jeżeli $f$ wyrażona w mapach $(U,\phi),(V,\psi)$ jest $C^r$-różniczkowalna w punkcie $\phi(p)$, to wyrażona w dowolnych mapach $(U',\phi'),(V',\psi')$ gładko zgodnych z mapami poprzednimi jest $C^r$-różniczkowalna.
\end{fact}

\begin{proof}

Niech $\hat{f}=\psi f \phi^{-1}$, $\overline{f}=\psi' f(\phi')^{-1}$. Niech $\phi(\phi')^{-1}=\alpha$, $\psi'\psi^{-1}=\beta$ będą odwzorowaniami przejścia.

Zauważmy, że $\overline{f}=\beta \hat{f}\alpha$, bo każdy umie rozpisać to sobie. Ponieważ wszystkie te funkcje są $C^r$ lub gładkie, to i całość jest $C^r$. Oczywiście pomijamy dowodzenie, że wszystkie te złożenia są dobrze określone na odpowiednich wzorach.
\end{proof}

\begin{definition}[globalna $C^r$-różniczkowalność]
    Odwzorowanie $f:M\N$ to jest [wszędzie] $C^r$-różniczkowalne, jeżeli jest $C^r$ różniczkowalne na otoczeniu każdego punktu $p\in M$.
\end{definition}

\begin{fact}[równoważna def globalnej $C^r$-różniczkowalności]\label{fact4:4}
    $f$ jest globalnie $C^r$-różniczkowalna $\iff$ dla dowolnych $(U,\phi)$ na $M$ i $(V,\psi)$ na $N$ $\psi f\phi^{-1}$ jest różniczkowalne na całej swojej dziedzinie określoności.
\end{fact}
\begin{proof}
Trywialne i pozostawiamy jako ćwiczenie.
\end{proof}

\begin{remark}[weryfikowalnie $C^r$]
    $C^r$-różniczkowalność $f$ wystarczy weryfikować tylko dla map z ustalonych atlasów na $M$ i $N$, co wynika z faktu \ref{fact4:4}
\end{remark}

\begin{fact}[złożenie gładkich jest gładkie]
    Złożenie gładkich odwzorowań pomiędzy gładkimi rozmaitościami jest gładkie.
\end{fact}

\begin{proof}
    Ustalmy, z czym tu mamy doczynienia. Niech $f:M\to N$ i $g:N\to P$ będą gładkimi odwzorowaniami między rozmaitościami. Niech $p\in M, q=f(p)\in N, s=g(q)=g(f(p))\in P$. Niech $(U,\phi),(V,\psi),(W,\xi)$ będą mapami wokół $p,q,s$. Wiemy, że $\psi f\phi^{-1}$ i $\xi g\psi^{-1}$ są gładkie. 

    Zauważmy, ze na odpowiednio mniejszym otwartym otoczeniu punktu $\phi(p)$ zachodzi następująca równość odwzorowań. Mianowicie, jeśli wyrazimy to złożone odwzorowanie $g\circ f$ w mapach $(U,\phi),(W,\xi)$, to zachodzi równość:
    $$\xi(f\circ g)\phi^{-1}=(\xi g\psi^{-1})(\psi f\phi^{-1})$$
    i to jest w jakimś podzbiorze $\R^n$, więc jest gładkie i rzeczywiste. Stąd złożenie dwóch takich funkcji jest gładkie na pewnym otoczeniu otwartym $p$. Ale to zachodzi dla dowolnego punktu $p\in M$, skąd wynika globalna gładkość.
\end{proof}

Im dalej w las będziemy coraz bardziej leniwi i zamiast pisać dowody pokroju tego co wyżej, będziemy widzieć że to z definicji i nie pisać dowodów $\star\star\star$.

\begin{fact}[rząd jakobianu jest dobrze określony]
    Dla gładkiego odwzorowania $f:M\to N$, rząd macierzy pierwszych pochodnych cząstkowych 
    $$\left(\frac{\partial(\psi f\phi^{-1})_i}{\partial x_j}(\phi(p))\right)_{i,j}$$ 
    nie zależy od wyboru map $(U,\phi),(V,\psi)$ wokół $p$ i $f(p)$.
\end{fact}
\begin{proof}Ćwiczenia\end{proof}

\begin{definition}[rząd funkcji]
\acc{Rzędem $f$ w punkcie $p\in M$} nazywamy rząd macierzy pierwszych pochodnych cząstkowych w punkcie $\phi(p)$.

Rząd w $p$ równy zero określamy też terminem, że \acc{pochodna $f$ w $p$ jest zerowa}.
\end{definition}

\begin{definition}[dyfeomorfizm]
    Gładkie odwzorowanie $f:M\to N$ jest \important{dyffeomorfizmem}, jeśli jest bijekcją i odwzorowanie odwrotne jest także gładkie. Rozmaitości między którymi istnieje dyfeomorfizm nazywamy \acc{dyfeomorficznymi} i traktujemy je jako jednakowe.
\end{definition}

\begin{fact}[wymiar dyfeomorficznych]
    Dyfeomorficzne rozmaitości mają ten sam wymiar.
\end{fact}
\begin{proof}Ćwiczenia
\end{proof}

\begin{remark}[dygresja o dyfeomorfizmach]
$ $\newline
    \begin{enumerate}
        \item $C^1$ vs $C^\infty$: pojęcia dyfeomorfizmu można zmodyfikować do $C^r$-dyfeomorfizmu.
        
        Wcześniej pokazaliśmy, że $C^1$-rozmaitość posiada $C^1$-zgodną $C^\infty$ strukturę. Jeśli dwie $C^\infty$-rozmaitości są $C^1$-dyfeomorficzne, to są również $C^\infty$-dyfeomorficzne. Stąd klasyfikacja $C^1$-rozmaitości (z dokładnością do $C^1$-dyfeomorfizmu) pokrywa się z klasyfikacją $C^\infty$-dyfeomorfizmów.
        \item $C^0$ vs $C^\infty$: $C^0$ dyfeomorfizm to po prostu homeomorfizm. 
        
        Wiemy już, że istnieją $C^0$-rozmaitości nieposiadające żadnej $C^\infty$-struktury. Istnieją $C^0$-rozmaitości posiadające wiele (parami niedyfeomorficznych) $C^\infty$ struktur. 

        Milnov pokazał, że istnieją $S^n$ dla $n\geq7$ takie, że istnieją takie parami niedyfeomorficzne strutktury. Otóż można sobie z tym zjechać do jeszcze niższych wymiarów, mianowicie Freedman i niezaleznie Donadson, że na $\R^4$ mamy nieprzeliczalnie wiele parami niedyfeomorficznych gładkich struktur. Dla wymiaróww $\leq3$ pokazano, że tak nie można egzotykować. 
    \end{enumerate}
\end{remark}

\subsection{Dyskretne ilorazy rozmaitości gładkich przez grupy dyffeomorfizmów}

\begin{definition}[grupa dyfeomorfizmów]
    \acc{Grupa $G$ dyfeomorfizmów rozmaitości $M$} to dowolny niepusty zbiór dyfeomorfizmów $g:M\to M$, który jest zamknięty na operację składania i brania odwrotności. Elementem identycznym jest $id_M$, a odwrotne to dyfeomorfizmy odwrotne. Grupa $G$ działa przez dyfeomorfizmy na rozmaitość $M$.
\end{definition}

\begin{definition}[orbita, rozbicie]
    \acc{Orbitą} punktu $x\in M$ względem działania $G$ na $M$ nazywamy zbiór
    $$\color{blue}G(x)=\{g(x)\;:\;g\in G\}$$
    Rodzina wszystkich orbit tworzy \acc[b]{rozbicie rozmaitości $M$} na podzbiory.
\end{definition}
Dwie orbity są albo całkiem rozłączne, albo pokrywają się.

\begin{definition}
    Zbiór orbit to $M/G$. $M/G$ tak naprawdę oznacze przestrzeń ilorazową działania $G$ na $M$, czyli przestrzeń topologiczną której elementami są orbity działania $G$ na $M$, zaś topologia jest \acc{ilorazowa}. To znaczy, że zbiór orbit jest otwarty w tym ilorazie $\iff$ suma tych orbit tworzy otwarty zbiór w $M$.
\end{definition}

Na przykład, jeśli $U\subseteq M$ jest otwarty, to $G(U)/G:=\{G(x)\;:\;x\in U\}$, to ten zbiór jest zbiorem otwarty w $M/G$. Co więcej, każdy otwarty zbiór w $M/G$ ma postać $G(U)/G$ jak wyżej. Czyli jeśli $\set{B}$ jest bazą na $M$, to wtedy 
$$\{G(U)/G\;:\;U\in\set{B}\}$$
jest bazą w $M/G$ [ćwiczenia].

\begin{conclusion}
    Iloraz $M/G$ zawsze posiada przeliczalną bazę na topologii.
\end{conclusion}




























\newpage 

\section{Dyskretne ilorazy rozmaitości gładkich przez grupy dyffeomorfizmów}

\begin{definition}[grupa dyfeomorfizmów]
    \acc{Grupa $G$ dyfeomorfizmów rozmaitości $M$} to dowolny niepusty zbiór dyfeomorfizmów $g:M\to M$, który jest zamknięty na operację składania i brania odwrotności. Elementem identycznym jest $id_M$, a odwrotne to dyfeomorfizmy odwrotne. Grupa $G$ działa przez dyfeomorfizmy na rozmaitość $M$.
\end{definition}

\begin{definition}[orbita, rozbicie]
    \acc{Orbitą} punktu $x\in M$ względem działania $G$ na $M$ nazywamy zbiór
    $$\color{blue}G(x)=\{g(x)\;:\;g\in G\}$$
    Rodzina wszystkich orbit tworzy \acc[b]{rozbicie rozmaitości $M$} na podzbiory.
\end{definition}
Dwie orbity są albo całkiem rozłączne, albo pokrywają się.

\begin{definition}
    Zbiór orbit to $M/G$. $M/G$ tak naprawdę oznacze przestrzeń ilorazową działania $G$ na $M$, czyli przestrzeń topologiczną której elementami są orbity działania $G$ na $M$, zaś topologia jest \acc{ilorazowa}. To znaczy, że zbiór orbit jest otwarty w tym ilorazie $\iff$ suma tych orbit tworzy otwarty zbiór w $M$.
\end{definition}

Na przykład, jeśli $U\subseteq M$ jest otwarty, to $G(U)/G:=\{G(x)\;:\;x\in U\}$, to ten zbiór jest zbiorem otwarty w $M/G$. Co więcej, każdy otwarty zbiór w $M/G$ ma postać $G(U)/G$ jak wyżej. Czyli jeśli $\set{B}$ jest bazą na $M$, to wtedy 
$$\{G(U)/G\;:\;U\in\set{B}\}$$
jest bazą w $M/G$ [ćwiczenia].

\begin{conclusion}
    Iloraz $M/G$ zawsze posiada przeliczalną bazę na topologii.
\end{conclusion}

\textbf{Przykład}: Działanie grupy $\Z$ na $\R$ określone przez: dla $k\in\Z$ $k(x)=x+k$. Wtedy 
$$\R/\Z\cong S^1$$
(sklejamy odcinki długości $1$).

\begin{definition}[działanie nakrywające]
$G$ działa na $M$ \acc{nakrywająco}, jeśli dla każdego $p\in M$ istnieje otwarte  otoczenie $p\in U\subseteq M$ takie, że rodzina obrazów $g(U)$ po $g\in G$ jest parami rozłączna $g_1(U)\cap g_2(U)=\emptyset$ dla $g_1\neq g_2$.
\end{definition}

Dla $U$ jak wyżej odwzorowanie $U\to G(U)/G$ zadane przez $x\mapsto G(x)$ jest homeomorfizmem. Z tego wynika, że dla działania nakrywającego rozmaitość $M$, \acc[i]{iloraz $M/G$ jest przestrzenią lokalnie euklidesową} tego samego wymiaru co wymiar rozmaitości $M$.

Iloraz zadany przez działanie nakrywające niekoniecznie jest rozmaitością topologiczną:

\textbf{Przykład:} Działanie $\Z$ na $\R^2\setminus\{(0,0)\}$ przez potęgi przekształcenia liniowego zadanego macierzą
$$\begin{pmatrix}2&0\\0&\frac{1}{2}\end{pmatrix}$$
jest nakrywające. Orbity wyglądają
\begin{illustration}
\draw(-0.3, 0)--(5.2, 0);
\draw(0,-0.3)--(0,5.2);
\end{illustration}
Natomiast taka przestrzeń nie jest przestrzenią Hausdorffa.

\begin{definition}
Działanie $G$ na $M$ przez dyfeomorfizmy jest
\begin{itemize}
    \item[\PHtunny] \important{wolne}, gdy dla każdego $g\in G\setminus\{id\}$ i dla każdego $x\in M$ jest $g(x)\neq x$,
    \item[\PHtunny] \important{właściwie nieciągłe} [properly dicontinuous], gdy dla każdego zwartego $K\subseteq M$ zbiór $g\in G$, że $g(K)\cap K\neq\emptyset$ jest skończony.
\end{itemize}
\end{definition}

\subsection{Kilka szybkich własności}

\begin{definition}
Dla $x\in M$ \acc{stabilizator} (podgrupa stabilizująca) to
$$Stab(x)=\{g\in G\;:\;g(x)=x\}.$$
\end{definition}

\begin{remark}
Działanie $G$ jest wolne $\iff$ dla każdego $x\in M$ $Stab(x)=\{id\}$.
\end{remark}

\textbf{Przykład:} Działanie $\Z$ na $\R^2$ przez potęgi orotów o $\frac{\pi}{n}$ nie jest wolne, bo $(0,0)$ zostaje na swoim miejscu. Natomiast to samo działanie na $\R^2$ jest już wolne.

\begin{fact}
Działanie grupy $G$ jest wolne $\iff$ dla każdego $x\in M$ odwzorowanie $G\to G(x)$ zadane przez $g\mapsto g(x)$ jest bijekcją.
\end{fact}

\begin{fact}
Gdy działanie $G$ przez homeomorfizmy na lokalnie zwartej przestrzeni topologicznej $X$ jest właściwie nieciągłe, to wówczas każda orbita $G(x)$ jest dyskretnym podzbiorem $X$ (każdy punkt z orbity posiada otoczenie otwarte $z\in U$ takie, że $U\cap G(x)=\{x\}$)
\end{fact}

Jeśli ponadto działanie to jest wolne, to jest ono nakrywające.

\begin{fact}(ważny i trudny) 
Jeśli $G$ działa przez homeomorfizmy na lokalnie zwartej przestrzeni $X$ w sposób właściwie nieciągły, to iloraz $X/G$ jest przestrzenią Hausdorffa.
\end{fact}

\textbf{Uwaga:} Działanie $\Z$ na $\R^2\setminus\{(0,0)\}$ przez potęgi $\begin{pmatrix}2&0\\0&\frac{1}{2}\end{pmatrix}$ nie jest właściwie nieciągłe. 

\begin{fact}
Jeśli $G$ działa na $M^n$ przez dyfeomorfizmy w sposób wolny, włąściwie nieciągły, to iloraz $M/G$ jest $n$-wymiarową rozmaitością topologiczną.
\end{fact}

Oznaczenie $M^n$ mówi, że $M$ jest rozmaitością $n$-wymiarową.

\subsection{Gładki atlas na $M/G$}

\phantomsection\label{gw:5:2:1}
\begin{bbox}
(\Coffeecup) $U$ jest otwarty i mapowy oraz dla każdych $g_1,g_2\in G$ różnych $g_1(U)\cap g_2(U)=\emptyset$.
\end{bbox}
\begin{itemize}
    \item[\PHtunny] Każdy $x\in M$ ma otoczenie $U$ spełniające (\Coffeecup), a stąd każda orbita $G(p)\in M/G$ ma otoczenie otwarte postaci $G(U)/G$ ze zbiorem $U$ spełniającym \hyperref[gw:5:2:1]{(\Coffeecup)}
    \item[\PHtunny] Jeśli $U$ spełnia \hyperref[gw:5:2:1]{(\Coffeecup)}, to odwzorowanie $i_u:U\to G(U)/G$ $p\mapsto G(p)$ jest homeomorfizmem. Wtedy $\phi_G:G(U)/G\to\overline{U}\subseteq\R^n$ określone przez $\phi_G=\phi\circ i_U^{-1}$ jest kandydatem na mapę na $M/G$.
\end{itemize}

\phantomsection\label{atlas:gladki:iloraz}
Niech $\set{A}$ będzie atlasem na $M$. Rozważmy rodzinę
$$A_G:=\{(G(U)/G,\phi_G)\;:\;U\text{ spełnia (\Coffeecup)},\;(U,\phi)\in\set{A}\}$$


\begin{itemize}
    \item[\PHtunny] $A_G$ jest gładko zgodny, więc jest gładkim atlasem na $M/G$
    \item[\PHtunny] odwzorowanie ilorazowe $q_G:M\to M/G$ zadane przez $q_G(x)=G(x)\in M/G$ jest gładkie
\end{itemize}

\begin{definition}[DUPAAA]
$f:M\to N$ jest \acc{lokalnym dyfeomorfizmem}, gdy każdy $x\in M$ posiada otwarte otoczenie $x\in U\subseteq M$ takie, że $f\restriction U:U\to f(U)$ jest dyfeomorfizmem na otwarty podzbiór $f(U)$.

W szczególności wymiary tych dwóch rozmaitości muszą się zgadzać.
\end{definition}

Gładka zgodność map z \hyperref[atlas:gladki:iloraz]{atlasem}. Niech $(G(U)/G,\phi_G)$ oraz $(G(V)/G,\psi_G)$ będą mapami zgodnymi z mapami $(U,\phi)$ oraz $(V,\psi)$ na $M$. Wtedy
$$\phi_G=\phi\circ i_U^{-1},\quad\psi_G=\psi\circ i_V^{-1}$$
i odwzorowanie przejścia to
$$\psi_G\circ\phi_G^{-1}:\phi_(G(U)/G\cap G(V)/G)\to\psi_G(G(U)/G\cap G(V)/G)$$
i zachodzi
$$\psi_G\psi_G^{-1}=\psi\circ i_V^{-1}\circ(\phi\circ i_u^{-1})^{-1}=\psi\circ i_V^{-1}\circ i_u\circ\phi^{-1}$$
Przyglądamy się przekształceniu
$$i_V^{-1}\circ i_U:U\cap i_U^{-1}(G(V)/G)\to V\cap i_v^{-1} (G(U)/G)$$
dla $y=i_V^{-1}\circ i_U(x)$ zachodzi $G(x)=i_U(x)=i_V(y)=G(y)$. Zatem $y=g_x(x)$ dla pewnego (jedynego, bo wolne) $g_x\in G$. 

Wiemy też, że $i_V^{-1}i_U$ jest homeomorfizmem, więc w szczególności jest ciągłe.
Z tej ciągłości wynika, że $x\mapsto g_x$ musi być stałe na komponentach spójności. Komponenty spójności $U\cap i_U^{-1}(G(V)/G)$ są otwarte w $M$. Na każdej z nich mamy $i_V^{-1}\circ i_U(x)=g(x)$ dla ustalonego $g\in G$ zależnego od komponentu.

Zatem $\psi_G\circ\phi_G^{-1}(x)=\psi\circ g\circ\phi^{-1}(x)$ dla $x$ z komponenty. Więc na tym zbiorze otwartym $\psi_G\circ\phi_G^{-1}$ jest gładkie.

Sprawdzamy własności $q_G$ w mapach $(U,\phi)$ na $M$ oraz $(G(U)/G,\phi_G)$ na $M/G$.

\textbf{Przykłady:} 
\begin{enumerate}
\item $\Z^n$ na $\R^n$ przez przesunięcia: $(k_1,...,k_n)(x_1,...,x_n)=(x_1+k_1,...,x_n+k_n)$. Można sprawdzić, że jest to działanie wolne i właściwie nieciągłe. Iloraz $\R^n/\Z^n$ jest nazywane $n$-wymiarowym torusem i jest homeomorficzne z $S^1\times...\times S^1$.
\item $\Z$ na $S^1\times\R$ (współrzędne $\theta$ na $S^1$, $n$ na $\R$. $k\in\Z$ działa przez $k(\theta,t)=((-1)^k\theta,t+k)$. To jest nic innego jak butelka Kleina.
\item $\Z$ działa na $[-1,1]\times\R$ przez $k(x,y)=((-1)^k,y+k)$ [wstęga mobiusa z brzegiem]
\item $Conf_n(M)$ - \acc{przestrzeń konfiguracyjna} $n$-elementowych podzbiorów w rozmaitości $M$. Rozważmy $\underbrace{M\times M...\times M}_{n\;razy}$ i rozważmy uogólnioną przekątną 
$$\Delta^n(M)=\{(x_1,...,x_n)\;:\;x_i=x_j\text{ dla pewnego }i\neq j\}$$
    \begin{itemize}
        \item nie wiem co się dzieje
        \item grupa permutacji $S_n$ działa na $(M\times...\times M\setminus\Delta^n(M)$ przez 
        $$\sigma(x_1,...,x_n)=(x_{\sigma(1)},...,x_{\sigma(n)})$$
        i jest to działanie wolne, właściwie nieciągłe (bo $S_n$ jest skonczona) i $S_n$ działa w ten sposób przez dyfeomorfizmy
    \end{itemize}

Mapa w $Conf_n(M)$ wokół punktu $p=(x_1,...,x_n)$: rozważmy w $M$ wszyskie parami rozłączne otoczenia $U_i$ punktów $x_i$, zbiór 
$$U_1\times...\times U_n\subseteq M\times...\times M\setminus(\Delta^n(M))$$
jest naturalnym zbiorem mapowym.

{\large\color{orange}Nie wieeem co się dzieje, nie słuuucham}
\end{enumerate}

\begin{remark}
Dla gładkiego odwzorowania
$$f:M/G\to N$$
możemy rozważać jego "podniesieni" do odwzorowania
$$\overline{f}=f\circ g:M\to N$$
Odwzorowanie $\overline{f}$ jest $G$-niezmiennicze: dla każdego $g\in G$ mamy
$$\overline{f}(x)=\overline(g(x)).$$
Odwrotnie, mając $G$-niezmiennicze gładkie $\overline{h}:M\to N$ możemy indukować z niego gładkie odwzorowanie z ilorazu $h:M.G\to N$ poprzez:
$$h(p)=\overline{h}(p'),$$
gdzie $p'\in g^{-1}(p)$ to dowolny punkt z tej $G$-orbity która jest punktem $p$ w ilorazie.
\end{remark}

\subsection{Klejenie rozmaitości}


\important{Otoczenie kołnierzowe}

\begin{theorem}
$M$-gładka $n$-rozmaitość, a $B$ to komponenta spójności brzegu $\partial M$. Wtedy istnieje dyfeomorficzne włożenie (czyli dyfeomorfizm na obraz)
$$k:B\times[0,1)\to M$$
na otwarte otoczenie $U$ komponenty $B$ w $M$, taki, że
$$k(x,0)=x$$
dla $x\in B$
\end{theorem}

\begin{definition}[klejenie rozmaitości]
$M_1,B_1\subseteq\partial M$ oraz $M_2,B_2\subseteq\partial M_2$ i $f:B_1\to B_2$ jest dyfeomorfizmem (po cichutku: $dim(M_1)=dim(M_2),dim(B_1)=dim(B_2)$)

Definiujemy
$$M_1\cup_f M_2:=M_1\sqcup M_2/\sim$$
gdzie relacja równoważności $\sim$ jest indukowana przez relację równoważności $x\sim f(x)\iff x\in B_1$
\end{definition}






















\newpage
 
\section{Pomocnik idiotów:}
\begin{longtable}{ p{8cm} p{8cm} }
    \renewcommand{\listtheoremname}{Skorowidz definicji}
    \listoftheorems[ignoreall, show={definition}]
    
    &
    
    \renewcommand{\listtheoremname}{Twierdzonkowa zabawa}
    \listoftheorems[ignore={definition}]
\end{longtable}
\end{document}
