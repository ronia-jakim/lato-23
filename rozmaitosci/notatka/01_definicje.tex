\section{Definicja rozmaitości}

%Zanim podany dokładną definicję, możemy rozważyć kilka przykładów rozmaitości różniczkowalnych:
%
%\indent \point powierzchnia, domknięta lub nie,
%
%\indent \point przestrzeniach opisanych (lokalnie) skończoną liczbą parametrów,
%
%\indent \point podzbiory $\R^n$ lub $\C^n$ zapisywalne równaniami algebraicznymi (np. $z_1^2+z_2^2+z_3^1$ w $\C^3$).
%
%Cały wykład będzie wstępnym słownikiem wokół pojęcia rozmaitości różniczkowalnej.
%

Definicję rozmaitości będziemy budowali warstwami: najpierw położymy fundamenty topologiczne, potem naniesiemy na to strukturę gładką, a na koniec rozszerzymy do pojęcia rozmaitości z brzegiem.

Zanim zajmiemy się konkretnymi definicjami, popatrzmy na kilka prostych przykładów rozmaitości:
\begin{itemize}
  \item powierzchnia, domknięta lub nie,
  \item przestrzenie opisane (lokalnie) skończoną liczbą parametrów,
  \item podzbiory $\R^n$ lub $\C^n$ zapisywane równaniami algebraicznymi (np. $z_1^2+z_2^2+z_3^2$ w $\C^3$).
\end{itemize}

\subsection{Rozmaitości topologiczne}

\begin{definition}[rozmaitość topologiczna]
Przestrzeń topologiczna $M$ jest $n$-wymiarową \important{rozmaitością topologiczną} [$n$-rozmaitością], jeżeli spełnia:
\begin{enumerate}
    \item jest Hausdorffa
    \item ma przeliczalną bazę
    \item jest \acc{lokalnie euklidesowa} wymiaru $n$, czyli każdy punkt z $M$ posiada otwarte otoczenie w $M$ homeomorficzne z otwartym podzbiorem w $\R^n$.
\end{enumerate}
\end{definition}

Warunkiem równoważnym do lokalnej euklidesowości jest istnienie otwartego otoczenia dla każdego punktu $p\in U\subseteq M$ takiego, że istnieje homeomorfizm $U\isomorphism B_r\subseteq\R^n$ [ćwiczenia].
\medskip

{\acc{\large Konsekwencje Hausdorffowości:}}
\begin{itemize}
    \item Mamy wykluczone pewne patologie, na przykład przestrzeń
\begin{illustration}
    \draw[very thick] (-3, 0)--(0, 0);
    \draw[very thick] (0, 0.5)--(3, 0.5);
    \draw[very thick] (0, -0.5)--(3, -0.5);
    \filldraw[very thick] (0, 0) circle (2pt);
\end{illustration}
nie jest rozmaitością topologiczną.
    \item Dla dowolnego punktu $p\in U\subseteq M$ i homeomorfizmu $\phi:U\to\overline U\subseteq\R^n$, jeśli $\overline{K}\subseteq\overline{U}$ jest zwartym podzbiorem $\R^n$, to $K=\phi^{-1}[\overline{K}]\subseteq M$ jest domknięty i zawarty w $M$ [ćwiczenia].
    \item Skończone podzbiory są zamknięte, a granice zbieżnych ciągów są jednoznacznie określone.
\end{itemize}

{\acc{\large Konsekwencje przeliczalności bazy:}}
\begin{itemize}
    \item \important{Warunek Lindel\"ofa:} każde pokrycie rozmaitości zbiorami otwartymi zawiera przeliczalne podpokrycie [ćwiczenia].
    \item Każda rozmaitość jest wstępującą sumą otwartych podzbiorów
    $$U_1\subseteq U_2\subseteq...\subseteq U_n\subseteq...$$
    które są po domknięciu zawarte w $M$.
    \item \acc{Parazwartość}, czyli każde pokrycie $M$ posiada lokalnie skończone rozdrobnienie.
    \begin{itemize}
        \item Rodzina $\set{X}$ podzbiorów $M$ jest \emph{lokalnie skończona} [locally finite], jeżeli każdy punkt $p\in M$ ma otoczenie, które przecina się co najwyżej ze skończenie wieloma elementami $\set{X}$.
        \item Jeśli mamy pokrycie $M$ zbiorami $\set{U}$ i bierzemy drugie pokrycie $\set{V}$ takie, że dla każdego $V\in\set{V}$ znajdziemy $U\in\set{U}$ takie, że $V\subseteq U$, to $\set{U}$ jest \acc{pokryciem włożonym/rozdrobnieniem}
    \end{itemize}
    \item Każdą rozmaitość jesteśmy w stanie zanurzyć w $\R^n$ dla odpowiednio dużego $n$.
\end{itemize}

{\acc{\large Konsekwencje lokalnej euklidesowości:}}
\begin{itemize}
    \item \ul{Twierdzenie Brouwer'a}: niepusta $n$ wymiarowa rozmaitość topologiczna nie może być homeomorficzna z żadną $m$ wymiarową rozmaitością gdy $m\neq n$.
    \item Liczba $n$ w definicji jest jednoznaczna, możemy więc określić \important{wymiar rozmaitości} jako $\dim M=n$.
\end{itemize}

Tutaj warto zaznaczyć, że zbiór pusty zaspokaja definicję rozmaitości topologicznej dla dowolnego $n$. Wygodnie jest jednak móc go czasem użyć, więc w definicji niepustość $M$ nie jest przez nas wymagana.

\begin{remark}[podzbiory to też rozmaitości]
    Każdy otwarty podzbiór $n$-rozmaitości topologicznej jest $n$-rozmaitością topologiczną [ćwiczenia].
\end{remark}

\subsection{Mapy, lokalne współrzędne}

\begin{definition}[mapa]
    Parę $(U,\phi)$, gdzie $U$ jest otwartym podzbiorem $M$, a $\phi$ to homeomorfizm
    $$\phi:U\to\overline{U}\subseteq\R^n.$$
    nazywamy \important{mapą} lub \important{lokalną parametryzacją} [coordinate chart] na rozmaitości $M$. Zbiór $U$ taki jak wyżej nazywamy \acc{zbiorem mapowym} [coordinate domain/neighborhood]. Z lokalnej euklidesowości wiemy, że \textbf{zbiory mapowe pokrywają całą rozmaitość}.
\end{definition}

Jeżeli $(U,\phi)$ jest mapą i dla $p\in M$ mamy $\phi(p)=0$, to mówimy, że mapa jest \emph{wyśrodkowana na $p$} [centered at $p$].

\begin{fact}[$n$-rozmaitość $\iff$ rodzina map pokrywających]
    Hausdorffowska przestrzeń $X$ o przeliczalnej bazie jest $n$-rozmaitością $\iff$ posiada rodzinę map $n$-wymiarowych dla której zbiory mapowe pokrywają cały $X$.
\end{fact}

\textbf{Przykład:} 

Rozważmy $S^n=\{(x_1,...,x_n)\in\R^{n+1}\;:\;\sum x_i^2=1\}\subseteq \R^{n+1}$ z dziedziczoną topologią. Z racji, że $\R^{n+1}$ jest Hausdorffa i ma przeliczalną bazę, to $S^n$ tęż spełnia te dwa warunki. Wystarczy teraz wskazać odpowiednią rodzinę map, która pokryje całe $S^n$. Dla $i=1,..., n+1$ określmy otwarte podzbiory w $S^n$
$$U_i^+=\{x\in S^n\;:\;x_i>0\}$$
$$U_i^-=\{x\in S^n\;:\;x_i<0\}$$

\begin{illustration}
    \shade[ball color=yellow, opacity=0.3] (2.3,0.3) arc (0:-180:2.3 and 0.6) arc (180:0:2.3 and 2.3);
    \shade[ball color=green, opacity=0.3] (2.3, -0.3) arc (0:-180:2.3 and 2) arc (180:0:2.3 and 0.6);
    \draw[color=yellow, opacity=0.5] (2.3,0.3) arc (0:-180:2.3 and 0.6) arc (180:0:2.3 and 2.3);
    \draw[color=green, opacity=0.5] (2.3, -0.3) arc (0:-180:2.3 and 2);
    \draw[color=green, opacity=0.5] (2.3, -0.3) arc (0:-180:2.3 and 0.6);
    \draw (0,0) circle (2);
    \draw (-2,0) arc (180:360:2 and 0.6);
    \draw[dashed] (2,0) arc (0:180:2 and 0.6);
    \node at (2.5, 2) {$\color{yellow}U_i^+\cap S^n$};
    \node at (2.5, -2) {$\color{green}U_i^-\cap S^n$};

    \draw (-2,-4) arc (180:360:2 and 0.6);
    \draw[dashed] (2,-4) arc (0:180:2 and 0.6);
    %\draw (-3, -5)--(3, -3);
    %\draw (-3, -4)--(3, -4);
    \draw[->] (0, -2.5)--(0, -3) node [midway, left] {$\phi_i^\pm$};
\end{illustration}

Określmy odwzorowania $\phi^\pm_i\;:\;U_i^\pm\to\R^n$
$$\phi_i^\pm(x)=(x_1,...,x_{i-1},\hat{x_i},x_{i+1},...,x_n).$$
Obraz tego odwzorowania to
$$\overline U_i^\pm=\phi_i^\pm(U_i^\pm)=\{(x_1,...,x_n)\in \R^n\;:\;\sum x_i^2<1\}.$$
Odwzorowanie $\phi_i^\pm: U_i^\pm\to\overline U_i^\pm$ jest wzajemnie jednoznaczne [bijekcja], bo
$$(\phi_i^\pm)^{-1}(x_1,...,x_n)=(x_1,...,x_{i-1}, \pm\sqrt{1-\sum x_j^2}, x_{i+1},...,x_n).$$
Mamy w obie strony odwzorowanie ciągłe, więc jest to homeomorfizmy z odpowiednimi zbiorami $\R^n$.

{\Large\color{orange} PRZYKŁADY Z LEE}

\subsection{Własności rozmaitości topologicznych}

Przypomnijmy najpierw kilka definicji z topologii i je poszerzmy. Mówimy, że przestrzeń topologiczna $X$ jest
\begin{itemize}
    \item \acc{spójna}, gdy nie można jej rozłożyć na sumę dwóch rozłącznych, otwartych i niepustych podzbiorów,
    \item \acc{drogowo spójna}, gdy każde dwa punkty można połączyć ciągłą ścieżką,
    \item \acc{lokalnie drogowo spójna}, gdy ma bazę zbiorów spójnych drogowo.
\end{itemize}

\begin{remark}[spójność rozmaitości topologicznych]
Jeśli przestrzeń $M$ jest rozmaitością topologiczną, to
\begin{enumerate}
    \item $M$ jest lokalnie spójna drogowo,
    \item $M$ jest spójna $\iff$ jest drogowo spójna,
    \item spójne składowe $M$ są takie same jak dorogowe spójne składowe,
    \item $M$ ma przeliczalnie wiele składowych, każda będąca otwartym podbiorem $M$ (a więc i spójną rozmaitością)
\end{enumerate}
\begin{proof}
    Punkt (1) jest prostą konsekwencją tego, że otwarte kule są spójne łukowo w $\R^n$ [ćwiczenia]. Punkty (2) i (3) wynikają w prosty sposób z (1). Punkt (4) jest powodowany punktami poprzednimi i tym, że baza $M$ jest przeliczalna. 
\end{proof}
\end{remark}

Przestrzeń topologiczna $X$ jest \important{lokalnie zwarta}, jeżeli każdy punkt ma bazę otoczeń których domknięcia są zwarte.

\begin{remark}[rozmaitości są lokalnie zwarte]
Każda rozmaitość topologiczna jest lokalnie zwarta.
\begin{proof}
    Zadanie na liście 1.
\end{proof}
\end{remark}

Przestrzeń zawierająca wszystkie homotopijne pętle zaczepione w $q\in X$ jest nazywana \acc{fundamentalną grupą} $X$ w $q$. Elementem neutralnym tej grupy jest funkcja stała $c_q(s)=q$. Dla rozmaitości topologicznych \emph{fundamentalne grupy} są przeliczalne.

