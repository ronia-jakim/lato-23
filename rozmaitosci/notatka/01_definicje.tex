\section{Definicja rozmaitości}

%Zanim podany dokładną definicję, możemy rozważyć kilka przykładów rozmaitości różniczkowalnych:
%
%\indent \point powierzchnia, domknięta lub nie,
%
%\indent \point przestrzeniach opisanych (lokalnie) skończoną liczbą parametrów,
%
%\indent \point podzbiory $\R^n$ lub $\C^n$ zapisywalne równaniami algebraicznymi (np. $z_1^2+z_2^2+z_3^1$ w $\C^3$).
%
%Cały wykład będzie wstępnym słownikiem wokół pojęcia rozmaitości różniczkowalnej.
%

Definicję rozmaitości będziemy budowali warstwami: najpierw położymy fundamenty topologiczne, potem naniesiemy na to strukturę gładką, a na koniec rozszerzymy do pojęcia rozmaitości z brzegiem.

Zanim zajmiemy się konkretnymi definicjami, popatrzmy na kilka prostych przykładów rozmaitości:
\begin{itemize}
  \item powierzchnia, domknięta lub nie,
  \item przestrzenie opisane (lokalnie) skończoną liczbą parametrów,
  \item podzbiory $\R^n$ lub $\C^n$ zapisywane równaniami algebraicznymi (np. $z_1^2+z_2^2+z_3^2$ w $\C^3$).
\end{itemize}

\subsection{Rozmaitości topologiczne}

\begin{definition}[rozmaitość topologiczna]
Przestrzeń topologiczna $M$ jest $n$-wymiarową \important{rozmaitością topologiczną} [$n$-rozmaitością], jeżeli spełnia:
\begin{enumerate}
    \item jest Hausdorffa
    \item ma przeliczalną bazę
    \item jest \acc{lokalnie euklidesowa} wymiaru $n$, czyli każdy punkt z $M$ posiada otwarte otoczenie w $M$ homeomorficzne z otwartym podzbiorem w $\R^n$.
\end{enumerate}
\end{definition}

Warunkiem równoważnym do lokalnej euklidesowości jest istnienie otwartego otoczenia dla każdego punktu $p\in U\subseteq M$ takiego, że istnieje homeomorfizm $U\isomorphism B_r\subseteq\R^n$ [ćwiczenia].
\medskip

{\acc{\large Konsekwencje Hausdorffowości:}}
\begin{itemize}
    \item Mamy wykluczone pewne patologie, na przykład przestrzeń
\begin{illustration}
    \draw[very thick] (-3, 0)--(0, 0);
    \draw[very thick] (0, 0.5)--(3, 0.5);
    \draw[very thick] (0, -0.5)--(3, -0.5);
    \filldraw[very thick] (0, 0) circle (2pt);
\end{illustration}
nie jest rozmaitością topologiczną.
    \item Dla dowolnego punktu $p\in U\subseteq M$ i homeomorfizmu $\phi:U\to\overline U\subseteq\R^n$, jeśli $\overline{K}\subseteq\overline{U}$ jest zwartym podzbiorem $\R^n$, to $K=\phi^{-1}[\overline{K}]\subseteq M$ jest domknięty i zawarty w $M$ [ćwiczenia].
    \item Skończone podzbiory są zamknięte, a granice zbieżnych ciągów są jednoznacznie określone.
\end{itemize}

{\acc{\large Konsekwencje przeliczalności bazy:}}
\begin{itemize}
    \item \important{Warunek Lindel\"ofa:} każde pokrycie rozmaitości zbiorami otwartymi zawiera przeliczalne podpokrycie [ćwiczenia].
    \item Każda rozmaitość jest wstępującą sumą otwartych podzbiorów
    $$U_1\subseteq U_2\subseteq...\subseteq U_n\subseteq...$$
    które są po domknięciu zawarte w $M$.
    \item \acc{Parazwartość}, czyli każde pokrycie $M$ posiada lokalnie skończone rozdrobnienie.
    \item Każdą rozmaitość jesteśmy w stanie zanurzyć w $\R^n$ dla odpowiednio dużego $n$.
\end{itemize}

{\acc{\large Konsekwencje lokalnej euklidesowości:}}
\begin{itemize}
    \item \ul{Twierdzenie Brouwer'a}: niepusta $n$ wymiarowa rozmaitość topologiczna nie może być homeomorficzna z żadną $m$ wymiarową rozmaitością gdy $m\neq n$.
    \item Liczba $n$ w definicji jest jednoznaczna, możemy więc określić \important{wymiar rozmaitości} jako $\dim M=n$.
\end{itemize}

Tutaj warto zaznaczyć, że zbiór pusty zaspokaja definicję rozmaitości topologicznej dla dowolnego $n$. Wygodnie jest jednak móc go czasem użyć, więc w definicji niepustość $M$ nie jest przez nas wymagana.

\begin{remark}[podzbiory to też rozmaitości]
    Każdy otwarty podzbiór $n$-rozmaitości topologicznej jest $n$-rozmaitością topologiczną [ćwiczenia].
\end{remark}

\subsection{Mapy, lokalne współrzędne}

\begin{definition}[mapa]
    Parę $(U,\phi)$, gdzie $U$ jest otwartym podzbiorem $M$, a $\phi$ to homeomorfizm
    $$\phi:U\to\overline{U}\subseteq\R^n.$$
    nazywamy \important{mapą} lub \important{lokalną parametryzacją} [coordinate chart] na rozmaitości $M$. Zbiór $U$ taki jak wyżej nazywamy \acc{zbiorem mapowym} [coordinate domain/neighborhood]. Z lokalnej euklidesowości wiemy, że \textbf{zbiory mapowe pokrywają całą rozmaitość}.
\end{definition}

Jeżeli $(U,\phi)$ jest mapą i dla $p\in M$ mamy $\phi(p)=0$, to mówimy, że mapa jest \emph{wyśrodkowana na $p$} [centered at $p$].

\begin{fact}[$n$-rozmaitość $\iff$ rodzina map pokrywających]
    Hausdorffowska przestrzeń $X$ o przeliczalnej bazie jest $n$-rozmaitością $\iff$ posiada rodzinę map $n$-wymiarowych dla której zbiory mapowe pokrywają cały $X$.
\end{fact}

\textbf{Przykład:} 

Rozważmy $S^n=\{(x_1,...,x_n)\in\R^{n+1}\;:\;\sum x_i^2=1\}\subseteq \R^{n+1}$ z dziedziczoną topologią. Z racji, że $\R^{n+1}$ jest Hausdorffa i ma przeliczalną bazę, to $S^n$ tęż spełnia te dwa warunki. Wystarczy teraz wskazać odpowiednią rodzinę map, która pokryje całe $S^n$. Dla $i=1,..., n+1$ określmy otwarte podzbiory w $S^n$
$$U_i^+=\{x\in S^n\;:\;x_i>0\}$$
$$U_i^-=\{x\in S^n\;:\;x_i<0\}$$

\begin{illustration}
    \shade[ball color=yellow, opacity=0.3] (2.3,0.3) arc (0:-180:2.3 and 0.6) arc (180:0:2.3 and 2.3);
    \shade[ball color=green, opacity=0.3] (2.3, -0.3) arc (0:-180:2.3 and 2) arc (180:0:2.3 and 0.6);
    \draw[color=yellow, opacity=0.5] (2.3,0.3) arc (0:-180:2.3 and 0.6) arc (180:0:2.3 and 2.3);
    \draw[color=green, opacity=0.5] (2.3, -0.3) arc (0:-180:2.3 and 2) arc (180:0:2.3 and 0.6);
    \draw (0,0) circle (2);
    \draw (-2,0) arc (180:360:2 and 0.6);
    \draw[dashed] (2,0) arc (0:180:2 and 0.6);
    \node at (2.5, 2) {$\color{yellow}U_i^+\cap S^n$};
    \node at (2.5, -2) {$\color{green}U_i^-\cap S^n$};

    \draw (-2,-4) arc (180:360:2 and 0.6);
    \draw[dashed] (2,-4) arc (0:180:2 and 0.6);
    %\draw (-3, -5)--(3, -3);
    %\draw (-3, -4)--(3, -4);
    \draw[->] (0, -2.5)--(0, -3) node [midway, left] {$\phi_i^\pm$};
\end{illustration}

Określmy odwzorowania $\phi^\pm_i\;:\;U_i^\pm\to\R^n$
$$\phi_i^\pm(x)=(x_1,...,x_{i-1},\hat{x_i},x_{i+1},...,x_n).$$
Obraz tego odwzorowania to
$$\overline U_i^\pm=\phi_i^\pm(U_i^\pm)=\{(x_1,...,x_n)\in \R^n\;:\;\sum x_i^2<1\}.$$
Odwzorowanie $\phi_i^\pm: U_i^\pm\to\overline U_i^\pm$ jest wzajemnie jednoznaczne [bijekcja], bo
$$(\phi_i^\pm)^{-1}(x_1,...,x_n)=(x_1,...,x_{i-1}, \pm\sqrt{1-\sum x_j^2}, x_{i+1},...,x_n).$$
Mamy w obie strony odwzorowanie ciągłe, więc jest to homeomorfizmy z odpowiednimi zbiorami $\R^n$.

{\Large\color{orange} PRZYKŁADY Z LEE}

\subsection{Własności rozmaitości topologicznych}

\subsection{Atlasy, rozmaitości gładkie [różniczkowalne]}

Na tym wykładzie nie będziemy poświęcać dużej uwagi rozmaitościom różniczkowalnym nie nieskończenie razy, więc pomimo lekkich niuansów między tymi dwoma słowami, dla nas zwykle znaczą one to samo.
\medskip

Dla funkcji $f:M\to\R$ chcemy określić, co znaczy, że \dyg{$f$ jest różniczkowalna}? Będziemy to robić za pomocą wcześniej zdefiniowanych map:
\begin{itemize}
    \item Funkcja $f$ \acc{wyrażona} w mapie $(U, \phi)$ to nic innego jak złożenie $f\circ \phi^{-1}:\overline U\to \R$. Teraz $f\circ\phi^{-1}$ jest funkcją zależącą od $n$ zmiennych rzeczywistych.
    \item Chciałoby się powiedzieć, że funkcja $f: M\to\R$ jest gładka, jeśli dla każdej mapy $(U, \phi)$ na $M$, ten fragment wyrażony w tej mapie $f\circ\phi^{-1}$ jest gładki. Niestety, tych map może być nieco za dużo.
    \item \deff{Odwzorowanie przejścia między dwoma mapami} $(U_1,\phi_1)$ i $(U_2,\phi_2)$ to funkcje $\phi_1\phi_2^{-1}$ i $\phi_2\phi_1^{-1}$ określone na $U_1\cap U_2$.
\end{itemize}

\begin{definicja}[zgodność map]
\deff{Mapy} $(U, \phi_1)$ oraz $(U, \phi_2)$ są \deff{zgodne} (gładko-zgodne), gdy odwzorowanie przejścia $\phi_1\phi_2^{-1}$ jest gładkie. Dla map \acc{$(U, \phi)$ i $(V, \psi)$} mówimy, że są one zgodne, jeśli 
\end{definicja}
\begin{itemize}
    \item $U\cap V=\emptyset$, albo
    \item $\phi\psi^{-1}:\psi(U\cap V)\to \phi(U\cap V)$ i $\psi\phi^{-1}(U\cap V)\to \psi(U\cap V)$ są gładkie.
\end{itemize}
\medskip

Warto zauważyć, że jeśli $(U, \phi)$ i $(V, \psi)$ są zgodne, to $f\circ\phi^{-1}\obciete (\phi(U\cap V))$ jest gładkie $\iff$

Odwzorowania przejściowe map są automatycznie \dyg{dyfeomorfizmami}.
\medskip

\begin{definicja}[atlas gładki]
    \deff{Gładkim atlasem} $\set{A}$ na topologicznej rozmaitości $M$ nazywamy dowolny taki zbiór map $\{(U_\alpha,\phi_\alpha)\}$ taki, że:
    \begin{enumerate}
        \item 1. zbiory mapowe $U_\alpha$ pokrywają całe $M$
        \item 2. każde dwie mapy z tego zbioru są zgodne.
    \end{enumerate}
\end{definicja}
\medskip

\textbf{Przykład:}  Rodzina map $\{(U_i^\pm,\phi_i^\pm)\;:\;i=1,2,...,n+1\}$ jak wcześniej na sferze $S^n\subseteq R^{n+1}$ tworzy gładki atlas. Wystarczy zbadać gładką zgodność tych map. Rozpatrzmy jeden przypadek: $(U_i^+,\phi_i^+),(U_j^+,\phi_j^+),i<j$. Po pierwsze, jak wygląda przekrój tych zbiorów?
$$U_i\cap U_j=\{x\in S^n\;:\;x_i>0,x_j>0\}$$
Dalej, jak wyglądają obrazy tego przekroju przez poszczególne mapy?
$$\phi_i^+(U_i\cap U_j)=\{x\in\R^n\;:\;|x|<1, x_{j-1}>0\}$$
$$\phi_j^+(U_i\cap U_j)=\{x\in\R^n\;:\;|x|<1,x_i<0\}$$
Odwzorowania przejścia to:
\begin{illustration}
    \node (L) at (0, 0) {$\phi_j^+(U_i^+\cap U_j^+)\ni(x_1,...,x_n)$};
    \node (C) at (4, -1) {$(x_1,...,x_{j-1},\sqrt{1-|x|^2},x_{j},...x_n)$};
\end{illustration}
$$\phi_i^+(\phi_j^+)^{-1}(x_1,...,x_n)=(x_1,...,x_{i-1}, x_{i+1},...,x_{j-1},\sqrt{1-|x|^2},x_j,...,x_n)$$
jest przekształceniem gładkim. Analogicznie dla drugiego odwzorowania przejścia.

\begin{definicja}[rozmaitość gładka]
    \deff{Rozmaitość gładka} to para $(M, \set{A})$ złożona z rozmaitości $M$ i gładkiego atlasu $\set{A}$ opisanego na $M$.
\end{definicja}

\acc{Uściślenie:} Często $(M,\set{A}_1)$ i $(M, \set{A}_2)$ będące rozmaitościami gładkimi określają tę samą rozmaitość.
\medskip

\begin{definicja}[zgodność map, atlasów]
Niech $\set{A}$ będzie gładkim atlasem na $M$.
\begin{enumerate}
    \item Mapa $(U,\phi)$ jest \deff{zgodna z atlasem $\set{A}$}, jeśli jest zgodna z każdą mapą z $\set{A}$.
    \item Dwa \deff{atlasy $\set{A}_1,\set{A}_2$ na $M$ są zgodne}, jeśli każda mapa z $\set{A}_1$ jest zgodna z atlasem $\set{A}_2$.
\end{enumerate}
\end{definicja}

\begin{tw}[zgodność to relacja równoważnośći]
    Relacja zgodności atlasów jest relacją równoważności.
\end{tw}

\textbf{Dowód:} Ćwiczenia.

Konwencja jest wtedy taka, że zgodne atlasy zadają tą samą strukturę gładką na $M$. W takim razie, zgodne atlasy można wysumować do jednego większego atlasu.

\begin{definicja}[atlas maksymalny]
$\set{A}$ jest \deff{atlasem maksymalnym} na $M$, jeśli każda mapa na $M$ z nim zgodna jest w nim zawarta. 
\end{definicja}

\begin{fakt}[dla każdego atlasu istnieje jedyny atlas maksymalny]
    Każdy atlas $\set{A}$ na $M$ zawiera się w dokładnie jednym atlasie maksymalnym na $M$. Zaś ten atlas maksymalny to zbiór wszystkich map na $M$ zgodnych z $\set{A}$.
\end{fakt}

\textbf{Dowód:} Ćwiczenia.

\dyg{Równoważna definicja rozmaitości gładkiej:} para $(M, \set{A})$, gdzie $M$ to rozmaitość topologiczna, zaś $\set{A}$ to pewien atlas maksymalny.
